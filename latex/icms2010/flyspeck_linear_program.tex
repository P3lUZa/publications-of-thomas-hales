
% File created 2010-06-25.
% Template for an AMS article style
\documentclass{llncs}
\usepackage{url}
\usepackage{xcolor}
\usepackage{tikz}
\usetikzlibrary{chains,shapes,arrows,positioning,backgrounds,fit}%
%
%
%%%%%%%%%%%%%%%%%%%%%%%%%%%%%%%%%%
%
\begin{document}
\mainmatter
\title{Linear Programs for the Kepler Conjecture (extended abstract)}
\author{Thomas C. Hales\thanks{Research supported by NSF grant 
0804189 and a grant from the Benter Foundation.  
The author places this abstract in the
public domain.}}
\institute{University of Pittsburgh}
%\email{hales@pitt.edu}}
%
%
\maketitle


%%%%%%%%%%%%%%%%%%%%%%



\begin{abstract} 
  The Kepler conjecture asserts that the densest arrangement of
  congruent balls in Euclidean three-space is the face-centered cubic packing, which is the
  familiar pyramid arrangement used to stack oranges at the market.
  The problem was finally solved in 1998 by a long computer proof.
  The Flyspeck project seeks to give a full formal proof of the Kepler
  conjecture.  This is an extended abstract for a talk in the formal proof session of ICMS-2010, which  will
  describe the linear programming aspects of the Flyspeck project.
\end{abstract}


\def\tikzfig#1#2#3{%
\begin{figure}[htb]%
  \centering
\begin{tikzpicture}#3
\end{tikzpicture}
  \caption{#2}
  \label{fig:#1}%
\end{figure}%
}
\def\op#1{{\operatorname{#1}}}
\newcommand{\ring}[1]{\mathbb{#1}}

The original proof of the Kepler conjecture in 1998 was about three
hundred pages long and relied long computer calculations that were
done by custom computer code~\cite{Hales:2006:DCG}.  The amount of custom computer code was
originally estimated to be about 40,000 lines, but later estimates
give a number closer to 180,000 lines~\cite{HHMNOZ}.  The larger figure includes a
large difficult-to-estimate number of duplicated lines.

The computer code consists of three separate
programs.  The first is a program that generates
all hypermaps up to isomorphism with prescribed
properties.  (A hypermap, defined below, is a combinatorial structure that 
conveniently encodes the structure of a planar graph.)  The output of this program
is a set $X$ of about $18$ thousand hypermaps.
The second is a program that proves nonlinear 
inequalities.  The third program generates and
solves linear programs.  

It is this third program
that is the subject of this extended abstract.  This third
program was considered from a formal proof
perspective in Obua's thesis~\cite{obua:phd}.
The thesis gives a formal verification of over 90\% of the cases.
Why did he not complete the remaining $10\%$ of the cases? He did not
encounter any difficulties with the technology.  Indeed, his work demonstrates
that the formal proof of linear programs is entirely feasible.

Rather, the difficulties are with the slopppy documentation in the
original proof of the Kepler conjecture.  The linear programming part
of the proof is most poorly documented part of the proof.  The
original linear programs were generated in interactive Mathematica
sessions~\cite{website:HalesKepler}.  There are voluminous notes about these interactive sessions,
but these notes are not in the form of executable code.  In the end, the remaining cases were not formalized because it was not clear what precisely was to be formalized.


\section{Linear Programs}

The linear programming part of the proof has been reworked from the
start with the formal proof in mind~\cite{website:FlyspeckProject}. The methods described in this extended
abstract treat 99.93\% of the cases (that is, all but $12$ hypermaps).

The linear programs are specified in the {\it MathProg} language,
which is a domain specific language designed for linear programming in the GNU Linear Programming Kit (GLPK)~\cite{website:GLPK}.
MathProg is a subset of the {\it AMPL} language~\cite{ampl}.  According to the
design of MathProg, the linear program is split into two parts: a
model and data.  The model contains the declarations of variables,
parameters, and indexing sets.  There is a single model file shared by
all linear programs.  There is a separate data file for each linear
program.  The model is further divided into a header and a body.

The computer code for this project has been written in Objective Caml.
The data files as well as the body of the model are automatically
generated.  The computer code has not yet been formalized, but it is now
in a formalization-ready state.  There are a few different parts to the code.

\begin{itemize}
\item (200 lines) interface with GLPK,
\item (200 lines) MathProg format model header,
\item (80 lines) model body code generator,
\item (300 lines) data generator and control flow.
\end{itemize}

The are two additional data files:

\begin{itemize}
\item (4MB) a set $X$ of about $18$ thousand hypermaps,
\item (1000 lines) archive of nonlinear inequalities (in HOL Light format).
\end{itemize}



This is a significant improvement over the 
original program from the 1998 proof of the Kepler
conjecture, which involved several thousand lines
of computer code, 3GB of data, and long interactive
sessions.

\section{The Main Theorem}

As we mentioned above, in the original proof of the Kepler conjecture,
it is difficult even to state the theorem that has been proved by the
linear programming part of the proof.  Here we give a simple
statement that captures the linear programming part of the proof of
the Kepler conjecture.

\begin{definition} A hypermap is a finite set $D$ with two
  permutations $n,f:D\to D$.  A third permutation $e$ is defined by
  the relation $e n f = I$.  We represent the hypermap as a tuple
  $x=(D,e,n,f)$.
\end{definition}

Let $X$ be the set of over $18$ thousand hypermaps that is mentioned above.
Each of these hypermaps can be augmented by various {\it markings} to
produce what we call marked hypermaps.  Let $Y$ be the finite set of
all marked hypermaps and let $Y_x\subset Y$ be those that come from
$x\in X$.  Certain subsets of $Y_x$ are called {\it covers} of $x$.  Associated with each marked hypermap $y\in Y$ is a polyhedron $P(y)$.
The main theorem about linear programs takes the following form.

\begin{theorem} For every hypermap $x\in X$, there exists a cover $U\subset Y_x$
such that for every $y\in U$, the polyhedron $P(y)$ is empty.
\end{theorem}

The proof is by a direct construction carried out by computer.  Each
polyhedron is explicitly presented as a finite system of linear
inequalities.  Linear programming methods show that each system of
linear inequalities has no solutions.  The existence of a cover is
established by a direct computer search.  
It takes about $2.5$ hours to run the program
on a laptop computer.

Finally, we describe the relationship between this theorem and the
Kepler conjecture.  The Kepler conjecture is initially expressed as a
statement about packings of congruent balls in an unbounded region of
space.  Various reduction arguments reduce the proof of the conjecture
to a conjecture about packings of finitely many balls.  

The proof of the Kepler conjecture is by contradiction.
If the Kepler conjecture is false, then there exists a finite packing
$V$ of at most $15$ balls that has various remarkable properties.  The
balls in the packing form the set of nodes of a graph $(V,E)$, and the
combinatorial properties of the graph can be encoded as a hypermap $x$
in the set $X$.  With a counterexample $V$ in hand, for every cover $U$ of
$x$, it is possible to find a marked hypermap $y\in U$ for which the
polyhedron $P(y)$ is nonempty.  The existence of a
counterexample $V$ contradicts the theorem, which asserts on the contrary that
$P(y)$ is empty.

\raggedright
\bibliographystyle{amsplain} %}plainnat}
\bibliography{../bibliography/all}
\end{document}




