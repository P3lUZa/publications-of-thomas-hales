%\documentclass[brochure,12pt]{bourbaki}% Pour un expose en franais
\documentclass[brochure,english,12pt]{bourbaki}% Pour un expos en anglais
\usepackage[matrix,arrow]{xy}
\usepackage{amssymb,amsfonts,amsmath,footnote}
\usepackage[francais]{babel}
\addressindent 100mm

\date{Avril 2011}
\bbkannee{63\`eme ann\'ee, 2010-2011}
\bbknumero{1035}
\title{TITLE}
\subtitle{after ...}
\author{Thomas HALES}
\address{University of Pittsburgh\\
Department of Mathematics\\
Pittsburgh, PA 15260 -- U.S.A.}
\email{tchales@gmail.com}


\begin{document}
\maketitle

\noindent{\bf INTRODUCTION}

\bigskip
...


\section{Tore standard et fusion}
\subsection{D\'efinitions et exemples}
Soient ...
\begin{prop}
\label{exposants}
...
\end{prop}

\begin{rema} ... \end{rema}

\subsection{Exemples de sous-groupes de...}

\subsection{Le tore standard et ...}

\begin{theo} [\cite{Se}]
\label{cascade}
\end{theo}

\noindent{\sc Preuve} (esquisse) --- 
\begin{defi}...
\end{defi}


\section{Sous-groupes finis (g\'en\'eralit\'es)}

\subsection{Un r\'esultat pr\'eliminaire}

\subsection{Classe anticanonique et ...}

\[
\xymatrix{
       0 \ar[r] & N \ar[r]^{i} \ar[d]_f & M \ar[r]^{p} \ar[d]_g &
                  P \ar[r] \ar[d]^h  & 0 \\
       0 \ar[r] & N' \ar[r]^{i'} & M' \ar[r]^{p'} & P' \ar[r]  & 0 }
\]


\begin{thebibliography}{666}

\bibitem[Se]{Se} J-P. SERRE -- {\it Le groupe de Cremona et ses sous-groupes finis},
S\'em. Bourbaki 2008/09, Exp. n$^\circ$~1000, Ast\'erisque {\bf 332} (2010), 75--100.

\end{thebibliography}
\end{document}
n{