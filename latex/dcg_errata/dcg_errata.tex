% DCG Kepler Errata
% Author: Thomas C. Hales
% Affiliation: University of Pittsburgh
% email: hales@pitt.edu
%
% latex format

% History:
% May 1, 2007
% Dec 18, 2007, Long deformation fix (biconnectedness proof).
% New Inequalities marked HYPOTHESIS.
% New Inequalities in file kep_ineq_bis.ml.

% Nov 2, 2008 New proof of biconnectedness is now in revision.tex.
% There is no need for the proof here. 
% Revision 646 contains old proof of biconnectedness.
% It is deleted in 647.


\documentclass[11pt]{amsart}
%\documentclass{llncs}
\usepackage{graphicx}
\usepackage{amsfonts}
\usepackage{amscd}
\usepackage{amssymb}
\usepackage{alltt}
\usepackage{url}
%\usepackage{amsmath,amsthm}


% Math notation.
\def\op#1{{\text{#1}}}
\newcommand{\mc}[1]{{\mathcal{#1}}}
\newcommand{\ring}[1]{\mathbb{#1}}
\def\to{{\quad\Longrightarrow\quad}}
\def\line{$\ell$}
\def\text{\hbox}

\newtheorem{definition}[subsubsection]{Definition}
\newtheorem{thm}[subsubsection]{Theorem}
\newtheorem{lemma}[subsubsection]{Lemma}
\newtheorem{assumption}[subsubsection]{Assumption}
\newtheorem{corollary}[subsubsection]{Corollary}
\newtheorem{remark}[subsubsection]{Remark}


\parindent=0pt
\parskip=\baselineskip

%%%%%%%%%%%%%%%%%%%%%%%%%%%%%%%%%%

\begin{document}

\title{Errata in ``The Kepler Conjecture''}

\author{Thomas C. Hales}

\address{Math Department, University of Pittsburgh}


\maketitle

\section{Introduction}

This article lists errata in the proof of the Kepler
conjecture.  A list of corrections already appears in~\cite{RKC}.
This article is a supplement to that list and does not repeat
corrections that already appear there.


\subsection{Relation between the Abridged and Unabridged Versions}

The abridged version of the Kepler conjecture
in the Annals \cite{A}
was generated by the same tex
files as the unabridged version in \cite{DCG}.


There are a few differences in wording that
were required between the two versions.
These are formatting
differences, different naming
conventions for the sections and subsections,
different conventions for references and citations,
and so forth.
The two articles also carry minor differences
in wording of transitional phrases that
accommodate the slightly different organizational
structure of the two documents.  A simple
tex macro was used to generate the occasional passage
that differs.

Because of the way these documents were produced
from the same tex files,
it seems that nearly every correction to
the abridged version will also be a correction to the unabridged version.
(So far, no errors have been found that are
unique to the abridged version.)
For that reason, we list the errata for the
unabridged version. The same list applies to corresponding 
passages in the abridged version.  




\subsection{Format}

Each correction gives its location in \cite{DCG}.
The location
\line+n counts down from the top of the page, or
if a section or lemma number is provided, it
counts from the top of that organizational unit.
The location \line-n counts up from the bottom
of the page. Footnotes are not included in the
count from the bottom.  Every line containing
text of any sort is included in the count,
including displayed equations, section headings,
and so forth.


\section{Errata}

All corrections to the main text have already been reported
in~\cite{RKC}.


\section{Index}

The index should have additional entries.
\begin{itemize}
	\item [p.128] distinguished edge
	\item [p.128] special
\end{itemize}

\section{Code}

\subsection{Inequalities}  

SPVI2002 refers to the 2002 version on the math arXiv of the paper
``Kepler Conjecture''  \url{http://arxiv.org/pdf/math/9811078}.
Since the inequalities are not listed in the DCG proof, we use the 2002 version as the standard reference for the inequalities.  The corrections are being maintained in the file {\tt kep\_inequalities.ml} in the repository at \url{https://github.com/flyspeck/flyspeck/}. 

\smallskip

181462710, SPVI2002, Group14.  The lower bound on $y_4$ should be 2.51 (typo).
(Thanks to STM.)

252231882, SPVI2002, Group 15.  The upper bound on $y_6$ is 3.51. (typo)
Also there is an implicit constraint $\eta_{456}\ge\sqrt2$.
(Thanks to STM.)

594246986, SPVI2002, Group 16, page 49.  There is an implicit constraint of $\eta_{456}\le \sqrt2$.  Also there is a bug in the verification code.  The domain is incorrectly listed (the constant $7.29$ should not appear).  (Thanks to STM.)

SPVI2002, SPVI2002, Group 16, page 49.  There is an implicit constraint of $\eta_{456}\le\sqrt2$ on this entire group of inequalities.
(Thanks to STM.)

256893386, SPVI2002, Group 17, page 49.  There is a difference between the verification code and the text.  In \url{http://www.math.pitt.edu/~thales/kepler98/interval/partK.cc}, the $y_6$ coefficient is $0.115$.  In the paper it is $0.12$.
(Thanks to STM.)

900212351, SPVI2002, Group 17, page 50.  The verification code has little relation to what is stated in the text.  This is a source of a potential error in the proof. (Thanks to STM.)

381970727, SPVI2002, Group 16, page 49.   There is a difference between the verification code and the text.  In \url{http://www.math.pitt.edu/~thales/kepler98/interval/partK.cc}, the lower bound on x4 is 7.29  In the paper it is 6.3001.  This is a potential source of error in the 1998 linear programs.  (Thanks to STM.)

319046543, SPVI2002, Group 7, page 44, number 23.  There is a sign error.
Replace $\nu_\Gamma$ with $-\nu_\Gamma$.  In the interval arithmetic
code partK.cc, it is stated correctly; and it seems to have been used correctly throughout the proof.  It is an isolated typo.   (Thanks to STM.)

900212351, SPVI2002, Group 17, page 50.  The interval arithmetic code
in partK.cc contains a bug.  
It was incorrectly copied from a different section.
STM provides the following counterexample :
  $$
  x = {6.3001, 4,4, 4.20260782962, 7.6729, 7.6729}.
  $$
  
621852152, SPVI2002, Group 18.4, page 60.
This inequality is false as stated.  
The problem seems to be a missing hypothesis,
that $|v_2-v_4|\ge\sqrt2$.  This can be
stated in terms of dihedral angles
$$
\op{dih}(0,v_3,v_2,v_1) + \op{dih}(0,v_3,v_1,v_5) + \op{dih}(0,v_3,v_5,v_4) \ge \op{dih}(a_3,a_2,a_4,\sqrt{8},2,2).
$$
This hypothesis has been inserted.
This extra hypothesis is satisfied
   when the inequality is used, so it doesn't affect
   the proof.
Thanks to STM.

207203174, SPVI2002, Group 18.4, page 60. 
Inequality is false at 
  $$\{a_1,a_2,a_3,a_4,a_5\}=\{2,2,2.51,2,2.51\}; \quad
   b_5=\sqrt{8};\quad  \{y,y'\}={3.2,3.9086};
   $$
   Note that 
   {\tt Solve[Delta[2, 2.51, 2.51, y, 2, 2] == 0, y]} gives a zero
   near $y = 3.90866$.
   tauVc drops rapidly as $x'$ increases in the range $[3.9,3.9086]$.
   It is still true by a considerable margin at $y'=3.9$.
   Ferguson's verification code appears at 
   source/section\_a46\_1c.c, but I haven't
   located the bug. (Thanks to STM.)
   McLaughlin also reports other values of $\{a_1,\ldots,a_5\}$ that give
   counterexamples.
   This inequality is corrected the same way
   as 621852152, with an added dihedral angle
   constraint.  This extra hypothesis is satisfied
   when the inequality is used, so it doesn't affect
   the proof.

354217730, SPVI2002, Group 18.9, page 60.
The constant $\sqrt2$ is a typo.  It should be $\sqrt8$.  (Thanks to STM.)
Ferguson's interval verification of this inequality appears
in source/section\_a46\_2c.c.

713930036, SPVI2002, Group 18.12, page 61.
There is a typo on the dihedral angles.  It should be
  $$
  \op{dih}(0,v_3,v_4,v_5)+\op{dih}(0,v_3,v_5,v_4)+\op{dih}(0,v_3,v_1,v_2)
  \ldots
  $$
The typo is entirely localized.

292827481, SPVI2002, Group 18.16, page 61.
McLaughlin notes a counterexample to this inequality at
$$
\{y_1,\ldots,y_6\} = \{2, 2., 2., 2., 3.2, 2.51\}.
$$
The interval verification code in partK.cc lists the constant term of
the inequality incorrectly.  This is a bug in the code.

640248153, SPVI2002, 17.19, Group 19, page 50.
This should include the constraint that $\Delta\ge0$. (Thanks to STM.)

131574415, SPVI2002, 17.27, Group 27, page 53.
This should include the constraint that $\Delta\ge0$. (Thanks to STM.)

559676877,  SPVI2002, 17.27, Group 27, page 53.
This should include the constraint that $\Delta\ge0$. (Thanks to STM.)

836331201, SPVI2002, 17.31, Group 31, page 54.
This should include the constraint that $\Delta\ge0$. (Thanks to STM.)

327474205, SPVI2002, 17.32, Group 32, page 55.
This should include the constraint that $\Delta\ge0$. (Thanks to STM.)

498007387, SPIII2002, A.6.7', page 19.
McLaughlin gives the counterexample:
$$
 y = \{2.13,2.13,2.13,2.08333,2.08333,2.08333\}.
$$
The interval verification is in part3.cc (numbered as 465988688).
The interval verification contains an extra disjunct in conclusion.
$$6.13 < y_1 +  y_2 + y_3$$
that is not stated in SPIII2002. 
Thus, we add the precondition
$$y_1 + y_2 + y_3 \le 6.13$$
to the inequality.


\subsection{Mathematica Code}

The Mathematica code at the Annals website
(\url{http://annals.math.princeton.edu/keplerconjecture/sphere.txt}) needs to be updated.

[sphere.txt,global]
The function $\op{Norm}$ defined in sphere.txt
has become a Mathematica built-in function.
We need to rename our function: 
	$$\op{Norm} \to \op{norm}$$
	
[sphere.txt,\line+262]
The file {\it more.m} is not part of the distribution:
	$$\ll \op{more.m} \to \op{(* -- *)}$$



%\input{dcg_errata_cut}


\begin{thebibliography}{}

%% References
\bibitem{A} {T. C. Hales}, A proof of the Kepler
	conjecture, Annals of Mathematics,
	2006.
	
\bibitem{DCG} {S. Feguson and T. Hales},
	The Kepler Conjecture, Disc. and Comput.
	Geom. 36 (1), July 2006.

\bibitem{RKC} {T. C. Hales, J. Harrison, S. McLaughlin, T. Nipkow, S. Obua, R. Zumkeller}, A Revision of the proof of the Kepler Conjecture, preprint, 2008.


\end{thebibliography}

Please report further errors to
Thomas C. Hales (\email{\url{http://www.math.pitt.edu/~thales}}).

\bigskip



\end{document}
