% 
% Author: Thomas C. Hales
% Affiliation: University of Pittsburgh
% email: hales@pitt.edu
%
% latex format

% History.  File started Oct 7, 2012
% Course notes for Math 2501, Algebra 2, University of Pittsburgh.
%


\documentclass{llncs}
\pagestyle{headings} 
\usepackage{verbatim}
\usepackage{graphicx}
\usepackage{amsfonts}
\usepackage{amscd}
\usepackage{amssymb}
\usepackage{amsmath}


\usepackage{alltt}
\usepackage{rotating}
\usepackage{floatpag}
 \rotfloatpagestyle{empty}
\usepackage{graphicx}
%\usepackage{multind}\ProvidesPackage{multind}
\usepackage{times}

% my additions
\usepackage{verbatim}
\usepackage{latexsym}
\usepackage{crop}
\usepackage{txfonts}
\usepackage[hyphens]{url}
\usepackage{setspace}
\usepackage{ellipsis} % 
% http://www.ctan.org/tex-archive/macros/latex/contrib/ellipsis/ellipsis.pdf 

\usepackage{amsfonts}
\usepackage{amsmath}
%tikz graphics
%\usepackage{xcolor} % to remove color.
\usepackage{tikz} % Needs pgf version >= 2.10.
\usetikzlibrary{chains,shapes,arrows,shadings,trees,matrix,positioning,intersections,decorations,
  decorations.markings,backgrounds,fit,calc,fadings,decorations.pathreplacing}
% fonts

% fonts
\usepackage[mathscr]{euscript} % powerset.
\usepackage{pifont} %ding
\usepackage[displaymath]{lineno}
\usepackage{rotating}

% theorems


% Math notation.
\def\op#1{{\hbox{#1}}} 
\def\tc{\hbox{:}}
%\newcommand{\ring}[1]{\mathbb{#1}}
\def\amp{\text{\&}}
\def\bq{\text{\tt {`}\,}}
\def\true{\text{true}}
\def\false{\text{false}}
\def\princ#1{\smallskip\hfill\break\smallskip\centerline{\it #1\hfill}}
\def\l{<}
% Flags

%%%%%%%%%%%%%%%%%%%%%%%%%%%%%%%%%%


% This file contains local settings and system dependencies

\def\displayallproof{t} 
% t (default): display all proofs.
% f: print documents without the proofs-- theorem statements only

\def\displayrating{f}
% t (default): display all ratings (verbose is also true)
% f : don't show them.

\def\verbose{t}
% f (default): do not display debugging information,
% t : display debug information and information about the formalization.



%% DEPRECATED

% Auxiliary directories -- deprecated.
%\def\dsp{/Users/thomashales/Pictures/mathFigures/DenseSpherePackings}  % flypaper graphics
%\def\pdf{/Users/thomashales/Pictures/collect_geom} % tarski graphics
%\def\pdfp{/Users/thomashales/Pictures/mathFigures/collect_geom} % kepler graphics
%\def\pdfp{.}

% deprecated.
%\def\showgraphics{f}  
% t: display graphics (not currently available), keep "f" for now.
% f (default): print a "no graphics logo" where graphics would normally go.

%% Macros

%%
%% Sam's Macros:
\def\myscorept{\text{ \sl pt}}
\def\qrtet{{quasi-regular tetrahedron}}
\def\qrtets{{quasi-regular tetrahedrons}}
\def\tomcite{{}}

 \DeclareMathOperator{\myscore}{\sigma}
%\DeclareMathOperator{\sol}{sol} \DeclareMathOperator{\dih}{dih}
%\DeclareMathOperator{\vor}{vor}
%\DeclareMathOperator{\octavor}{\sigma}
%\DeclareMathOperator{\octavor}{octavor}
 \DeclareMathOperator{\gma}{gma}
 \DeclareMathOperator{\score}{\sigma}
 \DeclareMathOperator{\sol}{sol}
 \DeclareMathOperator{\dih}{dih}
 \DeclareMathOperator{\vor}{vor}
%\DeclareMathOperator{\octavor}{\sigma}
%\DeclareMathOperator{\octavor}{octavor}
%\DeclareMathOperator{\gma}{gma}

%theorem definitions, etc.

%\theoremstyle{plain}% default


%\theoremstyle{remark}
%\newtheorem{calc}{Calculation}[subsection]
%\newtheorem*{remark}{Remark}
%\newtheorem{remark}{Remark}

%\theoremstyle{remark}
\newtheorem{calcf}{Calculation}[subsection]
%\newtheorem*{remark}{Remark}
%%TCH%%\newtheorem{remark}{Remark}


%% END Sam's Macros




\font\twrm=cmr8

% mathcal
\def\CalE{{\mathcal E}}
\def\CalB{{\mathcal B}}
\def\CalQ{{\mathcal Q}}
\def\CalW{{\mathcal W}}
\def\CalS{{\mathcal S}}
\def\CalR{{\mathcal R}}

% mathbb
\def\R{{\mathbb R}}
\def\N{{\mathbb N}}
\newcommand{\ring}[1]{\mathbb{#1}}
\def\A{{\mathbf A}}
\def\Rp{\ring{R}^{3\,\prime}}

% textsc
\def\calc#1{{\textsc{calc-#1}}}
\def\conseq#1{{\textsc{conseq-#1}}}

% operatorname
\def\op#1{{\operatorname{#1}}}
\def\sol{\operatorname{sol}}
\def\quo{\operatorname{quo}}
\def\anc{\operatorname{anc}}
\def\cro{\operatorname{crown}}
\def\vor{\operatorname{vor}}
\def\svor{\operatorname{s-vor}}
\def\sign{\operatorname{sign}}
\def\octavor{\operatorname{octavor}}
\def\dih{\operatorname{dih}}
\def\Adih{\operatorname{Adih}}
\def\arc{\operatorname{arc}}
\def\rad{\operatorname{rad}}
\def\gap{\operatorname{gap}}
\def\sc{{\operatorname{sc}}}
\def\geom{{\operatorname{g}}}
\def\anal{{\operatorname{an}}}
\def\PM{\operatorname{PM}}

\def\SA{A}
\def\SB{B}
\def\SC{C}
\def\SCp{C'}
\def\del{\partial}
\def\doct{\delta_{oct}}
\def\dtet{\delta_{tet}}
\def\pt{\hbox{\it pt}}
%\def\Vol{\hbox{vol}}
\def\scoregoal{8\,\pt}
\def\maxpi{\pi_{\max}}
\def\tausc{{\tau\!\operatorname{sc}}}
\def\piF{{\pi_F}}
\def\xiG{\xi_\Gamma}
\def\piG{\pi_\Gamma}
\def\xiV{\xi_V}
\def\xik{\xi_\kappa}
\def\xikG{\xi_{\kappa,\Gamma}}
\def\piV{\pi_V}
\def\tauLP{{\tau_{\hbox{\twrm LP}}}}
\def\DLP{\operatorname{D}_{\hbox{\twrm LP}}}
\def\ZLP{\operatorname{Z}_{\hbox{\twrm LP}}}
\def\tlp{\tau_{\hbox{\twrm LP}}}  % 2 args (p,q) tri, quad
\def\sLP{\sigma_{\hbox{\twrm LP}}}  % 2 args (p,q) tri, quad
\def\squander{(4\pi\zeta-8)\,\pt}
\def\tildeF{{\hbox{$\tilde F$}}}
\def\zloop{{z}_{loop}}
\def\dloop{{\delta}_{loop}}
\def\sqr{\sqrt}
\def\tildeV{V^{\CalS}}

\def\x{\relax}
\def\was#1{\relax}
\def\oldlabel#1{\label{x-#1}}
\def\refno#1{\hbox{}\nobreak\hfill {\tt (#1)}}
\def\marku#1{\hbox{\tt #1}}
\def\n#1{\quad $#1.$\quad}

% accent used in India citation in Overview.
\def\=#1{\accent"16 #1}
\def\i{I}

%\def\={\relax}


\def\verbose{f}

\begin{document}

\title{Course Notes on Gr\"obner bases}
\author{Thomas C. Hales}
\institute{University of Pittsburgh\\
\email{hales@pitt.edu}}
\maketitle

\def\ra{\rightarrow}
\def\ras{\rightarrow^*}
%\def\da{\downarrow}
\def\LT{\op{LT}}

This document contains supplementary course notes on Gr\"obner bases for Math 2501 (a second semester graduate course in Abstract Algebra at the University of Pittsburgh).  These notes are based on Dummit and Foote's {\it Abstract Algebra} and Harrison's {\it Handbook of Practical Logic and Automated Reasoning}.


\section{Terminating and Confluent Relations}


If $(\ra)$ is a relation on a set $X$, then $(\ras)$ means the reflexive transitive closure of $(\ra)$; that is, the intersection of all reflexive transitive relations that contain $(\ra)$.  
A relation $(\ra)$ on a set $X$ is {\it terminating}, if there are no infinite sequences $f_1\ra f_2\ra \cdots$.
An element $f\in X$ is
{\it reduced} if there does not exist $g$ such that $f\ra g$.
We say that $g$ is a {\it rewrite} of $f$ if $f\ra g$.


Given a relation $(\ra)$ on a set $X$, we say that $f, g\in X$ are {\it joinable} if there exists $h$ such that $f\ras h$ and $g\ras h$.
A relation $(\ra)$ is {\it confluent} if whenever we have $f\ras g_1$ and $f\ras g_2$, then $g_1$ and $g_2$ are joinable.  The relation is {\it weakly confluent} if whenever we have
$f\ra g_1$ and $f\ra g_2$, then $g_1$ and $g_2$ are joinable.  

Newman's lemma asserts that if a relation that is terminating and weakly confluent, then it is confluent.  In practical terms, it means that to check confluence of a terminating relation, it is enough to check the easier condition of weak confluence.   The idea of the proof is the following.   Let $Y\subset X$ be the set of elements $f$ that satisfy the confluence condition.  Assuming that  $f\in X$ has the property that whenever $f\ra g$, then $g\in Y$.   An easy check based on the definition of weak confluence then shows that $f\in Y$.  Newman's lemma follows from this by an easy induction on the terminating relation.


\section{Polynomial Relations}

We use the following notation.   Let $k$ be a field, and let $R=k[x_1,\ldots,x_n]$ be the polynomial ring in $n$ variables with coefficients in $k$.  Let $I$ be an ideal of $R$.  We generally use $F$ for a subset of $I$.  Let $\ring{N}=\{0,1,\ldots\}$.

The multi-degree of $x_1^{a_1} x_2^{a_2} \cdots x_n^{a_n}$ is $(a_1,\ldots,a_n)\in \ring{N}^n$.
We write $\op{md}:R\to \ring{N}^n$ for the multi-degree.

We fix an ordering $\l$ on variables; for example $x_n < x_{n-1} < \cdots < x_1$.  We put a well-ordering $(<)$ on $\ring{N}^n$ (viewed as the multi-degrees)  first by ordering
by total degree $a_1+\cdots+a_n$, then lexicographically among those elements with the same total degree (using the ordering of variables).

Each nonzero $f\in R$ can be uniquely written as a sum of nonzero monomials with distinct multi-degrees.  Each nonzero $f$ has a leading term $\LT(f)=\ell$, being that monomial term of $f$ with largest multi-degree (w.r.t. well-ordering).  We write $f = \ell - r$, where $-r$ is the sum of the remaining monomial terms.


Let $F\subset R$ be a set of polynomials.  We define a relation $(\ra_F)$ on $R$ as follows.
We write $f \ra_F g$ to mean that $f$ and $g$ are identical, except that some monomial term $m$ of $f$ of the form $m=m' \ell$ has been replaced with $m' r$ in $g$, for some
$\ell-r\in F$, where as usual $\ell$ is the leading term of $\ell-r$.
For example, if $F = \{x^2 + xy - 1\}$, and if $x$ is the largest variable,
we have $\ell=x^2$ and $r= - x y +1$. We then
have
\[
x^7 + y^3 + x^3 y = x^5 x^2 + y^3 + x^3 y \ra_F x^5 (- x y + 1) + x^3 y = -x^6 y - x^5 + x^3 y
\]
and
\[
x^6 + y^3 + x^3 y = x^7 + y^3 + x y (x^2) \ra_F x^6 + y^3 + x y (-x y + 1) = x^6 + y^3 - x^2 x^2 + x y.
\]
We drop the subscript $F$ on $(\ra_F)$, when the context makes $F$ clear.

Since $m' \ell - m' r = m'(\ell-r)\in (F)$ (the ideal generated by $F$), 
whenever $f\ra g$, then $f- g\in (F)$.
In other words, the relation $f \ra_F g$ simplifies the coset representative
$f+(F) = g + (F)$.  Similarly, by an easy induction, $f\ras_F g$ implies $f - g \in (F)$.

We claim that for any set $F\subset R$, the relation $(\ra_F)$ is terminating.  Otherwise, there is some $f_1\in R$ which is as small as possible (with respect to the well-ordering of multi-degree of its leading term) and such that $f_1\ra f_2\cdots$ begins some non-terminating sequence.  If at some stage in the sequence, the leading term of $f_1$ get rewritten, then by minimality of $f_1$,  the result is a term that cannot begin a non-terminating sequence and 
the sequence terminates.  If the leading term $\ell_1$ of $f_1$ never gets rewritten, then it is irrelevant, and  the sequence $f_i \ra f_{i+1}$ terminates because the sequence $f_i - \ell_1 \ra f_{i+1}-\ell_1$ terminates.  Either way, the relation is terminating.

\section{Characterizations of Gr\"obner bases}

We say that a set $F\subset R$ is a {\it Gr\"obner basis} (GB for short)
of the ideal $(F)$ it generates if $(\ra_F)$ is confluent.  What this means is that every element of $R$ has a uniquely determined reduced form with respect to the relation $(\ra_F)$.

\begin{lemma} Let $F$ be a subset of $R$ with rewrite relation $(\ra)$.  
The following are equivalent.
\begin{enumerate} 
\item $F$ is a GB of $(F)$; that is, $F$ is confluent.
\item $\forall f$.~  $f\ras 0$ iff $f\in (F)$.
\item $\forall f,g$.~ $f$ and $g$ are joinable iff $f-g\in (F)$.
\end{enumerate}
\end{lemma}

\begin{proof}
$(1\Rightarrow 2)(\Rightarrow)$:  It was observed above that for any set $F$,
$f\ras_F g$ implies $f - g\in (F)$.  Take $g=0$.

$(1\Rightarrow 2)(\Leftarrow)$:  Assume that $F$ is confluent and that $f\in (F)$.
We need to show that $f\ras0$.    We define $(F)$ as an increasing union over subsets $(F)_\alpha$, for each multi-degree $\alpha\in\ring{N}^n$:
\[
(F)_\alpha = \{\sum_i m_i (\ell_i-r_i) \mid \op{md} (m_i \ell_i) \le \alpha,\ \ell_i-r_i\in F,\ m_i~ \text{monomial} \},
\]
where $\op{md}$ denotes the multi-degree of a monomial.
We prove by strong induction over the well-ordered set $\ring{N}^n$ that
for all $\alpha$,  if $f\in (F)_\alpha$, then $f\ras 0$.  By the induction hypothesis,
we may assume that for all $\beta < \alpha$, and all  $g\in (F)_\beta$, then $g\ras 0$.
Pick $f\in (F)_\alpha$.  We may assume that $f$ is reduced.  For a contradiction,
we assume that $f$ is nonzero.   Write
\[
f = \sum_i m_i (\ell_i-r_i) \in (F)_\alpha,
\]
with $\op{md} (m_i \ell_i) \le \alpha$.  By the induction hypothesis, we may assume
that $\op{md}(m_j \ell_j) = \alpha$ for some fixed index $j$.  Clearly, $\op{md}\,\op{LT}(f)\le\alpha$.
If this is an equality, then we can rewrite the leading term of $f$ via $\ell_j - r_j\in F$,
contrary to the assumptiont that $f$ is reduced.  Hence 
\begin{equation}\label{eqn:alpha}
\op{md}\,\op{LT}(f) < \alpha.
\end{equation}

Let $I = \{i \mid \op{md}(m_i\ell_i) = \alpha\}$.  We have $j\in I$.
For each $i\in I$, we have $m_i \ell_i = c_i m_j \ell_j$, for some scalar $c_i\in k$.
By confluence, we may join the two rewrites: $m_i\ell_i\ra m_i r_i\ras s_i$, $c_i m_j \ell_j\ra c_i m_j r_j\ras s_i$.  This means $m_i r_i = t_i + s_i$, $c_i m_j r_j = t'_i +s_i$, where $t_i,t_i'\in (F)_\beta$ for some $\beta<\alpha$.

By \eqref{eqn:alpha}, we have
\begin{align*}
0 &= \sum_{i\in I} m_i \ell_i, = (\sum_{i\in I} c_i) m_j \ell_j,\\
0 &= \sum_{i\in I} c_i.
\end{align*}
and
\begin{align*}
\sum_{i\in I} m_i f_i 
  &= \sum_{i\in I} (m_i f_i - c_i m_j f_j)\\
  &= \sum_{i\in I} (m_i r_i - c_i m_j r_j)\\
  &= \sum_{i\in I} (t_i + s_i - t'_i - s_i)\\
  &=\sum_{i\in I} (t_i - t'_i) \in (F)_\beta.
\end{align*}
It follows that $f\in (F)_\beta$, so that $f\ras 0$, completing the proof by induction,
and completing the proof that $(1\Rightarrow 2)$.

$(2\Rightarrow 3)(\Rightarrow)$:  Assume (2) and that $f$ and $g$ are both joinable to some $h$.  Then $f-h\in (F)$ and $g-h\in (F)$.  So $f-g\in (F)$.

$(2\Rightarrow 3)(\Leftarrow)$:  Assume (2) and that $f-g\in F$.  We use a ``separation ofmonomials argument.''   By (2), $f-g\ras 0$.  Expand out the sequence of rewrites on monomials of $f-g$ keeping separate the rewrites on monomials in $f$ and monomials in $g$, never combining terms between the $f$ and the $g$ side of the ledger.  In the end, $f$ has been rewritten to $h$; that is $f\ras h$.  Since $f-g\ras 0$, we find that $g$ must have been rewritten to the same polynomial $h$; that is $g\ras h$.  This means that $f$ and $g$ are joinable.

$(3\Rightarrow 1)$:  Assume condition (3).  We prove confluence of $F$.  Let $f\ras g_1$ and $f\ras g_2$.  Then $f-g_1\in (F)$, $f-g_2\in (F)$, and taking differences $g_1-g_2\in (F)$.  By (3), we have that $g_1$ and $g_2$ are joinable.  This proves confluence.\qed
\end{proof}

\section{$S$-polynomial}

Let $F\subset R$.
Let $f_1=\ell_1-r_1$ and $f_2=\ell_2-r_2$ be elements in $F$ with leading terms $\ell_i$.
let $m$ be the least common multiple of $\ell_1$ and $\ell_2$; that is,
$m = m_1\ell_1 = m_2\ell_2$.  Set 
\[
S(f_1,f_2) = m_1 f_1 - m_2 f_2 = m_1 r_1 - m_2 r_2.
\]
Note that if $f_1,f_2\in F$, then $S(f_1,f_2)\in (F)$.

\begin{lemma}[Buchberger's criterion] In this context, $F$ is a GB
of $(F)$ iff $S(f_1,f_2)\ras  0$ for all $f_1,f_2\in F$.
\end{lemma}

\begin{proof}
$(\Rightarrow)$: If $F$ is a GB, and if $f_1,f_2\in F$, then $S(f_1,f_2)\in (F)$ and by
the equivalent conditions of the previous section, we have $S(f_1,f_2)\ras 0$.

$(\Leftarrow)$: 
By Newman's lemma, it is enough to prove local confluence of $(\ra)$.  That is, we
start with two rewrites $f\ra g_1$ and $f\ra g_2$ via $\ell_1-r_1, \ell_2-r_2\in F$, respectively.  We need to show that $g_1$ and $g_2$ can be joined.  By hypothesis, we may assume that 
$S(\ell_1-r_1,\ell_2-r_2) = m_1 r_1 - m_2 r_2 \ras 0$.

We consider two cases, depending on whether the two rewrites are applied to the same monomial term $m$ or to two distinct monomial terms $m_1$ and $m_2$ of $f$.

In the first case, $f$ has the form $f = m + r$, where $m = m' m_1 \ell_1 = m' m_2 \ell_2$, for some monomial $m'$ and $r$ is the sum of the other monomial terms of $f$.  Then $g_i = m' m_i r_i + r$ and
$g_1 - g_2 = m' S$.  Since $S\ras 0$, we also have $m' S\ras 0$ by the same sequence of rewrites.  By the separation of monomials argument in the previous section, the condition $g_1-g_2 \ras 0$ implies $g_1$ and $g_2$ are joinable.

In the second case, $f$ has the form $f = m_1 + m_2 + r$, where $m_1 = m'_1 \ell_1$ and $m_2 = m'_2 \ell_2$, the $m$s are monomials, 
and $r$ is the sum of the remaining monomials.   The multi-degrees of $m_1$ and $m_2$ are distinct, because otherwise they would be combined into a single monomial.  In this case,
$g_1 = m'_1 r_1 + m'_2 \ell_2 + r$ and $g_2 = m'_1 \ell_1 + m'_2 r_2 + r$.
  We have 
$g_1 - g_2 = -m'_1 (\ell_1-r_1) + m'_2 (\ell_2 - r_2)$.  The leading term is either
$-m'_1 \ell_1$ or $m'_2 \ell_2$.  Rewriting the leading term gives
$g_1- g_2 \ra \pm m'_i (\ell_i - r_i)$ for some $i$.   Rewriting the leading term again, gives $g_1-g_2\ras 0$.  By the separation of monomials argument, $g_1-g_2\ras0$ implies $g_1$ and $g_2$ are joinable.
\qed\end{proof}

\section{Existence of Gr\"obner bases}

\begin{lemma}  Let $I$ be an ideal of $R$.  Then $I$ has a GB.
\end{lemma}

\begin{proof}  A finite set $F_0$
of generators of $I$ exists
by Hilbert's basis theorem.  

Compute $S(f,g)$ for $f,g\in F_0$ and reduce each $S$-polynomial as much as possible with respect to $(\ra_{F_0})$.  If no nonzero reduced element can be obtained, then by Buchberger's criterion $F_0$ is a GB, and we're done.  Otherwise let $h_0\ne 0$ be obtained from some $S$ by reduction as far as possible and
set $F_1 = F_0\cup \{h_0\}$.  Note that since $f,g\in F_0\subset I$, we also have $h_0\in I$, so that $F_0$ and $F_1$ generate the same ideal.

Now iterate the procedure of the previous paragraph either to produce at some stage a GB, or failing that, an infinite sequence
\[
F_0 \subset F_1 \subset F_2 \cdots.
\]

We claim that the infinite sequence is impossible and hence that the procedure must terminate.  
Corresponding to an infinite sequence is a sequence of ideals
\[
J_0 \subset J_1 \subset J_2 \subset \cdots
\]
where $J_i$ is the ideal generated by leading terms of the elements of $F_i$.
We have $\op{LT}(h_i) \not\in J_i$, for otherwise we can rewrite using $F_i$ on the leading term of $h_i$, which is contrary to our assumption that $h_i$ is reduced.  Hence, sequence of ideals is a strictly increasing sequence.  This contradicts Hilbert's basis theorem.
\qed
\end{proof}

\section{Equivalence Definition of Gr\"obner basis}

If $X\subset R$, let $\LT_X = \{\op{LT}(f) \mid f\in X\}$, the set of leading terms.
As usual, $(\LT_X)$ denotes the ideal generated by $\LT_X$.

\begin{lemma} Let $I$ be an ideal of $R$ and let $F$ be a subset of $I$.
Then $F$ is a GB of $I$ if and only if $(\LT_I) = (\LT_F)$.
\end{lemma}

\begin{proof}
$(\Leftarrow)$:  First we show that $F$ generates $I$.  Suppose
for a contradiction that  $f\in I\setminus (F)$.  We may assume that $f$
is minimal in the sense that that multi-degree of the leading term $m$ of $f$ is as
small as possible.  Since $(\LT_I) = (\LT_F)$,
$m = \sum_i m_i \LT(f_i)$, for some monomials $m_i$ and some $f_i\in F$.
Equating monomial terms on the two sides, we have $m = c m_j \LT(f_j)$ for some
$f_j\in F$ and some scalar $c\in k$.  Thus, $f$ can be rewritten at its leading term via $f_j$ to get 
$f\ra g\in I\setminus (F)$, where $g$ has smaller leading term.  This contradicts
the minimality of $f$ and shows that $I=(F)$.

By equivalent conditions on GB, it is enough to show that $f\ras 0 $
for all $f\in (F)$.  The argument is identical to the previous paragraph, replacing
the condition $f\in I\setminus (F)$ with the conditon: $f\in (F)$ and $f\not\ras 0$.

$(\Rightarrow)$:  We prove the contrapositive.  Assuming the existence 
$f\in (\LT_I)\setminus (\LT_F)$, we
show that $F$ is not a GB of $I$.  We may assume that $f$ is minimal in
the sense that the multi-degree of the leading term $m$ of $f$ is as small
as possible.   Since $(\LT_I)$ is generated by monomials $\LT(g)$, for $g\in I$,
we may assume without loss of generality that $f$ itself is a monomial $f=m$ and
that $m = \LT(g)$, for some $g \in I$.  Since $f=m=\LT(g)\not\in (\LT_F)$, no rewrite
can take place on the leading term of $g$.  Hence $g\not\ras 0$, which shows that
$F$ is not a GB
\qed
\end{proof}

\section{Minimal Gr\"obner basis}

We are aiming for the uniqueness statement of a reduced GB.  To help us get there, we introduce minimal GB.
A GB $F$ of $(F)$ is {\it minimal} provided
\begin{enumerate}
\item the leading term of every $f\in F$ is monic, and
\item for every $f\in F$, the leading term of $f$ is not divisible by the leading term 
of any other $g\in F$.
\end{enumerate}

\begin{lemma} Every ideal $I$ has a minimal GB
\end{lemma}

\begin{proof}  Among all GBs of $I$ in which every leading term is monic, 
pick one of smallest cardinality.  We claim it is minimal.  
Otherwise $\LT(f_2)$ divides $\LT(f_1)$ for some distinct $f_1,f_2\in F$.  Let $G=F\setminus\{f_1\}$.  Then
\[
(\LT_{G}) = (\LT_F) = (\LT_I),
\]
which exhibits $G$ as a GB of smaller cardinality.
\qed
\end{proof}

\begin{lemma}  Any two minimal GBs of $I$ have the same distinct leading terms (and the same cardinality).
\end{lemma}

\begin{proof}  Let $F = \{f_1,\ldots\}$ and $G=\{g_1,\ldots\}$ be two minimal GBs.
Index the elements by increasing leading term: $\LT(f_i) < \LT(f_{i+1})$ and $\LT(g_i)<\LT(g_{i+1})$.  Assume by induction and for a contradiction that
$\LT(f_i)=\LT(g_i)$, for $i<k$, but that $\LT(f_k) < \LT(g_r)$ for all $k\le r$.  
Since $G$ is a GB, $f_k \ras_G 0$.  In particular, some $\LT(g_i)$ divides the leading
term of $f_k$.  However, by the induction hypothesis, some $\LT(g_i)=\LT(f_i)$ 
divides the leading
term of $f_k$, for $i<k$.  Thus, $F$ is not reduced, a contradiction.
\qed
\end{proof}

\section{Reduced Gr\"obner basis}

A GB $F$ of $(F)$ is said to be reduced, provided
\begin{enumerate}
\item  the leading term of every $f\in F$ is monic, and
\item  for every $f\in F$, no monomial term of $f$ is divisible by the leading term of a distinct $g\in F$.
\end{enumerate}

A reduced GB is in particular a minimal GB.

\begin{lemma} Every ideal $I$ has a reduced GB
\end{lemma}

\begin{proof}
Start with any minimal GB $F_0$.  If $\LT(f)$, $f\in F_0$, 
divides a (nonleading) monomial $m$ of $g\in F_0$, then 
replace $g$ with $h$, where $g\ra h$
is obtained by rewriting $g$ at the monomial $m$ via $f$.   Let $F_1$ be obtained by adding $h$ to and removing $g$ from $F_0$.  The leading terms are unchanged so
\[
(\LT_I) = (\LT_{F_0})=(\LT_{F_1})
\]
and this shows that $F_1$ is again a minimal GB  Upon interation, 
this process must terminate
by the well-ordering of monomials.  The terminal set $F_k$ is a reduced GB
\qed
\end{proof}

\begin{lemma}  The reduced GB of an ideal is unique.
\end{lemma}

\begin{proof}  Let $F=\{f_1,\ldots\}$ and $G=\{g_1,\ldots\}$ 
be two reduced GBs of an ideal $I$.  As in the previous section, we index them
by increasing leading term.  By the uniqueness for minimality $\LT(f_i)=\LT(g_i)$ for all
$i$.    Assume by induction and for a contradiction  that
$f_i = g_i$, for $i<k$, but $f_k\ne g_k$.  We have $f_k-g_k\in I$, hence
$f_k-g_k\ras_F 0$.  This means that $\LT(f_i)=\LT(g_i)$ divides $m = \LT(f_k-g_k)$ for some
$i$.  Necessarily, $i<k$, so $i\ne k$.   Up to a scalar, the monomial $m$ 
appears as a monomial term in $f_k$ or $g_k$.  Hence either $f_k$ or $g_k$ does not satisfy the defining property of a reduced GB
\qed
\end{proof}

We remark that the reduced GB is unique given the ordering on variables fixed
at the beginning of these notes.   A different ordering on variables will give a different
reduced GB

\end{document}



