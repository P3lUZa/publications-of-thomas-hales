%% INTRO SPIV

This paper contains the technical heart of the proof of the Kepler
conjecture.  Its primary purpose is to obtain good bounds on the
score $\sigma_R(D)$ when $R$ is an arbitrary standard region of a
decomposition star $D$.  This is particularly challenging, because
we have no {\it a priori} restrictions on the combinatorial type of
the standard region $R$.  It is not known to be bounded by a simple
polygon.  It is not known to be simply connected. Moreover, there
are multitudes of possible geometrical configurations of upright and
flat quarters, each scored by a different rule.  This paper will
deal with these complexities and will bound the score $\sigma_R(D)$
in a way that depends on a simple numerical invariant $n(R)$ of $R$.
When $R$ is bounded by a simple polygon, the numerical invariant is
simply the number of sides of the polygon. This bound on the score
of a standard region represents the turning point of the proof, in
the sense that it caps the complexity of a contravening
decomposition star, and restrains the combinatorial possibilities.
Later in the proof, it will be instrumental in the complete
enumeration of the plane graphs attached to contravening stars.

The first section will prove a series of approximations for the
score of upright quarters.  The strategy is to limit the number of
geometrical configurations of upright quarters by showing that a
common upper bound (to the scoring function) can be found for quite
disparate geometrical configurations of upright quarters. When a
general upper bound can be found that is independent of the
geometrical details of upright quarters, we say that the upright
quarters can be {\it erased.}  (A precise definition of what it
means to erase an upright quarter appears below.)  There are some
upright quarters that cannot be treated in this manner; and this
adds some complications to the proofs in this paper

The second section states the main result of the paper
(Theorem~\ref{thm:the-main-theorem}).  An initial reduction reduces
the proof to the case that the boundary of the given standard region
is a polygon.  A further argument is presented to reduce the proof
to a convex polygon.

The third section completes the proof of the main theorem.  This
part of the proof relies on a new geometrical decomposition of the
part of a $V$-cell over a standard region. The pieces in this
decomposition are called {\it truncated corner cells}.

A final section in this paper collects miscellaneous further bounds
that will be needed in later parts of the proof of the Kepler
conjecture.
