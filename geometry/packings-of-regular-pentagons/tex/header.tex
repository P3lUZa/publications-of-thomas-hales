%%
% Author Thomas C. Hales et al.
% LaTeX Format


%!TEX TS-program = latex    
%% This line is for TexShop. 

% started June 2015

\documentclass{amsart}
\usepackage{graphicx}
\usepackage{amsfonts}
\usepackage{amscd}
\usepackage{amssymb}
\usepackage{alltt}
\usepackage{multicol}
\usepackage{amsmath}
\usepackage{amsthm}
\usepackage{amscd}
%\usepackage{mnsymbol}

%%\documentclass[spanningrule]{cambridge7A}
%% Cambridge University Press Macros from
%% https://authornet.cambridge.org/information/productionguide/laTex_files/


% required by CUP
\usepackage[numbers]{natbib}
\usepackage{rotating}
\usepackage{floatpag}
 \rotfloatpagestyle{empty}
\usepackage{amsthm}
\usepackage{graphicx}
\usepackage{multind}\ProvidesPackage{multind}
\usepackage{times}


% my additions
\usepackage{verbatim}
\usepackage{latexsym}
\usepackage{amsfonts}
\usepackage{amsmath}
\usepackage{crop}
\usepackage{fancyhdr}
\pagestyle{fancy}
\usepackage{txfonts}
%\usepackage[hyphens]{url}
%\usepackage{url}
\usepackage{setspace}
\usepackage{ellipsis} % 
% http://www.ctan.org/tex-archive/macros/latex/contrib/ellipsis/ellipsis.pdf 
% wrapping graphics
% http://en.wikibooks.org/wiki/LaTeX/Floats,_Figures_and_Captions
\usepackage{wrapfig}
\usepackage{listings}


%tikz graphics
%\usepackage{xcolor} % to remove color.
\usepackage{tikz} % Needs pgf version >= 2.10.
\usetikzlibrary{chains,shapes,arrows,trees,matrix,positioning,backgrounds,fit,calc,fadings}

% fonts
\usepackage[mathscr]{euscript} % powerset.
\usepackage{pifont} %ding
\usepackage[displaymath]{lineno}


%%%%%%%%%%%%%%%%%%%%%%%%%%%%%%%%%%%%%%%%%

% This file contains local settings and system dependencies

\def\displayallproof{t} 
% t (default): display all proofs.
% f: print documents without the proofs-- theorem statements only

\def\displayrating{f}
% t (default): display all ratings (verbose is also true)
% f : don't show them.

\def\verbose{t}
% f (default): do not display debugging information,
% t : display debug information and information about the formalization.



%% DEPRECATED

% Auxiliary directories -- deprecated.
%\def\dsp{/Users/thomashales/Pictures/mathFigures/DenseSpherePackings}  % flypaper graphics
%\def\pdf{/Users/thomashales/Pictures/collect_geom} % tarski graphics
%\def\pdfp{/Users/thomashales/Pictures/mathFigures/collect_geom} % kepler graphics
%\def\pdfp{.}

% deprecated.
%\def\showgraphics{f}  
% t: display graphics (not currently available), keep "f" for now.
% f (default): print a "no graphics logo" where graphics would normally go.

%% Macros

%%
%% Sam's Macros:
\def\myscorept{\text{ \sl pt}}
\def\qrtet{{quasi-regular tetrahedron}}
\def\qrtets{{quasi-regular tetrahedrons}}
\def\tomcite{{}}

 \DeclareMathOperator{\myscore}{\sigma}
%\DeclareMathOperator{\sol}{sol} \DeclareMathOperator{\dih}{dih}
%\DeclareMathOperator{\vor}{vor}
%\DeclareMathOperator{\octavor}{\sigma}
%\DeclareMathOperator{\octavor}{octavor}
 \DeclareMathOperator{\gma}{gma}
 \DeclareMathOperator{\score}{\sigma}
 \DeclareMathOperator{\sol}{sol}
 \DeclareMathOperator{\dih}{dih}
 \DeclareMathOperator{\vor}{vor}
%\DeclareMathOperator{\octavor}{\sigma}
%\DeclareMathOperator{\octavor}{octavor}
%\DeclareMathOperator{\gma}{gma}

%theorem definitions, etc.

%\theoremstyle{plain}% default


%\theoremstyle{remark}
%\newtheorem{calc}{Calculation}[subsection]
%\newtheorem*{remark}{Remark}
%\newtheorem{remark}{Remark}

%\theoremstyle{remark}
\newtheorem{calcf}{Calculation}[subsection]
%\newtheorem*{remark}{Remark}
%%TCH%%\newtheorem{remark}{Remark}


%% END Sam's Macros




\font\twrm=cmr8

% mathcal
\def\CalE{{\mathcal E}}
\def\CalB{{\mathcal B}}
\def\CalQ{{\mathcal Q}}
\def\CalW{{\mathcal W}}
\def\CalS{{\mathcal S}}
\def\CalR{{\mathcal R}}

% mathbb
\def\R{{\mathbb R}}
\def\N{{\mathbb N}}
\newcommand{\ring}[1]{\mathbb{#1}}
\def\A{{\mathbf A}}
\def\Rp{\ring{R}^{3\,\prime}}

% textsc
\def\calc#1{{\textsc{calc-#1}}}
\def\conseq#1{{\textsc{conseq-#1}}}

% operatorname
\def\op#1{{\operatorname{#1}}}
\def\sol{\operatorname{sol}}
\def\quo{\operatorname{quo}}
\def\anc{\operatorname{anc}}
\def\cro{\operatorname{crown}}
\def\vor{\operatorname{vor}}
\def\svor{\operatorname{s-vor}}
\def\sign{\operatorname{sign}}
\def\octavor{\operatorname{octavor}}
\def\dih{\operatorname{dih}}
\def\Adih{\operatorname{Adih}}
\def\arc{\operatorname{arc}}
\def\rad{\operatorname{rad}}
\def\gap{\operatorname{gap}}
\def\sc{{\operatorname{sc}}}
\def\geom{{\operatorname{g}}}
\def\anal{{\operatorname{an}}}
\def\PM{\operatorname{PM}}

\def\SA{A}
\def\SB{B}
\def\SC{C}
\def\SCp{C'}
\def\del{\partial}
\def\doct{\delta_{oct}}
\def\dtet{\delta_{tet}}
\def\pt{\hbox{\it pt}}
%\def\Vol{\hbox{vol}}
\def\scoregoal{8\,\pt}
\def\maxpi{\pi_{\max}}
\def\tausc{{\tau\!\operatorname{sc}}}
\def\piF{{\pi_F}}
\def\xiG{\xi_\Gamma}
\def\piG{\pi_\Gamma}
\def\xiV{\xi_V}
\def\xik{\xi_\kappa}
\def\xikG{\xi_{\kappa,\Gamma}}
\def\piV{\pi_V}
\def\tauLP{{\tau_{\hbox{\twrm LP}}}}
\def\DLP{\operatorname{D}_{\hbox{\twrm LP}}}
\def\ZLP{\operatorname{Z}_{\hbox{\twrm LP}}}
\def\tlp{\tau_{\hbox{\twrm LP}}}  % 2 args (p,q) tri, quad
\def\sLP{\sigma_{\hbox{\twrm LP}}}  % 2 args (p,q) tri, quad
\def\squander{(4\pi\zeta-8)\,\pt}
\def\tildeF{{\hbox{$\tilde F$}}}
\def\zloop{{z}_{loop}}
\def\dloop{{\delta}_{loop}}
\def\sqr{\sqrt}
\def\tildeV{V^{\CalS}}

\def\x{\relax}
\def\was#1{\relax}
\def\oldlabel#1{\label{x-#1}}
\def\refno#1{\hbox{}\nobreak\hfill {\tt (#1)}}
\def\marku#1{\hbox{\tt #1}}
\def\n#1{\quad $#1.$\quad}

% accent used in India citation in Overview.
\def\=#1{\accent"16 #1}
\def\i{I}

%\def\={\relax}


% line numbers
\def\lll{\resetlinenumber[1]}
\def\linenumberfont{\normalfont\small\sffamily}

\def\tocpart#1{
  \addcontentsline{toc}{part}{\Large{#1}}}
% (or even \LARGE)
  
\crop
%\makeindex
%\makeindex{index/Notation}
%\makeindex{index/Index}

\def\linput#1{\lll\input{#1}}


\raggedbottom  % for now.
%\raggedright  % don't worry for now.

%% \renewcommand{\baselinestretch}{2}  %% turns on double spacking

%%% end notes -kill this.
%%\maketextnotes

%%%%%%%%%%%%%%%%%%%%%%%%%%%%%%%%%%%%%%%%

% new theorems

\theoremstyle{plain}
\newtheorem{theorem}[equation]{Theorem}
\newtheorem{theorem*}[equation]{Theorem$^*$}
\newtheorem{lemma}[equation]{Lemma}
\newtheorem{lemma*}[equation]{Lemma$^*$}
\newtheorem{background}[equation]{Background}
\newtheorem{corollary}[equation]{Corollary}
\newtheorem{example}[equation]{Example}
\newtheorem{conjecture}[equation]{Conjecture}
%---
\theoremstyle{definition}
\newtheorem{definition}[equation]{Definition}
%---
\theoremstyle{remark}
\newtheorem{remark}[equation]{Remark}
\newtheorem{notation}[equation]{Notation}
\newtheorem{exer}[equation]{Exercise}

\renewcommand\footnotemark{}
%\renewcommand\footnoterule{}

%%%%%%%%%%%%%%%%%%%%%%%%%%%%%%%%%%%%%%%%%

\begin{document}
\title
    {Packings of Regular Pentagons in the Plane}
\author{Thomas Hales and W\"oden Kusner}
\date{February 21, 2016}

\begin{abstract}  We show that every packing of regular pentagons in the Euclidean plane has
density less than $0.9611$.   Our proof is computer-assisted.  We also 
give a detailed strategy for proving the Kuperberg-Kuperberg
conjecture, which asserts that the optimal packing of regular pentagons in the plane is a double lattice,
formed by aligned vertical columns of upward pointing pentagons alternating
with aligned vertical columns of downward pointing pentagons.  The strategy is based on estimates
of the areas of Delaunay triangles.  Our strategy reduces the Kuperberg conjecture to
area minimization problems that involve at most four acute Delaunay triangles.
\end{abstract}

 \lhead{Hales and Kusner}
\rhead{Pentagon Packings}

\parskip=\baselineskip

 \maketitle

%%%


    
%%%%%%%%%%%%%%%%%%%%%%%%%%%%%%%%%%%%%%%%%
%%% FRONT

      
% The Optimal Packing of Regular Pentagons in the Plane, I
% Tex file started June 23, 2015.
% Notation: c = rho = cos pi /5.
% 

%Title: Packings of Regular Pentagons in the Plane,
%Part I, preliminaries.
% draft Nov, 2015.
% version 2, June 2016xs

%Authors: Thomas Hales, Woden Kusner.

\def\area{\op{area}}
\def\areta{\op{areta}}
\def\c{{\mathbf c}}
\def\K{\op{K}}
\def\aK{a_K}
%\def\rat{\Rightarrow}
\def\Ra{\Rightarrow}
\def\nRa{\nRightarrow}
\def\ra#1{\Rightarrow_{#1}}
\def\rab{\ra{b}}
\def\na#1{{\nRightarrow}_{#1}}

\def\epso{\epsilon_0}
%\def\expo{\epsilon'_0}
\def\C{\mathcal C}
\def\S{\mathcal S}
\def\N{\mathcal N}
\def\M{\mathcal M}
\def\T{\mathcal T}
\def\PD{\Psi D}.
\def\cong{\equiv}
\def\r{{\mathbf r}}
\def\bl{\bar\ell}

\def\dd#1#2{\norm{\c_{#1}}{\c_{#2}}}
\def\tb{\tilde b}
\def\tn{\tilde n}

\def\libel#1{{\sc [#1]}\label{#1}}
\def\rif#1{(\ref{#1}-{\text{\sc #1})}}
%\def\raf#1{\rif{#1}}

% tikz.
\def\smalldot#1{\draw[fill=black] (#1) node [inner sep=0.8pt,shape=circle,fill=black] {}}
\def\graydot#1{\draw[fill=gray] (#1) node [inner sep=1.3pt,shape=circle,fill=gray] {}}
\def\whitedot#1{\draw[fill=gray] (#1) node [inner sep=1.3pt,shape=circle,fill=white,draw=black] {}}
\tikzset{dartstyle/.style={fill=black,rotate=-90,inner sep=0.7pt,dart,shape border uses incircle}}
\tikzset{grayfatpath/.style={line width=1ex,line cap=round,line join=round,draw=gray}}

% parameters (x,y) center coords, theta vertex angle, rho radius in cm.
\def\pent#1#2#3#4{%
\draw[red] (#1,#2) + (#3:#4cm) -- + (#3+72:#4cm) -- +(#3+144:#4cm) -- +(#3+216:#4cm) -- + (#3+288:#4cm) -- cycle
}

% parameters (x,y) center coords, theta vertex angle.
\def\pen#1#2#3{%
\draw[red] (#1,#2) + (#3:1cm) -- + (#3+72:1cm) -- +(#3+144:1cm) -- +(#3+216:1cm) -- + (#3+288:1cm) -- cycle
}

\def\threepent#1#2#3#4#5#6#7#8#9{%
\pen{#1}{#2}{#3};
\pen{#4}{#5}{#6};
\pen{#7}{#8}{#9};
\draw[blue] (#1,#2) -- (#4,#5) -- (#7,#8) -- cycle
}

\def\threepentnoD#1#2#3#4#5#6#7#8#9{%
\pen{#1}{#2}{#3};
\pen{#4}{#5}{#6};
\pen{#7}{#8}{#9}
}


\centerline{\it We dedicate this article to W. Kuperberg.}

\section{Introduction}

A fundamental problem in discrete geometry is to determine the highest density of a packing
in Euclidean space by congruent copies of a convex body $C$.   For example, when $C$ is a ball
of given radius, this problem reduces to the sphere-packing problem in Euclidean space.
After a sphere, the simplest shape to consider is a regular polygon $C$ in the plane.
When the regular polygon is an equilateral triangle, square, or hexagon, then copies of $C$ tile
the plane.  In these cases, the packing problem is trivial.  The simplest packing problem
that has remained unsolved until now is the packing problem of regular pentagons in the plane.
In this article, we solve the regular pentagon packing problem.

Kuperberg and Kuperberg
 have conjectured that the
densest packing of congruent regular pentagons in the plane is
achieved by a double-lattice arrangement: verticals column of aligned
pentagons pointing upward, alternating with vertical columns of
aligned pentagons pointing downward (Figure~\rif{fig:double-lattice}) \cite{Kup}.  We call this the {\it Kuperberg
  conjecture,}  and we call this packing of pentagons the {\it Kuperberg packing}.
This packing has density
\[
\frac{5 - \sqrt{5}}3 \approx 0.921311.
\] % Checked 2016/2/18. In Vallentin, p.3.  calcs.ml.
Before our work, the best known bound on the density of packings of regular
pentagons was $0.98103$, obtained through the representation theory of the group of rigid motions of the plane \cite{Val}.
Our research is a continuation of W. Kusner's thesis \cite{Kus}, which proves
the local optimality of the Kuperberg packing.
% 0.98103 page 27 of Vallentin. Checked 2016/2/18.


\tikzfig{double-lattice}{Kuperbergs' double-lattice packing of regular pentagons.  All figures show
pentagons in red and Delaunay triangles in blue.}
{
\pent{0.0}{0.0}{-90}{0.4};  % s = 0.4
\pent{0.0}{0.724}{-90}{0.4};  % (1+kappa)*s
\pent{0.0}{2*0.724}{-90}{0.4};  
\pent{0.57}{-0.3854}{90}{0.4};  % 3kappa*sigma*s,(3sigma^2-2)*s
\pent{0.57}{-0.3854+0.724}{90}{0.4}; 
\pent{0.57}{-0.3854+2*0.724}{90}{0.4};  
\pent{0.57}{-0.3854+3*0.724}{90}{0.4};  
\pent{-0.57}{-0.3854}{90}{0.4};  % 3kappa*sigma*s,(3sigma^2-2)*s
\pent{-0.57}{-0.3854+0.724}{90}{0.4}; 
\pent{-0.57}{-0.3854+2*0.724}{90}{0.4};  
\pent{-0.57}{-0.3854+3*0.724}{90}{0.4};  
\pent{2*0.57}{0.0}{-90}{0.4};  % s = 0.4
\pent{2*0.57}{0.724}{-90}{0.4};  % (1+kappa)*s
\pent{2*0.57}{2*0.724}{-90}{0.4}; 
\pent{3*0.57}{-0.3854}{90}{0.4};  % 3kappa*sigma*s,(3sigma^2-2)*s
\pent{3*0.57}{-0.3854+0.724}{90}{0.4}; 
\pent{3*0.57}{-0.3854+2*0.724}{90}{0.4};  
\pent{3*0.57}{-0.3854+3*0.724}{90}{0.4};   
}


%\section{Introduction}

This paper gives a computer-assisted proof of the Kuperberg conjecture.


\begin{theorem}\libel{thm:main}  No packing of congruent regular pentagons in the Euclidean
plane has density greater than that of
the Kuperberg-Kuperberg packing.   The Kuperberg-Kuperberg packing is the
unique periodic packing of congruent regular pentagons that attains optimal density.
\end{theorem}



In this paper, we consider packings of congruent regular pentagons in
the Euclidean plane.   Density is scale-invariant.  Without loss of generality,
we may assume that all pentagons are regular pentagons of fixed
circumradius $1$.  The inradius of each pentagon is $\kappa:= \cos
(\pi/5) = (1+\sqrt{5})/4 \approx 0.809$. We set $\sigma := \sin(\pi/5)
\approx 0.5878$.  The length of each pentagon edge is $2\sigma$.
% checked 2016/2/18 in calcs.ml.

All pentagon packings will be assumed to be saturated; that is, no
further regular pentagons can be added to the packing without overlap.
The assumption of saturation can be made without loss of generality,
because our ultimate aim is to give upper bounds on the density of
pentagon packings, and the saturation of a packing cannot decrease its
density.

We form the Delaunay triangulations of the pentagon packings.  The
vertices of the triangles are alway taken to be the centers of the
pentagons.  The saturation hypothesis implies that no Delaunay
triangle has circumradius greater than $2$.  This property of Delaunay
triangles is crucial.  It also follows that every edge of a Delaunay
triangle has length at most $4$.

For most of the paper, we consider a fixed saturated packing and its
Delaunay triangulation.  Generally, unless otherwise stated,
every triangle is a  Delaunay triangle.
Statements of lemmas and theorems implicitly
assume this fixed context.

Initially, pentagons in a packing play two roles: they constrain the
shapes of the Delaunay triangles and they carry mass for the density.
We prefer to we change our model slightly so pentagons are only used
to constrain the shapes of triangles.  We replace the mass of each
pentagon by a small, massive, circular disk (each of the same small
radius) centered at the center of the pentagon, and of uniform density
and the same total mass as the pentagon.  In this model, each Delaunay
triangle contains exactly one-half the mass of a pentagon.  By
distributing the pentagon mass uniformly among the Delaunay triangles,
we may replace density maximization with Delaunay triangle area
minimization.

We write 
\[
\aK := \frac{3}{2}{ \sigma \kappa(1+\kappa)} \approx 1.29036
\] % checked 2016/2/18 in calcs.ml
for the common area of every Delaunay triangle in the 
Kuperberg packing.  

\section{Clusters of Delaunay triangles}

The area of a Delaunay triangle in a saturated pentagon packing can be
smaller than $\aK$.  Our strategy for proving the Kuperberg conjecture
is to collect triangles into finite clusters such that the average
area over each cluster is at least $\aK$.  

We say that a Delaunay triangle (in any pentagon packing) is {\it
  subcritical} if its area is at most $\aK$.  We will obtain a lower
bound $a_0 := 1.237$ on the area of a nonobtuse Delaunay triangle
(Lemma~\rif{lemma:a0}).  This is a very good bound.  It is very close
to the numerically smallest achievable area, which is approximately
$1.23719$.\footnote{In the notation of the appendix, the numerical 
minimum is achieved by a pinwheel with parameters $\alpha=\beta=0$
and $x_\gamma\approx 0.16246$.}  
We write $\epso := \aK - a_0 \approx 0.05336$, for the
difference between the desired bound $\aK$ on averages of triangles
and the bound $a_0$ for a single nonobtuse triangle.
Let $\epso' = 0.008$.  
% Numerical page 8. of nonobtuse notes.
% numerical min checked 2016/2/18 in Mathematica.

\subsection{examples}

To understand this paper, it is necessary to understand a series of examples.
Some of these examples serve as counterexamples to naive approaches to this problem.

[XX to do: add graphics]

\begin{example} All Delaunay triangles in the Kuperberg-Kuperberg packing are congruent and have
area $\aK$.
We call them Kuperberg-Kuperberg triangles.
\end{example}

\begin{example}  The area of an obtuse Delaunay triangle can have area
significantly less than $\aK$.  That is, obtuse subcritical triangles exist.
\end{example}

\begin{example} By the Delaunay condition, a triangle adjacent to an obtuse subcritical
triangle along its longest edge tends to be large.
\end{example}

\begin{example} The area of an acute Delaunay triangle can have area less than
$\aK$.  We show various examples.
\end{example}

\begin{example} The longest edge of an acute subcritical Delaunay triangle cannot be too short.
(See Lemma~\rif{XX}.)
\end{example}

\begin{example}  The Kuperberg-Kuperberg triangle is not a local minimum of the area function.
Numerically, it appears that the Kuperberg-Kuperberg triangle can be continuously deformed, along an area-decreasing
path, into the acute Delaunay triangle of minimum area.
\end{example}

\begin{example}  The Kuperberg-Kuperberg packing can be partitioned into pairs
of Delaunay triangles (called $K$-dimers) in which the two triangles
in each dimer share their common longest edge.  It is known that the $K$-dimer is a local
minimimum of the area function \cite{Kus}.
\end{example}

\begin{example} The $K$-dimer admits a shear motion that preserves all edges of contact.
It is obvious by symmetry that total area of the dimer (that is, the sum of the areas of the two triangles)
is preserved to first order by a shear transformation, 
because a left shear is equivalent under a reflectional symmetry to a right shear.
\end{example}

\begin{example} There exist pairs of acute triangles $(T_1,T_0)$ such that the sum of their two areas
 is at most $2\aK$ and such that (1) $T_1$ is subcritical and is adjacent to $T_0$ 
along the longest edge of $T_1$, but such that (2) the longest edge of $T_0$ is 
not the edge shared with $T_1$.  We will call such pairs {\it pseudo-dimers}.
 Pseudo-dimers add significant complications to the proof.
\end{example}

\subsection{attachment}

As mentioned above, our strategy for proving the Kuperberg conjecture
is to collect triangles into finite clusters such that the average
area over each cluster is at least $\aK$.  The clusters are defined by
various equivalence relations.  These equivalence relations, in turn,
are defined as the reflexive, symmetric, transitive closure of further
relations on the set of Delaunay triangles in a pentagon packing.

A second strategy is to replace the area function $\area(T)$ on Delaunay
triangles with a modified area function $b(T)$.  The modification steals area from
nearby triangles that have area to spare and gives to triangles in need.  It will 
be sufficient to prove that the average of $b(T)$ over each cluster is at least $\aK$.


In more detail, below, we define various relations $(\ra{x})$, viewing
relations in the usual way as sets of ordered pairs.  For each
relation $(\ra{x})$, we write ${(\equiv_{x})}$ for the equivalence
relation obtained as the reflexive, symmetric, transitive closure of
$(\ra{x})$.  We call a corresponding equivalence classes $\C$ an
$x$-cluster, or simply a {\it cluster}, when the relation $(\ra{x})$
has been fixed.  In other words, each relation defines a directed
graph whose nodes are the Delaunay triangles of a pentagon packing,
with directed edges given as arrows $T_1\ra{x} T_2$.  An $x$-cluster
is the set of nodes in a connected component of the underlying
undirected graph.

We say that Delaunay triangle $T_1$ {\it attaches to Delaunay
  triangle} $T_2$, provided that the following condition holds: $T_2$
is the adjacent triangle to $T_1$ along the longest edge of $T_1$.
(If the triangle $T_1$ has more than one equally longest edge, fix
once and for all a choice among them, and use this choice to determine
the triangle that $T_1$ attaches to.  Thus, $T_1$ always attaches to
exactly one triangle $T_2$.  We can assume that the tie-breaking
choices are made according to a translation invariant rule.)  We write
$T_1 \Ra T_2$ for the attachment relation.


We define the relation $(\ra{a})$ by $T_1 \ra{a} T_2$ iff
$T_1\Ra T_2$ and $T_1$ is subcritical.

If $\T$ is a finite set of Delaunay triangles, we set $\area(\T):=\sum_{T\in\T} \area(T)$.

\begin{definition}[dimer pair]
We define a dimer pair to be an ordered pair $(T_1,T_0)$ of Delaunay triangles
such that 
\begin{enumerate}
\item $T_0$ and $T_1$ are both nonobtuse.
\item $T_1\ra{a} T_0$.  (In particular $T_1$ is subcritical.)
\item $T_0 \Ra T_1$.
\item $\area\{T_1,T_0\} \le 2\aK$.
\end{enumerate}
We write $DP$ for the set of dimer pairs.
\end{definition}

\begin{definition}[pseudo-dimer]
We define a pseudo-dimer to be an ordered pair $(T_1,T_0)$ of Delaunay triangles
such that
\begin{enumerate}
\item $T_0$ and $T_1$ are both nonobtuse;
\item $T_1\ra{a} T_0$;  (In particular $T_1$ is subcritical.)
\item $T_0 \nRightarrow T_1$;
\item $\area\{T_1,T_0\} \le 2\aK$;
\end{enumerate}
We write $\PD$ for the set of pseudo-dimers.
\end{definition}

We observe that dimer pairs differ from pseudo-dimers in the third defining condition, which is a condition
on the location of the longest edge of $T_0$.   Every pseudo-dimer determines a third triangle $T$ by
the condition $T_0\Ra T$.  The shared edge of $T_0$ and $T$, leading out of the pseudo-dimer
is called the egressive edge of the pseudo-dimer or $T_0$.


For any  set $\S$ of ordered pairs, let  $n_1(T,\S) 
= \card \{T_2\mid (T,T_2)\in \S\}$ and $n_2(T,\N) = \card \{T_1\mid
(T_1,T)\in \S\}$. 

Let $\N_+$ be the set of pairs $(T_1,T_2)$ of Delaunay triangles such that
\begin{enumerate}
\item $T_1$ is nonobtuse and the longest edge of $T_1$ has length at least $1.72$;
\item $T_2$ is obtuse;
\item $T_1\Ra T_2$.
\end{enumerate}

Let $\N = \N_+ \cup \N_-$, where $\N_-$ is the set of pairs $(T_1,T_2)$ of Delaunay triangles such that
\begin{enumerate}
\item $T_1$ is nonobtuse and the longest edge of $T_1$ has length at least $1.72$;
\item $T_2$ is nonobtuse;
\item $T_1\Ra T_2$;
\item there exists an obtuse triangle $T$ such that $T\Ra T_2$ and 
$\area(T) \le \aK+  n_2(T,\N_+)\epso$.
\end{enumerate}

Let $\M$ be the set of pairs $(T_1,T_2)$ of Delaunay triangles such that
\begin{enumerate}
\item $(T_1,T_2)\not\in \N$;
\item $T_1\Ra T_2$;
\item The longest edge of $T_1$ has length  at least $1.72$.
\item There exists a unique $T$ such that $(T_1,T)\in \PD$.
\end{enumerate}

We abbreviate $m_1(T) = n_1(T,\M)$, $m_2(T) = n_2(T,\M)$, $n_1(T) = n_1(T,\N)$, $n_2(T) = n_2(T,\N)$,
and $n_2^+(T) = n_2(T,\N_+)$.

We define
\[
b(T) := \op{area}(T) + \epso (n_1(T) - n_2(T)) + \epso' (m_1(T) - m_2(T)).
\]
We say that $T$ is {\it $b$-subcritical} if $b(T) \le \aK$.  We write
$T_1\rab T_2$, if $T_1\Ra T_2$ and $T_1$ is $b$-subcritical.
The equivalence classes of triangles under the corresponding
equivalence relation $(\equiv_b)$ are called {\it $b$-clusters}.

We note that $\rab$ is given by a translation-invariant rule.  The function $b(T)$ is also
translation invariant and depends only on local information in the pentagon packing near  the triangle $T$.

We remark that the modification $b(T)$ of the area function has two correction terms.
The first term $\epso (n_1(T) - n_2(T))$ takes away from obtuse triangles (and their neighbors) and gives 
 to nonobtuse triangles.  
The intuition behind this correction terms is that the triangle adjacent to an obtuse triangle along its long edge
has a very large surplus area that can be beneficially redistributed.  It allows us to make a clean separation of
the proof of the main inequality into two cases: clusters that contain an obtuse triangle and clusters that do not.
The second term $\epso' (m_1(T) - m_2(T))$ augments the area of pseudo-dimers, by taking from
their neighbor.   The intuition behind this correction term is that a pseudo-dimer can have area strictly less than
the $K$-dimer, and we need to boost its area with a correction term to make it satisfy the main
inequality.  A calculation given below shows that the neighbor of the pseudo-dimer has area to spare.

The correction terms allow us to keep the size of each cluster small.  Eventually, we show that each
cluster contains at most four triangles.  This small size will be helpful when we turn to the computer calculations.

We give  some simple consequences of our definitions.

\begin{lemma} If $(T_1,T_2)\in \M$, then both $T_1$ and $T_2$ are nonobtuse.
\end{lemma}

\begin{proof}  $T_1$ is nonobtuse, becaue $(T_1,T)\in\PD$ for some $T$.
If $T_2$ were obtuse, then we would have $(T_1,T_2)\in \N_+$, which is impossible
because $\N\cap \M = \emptyset$.
\end{proof}

\begin{lemma}[obtuse $b$]  If $T$ is an obtuse Delaunay triangle, then $m_1(T)=m_2(T)=n_1(T)=0$.
Thus, $b(T) = \area(T) - \epso n_2(T)$.
\end{lemma}

\begin{proof}  The previous lemma gives $m_1(T)=m_2(T)=0$.  The first component of
$\N$ is nonobtuse by definition, so $n_1(T)=0$.
\end{proof}

\begin{remark}
  Note also that $n_1(T)\le 1$, because each triangle attaches to
  exactly one other triangle.  Also, $n_2(T)\le 3$, because each
  attachment forms along an edge of the triangle $T$.  Similarly,
  $m_1(T)\le 1$ and $m_2(T)\le 3$.
\end{remark}

\begin{lemma}\libel{lemma:n1m1}  Suppose that $n_1(T)>0$. Then $m_1(T)=0$.
\end{lemma}

\begin{proof} This follows directly from the disjointness of $\M$ and $\N$.
\end{proof}

\begin{lemma}\libel{lemma:n2m2}  Suppose $n_2(T_2)>0$.  Then $m_2(T_2)=0$.
\end{lemma}

\begin{proof} If $T_2$ is obtuse, then $m_2(T_2)=0$ by Lemma~\rif{XX}.
We may assume that $T_2$ is nonobtuse.
By the definition of $\N$, the condition $n_2(T_2)$ implies the existence of an obtuse $T$ with $T\Ra T_2$.
Then the result follows from the definition of $\M$ and the disjointness $\N\cap\M = \emptyset$.
\end{proof}



\subsection{The main inequality}

\begin{lemma}\libel{lemma:main}  
If  every $b$-cluster $\C$ in every saturated packing of regular
  pentagons is finite, and if for some $a$ every $b$-cluster average
  satisfies
\begin{equation}\libel{eqn:main}
\frac{\sum_{T\in \C} b(T)}{\card(\C)} \ge a,
\end{equation}
then the density of a packing of regular pentagons never exceeds 
\[
\frac{\area_P} {2 a},
\]
where $\area_P = 5\kappa\sigma$ is the area of a regular pentagon of
circumradius $1$.  In particular, if the inequality holds for $a=\aK$,
then the density never exceeds
\[
\frac{\area_P}{2 \aK} = \frac{5 - \sqrt{5}}{3},
\] % checked 2016/2/18 in calcs.ml
the density of the Kuperberg packing.
\end{lemma}

For any finite set $\C$ of triangles, we will call the inequality
(\rif{eqn:main}) with the constant $a=\aK$ the {\it main inequality}
(for $\C$).  We call the strict inequality,
\begin{equation}\libel{eqn:strict-main}
\frac{\sum_{T\in \C} b(T)}{\card(\C)} > \aK,
\end{equation}
the {\it strict main inequality} (for $\C$).


\begin{proof} The maximum density can be obtained as the limit of the
  densities of a sequence of saturated periodic packings.  Thus, it is
  enough to consider the case when the packing is periodic.  A
  periodic packing descends to a packing on a flat torus
  $\ring{R}^2/\Lambda$, for some lattice $\Lambda$.  The rule defining
  $\N$ is translation invariant, and $\N$ descends to the torus.  On
  the torus, the set of pentagons, the set of triangles, and the set
  $\N$ are finite.  The equivalence relation $(\equiv_b)$ defining
  clusters is translation invariant, and each cluster is finite, so
  that no cluster contains both a triangle and a translate of the
  triangle under a nonzero element of $\Lambda$.  Thus, each cluster
  $\C$ in $\ring{R}^2$ maps bijectively to a cluster $\C$ in the flat
  torus.  Let $m$ be the number of pentagons in the torus.  By the
  Euler formula for a torus triangulation, the number of Delaunay
  triangles is $2m$.  We have
\[
\sum_{T} n_2(T) =  \sum_{T} n_1(T) = 
\card(\N).
\]
Thus, the terms in $b(T)$ involving $n_1(T)$ and $n_2(T)$ cancel:
\[
\area(\ring{R}^2/\Lambda) = \sum_T \area(T) = \sum_T b(T).
\]    
Let $\area_P$ be the area of a regular pentagon.  Making use of the
hypothesis of the lemma, we see that the density is
\[
\frac{m\, \area_P}{\sum_T \area(T)} 
=\frac{m\, \area_P}{\sum_T b(T)} \le\frac{m\, \area_P}{2 m\, a} 
= \frac{\area_P}{2\, a}.
\]
When $a=\aK$, the term on the right is the density of the Kuperberg
packing, as desired.
\end{proof}

This article gives a proof of the following theorem. In view of Lemma \rif{lemma:main},
it implies the main result, Theorem \rif{thm:main}.  The proof of this result appears
at the end of Section~\rif{lemma:nonobtuse}.

\begin{theorem}\libel{conj:main}
  Let $\C$ be a $b$-cluster of Delaunay triangles in a saturated
  packing of regular pentagons.  Then $\C$ is finite and the average
  of $b(T)$ over the cluster is at least $\aK$.  That is, $\C$
  satisfies the main inequality.  Equality holds exactly when $\C$
  consists of two adjacent Delaunay triangles from the Kuperberg
  packing, attached along their common longest edge.
\end{theorem}

\begin{remark}\libel{rem:equal}
Analyzing the proof of Lemma~\rif{lemma:main}, we see that for a
periodic packing, the maximum density is achieved exactly when each
 cluster in the packing gives exact equality in the main inequality.
Thus, the theorem implies that the Kuperberg packing is the
unique periodic packing that achieves maximal density.
\end{remark}



\section{Pentagons in contact}


\subsection{notation}

By way of general notation, we use uppercase $A,B,C,\ldots$ for
pentagons; $\v_A,\v_B,\ldots$ for the vertices of pentagons;
$\c_A,\c_B,\ldots$ for centers of pentagons; $\p,\q,\ldots$ for
general points in the plane; $\normo{\p}$ for the Euclidean norm;
$\dd{A}{B}$ for center-to-center distances;
$\alpha,\beta,\gamma,\phi,\psi,\ldots$ for angles; and $T_1,T_2,\ldots$
for Delaunay triangles.

We let $\eta(T) = \eta(d_1,d_2,d_3)$ be the circumradius of a triangle
$T$ with edge lengths $d_1,d_2$, and $d_3$.

Let $\angle(\p,\q,\r)$ be the angle at $\p$ of the triangle with
vertices $\p$, $\q$, and $\r$.  Let $\arc(d_1,d_2,d_3)$ be the angle
of a triangle (when it exists) with edge lengths $d_1$, $d_2$, and
$d_3$, where $d_3$ is the edge length of the edge opposite the
calculated angle.



% page 7 pdf obtuse.

We write $\area(T)= \area(d_1,d_2,d_3)$ for the area of triangle $T$
with edge lengths $d_1,d_2,d_3$.  We write
$\areta(d_1,d_2,\eta)$ for the area of a triangle with two edges
$d_1,d_2$ and circumradius $\eta$.  There are in general two
noncongruent triangles with data $d_1,d_2,\eta$.  We choose
$\areta(d_1,d_2,\eta)$ to give the area of the triangle such that its
third edge $d_3$ is as long as possible.

A Delaunay triangle has edge lengths at least $2\kappa$.  This minimum
Delaunay edge length is attained exactly when the the two
corresponding pentagons have an edge in common.  The following lemma
shows that under quite general conditions, we are justified in our
decision to choose $d_3$ as long as possible in the definition of the
function $\areta(d_1,d_2,\eta)$.  It is justified in the sense that
the other choice does not usually give a Delaunay triangle of a pentagon
packing, according to the following simple test.

\begin{lemma}\libel{lemma:areta}  Let $d_1$, $d_2$
and  $\eta$ be positive real numbers.
Assume that $T$ and $T'$ are triangles with edge lengths
$d_1,d_2,d_3$ and $d_1,d_2,d_3'$,  and with the same circumradius
$\eta$. Assume $2\kappa\le d_1\le d_2$.  Set
 $\alpha = \arc(\eta,\eta,d_1)+\arc(\eta,\eta,2\kappa)$.
If $2\kappa \le d_3' < d_3$, then $\alpha < \pi$ and $2\eta\sin(\alpha/2) \le d_2$.
\end{lemma}

In contraposition, if $\alpha\ge\pi$ or if $d_2 < 2\eta\sin(\alpha/2)$,
then the triangle $T'\ne T$, with $d_3' < d_3$,
cannot satisfy the constraint $2\kappa\le d_3'$ of a  Delaunay triangle.


\begin{proof} See Figure \rif{fig:areta}.
Let $\p$, $\q$, and $\r$ (resp. $\p$, $\q$, and $\r'$) be the vertices of $T$
(resp. $T'$) on a common  circle, with
\[
\norm{\p}{\q} =d_1, \quad\norm{\p}{\r}=\norm{\p}{\r'}=d_2, \quad\text{and } 
\norm{\q}{\r'} = d_3' \le d_3=\norm{\q}{\r}.
\]
If $\alpha\ge\pi$, a point $\r'\ne \r$ satisfying the constraints does not exist.
Assume $\alpha < \pi$. 
As the figure indicates, $\norm{\p}{\r'}$ is minimized (as a function of $d_2$)
when $d_3' = 2\kappa$, the lower constraint.  Then
$d_2 = \norm{\p}{\r'}\ge 2\eta\sin(\alpha/2)$.
\end{proof}

\tikzfig{areta}{There can be two positions $\r,\r'$ on the circumcircle for
the third vertex of the triangle.}{
[scale=1]
\draw (0,0) circle (1cm);
\draw (1,0) node[anchor=west] {$\p$} --  (70:1cm) node[anchor=south] {$\q$};
\draw (1,0) -- node[above=1pt] {$d_2$} (130:1cm) node[anchor=south] {$\r'$};
\draw (1,0) -- node[below=1pt] {$d_2$} (-130:1cm) node[anchor=north] {$\r$};
\draw (130:1cm) -- (-130:1cm);
\draw[thin,gray] (1,0) -- (-1,0);
}



\subsection{triple contact}

In this subsection, we describe possible contacts between pentagons.

When two pentagons touch each other, some vertex of one touches an
edge of the other.  We call the pentagon with the vertex contact the
{\it pointer} pentagon, and the pentagon with the edge contact the
{\it receptor} pentagon (Figure~\rif{fig:receptor}).  We also call the
vertex in contact the {\it pointer vertex} of the pointer pentagon. There are
degenerate cases, when the contact set between two pentagons contains
of a vertex of both pentagons.  In these degenerate cases, the
designation of one pentagon as a pointer and the other as a receptor
is ambiguous.

\tikzfig{receptor}{Pointer and receptor pairs of pentagons.  In each pair, the pentagon on the
left can be considered a pointer pentagon, with receptor on the right.  The first pair is nondegenerate,
and the other two pairs are degenerate.}{
[scale=0.5]
\pent{0}{0}{0}{1};
\pent{1.8}{0}{0}{1};
\pent{5}{0}{0}{1};
\pent{7}{0}{180}{1};
\pent{10+0.176}{0.243}{0}{1};  % 0.3 {Cos [54^o],Sin[54^o]}
\pent{12-0.172}{-0.243}{180}{1};
}


% nonobtuse.
We consider a single Delaunay triangle and the three nonoverlapping
pentagons centered at the triangle's vertices
(Figure~\rif{fig:3C-type}).  We call such a configuration a {\it
  $P$-triangle}.  A $P$-triangle is determined up to congruence by six
parameters: the lengths of the edges of the Delaunay triangle and the
rotation angles of the regular pentagons.  When we refer to the area
or edges of a $P$-triangle, we mean the area or edges of the
underlying Delaunay triangle.  More generally, we allow $P$-triangles
to inherit properties from Delaunay triangles, such as obtuseness or
nonobtuseness, the relations $(\ra{x})$, clusters, and so forth. In
clusters of $P$-triangles it is to be understood that the pentagons
coincide at coincident vertices of the triangles.  When there is a
fixed backdrop of a Delaunay triangulation of a pentagon packing, it
is not necessary to make a careful distinction between a
$P$-triangle and its underlying Delaunay triangle.

We say that a $P$-triangle is $3C$ (triple contact),
if each of the three pentagons contacts the other two.

We may direct the edges of a $3C$ triangle by drawing an arrow from
the pointer pentagon to the receptor pentagon.  We may classify $3C$
triangles according to the types of triangles with directed edges.
There are two possibilities for the directed graph.
\begin{enumerate}
\item Some vertex is a source of two directed edges and another vertex
is the target of two directed edges  ($L$-junction, $T$-junction or
  $\Delta$-junction).
\item Every vertex is both a source and a target (pinwheel, pin-$T$).
\end{enumerate}

As indicated in parentheses, we have named each of the various contact
types.  An example of each of the contact types is shown in
Figure~\rif{fig:3C-type}.  An exact description of these contact types
appears later.  The name $L$-junction is suggested by the $L$-shaped
region bounded by the three pentagons.  Similarly, the name
$T$-junction is suggested by the $T$-shaped region bounded by the
three pentagons.  Similarly, for $\Delta$-junctions.  We will see
below that the types in the figure exhaust the geometric types of
$3C$ contact.


\tikzfig{3C-type}{Types of $3C$-contact from left-to-right:
a pinwheel, a pin-$T$ junction, a $\Delta$-junction, an $L$-junction, and a $T$-junction.}
{
[scale=0.6]
\threepentnoD{0.00}{0.00}{46.69}{0.82}{1.53}{218.09}{1.73}{0.00}{163.28};
\begin{scope}[xshift=4.5cm,yshift=1.5cm]
\threepentnoD{0.00}{0.00}{90}{1.40}{-1.377}{0}{-0.35}{-1.628}{-18.89};
\end{scope}
\begin{scope}[xshift=8cm]
\threepentnoD{0.00}{0.00}{-5.16}{0.99}{1.70}{235.38}{1.98}{0.00}{183.43};
\end{scope}
\begin{scope}[xshift=12cm]
\threepentnoD{0.00}{0.00}{80.86}{0.97}{1.58}{232.21}{1.82}{0.00}{223.24};
\end{scope}
\begin{scope}[xshift=16cm]
\threepentnoD{0.00}{0.00}{114.48}{0.90}{1.59}{237.18}{1.66}{0.00}{219.24};
\end{scope}
}

% formatpent3deg (delToPent3 (delAf pinwheeldelA (0.15) (0.15) (0.5)));;  
% formatpent3deg (delToPent3 (delAf pinwheeldelA (0.15) (0.15) (0.0)));;  

% formatpent3deg (delToPent3 (delAf deltajdelA (0.2) (0.15) (0.05)));;  
% formatpent3deg (delToPent3 (delAf ljdelA (0.5) (0.6) (0.90)));;  
% formatpent3deg (delToPent3 (delAf tjdelA (1.0) (1.2) (1.3*. ee)));;  

% graphics for pin-T brute forced in Mathematica.

%Note that the figures give two different geometric types
%associated to the combinatorial structure of a vertex that
%is the target of two edges.  

A {\it cloverleaf} arrangement is a $3C$ triangle that has a point at
which vertices from all three pentagons meet
(Figure~\rif{fig:clover}).  This is degenerate because this shared
vertex can be considered as a pointer or receptor.

\tikzfig{clover}{A cloverleaf (degenerate pinwheel)}
{
[scale=0.6]
\begin{scope}[xshift=4cm]
\threepentnoD{0.00}{0.00}{31.70}{0.76}{1.52}{203.11}{1.70}{0.00}{148.30};
\end{scope}
}


In general, in this paper, a {\it non-anomaly} lemma refers to a
geometrical lemma that shows that certain geometric configurations are
impossible.  Generally, it is obvious from the informal pictures that
various configurations cannot exist.  The non-anomaly lemmas then
translate the intuitive impossibilities into mathematically precise
statements.  We give a few non-anomaly lemmas as follows.
They are expressed as separation results, asserting that
two pentagons $A$ and $C$ do not touch.

\begin{lemma}\libel{lemma:sep1} 
Let $T$ be a $3C$-triangle with pentagons $A$, $B$, and $C$ such that
$B$ is a pointer to both of the other pentagons $A$ and $C$.  
Assume that $T$ is not a cloverleaf.
Then the two pointer vertices $\v_B$ and $\v_B'$ are adjacent
vertices of $B$.
\end{lemma}

\begin{lemma}\libel{lemma:sep2}  
Let $T$ be a $3C$-triangle with pentagons $A$, $B$, and $C$.
Suppose that pentagon
$A$ is a pointer to $B$ at $\v_A$ and that $B$ is a pointer to $C$
at $\v_B$.  Then on $B$, the vertex
$\v_B$ is not opposite to the edge of $B$ containing
$\v_A$.
\end{lemma}

\begin{lemma}\libel{lemma:sep3} 
Let $T$ be a $3C$-triangle with pentagons $A$, $B$, and $C$ 
such that $B$ is a receptor of both of
  the other pentagons.  Then the two pointer vertices
  $\v_A$ and $\v_C$ lie on the same edge or adjacent pentagon edges of
  $B$. 
\end{lemma}



\tikzfig{sep}{A line through the center 
of the middle pentagon $B$
through one of its vertices 
separates the two extremal 
pentagons $A$ and $C$.}{
[scale=0.5]
\pen{0}{0}{90};
\pen{1.72}{0.56}{90};  % (1+kappa) {Cos pi/10 , sin pi/10}
\pen{-1.72}{0.56}{90};  
\draw (0,-1.5) -- (0,1.5);
\begin{scope}[xshift=6cm]
\pen{0}{0}{90};
\pen{1.72}{0.56}{90};  
\pen{-1.72}{-0.56}{90};  
\draw (0,-1.5) -- (0,1.5);
\end{scope}
\begin{scope}[xshift=12cm]
\pen{0}{0}{90};
\pen{1.72}{-0.56}{90};  
\pen{-1.72}{-0.56}{90};  
\draw (0,-1.5) -- (0,1.5);
\end{scope}
}


\begin{proof} The Lemmas~\rif{lemma:sep1}, \rif{lemma:sep2}, and
  \rif{lemma:sep3} can be proved in the same way.  In each case, we
  prove the contrapositive, assuming the negation of the geometric
  conclusion, and proving that the configuration is not $3C$.  We show
  that the configuration is not $3C$ by constructing a separating
  hyperplane between the pentagons $A$ and $C$.  In each case, the
  separating hyperplane is a line through the center of the middle
  pentagon $B$ and passing through a vertex $\v$ of that pentagon.  See
  Figure~\rif{fig:sep}.  In the case of Lemma~\rif{lemma:sep1}, there
  is a degenerate case of a cloverleaf, where all three pentagons meet
  at the vertex $\v$ on the separating line.
\end{proof}

\begin{definition}[$\Delta$]
  We say that a $3C$-triangle has type $\Delta$ if we are in the first
  case of Lemma~\rif{lemma:sep3} (the two pointer vertices $\v_A$ and
  $\v_C$ of $A$ and $C$ lie on the same edge of $B$) provided the line $\lambda$
  through that edge of $B$ separates $B$ from $A$ and $C$. (See
  Figure~\rif{fig:delta}.)
\end{definition}

\tikzfig{delta}{In type $\Delta$, a line separates pentagon 
$B$ from
the other two pentagons. The second figure (which is degenerate of type $L$) does not have
type $\Delta$.}{
[scale=0.5]
\pen{0}{0}{90};
\draw (0,0) node {$B$};
\draw (-2.0,-0.809) node[above] {$\lambda$} --  (2.0,-0.809);
\pen{-1.05}{-2*0.809}{-90};
\pen{1.9 - 1.05}{-2*0.809}{-90};
\begin{scope}[xshift=6cm]
\pen{0}{0}{90};
\draw (0,0) node {$B$};
\draw (-2.0,-0.809) node[above] {$\lambda$} --  (2.0,-0.809);
\pen{-0.428}{-2*0.809}{-90};
\pen{1.36}{ -1.44}{ 212.704};
\end{scope}
}


In type $\Delta$, say $A$ is a pointer into $C$ at $\v$.  Then $\v_A$
and $\v$ are the two endpoints of some edge of $A$.  Also, $\v_C$ and
$\v$ lie on the same edge of $C$.  If the line $\lambda$ does not
separate $B$ from $A$ and $C$, then $\v_C$ is a shared vertex of $B$
and $C$, and we have a degeneracy that can also be viewed as $\v_A$
and $\v_C$ on adjacent pentagon edges of $B$.  This case will be
classified as a degenerate $L$-junction below.

\begin{definition} Let $T$ be a $P$-triangle with pentagons $A$, $B$,
  and $C$.  Suppose that $A$ is in contact with $B$ and $C$ at points
  $\v_B$, $\v_C$ of $A$ that lie on adjacent edges of $A$.  Then we
  call the common endpoint $\v$ of those adjacent edges an {\it inner
    vertex} of $A$.  (The definition of inner vertex includes the
  degenerate case when a vertex of $A$ is $\v_B$ or $\v_C$.)
\end{definition}

\begin{definition} We say that a $3C$-triangle $T$ is type {\it
    pin}-$k$, for $k\in \{0,1,2,3\}$ if $A$ points into $B$, $B$
  points into $C$, $C$ points into $A$, and if there are exactly $k$
  pentagons (in $\{A,B,C\}$) that have an inner vertex.  We use {\it
    pinwheel} as a synonym for pin-$0$ and {\it pin-T} as a synonym
  for pin-$2$.
\end{definition}

%\begin{definition}
%  A {\it pinwheel} is a $3C$-triangle in which all three pentagons are
%  both pointers and receptors.  The following lemma identifies the
%  geometric structure of such a $P$-triangle.  The two contact ponts
%  on each pentagon are a vertex and another point on the same pentagon
%  edge.
%\end{definition}

% page 23-nonobtuse.
\begin{lemma} Let $T$ be a $3C$-triangle of type pin-$1$ with
  pentagons $A$, $B$, and $C$.  Then $T$ is degenerate of type
  $\Delta$.
% or type pin-$2$.
%  Assume that $C$ points to $A$ at $\v_{CA}$, that $A$ points to $B$
%  at $\v_{AB}$, and that $B$ points to $C$ at $\v_{BC}$.  Assume that
%  $T$ is not (degenerately) type $\Delta$ (Figure~\rif{fig:delta}).
%  Then $\v_{CA}$ and $\v_{BC}$ lie on the same edge of $C$.
\end{lemma}

\tikzfig{pinwheel}{A pin-$1$ configuration.}{
\begin{scope}[scale=1.0]
\pen{0}{0}{-90};
\pen{-0.974}{ 1.543}{ 32.70422};
\draw[red] ++ (-0.974,1.543) 
  ++ (0.842,0.532) node[black,anchor=west] {$\v_{CA}$} -- 
  ++ (115:0.2) -- ++ (-65:1.6) node[black,anchor=south west] {$\v_{AB}$}
   -- ++ (7:1.17);
\draw (0,0) node {$B$};
\draw (0,0) + (3*72-90:1) node[anchor=north west] {$\v_{BC}$};
\draw (-0.974,1.543) node {$C$};
\draw (1.1,1.543) node {$A$};
%\draw (0.57,0.809) -- (2.1,0.809) node[anchor=south] {$\gamma$};
\end{scope}
}

\begin{proof}  We draw a (distorted) picture of a
  pinwheel that violates the conclusion (Figure~\rif{fig:pinwheel}).
 %(There are
 % other configurations (and associated figure) that must be ruled out,
 % when additionally $\v_{AB}$ and $\v_{BC}$ lie on adjacent edge of
 % pentagon $B$ or when $\v_{CA}$ and $\v_{AB}$ lie on adjacent edges
 % of pentagon $A$.  We leave these cases to the reader's imagination.)
%XX
  We let $\p$ be the inner vertex of $C$; that is, the vertex that is
  interior to the triangle $(\v_{AB},\v_{BC},\v_{CA})$.  It is an
  endpoint of the edge of the pentagon $C$ containing $\v_{BC}$.  We have
\[
\angle (\p,\v_{BC},\v_{AB}) \le \pi,\quad \angle(\p,\v_{CA},v_{BC}) = 3\pi/5,\quad
\angle(\v_{AB},\v_{CA},\v_{BC})\le 2\pi/5.
\]
(The last inequality uses the fact that $T$ is not pin-$2$, so
that $\v_{AB}$ is not a vertex of $B$.)
We also have
\[
\angle(\p,\v_{CA},\v_{AB}) \ge 2\pi /5 \ge \angle(\v_{AB},\v_{CA},\v_{BC}) \ge \angle(\v_{AB},\v_{CA},\p).
\]
The law of sines applied to the triangle $(\p,\v_{CA},\v_{AB})$ then gives
\[
2\sigma =\norm{\v_{CA}}{\p}\le\norm{\v_{CA}}{\v_{AB}}\le 2\sigma.
\]
%\[
%2e\le \norm{\p}{\v_{CA}} \le \norm{\v_{CA}}{\v_{AB}} = 2e.
%\]
Thus, we have equality everywhere.  In particular, $\angle(\p,\v_{BC},\v_{AB})=\pi$.  This
has type $\Delta$.
\end{proof}

\begin{lemma}  The type pin-$3$ does not exist.  
\end{lemma}

\begin{proof} Suppose for a contradiction that a $P$-triangle $T$ of
  type pin-$3$ exists.  The region $X$ bounded by the three pentagons
  is a nonconvex star-shaped hexagon, with interior angles $\alpha'$,
  $7\pi/5$, $\beta'$, $7\pi/5$, $\gamma'$, and $7\pi/5$.  The vertices
  of $X$ with angles $7\pi/5$ are the inner vertices of the three
  pentagons of $T$.  The sum of the interior angles in a hexagon is $4\pi$:
\[
4\pi = \alpha'+\beta'+\gamma' + 3 (7\pi/5),
\]
which implies that $\alpha'+\beta'+\gamma' = -\pi/5$, which is
impossible.
\end{proof}

\begin{definition}[$T$ and $L$-junction]
  We say that a $3C$-triangle is a type $T$- or $L$-junction if it is
  not type $\Delta$ and if we are in the second case of
  Lemma~\rif{lemma:sep3} (the two pointer vertices $\v_A$ and $\v_C$
  lie on adjacent edges of $B$).  Say $A$ is a pointer into $C$ at
  $\v$.  We say that it has type {\it $L$-junction} if $\v_C$ and $\v$
  lie on the same pentagon edge of $C$, and otherwise we say it has
  type {\it $T$-junction}.
\end{definition}

We can be more precise about the structure of a $T$-junction.  In the
context of the definition, Lemma~\rif{lemma:sep2} implies that $\v$
and $\v_C$ lie on adjacent pentagon edges of $C$.

This completes the classification of $3C$-triangles.  Useful
coordinate systems for the various types can be found in the appendix
(Section~\rif{sec:appendix}).

\section{Delaunay triangle areas}

As an application of the classification from the previous section,
this section makes a computer calculation of a lower bound on the
longest edge length of a subcritical triangle.  We also obtain a lower
bound on the area of a nonobtuse Delaunay triangle.


%The next lemma gives a slight improvement on the edge length bound
%$\kappa\sqrt{8}\approx 2.288$.

\begin{lemma}\libel{lemma:21} A nonobtuse subcritical triangle has edge lengths
  at most $2.1$.
\end{lemma}

\begin{proof} By the monotonicity of area as a function of edge length
  for nonobtuse triangles, a triangle with an edge length at least
  $2.1$ has area at least
\[
\area(2.1,2\kappa,2\kappa) > \aK,
\] % checked 2016/2/18 in Mathematica.
which is not subcritical.
\end{proof}

\begin{lemma}\libel{lemma:right} 
  A nonobtuse subcritical triangle has edge length less than
  $\kappa\sqrt8$.  In particular, a right-angled Delaunay triangle is
  not subcritical.
\end{lemma}

\begin{proof}  
This is a corollary of the previous lemma, because $\kappa\sqrt8 >
2.1$.
%
%  The area of a nonobtuse triangle is monotonic increasing in its edge
%  lengths.  A nonobtuse Delaunay triangle with an edge of length at
%  least $\kappa\sqrt{8}$ has area at least that of the isosceles right
%  triangle
%\begin{equation}\libel{eqn:right}
%\area(T) \ge \area(2 \kappa,2 \kappa,\kappa\sqrt{8}) = 2 \kappa ^2 \approx 1.30902 > \aK.
%\end{equation} % checked 2016/2/18 in calcs.ml
%This is not subcritical.  
\end{proof}


\begin{definition}
In a $P$-triangle, we say that a pentagon $A$ has {\it primary
  contact} if one or more of the following three conditions hold:
\begin{enumerate}
\item (slider contact) The pentagon $A$ and one $B$ of the other two pentagons
share a positive length edge segment;
\item (midpointer contact)  A vertex of one of the other two pentagons
is  the midpoint of one of the edges of the pentagon $A$; or
\item (double contact) The pentagon $A$ is in contact
with both of the other pentagons.
\end{enumerate}
\end{definition}

The next lemma is used to give area
estimates when an edge has length at most $\kappa\sqrt{8}$.

\begin{lemma}\libel{lemma:primary-contact} 
  Let $A$ be a pentagon in a nonobtuse $P$-triangle.  Assume that the
  triangle edge opposite $\c_A$ has length at most $\kappa\sqrt{8}$.
  Then the $P$-triangle can be continuously deformed until $A$ is in
  primary contact, while preserving the following constraints: the
  deformation (1) maintains nonobtuseness, (2) is non-increasing in
  the edge lengths, and (3) keeps fixed the other two pentagons $B$
  and $C$.
\end{lemma}

\begin{proof} Fixing $B$ and $C$, we move $A$ to contract the two
  edges of the triangle at $\c_A$.  For a contradiction, assume that
  none of the primary contact conditions occur throughout the
  deformation.  Continue the contractions, until $A$ contacts another
  pentagon, then continue by rotating $A$ about its center $\c_A$ to
  break the contact and continue.  Eventually, the assumption of
  nonobtuseness must be violated.  However, this triangle cannot be
  obtuse at $\c_A$, because the triangle edge lengths are at least $2
  \kappa$, $2 \kappa$ with opposite edge at most
  $\kappa\sqrt{8}$. This is a contradiction.
\end{proof}

% page 30.
\begin{lemma}\libel{lemma:delta} A subcritical $3C$-triangle does not have type $\Delta$.
  In fact, such a $P$-triangle $T$ has area greater than $1.5$.
\end{lemma}

\begin{proof} The proof is computer-assisted.  The $3C$-triangles of
  type $\Delta$ form a three-dimensional configuration space.  The
  appendix (Section~\rif{sec:appendix}) introduces good coordinate
  systems for each of the various $3C$-triangle types.  We make a
  computer calculation of the area of the Delaunay triangle as a
  function of these coordinates.  We use interval arithmetic to
  control the computer error.  The lemma follows from these computer
  calculations.
\end{proof}



\begin{lemma}\libel{lemma:mid-172}  
  If a pentagon $A$ has midpointer contact with a pentagon $B$, then
  $\dd{A}{B} > 1.72$.
\end{lemma}

\begin{proof} Suppose a pointer vertex $\v_A$ of $A$ is the midpoint of an
  edge of pentagon $B$.  Rotating $A$ about the vertex $\v_A$, keeping
  $B$ fixed, we may decrease $\dd{A}{B}$ until $A$ and $B$
  have slider contact.  This determines the configuration of $A$ and
  $B$ up to rigid motion.  By the Pythagorean theorem, the
  distance between pentagon centers is
\[
\dd{A}{B} = \sqrt{(2\kappa)^2 + \sigma^2} \approx 1.72149 > 1.72.
\] % checked 2016/2/18 in calcs.ml
\end{proof}

\begin{lemma}\libel{lemma:172}
  If every edge of a $P$-triangle $T$ is at most $1.72$, then the
  triangle is not subcritical.
\end{lemma}

\begin{proof} This is a computer-assisted proof.\footnote{The constant
    $1.72$ is nearly optimal.  For example, in the notation of the
    appendix, the pinwheel with parameters $\alpha=\beta=\pi/15$,
    $x_\gamma = 0.18$ is subcritical equilateral with edge lengths
    approximately $1.72256$.} Such a triangle is nonobtuse.  By
  Lemma~\rif{lemma:primary-contact}, we may deform $T$, decreasing its
  edge lengths and area, until each pentagon is in primary contact
  with the other two.  By the previous lemma, we may assume that the
  contact is not midpointer contact.  Thus, each pentagon has double
  contact or slider contact with the other pentagons.

  If the $P$-triangle does not have $3C$ contact, then obvious
  geometry forces one pentagon to have double contact and the other
  two pentagons to have slider contact (Figure~\rif{fig:172-slider}).
  The nonoverlapping of the pentagons forces one of slider contacts to
  be such that a sliding motion along the edges of contact decreases
  area and edge lengths of $T$.  Thus, the $P$-contact can be deformed
  until $3C$ contact results.

\tikzfig{172-slider}{We can slide pentagons to decrease
lengths and the area of the Delaunay triangle}{
[scale=0.5]
\threepent{0}{0}{-90}{0.63}{-1.54}{90}{-1.27}{ -1.07}{90};
}


Now assume that the $P$-triangle has $3C$ contact.
We have classified all $3C$ triangles.  We obtain the proof
by expressing each family of triangles in terms of explicit
coordinates from the appendix (Section~\rif{sec:appendix}) and
computing bounds on the areas and edge lengths of the triangles using
interval arithmetic.  The result follows.
\end{proof}

\begin{lemma}\libel{lemma:2C} 
  Let $T$ be a subcritical nonobtuse $P$-triangle with pentagons $A$,
  $B$, and $C$.  Then fixing $B$ and $C$, we may deform $T$ by moving
  the third pentagon $A$, without increasing the area of $T$, until
  $A$ has double contact (with $B$ and $C$).
\end{lemma}

\begin{proof}
  By Lemma~\rif{lemma:right}, the edge lengths of $T$ are at most
  $\kappa\sqrt8$.  By Lemma~\rif{lemma:primary-contact}, we may assume
  that the pentagon $A$ has primary contact.  If the primary contact
  of a pentagon $A$ is slider contact, we may slide $A$ along the edge
  of contact in the direction to decrease the area of $T$ until it has
  double contact.  If the contact of $A$ is midpointer contact, then
  we may rotate $A$ about the point of contact with a second pentagon
  $B$, in the direction to decrease the area of $T$ until it has
  double contact.  These area-decreasing deformations never transform
  the nonobtuse subcritical triangle into a right triangle
  (Lemma~\rif{lemma:right}).
\end{proof}


\begin{lemma}\libel{lemma:a0}  
A nonobtuse $P$-triangle $T$ has area at least $a_0$.
\end{lemma}

\begin{proof} This is a computer-assisted proof.  We may assume for a
  contradition that $T$ has area less than $a_0$.  In particular, it
  is subcritical.  By Lemma~\rif{lemma:2C}, we may assume that each
  pentagon has double contact with the other two, and that $T$ is
  $3C$.  We have classified all $3C$ triangles.  We obtain the proof
by expressing each family of triangles in terms of explicit
coordinates from the appendix (Section~\rif{sec:appendix}) and
computing bounds on the areas and edge lengths of the triangles using
interval arithmetic.  The result follows.
\end{proof}

\section{Computer Calculations}\libel{sec:calc}

The proofs of the theorems in this article rely heavily on computer calculations.
These computer calculations are stated and proof at the end of the paper.
In this section, we make use of the following lemmas, which are proved by
computer.  (We use results that have been pushed to the end of the paper, but
there is no circular reasoning involved in using those results here.)  We state
a few calculations that will not be needed until later to keep all  the calculations
together.  They will be proved together using a meet-in-the-middle algorithm in
Section~\rif{XX}.

\begin{lemma}[computer-assisted]\libel{NKQNXUN} \libel{calc:pseudo1} 
Let $(T_1,T_0)\in \PD$.  The edge shared between
$T_0$ and $T_1$ has length less than $1.8$. 
\end{lemma}

\begin{lemma}[computer-assisted]\libel{RWWHLQT}\libel{calc:pseudo2} 
Let $(T_1,T_0)\in \PD$.  The longest edge of $T_0$
has length greater than $1.8$.
\end{lemma}

\begin{lemma}\libel{calc:pseudo3}  
Let $(T_1,T_0)\in \PD$.  The two edges of $T_0$ other than the longest edge have lengths less
than $1.8$.
\end{lemma}

\begin{proof}
The shared edge has length less than $1.8$ by Lemma~\rif{hyp1}.  
It has length at least $1.72$ by Lemma~\rif{XX}.
If the third edge has length at
least $1.8$, then by the previous lemmas, its three edges have lengths at least $1.72$, $1.8$, $1.8$.
Then 
\[
\area\{T_1,T_0\} > a_0 + \area(1.72,1.8,1.8) > a_0 + (\aK + \epso) = 2\aK.
\]
This area inequality contradicts a defining property of pseudo-dimers.
\end{proof}

\begin{lemma}[computer-assisted]\libel{BXZBPJW}\libel{calc:pseudo-area}
Let $(T_1,T_0)\in\PD$.  Then $\area\{T_0,T_1\} \ge 2\aK - \epso'$.
\end{lemma}

\begin{lemma}[computer-assisted]\libel{JQMRXTH}\libel{calc:pseudo-area3}
Let $(T_1,T_0)\in\PD$. Assume $T_0\Ra T$.  Then $\area\{T_0,T_1,T\} > 3\aK + \epso'$.
\end{lemma}

\begin{corollary}\libel{lemma:egress'} 
Let $(T_1,T_0)\in\PD$.  Assume that $T_0\Ra T$.  Then
$\area\{T\} > \aK+\epso'$.
\end{corollary}

\begin{proof}  
\[
\area(T) = \area\{T_0,T_1,T\} - \area\{T_0,T_1\} > (3\aK + \epso') - 2\aK = \aK + \epso'.
\]
\end{proof}

\begin{corollary}\libel{lemma:m2-area}
Suppose that $m_2(T)>0$.  Then $\area(T) > \aK + \epso'$.
\end{corollary}

\begin{proof} If $m_2(T)>0$, there exists $(T_1,T_0)\in\PD$ such that $T_0\Ra T$.
The result follows from the previous corollary.
\end{proof}


\begin{definition}[long isosceles]
We say that a triangle is long isosceles if the two longest edges of the triangle have
equal length.  We include equilateral triangles as a special case of long isosceles.
\end{definition}

\begin{definition}[O2C]
We say that a triangle $T=T_0$ or $T=T_1$ in a dimer pair or a pseudo-dimer pair has outside double
contact $(O2C)$ if the pentagon $A$ at the vertex that is not shared with the other triangle in the pair
has double contact.
\end{definition}

\begin{lemma}[computer-assisted]\libel{KUGAKIK}\libel{calc:dimer-isosc}
Let $(T_1,T_0)\in DP$.  Then $T_1$ is not both $O2C$  and long isosceles.
\end{lemma}

\begin{definition}[good angle]\libel{def:good}  Let $T$ be a $P$-triangle.  Let $e$ be an edge of the triangle with pentagons
$A$ and $B$ at its endpoints.
Let $\alpha = \alpha(T,e)$ be the angle between the two pentagons $A$ and $B$.  (See Figure~\rif{XX}.)  Modulo
$2\pi/5$, we can assume that $\alpha\in  [0,2\pi/5]$.  We say that the angle is {\it good} along $(T,e)$ if $\alpha>\pi/5$.
(XX What about degenerate caes $\alpha = 0 \equiv 2\pi/5$?)
\end{definition}

Let $T$ be a $P$-triangle and let $T'$ be the adjacent $P$-triangle along edge $e$.  
We have the invariant $\alpha(T,e)+\alpha(T',e)=2\pi/5$.
Hence if the angle is good along $(T,e)$ then it is not good along $(T',e)$.


\begin{lemma}[computer-assisted]\libel{FHBGHHY}\libel{calc:good}
Let $\{T_0,T_1\}$ be given $P$-triangles (not necessarily a pseudo-dimer) such that $\area(T_1)\le \aK$ and $T_1\Ra T_0$.
Assume that there is a nonshared edge $e$ of $T_0$ of length greater than $1.8$ and that the angle is not good along $(T_0,e)$.
Then $\area\{T_0,T_1\} > 2\aK + \epso'$.
\end{lemma}

\begin{corollary}\libel{lemma:good} If $(T_1,T_0)\in \PD$ and $e$ is the longest (that is, egressive) edge of $T_0$.  Then the angle
is good along $(T_0,e)$.
\end{corollary}

\begin{proof} If the angle is not good, then we can apply the lemma to find tht the area of the pseudo-dimer is
greater than $2\aK + \epso'$, which contradicts one of the defining properties of a pseudo-dimer.
\end{proof}


\begin{lemma}[computer-assisted]\libel{HUQEJAT}\libel{calc:pent3}
Let $T_1^i\Ra T_0$ and $\area(T_1^i)\le\aK$ for $i=0,1$ for distinct $P$-triangles $T^1_0$ and $T^1_1$.
Then $\area\{T_0,T^0_1,T^1_1\} > 3\aK + \epso'$.
\end{lemma}


\begin{lemma}[computer-assisted]\libel{QPJDYDB}\libel{calc:pent4}
Let $T_1^i\Ra T_0$ and $\area(T_1^i)\le\aK$ for $i=0,1,2$ for distinct $P$-triangles $T^1_0$, $T^1_1$, and $T^2_1$.
Then $\area\{T_0,T^0_1,T^1_1,T^2_1\} > 4\aK$.
\end{lemma}


\section{Dimer Pairs}

The purpose of this section is to give a proof of the following theorem.  


\begin{theorem}
Let $(T_1,T_0)$ be a dimer pair.  (In particular, we assume that $T_1$ is subcritical,
that $\area\{T_0,T_1\}\le 2\aK$ and that $T_0$ and $T_1$ share a common longest edge.)
Then $(T_1,T_0)$ is the $K$-dimer of area exactly $2\aK$.
\end{theorem}

The proof will fill the entire section.
 The strategy
of the proof is to give a sequence of area decreasing deformations to $(T_1,T_0)$,
until the $K$-dimer is reached.

We fix notation that will be used throughout this section $(T_1,T_0)$ is a dimer pair.
The $P$-triangle $T_1$ has a pentagon centered at each vertex.  We label the pentagons of $T_1$ as
$A$, $B$, $C$, with $A$ and $C$ shared with $T_0$.  We call $B$ the outer pentagon of $T_1$.
Similarly, we label the pentagons of $T_0$ as $A$, $C$, $D$, with outer pentagon $D$ of $T_0$.

A motion of a pentagon in the plane can be described by an element of the isometry group of
the plane, which is a semidirect product of a translation group and an orthogonal group.  
Because of the dihedral
symmetries of the regular pentagon, each motion can be realized as a translation followed
by a rotation by angle between $0$ and $2\pi/5$.   In particular, a translation of a pentagon is a motion of
a pentagon such that the rotational part is the identity.

\begin{lemma}  Let $(T_1,T_0)$ be a dimer pair, then every edge of $T_1$ and $T_0$ has length
less than $\sqrt8\kappa$.  In particular $T_1$ and $T_0$ are both acute (and not just merely nonobtuse).
\end{lemma}

\begin{proof}  The shared edge between $T_0$ and $T_1$ is the common longest edge of the
two triangles.  It is enough to show that this edge has length less than $\sqrt8\kappa$.
This is an edge of a subcritical triangle $T_1$. The result follows from Lemma~\rif{lemma:right}.
\end{proof}

In particular, area non-increasing deformations of a dimer pair, never transform an acute
triangle into a right or obtuse triangle.  In other words, the nonobtuseness constraint in the
definition of a dimer pair is never a binding constraint in a deformation.

The deformation of a general dimer pair to the Kuperberg-Kuperberg pair will take place in several
stages.  We give a summary of the stages here, before going into details.  Here is the proof sketch:

\begin{enumerate}
\item We deform the dimer pair so that each of $T_0$ and $T_1$ is $O2C$
or long isosceles.
\item We deform so that each of $T_0$ and $T_1$ is $O2C$.
%\item We deform so that both triangles are triple contact, or so that both
%are $O2C$ and $T_1$ is long isosceles.
\item We show that both triangles are triple contact.
\item Working with triple contact triangles, 
we compute that the condition $\area\{T_0,T_1\}\le 2\aK$ implies that
$(T_0,T_1)$ lies in a small explicit neighborhood of the $K$-dimer.
\item We construct a curve $\Gamma$ in the space of dimer pairs, with parameter $t$
such that $t=0$ defines the $K$-dimer.
\item Working with triple contact triangles in a small explicit neighborhood of the
$K$-dimer, and for some small explict constant $M$,
each dimer pair can be connected by a path (in the dimer configuration
space) to a dimer on the curve $\Gamma$ with parameter $|t|<M$.
A computation shows that the area of the dimer decreases along the path to $\Gamma$.  
\item The unique global minimum of the area function along $\Gamma$ occurs at $t=0$;
that is, the $K$-dimer is the unique global minimizer.
\end{enumerate}

\subsection{reduction to $O2C$ or long isosceles}



In this section, we show that each of $T=T_0$ and $T=T_1$ can be deformed in an area
decreasing way until $T$ is either $O2C$ (that is, the outer pentagon has double contact)
or long isosceles (that is, the two longest edges of the triangle have the same length).

To show this, we assume that $T$ and its deformations are not long isosceles;  that is,
it and its deformations have a unique longest edge that is shared with the other triangle.  Fixing the two pentagons of $T$
($A$ and $C$) along the shared edge, we show we can deform the outer pentagon until it has $2C$ contact.

This is easy to carry out.  By Lemma~\rif{XX}, we can move $B$ in an area decreasing way until $B$ has primary
contact.  We can continue to move the outer pentagon by translation (meaning no rotation) that preserves contact with either
$A$ or $C$ and that is non-increasing in triangle area until double contact is achieved.  This is $O2C$.
We do this for both $T=T_0$ and $T=T_1$.

\subsection{reduction to $O2C$}\libel{sec:O2C}

In this subsection we show that each of $T=T_0$ and $T=T_1$ can be deformed in an area decreasing way
so that
it is $O2C$.  

We begin with the case $T=T_0$.  If the deformations in the previous section made $T_0$ into a long isosceles triangle,
then $(T_0,T_1)$ is a boundary case that  can also be classified as a pseudo-dimer.  By earlier calculations, the longest edge
of a pseudo-dimer is greater than $1.8$ and has strictly greater length than the shared edge between $T_0$ and $T_1$.
Thus, it is not long isosceles.

We now consider the case $T=T_1$.
In view of the reductions of the previous subsection, we may assume that $T_1$ has primary contact and
that $T_1$ is long isosceles.  We may further assume  that translation of the outer pentagon with fixed contact with its contact ($A$ or $C$)
in an area decreasing direction would violate the constraint that the longest edge of $T_1$ is the shared edge.  (In other words,
the translation that decreases area would increase the edge length of the long nonshared edge.)
For a contradiction, we may assume that the primary contact is not $O2C$.

We claim that these conditions constrain $T_1$ so that $T_1$ is not subcritical.  This is contrary to the defining
conditions of a dimer pair.  Thus, we complete this stage of the proof by proving non-subcriticality. For the rest of the proof, we disregard $T_0$.

The proof is computer assisted.  We deform $T_1$ in an area decreasing way into
a form that can be easily computed.  We assume that the outer pentagon $B$ is in contact with pentagon $A$.
Since we are now disregarding $T_0$, we may deform the triangle $T_1$ by moving $T_0$, preserving the isosceles constraint and decreasing area,
until $C$ has primary contact.  In particular, $C$ is in contact with $A$ or $B$.

We consider two cases, depending on whether the primary contact of  $B$ has slider contact  or midpointer contact.  
We need two non-anomaly lemmas, one for slider contact and one for midpointer contact.  In both cases, the lemmas imply that the
edge of contact (whether slider or midpointer) is one of the long edges of the isosceles triangle.

\begin{lemma}[slider-non-anomaly]  Let $T$ be a subcritical $P$-triangle with pentagons $A$, $B$, and $C$.  Assume that $B$ has slider
contact with $A$.  Then the translation of $B$ along the slider contact in the direction to decrease the area also decreases
the distance $\dd{B}{C}$.
\end{lemma}

\begin{proof} Suppose to the contrary that the translation is increasing in $\dd{B}{C}$.  Choose coordinates so that the $x$-axis
passes through $\c_A$ and $\c_C$, with the origin at $\c_A$, with  $\c_C$ in the right half-plane, and with $\c_B$ in the positive half-plane.
This means the the line $\lambda$ through the edge
of contact between $A$ and $B$ has positive slope.  Slider contact implies that 
the point $B$ lies on the line $\lambda'$ parallel to $\lambda$ at distance $2\kappa$ from the origin.
Every point on $\lambda'$ either has $y$-coordinate at least $2\kappa$ or negative $x$-coordinate.  If the $y$-coordinate
is at least $2\kappa$ the triangle is not subcritical.  If the $x$-coordinate of $B$ is negative, then the distance $\dd{C}{B} \ge\kappa\sqrt8$
and againt this implies that the triangle is not subcritical.
\end{proof}

\begin{lemma}[midpointer-non-anomaly]  Let $T$ be a subcritical $P$-triangle with pentagons $A$, $B$, and $C$.  Assume that $B$ has midpointer
contact with $A$, with $B$ pointing to $A$.  Then the rotation of $A$ about the point of contact in the direction to decrease the area also decreases
the distance $\dd{B}{C}$.
\end{lemma}



\begin{proof} Suppose to the contrary that the rotation is increasing in $\dd{B}{C}$.  Choose coordinates so that the $x$-axis passes
through $\c_A$ and $\c_C$, with origin at $\c_A$, with $\c_C$ in the right half-plane, and with $\c_B$ in the positive half-plane.
We claim that the distance from $\c_A$ to the the $x$-axis is at least $2\kappa$ so that the area of the triangle is at least $2\kappa^2 > \aK$,
contrary to the assumption that the triangle $T$ is subcritical.   To prove the claim, we disregard the point $\c_C$.  We rigidly rotate
$A$ and $B$ about $\c_B$, keeping their relative positions fixed, until the lower edge of $A$ is horizontal.  This rotation decreases the
distance of $\c_A$ to the $x$-axis.  Now the result is clear: the point has distance at least $\kappa$ to the $x$-axis, and the center $\c_C$ is
directly distance $\kappa$ above the point.
\end{proof}

We have reduced the long isosceles triangle to a triangle with double contact and such that one of those contacts is between $A$ and $B$, one of the
longest edges.  The contact type between $A$ and $B$ is either slider contact or midpointer contact.  The configuration space of such triangles is
three dimensional.  We choose coordinates and compute\footnote{calculations iso\_2C and iso\_2C'} 
with interval arithmetic to show that no such triangle is subcritical.  This completes this reduction.

\subsection{reduction to triple contact}

In this stage, we initially assume that both triangles $T_0$ and $T_1$ are $O2C$.
We deform so that both triangles are triple contact.

We briefly describe the argument.   Our deformations will preserve the $O2C$ contacts.  By an argument made in the second paragraph of Section~\rif{sec:O2C},
we may assume without loss of generality that the triangle $T_0$ is not long isosceles.   By Lemma~\rif{calc:dimer-isosc}, we have that $T_1$ is not long
isosceles.
Assuming that neither triangle can be long isosceles,
 we show that both triangles can be brought into triple contact.  Since both triangles are already $O2C$, this amounts
to decreasing the edge between $\c_A$ and $\c_C$ until the pentagons $A$ and $C$ come into contact.
This deformation consists of a translation of all four pentagons in a motion we call squeezing.  It suffices to describe the deformation separately on
each triangle $T_0$ and $T_1$ and to prove that this deformation decreases area.  

Let $T$ be a triangle with double contact at $B$.  We assume that $\c_A$ and $\c_C$ lie on the $x$-axis with $\c_A$ to the left of $\c_C$, and
with $\c_B$ in the upper half-plane.
If $\c_A$ is free to translate to the right or if $\c_C$ is free to translate to the left with overlapping pentagons, then we do so.  We assume that
we are not in this trivial case.

The squeezing deformation translates $B$ directly upward away from the $x$-axis, while translating $A$ to the right and $C$ to the left to maintain
double contact at $B$.  

\begin{lemma}  Let $T$ be a nonobtuse $P$-triangle with pentagons $A$, $B$, and $C$, where $B$ has double contact.
Assume that the  area of $T$ is at most $\aK+\epso$.  Assume that $T$ is not in the trivial situation of free translation mentioned above.
Then the squeezing deformation decreases area.
\end{lemma}

\begin{proof}  We analyze the effect on the area by moving $B$ directly upward by $\Delta y >0$, $A$ to the right by $\Delta x_A >0$ and $C$ to the left by
$\Delta x_C >0$.  Let $d_{XY} = \dd{X}{Y}$, and let $\arc_X$ be the angle of the triangle at $\c_X$.
The area of $T$ is $\area(T) = d_{AC} d_{AB} \sin(\arc_A)/2$.  The transformed area is $(d_{AC}-\Delta x_A - \Delta x_C)(d_{AB}\arc_A + \Delta y)/2$.  Thus,
Let $\sigma_A = \Delta y/\Delta x_A$ and $\sigma_C = \Delta y /\Delta x_C$.  Passing to the limit as $\Delta y \mapsto 0$, we find that squeezing
decreases the area exactly when
\[
d_{AC} < \left(\frac{1}{\sigma_A}+\frac{1}{\sigma_C}\right)d_{AB}\sin(\arc_A).
\]
It is enough to prove this inequality.  We do this with a computer calculation\footnote{calculation squeeze\_calc} using interval arithmetic.

We defined $\sigma_A$ and $\sigma_C$ as derivatives, but in fact no differentiation is required.  For example, consider $\sigma_A$.  The pentagon $B$
has contact with $A$.  Thus, $B$ points into $A$ or $A$ points into $B$.  If $A$ points into $B$, then the point of contact lies along an edge $e$ of $B$.
The squeeze transformation translates  $A$ and $B$ maintaining the contact.  
Viewed from a coordinate system that fixes $B$, the squeezing lemma translates $A$ parallel to the line through $e$.
That is, $\sigma_A$ is simply the absolute value of the slope of the line through $e$.
Elementary coordinate calculations described in the coordinate section of this paper give an explicit formula for the slope of this line.  There are two
cases, depending on which pentagon points to the other.  There are no difficulties in carrying out the computer calculations with these explicit formulas.

In a degenerate situations, the pentagons $A$ and $B$ might have vertex to vertex contact. But even in this degenerate case, the squeezing deformation
determines an edge $e$ of $A$ or $B$ that determines the slope $\sigma_A$.
\end{proof}

The triangle $T_1$ is subcritical.  The area of $T_0$ is given by
\[
\area(T_0) = \area\{T_0,T_1\} - \area(T_1) < 2\aK - a_0 = \aK+\epso.
\]
Thus, the assumption of the lemma holds.
We continue the squeezing deformation until $A$ comes into contact with $C$.
This completes the reduction to $3C$ contact.


\subsection{reduction to a small neighborhood of the $K$-dimer}

From this stage forward, $T_0$ and $T_1$ are both triangles with triple contact.
We show that the condition $\area\{T_0,T_1\}\le 2\aK$ implies that
$(T_0,T_1)$ lies in a small explicit neighborhood of the $K$-dimer.

The shared pentagons $A$ and $C$ are in contact.  By symmetry, we may assume that $A$ points into $C$.
We have classified all $P$-triangles with triple contact in Section~\rif{XX}.  In this stage, we rely heavily
on this classification.   The triple contact type $\Delta$ has area at least $1.5 > \aK+\epso$ by Lemma~\rif{lemma:delta}.
This is too large an area to be part of a dimer pair.   There are eight combinatorial ways that the pair $(A,C)$ with $A$ pointing
to $C$ can be extended to a  triple contact triangle: a pinwheel, a pin-$T$ junction, an $L$-junction (3 ways), and a $T$-junction (3 ways).
There are three ways of extending the edge along $(A,C)$ to an $L$-junction depending on which of the three triangle edges of the $L$-junction
is placed along $(A,C)$.  A similar remark applies to $T$-junctions.  A pinwheel has cyclic symmetry, so it gives rise to a single case.

We need an argument to show that only a single combinatorial type of pin-$T$ triangle needs to be considered.  
In this paragraph only, we shift notation and refer
to labels on pentagons in the pin-$T$ configuration shown in Figure~\rif{XX}.  We claim that if the area the triangle has area less than
$\aK+\epso$, then the edge length $\dd{B}{C}$ in that Figure is the unique longest edge of the triangle.   
This claim is established by computer calculation\footnote{calculation dimer.ml:one\_pintx.}.  
This means that the shared edge of the pin-$T$ triangle is determined.

Since $T_0$ and $T_1$ both have triple contact, the dimer pair $(T_0,T_1)$ lies in a $4$-dimensional configuration space.
The strategy of the proof is to check by computer\footnote{calculations in dimer.ml} 
that there does not exist a  dimer pair outside a small explicit neighborhood
of the $K$-dimer.  In other words, the area constrains $\area(T_1)\le \aK$ and $\area\{T_0,T_1\}\le2\aK$ are
impossible to satisfy when the longest edge on both triangles is the shared edge.  
We run over $64 = 8\times 8$ cases depending on the combinatorial types of $T_0$ and $T_1$.  In each case,
$T_1$ runs over a three-dimensional configuration space.  Most of the $64$ cases do not contain the $K$-dimer.
In these cases, it is not necessary to specify a small explicit neighborhood.  In the cases that do contain the $K$-dimer,
the neighborhood is desribed in local coordinates.

We say a word about the coordinate system used to carry out these calculations.
The triangles $T_0$ and $T_1$ separately have good coordinate systems described in Section~\rif{XX}.  Three variables
each running over a bounded closed interval parameterize the configuration space for each configuration type.
These coordinates are always numerically stable.  The quantities $\dd{A}{C}$ and $\alpha_{AC}$ can be computed from $T_0$ or $T_1$  alone.
A natural way to try to parameterize the dimer pairs of a given combinatorial type is to use the three variables $x_1,x_2,x_3$ from Section~\rif{XX} for $T_1$,
then to choose an appropriate quantity $x_4$ on $T_0$ such that $T_0$ is determined by $\dd{A}{C}$, $\alpha_{AC}$ (viewed as functions of $x_1,x_2,x_3$) and
$x_4$.  Then $x_1,\ldots,x_4$ give coordinates for the dimer pair that can be used to do the computer calculations.

Usually, this strategy works, but in a few situations, there is no obvious way to pick the fourth coordinate $x_4$ in a numerically stable way.
Fortunately, we can give a characterization of all situations where it is difficult to pick a numerically stable coordinate $x_4$.  These are
expresed as conditions on the combinatorial type of $T_0$ and as bounds on $\dd{A}{C}$ and $\alpha_{AC}$.  In each situation, we use good
coordinates provided by Section~\rif{XX} to show that
these conditions force $T_0$ to have area greater than $\aK+\epso$, which is incompatible with the conditions defining a dimer pair.  
These too are computer calculations\footnote{calculation dimer.ml}.
Thus, we are justified in excluding a few situations, where coordinates become unstable.  (The underlying source of numerical instability is
our use of the law of sines 
\[
\frac{a}{\sin\alpha} = \frac{b}{\sin\beta}
\]
to compute the length of one triangle edge $a$ in terms of another $b$, which encounters instability for $\beta$ near $0$.)

In terms of the local coordinates of Section~\rif{XX}, if $x_1=x_2=x_3=x_4=0$ defines the $K$-dimer, then the explicit
small neighborhood we exclude is given by $|x_i|\le 0.01$, for $i=1,2,3,4$.  For example, in the pinwheel type on $T_1$, 
we have $(x_1,x_2,x_3)=(\alpha,\beta,x_\gamma)$ as given in Section~\rif{XX}, and the fourth variable $x_4$ is determined
by the type of $T_0$.

\subsection{defining a curve}

In this stage, we define a curve $\Gamma$ in the configuration space of triple contact dimer pairs such that the $K$-dimer
is defined by parameter value $t=0$.  We represent the $K$-dimer by a pair of triple contact triangles with shared pentagons
$A$ and $C$, where $A$ points to $C$.  Assume that $\c_A$ and $\c_C$ lie along the $x$ axis, with $\c_A$ to the left
of $\c_C$.  Let $B$ and $D$ be the two outer pentagons.  We deform this dimer by fixing $A$,
and translating $C$ vertically (maintaining the pointer contact of $A$ to $C$).  We translate $B$ and $D$ to maintain
all edge to edge contacts: $(A,B)$, $(A,D)$, $(C,B)$, $(C,D)$.   The parameter $t$ is the vertical translation distance of the pentagon $C$.
See Figure~\rif{XX}.

In terms of the coordinates $(x_\alpha,\alpha)$ of Section~\rif{XX} for two pentagons in contact, the relative position of $A$ and $C$
is described by the curve $\alpha=\pi/5$ and $x_\alpha = \sigma+t$.


\subsection{reduction to the curve}

In this stage, we show that we can reduce to points on the curve in the following sense.
Working with triple contact triangles in a small explicit neighborhood of the
$K$-dimer ($|x_i|<0.01$ for $i=1,2,3,4$ in appropriate coordinates),
each dimer pair $(T_0,T_1)$ can be connected by a path (in the dimer configuration
space) to a dimer on the curve $\Gamma$ with parameter $|t|<0.01$.
The area function is decreasing along this path.
We have two tasks in the subsection.  First, we construct the path $P$, and 
then we show that the area function decreases along the path.  We will use
$s$ as a local parameter on the path ($s\mapsto P(s)$), 
which we need to keep separate from
the parameter $t$ for $\Gamma$.  We construct a path such that $s=0$ determines the
initial dimer pair $(T_0,T_1)$ and such that the path $P$ is defined for $s\in [0,s_0]$, for some $s_0>0$.

We show how to construct paths for $T_0$ and $T_1$ separately, and then
we show that the paths are compatible and give a path for the dimer pair $(T_0,T_1)$.
We consider cases, according to the type of the triple contact.
As mentioned above, there are $8$ types for each triangle, but most of them do not
approach a triangle in the $K$-dimer.  We may restrict our attention
to the four triangle types that approach the $K$-dimer.   These are the pinwheel, $L_1$-junction,
$L_2$-junction, and $T_3$-junction.  See Figure~\rif{XX}.
The paths are constructed explicitly in the coordinate section XX.

The argument that the paths for $T_0$ and $T_1$ fit together compatibly is elementary.
We assume that $A$ has contact with $C$ and that $A$ points into $C$.
The relative position of $A$ and $C$ is described by two variables $x_{AC}\in[0,2\sigma]$ and 
$\alpha_{AC}\in[0,2\pi/5]$.
The variable $\alpha_{AC}$ depends on an orientation provided by the triangle $T=T_0$ or $T=T_1$.
They have opposite orientation, and 
\begin{equation}\libel{eqn:alpha-sum}
\alpha_{AC,0}+\alpha_{AC,1} = 2\pi/5,\quad x_{AC,0} + x_{AC,1} = 2\sigma.
\end{equation}
where we have added a subscript to show the dependence on $T_i$.  These equations are the
compatibility conditions for the two $P$-triangles to fit together.
In every case, the path $P$ fixes $x_{AC}$, and  $\alpha_{AC}(s)$ as a function of $s$ is
\begin{equation}\libel{eqn:alpha-pi5}
\alpha_{AC}(s) = 
\begin{cases}
\alpha_{AC}(0) + s&\text{if } \alpha_{AC}(0) < \pi/5\\
\alpha_{AC}(0) - s&\text{if } \alpha_{AC}(0) > \pi/5\\
\end{cases}
\end{equation}
(If $\alpha_{AC}(0) = \pi/5$, then $\alpha_{AC}$ remains fixed along the path $P$.)
This gives compatibility. If for example, $\alpha_{AC,0}(0) < \pi/5$, then
\[
\alpha_{AC,0}(s) + \alpha_{AC,1}(s) = (\alpha_{AC,0}(0) + s) + (\alpha_{AC,1} - s) = 2\pi/5,
\]
as desired.

We need to check that the path $P$ terminates at some dimer pair $\Gamma(t)$.  This follows from
the following properties of the path that are true by construction.  The angle $\alpha_{AC}$ moves
towards $\pi/5$. (See Equation \rif{eqn:alpha-pi5}). The pentagons $B$ and $D$ maintain double contact
and rotate towards slider contact.   If $\alpha_{AC}=\pi/5$ and if $B$ (or $D$) has slider contact with $A$ or $C$,
then it has slider contact with both, and must therefore be a dimer pair on the curve $\Gamma$.

We claim that the parameter $t$ satisfies $|t|<0.01$.  In fact, 
because $t' := x_{AC}-\sigma$ is a coordinate defining the
coordinate neighborhood of the $K$-dimer so $|t'|<0.01$.  Also, $t'$ remains fixed along the path $P$,
and $t'$ restricts to the parameter $t$ on $\Gamma$.  Thus, at the terminal point of the path, $|t|<0.01$.

Finally, we need to check that the area function is decreasing. This, we prove by a computer calculation
of the derivative of the area function along the path at $s=0$.  For this, we use automatic differentiation algorithms,
as described in Section~\rif{XX}.  This completes the reduction to points on the curve $\Gamma(t)$.


\subsection{global minimization along the curve}

The final stage of the proof of Theorem \rif{XX} is the minimization of the area of a dimer pair along the
curve $\Gamma(t)$, for $|t|<0.01$.    We have now reduced to an optimization problem in a single variable 
that is relatively easy to solve.  The area function is obviously analytic.
By automatic differentiation, we take the second derivative of the area
function as a function of $t$ and calculate that it is always positive.  Thus, the area function has a unique
global minimum on $|t|\le 0.01$.  
By symmetry in the underlying geometry, 
it is clear that the area function has derivative zero along $\Gamma$ at
$t=0$.  The global minimum is therefore given at $t=0$, which is the $K$-dimer.

\section{Pseudo-Dimers}

In this section, we determine the structure of pseudo-dimers and specialize
the function $b$ to this context.

On a nonobtuse triangle, the area is a monotonic function of its edge lengths.
This makes it easy to give lower bounds on triangle areas.
Here are some simple    area calculations that will be used repeatedly.
\begin{align*}
\area(1.8,1.8,1.8) &> \aK + 0.112\\
\area(1.8,1.8,1.72) &> \aK + 0.069\\
\area(1.8,1.72,1.72) &> \aK + 3\epso'\\
\area(1.8,1.8,2\kappa) &> \aK + \epso'\\
\area(\kappa\sqrt8,2\kappa,1.72) &> \aK +\epso.
\end{align*}


\subsection{properties of pseudo-dimers}

\begin{lemma}\libel{lemma:pd-m2}  If $(T_1,T_0)\in\PD$, then $m_2(T_0)=0$.
\end{lemma}

\begin{proof} Assume for a contradiction that $m_2(T_0)>0$.  Then there exists $(T'_1,T_0')\in\PD$ and
$(T'_0,T_0)\in\M$.  The shared edge between $T_0$ and $T'_0$ has length greater than $1.8$.
This is the egressive edge $e$ of $T_0$.  This is impossible by Lemma~\rif{calc:good}, which states
that the condition of having a good angle is not symmetrical.
\end{proof}

\begin{corollary}\libel{lemma:m1m2}  For every triangle $T$, either $m_1(T)=0$ or $m_2(T)=0$.
\end{corollary}

\begin{proof}  Suppose $m_1(T)>0$, then $T=T_0$ for some $(T_1,T_0)\in\PD$.  The
lemma gives $m_2(T)=0$.
\end{proof}

\begin{lemma}[pseudo-dimer disjointness]  Assume $(T_1,T_0)\in \PD$ and $(T'_1,T'_0)\in\PD$ and
$\{T_0,T_1\}\cap\{T'_0,T'_1\} \ne \emptyset$.  Then $T_0 = T'_0$.
\end{lemma}

\begin{proof}  If we dismiss the other three cases $T_1=T'_1$ and $T_1=T'_0$ and $T_0=T'_1$ of a nonempty
intersection, then the conclusion $T_0 = T'_0$ will stand.

Assume first that $T_1 = T'_1$.  Then $T_1\Ra T_0$ and $T_1\Ra T'_0$, which gives $T_0=T'_0$.

Next assume that $T_1=T'_0$.  We have $T'_1\Ra T'_0 = T_1 \Ra T_0$.  By the calculations above,
this puts incompatible constraints on the length of the shared edge between $T_0$ and $T_1$.  It must have
length less than $1.8$ and greater than $1.8$.

The case $T_0=T'_1$ follows from the previous case by symmetry.
\end{proof}

\begin{lemma}[pseudo-dimer-obtuse]\libel{lemma:pd-obtuse}
Assume $(T_1,T_0)\in\PD$.  Then each edge of $T_0$ and $T_1$ has length less than $\sqrt8\kappa$.
In particular, if  $T_2$ is obtuse then  $T_2\nRa T_0$ and $T_2 \nRa T_1$.
\end{lemma}

\begin{proof}  The shared edge between $T_0$ and $T_1$ has length less than $1.8 < \sqrt8\kappa$.  The shared
edge is the longest edge of $T_1$, so that each edge of $T_1$ has length less than $1.8$.
The only possibility is the egressive edge of $T_0$.  But if the egressive edge of $T_0$ has length at least $\sqrt8\kappa$,
then
\[
\area\{T_1,T_0\} > a_0 + \area(\kappa\sqrt8,2\kappa,1.72) > a_0 + (\aK + \epso) = 2\aK.
\]
This contradicts the area condition in the definition of pseudo-dimer.  This completes the first claim of the lemma.

If $T_2$ is obtuse, then its longest edge has length at least $\sqrt8\kappa$.  If $T_2\Ra T$, then the shared edge
has length at least $\sqrt8\kappa$.
\end{proof}

\begin{corollary}\libel{lemma:pd-n2}
If $(T_1,T_0)\in\PD$, then $n_2(T_0)=0$.
\end{corollary}

\begin{proof} Suppose for a contradiction that $n_2(T_0)>0$.  Then $(T,T_0)\in \N$ for some $T$.  Since $T_0$ is
nonobtuse, this implies that there exists some obtuse triangle $T_2$ such that $T_2\Ra T_0$.  This is contrary
to the previous lemma.
\end{proof}

\begin{lemma}\libel{lemma:pd-1} Let $(T_1,T_0)\in\PD$.  Then $n_1(T_1)=n_2(T_1)=m_1(T_1)=m_2(T_1)=0$.  In particular,
$b(T_1) = \area(T_1)$.
\end{lemma}

\begin{proof}  We claim that $m_1(T_1)=m_2(T_1)=0$.  Otherwise $T_1$ shares an egressive edge with some
peudo-dimer of length greater than $1.8$.  However, the longest edge of $T_1$ is its shared edge with $T_0$, which
has length less than $1.8$.  This gives the claim.

Next we claim that $n_1(T_1)=0$.  Otherwise, $(T_1,T_0)\in \N$.  Since $T_0$ is nonobtuse, this implies
that there exists an obtuse triangle $T_2$ such that $T_2\Ra T_0$.  This contradicts Lemma~\rif{lemma:pd-obtuse}.

Finally, we claim that $n_2(T_1)=0$.  Otherwise, $(T,T_1)\in \N$ for some $T$.  Since $T_1$ is nonobtuse,
this implies that there exists an obtuse triangle $T_2$ such that $T_2\Ra T_1$.  This contradicts
Lemma~\rif{lemma:pd-obtuse}.
\end{proof}

\begin{lemma}\libel{lemma:pd-b}  Let $(T_1,T_0)\in\PD$.  Then $b(T_0) > \aK$.  That is, $T_0$ is not $b$-subcritical.
\end{lemma}

\begin{proof}
We recall from Corollary~\rif{lemma:pd-n2} that $n_2(T_0)=0$.  From Lemma~\rif{lemma:pd-m2} we
have $m_2(T_0)=0$.  Thus, $b(T_0) = \area(T_0) + \epso n_1(T_0) + \epso' m_1(T_0)$.
We have
\[
\area(T_0) = \area\{T_0,T_1\} - \area(T_1) > (2\aK-\epso') -\aK = \aK - \epso'.
\]

We first treat the case that $n_1(T_0)>0$.  Then we have $n_1(T_0)\ge 1$.
Thus, $b(T_0) \ge \area(T_0) + \epso n_1(T_0) > (\aK - \epso') + \epso > \aK$.
This completes this case.  

For the remainder of the proof, we assume that $n_1(T_0)=0$.  In particular, we have
$b(T_0) = \area(T_0) + \epso' m_1(T_0)$.

Next, we treat the case in which $T_1$ is the unique triangle $T$ such that $(T,T_0)\in\PD$.
In this case, $(T_0,T_1)\not\in\M$, and $(T_0,T_1)\in\M$, so that $m_1(T_0)=1$.  Thus
$b(T_0) > (\aK-\epso') + \epso' = \aK$.

In the remaining case, there exists $T'_1\ne T_1$ such that $(T'_1,T_0)\in\PD$.  The edges of
$T_0$ shared with $T_1$ and $T'_1$ have length at least $1.72$ and less than $1.8$.  The
egressive edge of $T_0$ has length at least $1.8$.  Thus,
\[
b(T_0) \ge \area (T_0) \ge \area(1.72,1.72,1.8) > \aK.
\]
This completes the proof.
\end{proof}


\section{Main inequality for dimers}

Recall that
the previous section shows that there exists a unique dimer pair (up to isometry).
It is the $K$-dimer.  In particular, if $(T_1,T_0)$ is a dimer, then 
$\area(T_0)=\area(T_1)=\aK$, and $T_0\Ra T_1$ and $T_1\Ra T_0$.
This section proves the following theorem.

\begin{theorem}\libel{thm:dimer} Let $(T_0,T_1)$ be a dimer pair.  Then for $T=T_0,T_1$,
we have $m_1(T)=m_2(T) = n_1(T)=n_2(T) = 0$; and $b(T) = \area(T)$.
Moreover, $\{T_0,T_1\}$ is a $b$-cluster, and the main inequality holds for
$\{T_0,T_1\}$.
\end{theorem}

The proof will occupy the entire section.  Before treating dimers, we treat the
easy case of singletons.

\begin{lemma}\libel{lemma:singleton}  Assume that all the edges of a triangle $T$ have length less than $1.72$.
Then the $b$-cluster of $T$ is the singleton $\{T\}$ and $b(T) =\area(T) > \aK$.  Generally,
if $\{T\}$ is a singleton $b$-cluster, then $b(T) > \aK$ and the main inequality holds.
\end{lemma}

\begin{proof}  By the definition of the constants $n,m$, we have $n_1(T)=n_2(T) = m_1(T)=m_2(T)=0$.
Thus, $b(T)=\area(T)$.  We have $\area(T) > \aK$ by Lemma~\rif{XX}.  Since $T$ is not $b$-subcritical,
there are no arrows $T\rab -$.  

We claim that there cannot exist an arrow $T'\rab T$.
Otherwise,  the longest edge of $T'$ has length less than $1.72$
and again $T'$ is not $b$-subcritical.  The claim follows and the $b$-cluster is a singleton.

In general, if $\{T\}$ is any singleton cluster, there is no arrow $T\rab -$.  This implies that $T$ is not
$b$-subcritical, so that $b(T) > \aK$.  The result follows.
\end{proof}

Next,
we show the disjointness of dimers from pseudo-dimers.  

\begin{lemma} Let $(T_1,T_0)\in DP$ and let $(T'_1,T'_0)\in\PD$.  Then for $T\in\{T_0,T_1\}$,
we have
$T'_0\nRa T$.
\end{lemma}

\begin{proof} Assume  $T'_0\Ra T$.  By Lemma~\rif{lemma:egress'},
we have $\area(T)>\aK$.  But if $T\in\{T_0,T_1\}$, we have $\area(T)=\aK$.
This gives the result.
\end{proof}

\begin{lemma}\libel{lemma:m1-dimer}  Let $(T_1,T_0)\in DP$ and let $(T'_1,T'_0)\in\PD$.  Then
$\{T_0,T_1\}\cap \{T'_0,T'_1\} = \emptyset$.  In particular, $m_1(T_0) = m_1(T_1)=0$.
\end{lemma}

\begin{proof}
We claim that $T'_0\not\in \{T_0,T_1\}$.  Otherwise, $T'_0\Ra T\in\{T_0,T_1\}$, which is contrary
to the previous lemma.

We claim that $T'_1\ne T_1$.  Otherwise, if $T_1=T'_1$, then $T'_1=T_1\Ra T_0$ and $T'_1\Ra T'_0$,
so that $T_0 = T'_0$, which have shown impossible.

Finally, we claim that $T'_1\ne T_0$.  Otherwise if $T_0 = T'_1$, then $T'_1 = T_0\Ra T_1$ and
$T'_1\Ra T'_0$, so $T_1 = T'_0$, which we have shown to be impossible.

If $m_1(T) >0$, then there exists $T'$ such that $(T',T)\in \PD$.  This is impossible for $T\in\{T_0,T_1\}$ by the
disjointness result.
\end{proof}

\begin{lemma}\libel{lemma:m2-dimer}  Let $(T_1,T_0)\in DP$ and let $(T'_1,T'_0)\in PD$.  Then $T'_0\nRa T_0$ and
$T'_0 \nRa T_1$.  In particular, $m_2(T_0)=m_2(T_1)=0$.
\end{lemma}

\begin{proof}  $T'_0\nRa T$, for $T\in\{T_0,T_1\}$ by an earlier lemma.

If $m_2(T)>0$, then by the definition of the set $\M$,  there exists $(T_1,T_0)\in \PD$, such that $T_0\Ra T$.
The result follows.
\end{proof}

\subsection{Interaction of dimers and obtuse triangles}




\begin{lemma}\libel{lemma:t2b-nonobtuse}
  If $T_1$ is obtuse, $T_2$ is nonobtuse, and if $T_1 \Ra T_2$, then $T_2$ is not
  $b$-subcritical.
\end{lemma}


\begin{proof} 
  If $T_1$ is obtuse, then its longest edge, which is shared with
  $T_2$, has length at least $\kappa\sqrt8$.

  Assume that $T_2$ is nonobtuse and $b$-subcritical.  Then $\area(T_2) > a_K$ by
  Lemma~\rif{lemma:right}.  Thus, we must have $n_2(T_2)>0$ or $m_2(T_2)>0$.
  We consider two cases, according to which of these conditions occur.

 Assume in the first case that $n_2(T_2)>0$; that is, $(T',T_2)\in \N$.  Recall that this
 implies $m_2(T_2)=0$ by Lemma~\rif{XX}.
  We have $T_1\Ra T_2$ and $(T_1,T_2)\not\in \N$, because $T_1$ is obtuse.  
  It follows that $n_2(T_2)\le 2$.  In fact, $n_2(T_2)$ is equal to the number of
  nonobtuse $T$ with longest edge of length greater than $1.72$ such that $T\Ra T_2$.
  If $n_2(T_2)=1$, we have
  \[
b(T_2) \ge \area(T_2) - \epso > \area(\kappa\sqrt8,1.72,2\kappa)  - \epso > \aK.
\]
If $n_2(T_2)=2$, we have
  \[
b(T_2) \ge \area(T_2) - 2\epso > \area(\kappa\sqrt8,1.72,1.72)  - 2\epso > \aK.
\]
  This completes the first case.

  Finally, we consider the case that $m_2(T_2)>0$ (and $n_2(T_2)=0$).
  We have $m_2(T_2)\le 2$, because $T_1\Ra T_2$, and $T_1$ is obtuse, and cannot be
part of a pseudo-dimer.
  The following area estimate gives the result.
  \[
b(T_2) \ge \area(T_2) - 2\epso'\ > \area(\kappa\sqrt8,1.8,2\kappa)  - 2\epso' > \aK.
\]
\end{proof}

\begin{corollary}\libel{lemma:n2b} If $n_2(T_2)>0$  with $T_2$ nonobtuse, then $b(T_2) > \aK$.
\end{corollary}  

\begin{proof}  By the definition of $\N$, There exists an obtuse triangle $T$ such that $T\Ra T_2$.
The result follows.
\end{proof}

\begin{lemma}\libel{lemma:m2b}  Let $T$ be $b$-subcritical and nonobtuse.
Then $n_2(T)=0$.  Moreover, assume that there exists $T'$ that is $b$-subcritical
such that $T'\rab T$ or $T\rab T'$.  Then  $\area(T)\le \aK$.
\end{lemma}

\begin{proof}  
By the contrapositive of the corollary, it follows that $n_2(T)=0$.

Assume for a contradiction that $\area(T) > \aK$.
We have
\begin{equation}\libel{eqn:2-ac}
\area(T) > \aK \ge b(T) \ge \area(T) - m_2(T) \epso'.
\end{equation}
Thus, $m_2(T) >0$.  
By the definition of $\M$, there exists  $(T'_1,T'_0)\in\PD$ with $(T'_0,T)\in\M$ and $T'_0\Ra T$.
%We have $T'_0\ne T_1$, because Lemma~\rif{XX} gives $b(T'_0) > \aK \ge b(T_1)$.
By Corollary~\rif{lemma:egress'}, we have $\area(T) > \aK+\epso'$. Combined with
Inequality ~\rif{eqn:2-ac}, this gives $m_2(T)\ge 2$.
Repeating the argument, we have $(T''_1,T''_0)\in \PD$ with $(T''_0,T)\in \M$ and $T''_0\Ra T$.
The triangles $T'$, $T'_0$, and $T''_0$ are distinct, because for example $b(T'_0) > \aK \ge b(T')$.  
Because of the arrow $T'\rab T$ or $T'\rab T$, the triangles $T$ and $T'$ belong to the same $b$-cluster.
Hence, the longest edge of $T'$ has length at least $1.72$.
Then
\[
b(T) \ge \area(T) - m_2(T)\epso' > \area(1.8,1.8,1.72) - 3\epso' > \aK.
\]
This  contradicts the assumption that $T$ is $b$-subcritical.
\end{proof}

\begin{lemma} \libel{lemma:rab-sequence} There does not exist a sequence
$T_1\rab T_2 \rab T_3$, with $T_1\ne T_3$, in which $T_1$ and $T_2$  are
both nonobtuse.
\end{lemma}

\begin{proof}  Assume for a contradiction that the sequence exists.  
By the Lemma~\rif{lemma:m2b}, $n_2(T_2)=0$ and $\area(T_2)\le \aK$.

We claim that $m_1(T_2)=0$.  Otherwise, there exists $(T'_1,T_2)\in\PD$, and by Lemma~\rif{XX},
$T_2$ is not $b$-subcritical, contradicting the assumptions of the lemma.

We claim that $m_2(T_2)=0$.  This follows by Lemma~\rif{lemma:m2-area} and the claim $\area(T_2)\le \aK$.

We have that $n_1(T_2)=0$.  Otherwise, we reach the contradiction,
\[
\aK \ge b(T_2) =\area(T_2) + \epso n_1(T_2) > a_0 + \epso = \aK.
\]
This shows that $\area(T_2) = b(T_2)$.

By Lemma~\rif{lemma:m2b}, we have $\area(T_1)\le \aK$.
It follows that $(T_1,T_2)\in\PD$, and by an earlier lemma, 
we reach a contradiction $\aK \ge \area(T_2) = b(T_2) > \aK$.
\end{proof}
  
We are finally in a position to prove  Theorem~\rif{thm:dimer}.

\begin{proof}[Proof of dimer theorem \rif{thm:dimer}.]
%Let $(T_0,T_1)$ be a dimer pair.  Then for $T=T_0,T_1$,
%we have $m_1(T)=m_2(T) = n_1(T)=n_2(T) = 0$; and $b(T) = \area(T)$.
%Moreover, $\{T_0,T_1\}$ is a $b$-cluster, and the main inequality holds for
%$\{T_0,T_1\}$.
Let $(T_0,T_1)$ be a dimer pair.   We have proved that $m_1(T)=m_2(T)=0$ in Lemmas~\rif{lemma:m1-dimer} and \rif{lemma:m2-dimer}.

We claim that $n_2(T_0)=n_2(T_1)=0$.  Otherwise, $T_1$ or $T_0$ has an edge of length at least $\kappa\sqrt8$, and the triangle
is not subcritical.

We claim that $n_1(T_0)=n_1(T_1)=0$.  Otherwise, say $(T_1,T_0)\in\N$ and we have the contradiction $n_2(T_0)>0$.

This shows that $b(T_i)=\area(T_i)$, for $i=0,1$.
Since $T_0$ and $T_1$ are both subcritical, they are also $b$-subcritical, and fall into the same $b$-cluster.
This is the full $b$-cluster, for otherwise, we would have say $T\rab T_0\rab T_1$, which is impossible by
Lemma~\rif{lemma:rab-sequence}, for $T$ nonobtuse.  And if $T$ is obtuse, this would give $T_0$ an edge of length at least $\kappa\sqrt8$.

This gives the proof.
\end{proof}





\section{Cluster Structure}


% Fig 1. 1-obtuse.pdf page 13b.


%\subsection{structure of  clusters}

We continue with our analysis of the clusters in a fixed saturated
packing of regular pentagons.  
%By an {\it obtuse cluster}, we mean any
%$b$-cluster of $P$-triangles that contains at least one obtuse
%triangle.

\begin{lemma}\libel{lemma:notlong}  
Let $T_1$ and $T_2$ be $P$-triangles, such that $T_1\Ra T_2$, where $T_1$ is nonobtuse subcritical and
  $T_2$ is obtuse.  Then the edge of attachment is not the longest
  edge of $T_2$.
\end{lemma}

\begin{proof} We have seen that each edge of a nonobtuse subcritical
  triangle has length less than $\kappa\sqrt8$ and that the longest edge of an
  obtuse Delaunay triangle has length at least $\kappa\sqrt{8}$.  These are
  incompatible conditions on a shared edge.
\end{proof}

\begin{lemma}  
  Let $(T_1,T_2)\in\N$, where $T_2$ is obtuse.  Then the edge shared between the triangles is
  not the longest edge of $T_2$.  In particular, $n_2(T_2)\le 2$.
\end{lemma}

\begin{proof} If $(T_1,T_2)\in \N$, then $T_1$ and $T_2$ satisfy the
  assumptions of Lemma~\rif{lemma:notlong}.
\end{proof}

\begin{lemma}\libel{lemma:3m2} 
If $m_2(T) = 3$, then $b(T) > \aK + \epso$.
\end{lemma}

\begin{proof}  By Lemma~\rif{lemma:n2m2}, $n_2(T)=0$.
There is a longest edge of a pseudo-dimer along each edge of $T$.
This gives
\[
b(T) \ge \area(T) - 3\epso' \ge \area(1.8,1.8,1.8) - 3\epso' > \aK + \epso.
\]
\end{proof}

\begin{lemma}\libel{lemma:Nb}  If $(T_1,T_0)\in \N$, then $b(T_1)>\aK$.
\end{lemma}

\begin{proof}  Otherwise,
\[
\aK \ge b(T_1) = \area(T_1) + \epso (1 - n_2(T_1)) + \epso' (m_1(T_1)-m2(T_1) > \aK - \epso n_2(T_1) - \epso' m_2(T_1).
\]
So $n_2(T_1) > 0$ or $m_2(T_1)>0$. This gives two cases.  

Suppose that $n_2(T_1) >0$.  Then $(T,T_1)\in\N$ for some nonobtuse $T$ whose shared edge with $T_1$ has length at least $1.72$.
Recall that $T_1$ is nonobtuse.  Thus, by the definition of $\N$, there exists $T'$ obtuse such that $T'\Ra T_1$.  The shared edge
has length at least $\kappa\sqrt8$.  Also, $n_2(T_1)\le 2$ (because $(T',T_1)\not\in \N$).  By Lemma~\rif{lemma:n2m2}, we have
$m_2(T)=0$.  Then
\[
b(T_1) \ge \area(T_1) + \epso (1 - 2) \ge \area (2\kappa,1.72,\kappa\sqrt8) > \aK.
\]
This completes this case.

Finally, suppose that $m_2(T_1)>0$ and $n_2(T_1)=0$.  There exists a pseudo-dimer $(T'_1,T'_0)$ such that $T'_0\Ra T_1$.
We have
\[
b(T_1) > (\aK + \epso') + \epso -\epso'  3 = \aK.
\]
\end{proof}



\begin{lemma}\libel{lemma:no-ao} 
  There is no arrow $T_1 \rab T_2$ with $T_1$ nonobtuse and $T_2$
  obtuse.
\end{lemma}

\begin{proof}  
  Assume for a contradiction that such a pair $(T_1,T_2)$ exists.
By Lemma \rif{lemma:Nb}, $(T_1,T_2)\not\in\N$.
By the definition of $\N$, the longest edge of $T_1$ has length
less than $1.72$.  By Lemma~\rif{lemma:singleton}, $T_1$ forms
a singleton cluster.  This contradicts $T_1\rab T_2$.
\end{proof}


\section{Obtuse clusters}\libel{sec:obtuse}

%Recall from above that we refine the $b$-clusters into $b$-clusters by
%defining a relation $(\ra{c})$ and corresponding equivalence relation
%$(\equiv_c)$, where $T_1 \ra{c} T_2$ iff ($T_1 \rab T_2$ and ($T_1$
%is obtuse or $T_2$ is not $b$-subcritical)).  Recall that a
%$b$-cluster $\C$ is an equivalence class of Delaunay triangles under
%the corresponding equivalence relation $(\equiv_c)$.

\begin{remark}\libel{rem:delaunay}
  Recall that the Delaunay property implies that two adjacent Delaunay
  triangles $T_1$ and $T_2$ have the property that the angle
  $\alpha_1$ of $T_1$ and $\alpha_2$ of $T_2$ satisfy $\alpha_1 +
  \alpha_2\le \pi$, where $\alpha_i$ is the angle of $T_i$ that is not
  on the shared edge of $T_1$ and $T_2$.  In particular, {\it two
    obtuse Delaunay triangles cannot be joined along an edge that is
    the longest on both triangles.}  The extreme case
  $\alpha_1+\alpha_2=\pi$ corresponds to the degenerate situation
  where $T_1$ and $T_2$ form a cocircular quadrilateral. When
  cocircular, either diagonal of the quadrilateral gives an acceptable
  Delaunay triangulation.
\end{remark}



\begin{lemma}\libel{lemma:t2b}
  If $T_1$ is obtuse, and $T_1 \Ra T_2$, then $T_2$ is not
  $b$-subcritical.
\end{lemma}


\begin{proof}  If $T_2$ is nonobtuse, then this is Lemma~\rif{lemma:t2b-nonobtuse}.

Assume that $T_2$ is obtuse.  By basic properties of Delaunay
triangles (Remark~\rif{rem:delaunay}), Delaunay triangles never join
along an edge that is the longest on both triangles.  Thus, $T_1$
attaches to $T_2$ along an edge adjacent to the obtuse angle of $T_2$.
To bound the area of $T_2$, we deform $T_2$ decreasing its area and
increasing its longest edge and its circumradius, until we obtain a
triangle of circumradius $\eta=2$, and shortest edges $2\kappa$ and
$\kappa\sqrt{8}$.  Then a numerical calculation gives
\[
b(T_2) = \area(T_2) - \epso n_2(T_2) \ge 
\areta(2\kappa,\kappa\sqrt{8},2) - 2\epso > \aK.
\] % checked 2016/2/18 in Mathematica. It holds by a margin 0.1856....
The use of the function $\areta$ is justified by
Lemma~\rif{lemma:areta} and the numerical estimate
\[
d_2 = \kappa\sqrt8 <  4 \sin(\arc(2,2,2\kappa)) = 2\eta\sin(\alpha/2).
\] % checked 2016/2/18 in Mathematica.
\end{proof}


\begin{lemma}\libel{lemma:sequence}  There does not exist a three
term sequence $- \rab -\rab -$ where  the three triangles  are distinct.
\end{lemma}

\begin{proof}  
Assume for a contradiction, that such a sequence exists.
By Lemma~\rif{XX}, there does not exist a sequence $- \rab - \rab -$,
where the first triangle is obtuse.  Thus, we may assume that the first
triangle is nonobtuse.
By Lemma~\rif{XX}, there does not exist $T_1\rab T_2$,
where $T_1$ is nonobtuse and $T_2$ is obtuse.  Thus, we may assume
that every triangle in the sequence is nonobtuse.  This is impossible by
Lemma~\rif{XX}.
\end{proof}

If $\C$ is a $b$-cluster that is not a singleton, then there is
some arrow $T_1\rab T_0$.  We have the following structure
lemma for $b$-clusters.

\begin{lemma}\libel{lemma:c-weak}
  Let $\C$ be a $b$-cluster, and let $T_1\rab T_0$ be an
  arrow between triangles in $\C$.
  Then $\C = \{T_0\}\cup \{T \mid T\rab T_0\}$.
\end{lemma}

\begin{proof}  There is no arrow out of $T_0$ because that
would produce a sequence $T_1\rab T_0\rab -$.  

Assume that $T\rab T_0$.  There is
no arrow into $T$,  becaue that would also produce a sequence
of length three: $- \rab T_1 \rab T_0$.  There is a unique arrow out of $T$.
Thus, we have accounted for all of the arrows in and out of $T_0$ and in and
out of $T$.
\end{proof}

\begin{corollary}\libel{lemma:card4}
  Every $b$-cluster is finite of cardinality at most $4$.
\end{corollary}


The following theorem and its corollary is the main result of this
section.  

\begin{theorem}\libel{lemma:obtuse}  
  Let $\C$ be any $b$-cluster that contains an obtuse triangle.  Then
  $\C$ is finite, and the strict main inequality
  (\rif{eqn:strict-main}) holds for $\C$.
\end{theorem}

\begin{lemma} Let $T\rab T_0$ be an arrow between two triangles in
a $b$-cluster that contains an obtuse triangle.  Then $T$ is obtuse.
Moreover, $b(T) = \area(T) -\epso n_2(T)$.
\end{lemma}

\begin{proof} Assume for a contradiction that $T$ is nonobtuse.  By Lemma~\rif{XX}
and the arrow $T\rab T_0$, the triangle $T_0$ is nonobtuse.  By assumption and the structure
theorem for $b$-clusters, there exists
$T'\rab T_0$, where $T'$ is obtuse.
The singleton lemma implies that the longest edge of $T$, which is shared with $T_0$ has length
at least $1.72$.  By the definition of $\N$, we have $(T,T_0)\in\N$.   By Lemma~\rif{lemma:Nb}, we
have $b(T)>\aK$, and $T$ is not $b$-subcritical.  Thus, we obtain a contradiction $T\nRa T_0$.

Moreover, $b(T) = \area(T) -\epso n_2(T)$ by Lemma~\rif{XX}.
\end{proof}

Let $T\rab T_0$ be an arrow between two triangles in a $b$-cluster, where $T$ is obtuse.
Let $\tilde n_2(T)\le 2$ be the number of edges of $T$ other than the longest that
have length at least $1.72$.  By definition of $\N$ and $n_2$, we have $\tilde n_2(T)\ge n_2(T)$.
Then $\tb(T) \le b(T)$, where $\tb(T) := \area(T)-\epso\tn_2(T)$.

If $T_0$ is obtuse, then again $\tb(T_0)\le b(T_0)$.
If $T_0$ is nonobtuse, then
\[
b(T) \ge \area(T_0) - \epso n_2(T_0) - \epso' m_2(T_0) \ge \area(T_0) - \epso (n_2(T_0)+m_2(T_)) \ge \tb(T).
\]
This completes the proof of the following lemma.

\begin{lemma} Let $\C$ be  a $b$-cluster that contains an obtuse triangle, and
let $T\in \C$.  Then  $\tb(T)\le b(T)$.
\end{lemma}






\begin{proof}[proof of Theorem~\rif{lemma:obtuse}.]
  The proof involves several relatively simple cases.  
  We recall that each Delaunay triangle has edge lengths at
  least $2\kappa$ and circumradius at most $2$.

  If the $b$-cluster is a singleton $\{T_1\}$, where $T_1$ is obtuse,
  then the singleton lemma gives the result.
  We now assume that $\C$ is not a singleton.


We break the proof into six cases depending on whether $T_0$ is
nonobtuse, and depending on $\card(\C)\in \{2,3,4\}$.
In each case we prove inequality 
\begin{equation}\libel{eqn:n'}
\sum_{T\in \C'} \area(T) > a_K \card(\C) + n'\epso.
\end{equation}
where $n'$ is the total number of external edges of the cluster of length at least $1.72$.
(By external edge, we mean edge not shared between two triangles in the cluster.)
This implies that
\[
\sum_{T\in\C} b(T) \ge \sum_{T\in\C} \tb(T) \ge \sum_{T\in\C}\area(T) - \sum_{T\in\C} \tn_2(T) > a_K \card(\C),
\]
which is the strong main inequality.

{\it Case 1. The triangle $T_0$ is a nonobtuse triangle, and
  $\C=\{T_0,T_1\}$.}  The triangle $T_0$ has a vertex $\v$ that is
not shared with $T_1$.  By the Delaunay property, $\v$ lies outside
the circumcircle of $T_1$.  The triangles $T_0$ and $T_1$ form a
quadrilateral $Q$ whose diagonal is the shared edge of $T_0$ and
$T_1$.  We deform the quadrilateral $Q$ to decrease its area while
maintaining the following constraints:
\begin{enumerate}
\item The vertex $\v$ lies on or outside the circumcircle of
  $T_1$. The circumradius of $T_1$ is at most $2$.
\item The edge length of the $i$th edge of $Q$ is at least
  $d_i\in\{2\kappa,1.72\}$,
where $n'$ is the number of $d_i$ that equal $1.72$; and
\item $T_1$ is not acute.
\end{enumerate}
We drop all other constraints as we deform. (In particular, we do not
enforce the nonoverlapping of pentagons in the $P$-triangles.)
We continue to deform $Q$ until one of the following two subcases hold:
\begin{enumerate}
\item $Q$ is cocircular; or
\item For all $i=1,2,3,4$, the $i$th edge of $Q$ has reached its lower
  bound $d_i$.
\end{enumerate}

In the first subcase (cocircularity), we drop the third constraint
(non-acuteness) and continue area decreasing deformations for $Q$
under the constraint of a fixed circumcircle.  We note that the area of
a cocircular quadrilateral $Q$ depends only on the lengths of the
edges and not on their cyclic order on $Q$.  We may thus rearrange the
edge order as we deform.  For a given circumcircle, the area is
minimized when three of the edges attain their lower bound $d_i$.  By
suitable reordering of the edges, we may assume that $Q$ is an
isosceles trapezoid and that the ``free'' edge is parallel to and
longer than its opposite edge on $Q$.  For such $Q$, the area as a
function of the circumradius is concave, so that the minimum occurs
when the circumradius is as small (that is, all edges attain the
minimum $d_i$) or as large (that is, $\eta(Q)=2$) as possible.
When $\eta(Q)=2$, we relax the edge lengths constraints further
to allow three edges to have length $2\kappa$.
Explicit numerical calculations in these two extremal configurations
show that the inequality (\rif{eqn:n'}) is satisfied (for each $n'$).

In the second subcase (every edge attains its minimal length $d_i$), the
four edge lengths are fixed.  The area of $Q$ is a concave function of
the length of the diagonal.  We thus minimize the area of $Q$ when the
diagonal is as small as possible (that is, $T_1$ is a right triangle
-- when this satisfies the other constraints) or as large as possible
(that is, $Q$ is cocircular).  The cocircular case has already been
considered.  Explicit numerical calculations of $Q$ when $T_1$ is
right gives the inequality (\rif{eqn:n'}) in each case.
% checked 2016/2/18 in Mathematica.




{\it Case 2. The triangle $T_0$ is a nonobtuse triangle, and $\C=\{T_0,T_1,T_1'\}$.}
The long edges of the obtuse triangles $T_1$ and $T_1'$ have length at least
$\kappa\sqrt{8}$.

We consider a subcase where $\eta(T_1)\le 1.7$ and $\eta(T_1') \le
1.7$.  Then calculations based on the monotonicity of the area
functions give
\[
\area(T_0) \ge \area(d,\kappa\sqrt{8},\kappa\sqrt{8}) > 
   \begin{cases}1.73,&\text{if } d \ge 2\kappa\\ 
     1.73+\epso, & \text{if } d \ge 1.72 \end{cases}.
\]
The areas of $T=T_1,T_1'$ are at least
\begin{equation}\libel{eqn:173}
\area(T) \ge \areta(d,2\kappa,1.7) >
   \begin{cases}1.08,&\text{if } d \ge 2\kappa\\
     1.08+2\epso, & \text{if } d \ge 1.72 \end{cases}.
\end{equation}
These bounds give
\[
\area(T_1) + \area(T_1') + \area(T_0) > 1.73 + 2(1.08) + n'\epso > 
3 \aK + n' \epso.
\]
% checked 2016/2/18 in Mathematica.

By symmetry, we may now assume that $\eta(T_1) \ge 1.7$.  By the
Delaunay condition, since $T_1$ is obtuse and $T_0$ is nonobtuse, this
forces $\eta(T_0)\ge 1.7$.  We minimize the area of $T_0$ subject to
the constraints that its circumradius is at least $1.7$, that it is
nonobtuse, and its edge lengths are at least $\kappa\sqrt{8}$,
$\kappa\sqrt{8}$, and $2\kappa$.  If two edges are $2\kappa$,
$\kappa\sqrt{8}$, then $T_0$ is obtuse, so the binding constraints for
the optimization become $\eta(T_0)=1.7$, $2\kappa$ edge length, and a
right triangle.  Such a triangle has area at least 
\[
2\kappa\sqrt{\eta^2 - \kappa^2} \ge 2.41.
\]  
The areas of $T_1$ and $T_1'$ are at least
\begin{equation}\libel{eqn:968}
\areta(2\kappa,2\kappa,2) > 0.968.
\end{equation}
This completes this case:
\[
\area(T_1) + \area(T_1') + \area(T_0) 
>
2(0.968) + 2.41 > 3\aK + 5 \epso \ge 3\aK + n'\epso.
\] %checked 2016/2/18 in Mathematica.

{\it Case 3. The triangle $T_0$ is a nonobtuse triangle, and $C'=\{T_0,T_1,T_1',T_1''\}$.}

This case is almost identical to case 2.  We use the same bounds
(Equations (\rif{eqn:173}) and (\rif{eqn:968})) on $\area(T)$ as
before, for $T = T_1, T_1', T_1''$.  We can improve the bound on the
area of $T_0$:
\[
\area(T_0) \ge \area(\kappa\sqrt{8},\kappa\sqrt{8},\kappa\sqrt{8}) > 2.2668.
\]
Moreover, in the subcase where $\eta(T_0)\ge 1.7$, we have
(even after dropping the nonobtuseness constraint):
\[
\area(T_0) \ge \areta(\kappa\sqrt{8},\kappa\sqrt{8},1.7) > 2.6.
\]
In this case, $n'\le 6$.  Proceeding as before, we get
\[
\area(T_0)  + \area(T_1) + \area(T_1') + \area(T_1'') > 
\begin{cases}
2.2668 + 3 (1.08) \\
2.6 + 3(0.968)
\end{cases}
> 4\aK + n' \epso.
\] %checked 2016/2/18 in Mathematica.


This completes the proof for cases involving a nonobtuse triangle
$T_0$.  In the remaining cases, we assume that $T_0$ is obtuse.  In
fact, every triangle in $\C$ is obtuse.

{\it Case 4. The triangle $T_0$ is an obtuse triangle, and $\C=\{T_0,T_1\}$.}  

In this case, $n'\le 3$.  It will not be necessary to create subcases
according to whether short edges are at least $2\kappa$ or $1.72$.  We
will show that we can relax the lower bound on the edge to $2\kappa$
and still obtain the bound (\rif{eqn:n'}).

We minimize area by flattening $T_0$ by stretching its long edge until
$\eta(T_0)=2$.  We further decrease area, keeping the circumradius
fixed, by contracting the shorter edge not shared with $T_1$, until
the edge has length $2\kappa$.

Next continue to minimize area by contracting an edge of $T_1$,
keeping its circumradius fixed, until an edge has length $2\kappa$.
Then, allowing the circumradius of $T_1$ to increase, we contine until
both shorter edges have length $2\kappa$ or until the circumradius
reaches $2$.

First assume that both shorter edges of $T_1$ have length $2\kappa$.
We have reduced to a one-parameter family of quadrilaterals.  We can
choose the parameter to be the length $x$ of the diagonal, the common
edge of $T_1$ and $T_0$.  The parameter $x$ ranges between
$\kappa\sqrt{8}$ and $x_{\max}\approx 2.9594$, determined by the
condition $\eta(2\kappa,2\kappa,x_{\max}) = 2$.  We check numerically
that
\[
\area(T_1) + \area(T_0) \ge \area(2\kappa,2\kappa,x) +
\areta(x,2\kappa,2) > 2\aK + 3 \epso \ge 2\aK + n'\epso.
\] % checked 2016/2/18 in Mathematica.

Next, assume the circumradius of $\eta(T_1)$ reaches $2$, then we
have a cocircular quadrilateral that can be treated as in Case 1.  In
particular, the minimizing cocircular quadrilateral has three edges of
length $2\kappa$ and circumradius $2$.  This is precisely the limiting
case of a quadrilateral with diagonal $x_{\max}$ considered above.

This completes the argument in this case.

{\it Case 5. The triangle $T_0$ is an obtuse triangle, and $\C=\{T_0,T_1,T_1'\}$.}  

By Remark~\rif{rem:delaunay}, there is no arrow $T \Ra T_0$ in $\C$
such that the shared edge is the long edge of $T_0$.  In particular,
there cannot exist (Case 6) with $\C=\{T_0,T_1,T_1',T_1''\}$ with
every triangle obtuse.  Thus, Case 5 is the last case to be
considered.

We have $n'\le 4$.
The area of $T_0$ is at least
$\areta(\kappa\sqrt{8},\kappa\sqrt{8},2) > 2.45$.  
Using our earlier estimates (\rif{eqn:968})
for $\area(T)$, for $T=T_1,T_1'$, we
have
\[
\area(T_1) + \area(T_1') + \area(T_0) > 2 (0.968) + 2.45 > 3\aK + n'\epso.
\] % checked 2016/2/18 in Mathematica.
%If some shorter edge of $T_1$ or $T_1'$ is at least $1.72$, then we
%again use the estimates from earlier cases to
%obtain % specify earlier cases XX
%\[
%\area(T_1) + \area(T_1') + \area(T_0) >  3\aK  + n'\epso.
%\]
This completes the proof of the theorem.
\end{proof}


\section{Nonobtuse Clusters}\libel{sec:nonobtuse}

In this section we prove the main inequality for $b$-clusters in which every
triangle is nonobtuse.

By Corollary \rif{lemma:n2b} and Lemma \rif{lemma:Nb}, 
if $T$ is $b$-subcritical and nonobtuse, then $n_1(T)=n_2(T)=0$.


\begin{lemma}\libel{lemma:m2} Let $T\rab T_0$. Assume that $T$ and $T_0$ are nonobtuse.
Then $m_2(T)=0$.
\end{lemma}

\begin{proof}  Assume that $m_2(T)>0$.  We have just observed that $n_1(T)=n_2(T)=0$.
By Lemma \rif{lemma:m1m2}, we have $m_1(T)=0$.
By Lemma \rif{lemma:3m2}, we have $m_2(T)\le 2$. 

%We have $n_1(T)=0$.  Otherwise, we have a contradiction:
%\[
%\aK \ge b(T) \ge \area(T) + \epso n_1(T) - \epso' m_2(T) \ge (\aK+\epso') +\epso - 3\epso' > \aK.
%\]

We have $m_2(T)=2$.  Otherwise, if $m_2(T)=1$, we have a contradiction:
\[
\aK \ge b(T) \ge \area(T) - \epso' > (\aK + \epso') - \epso' = \aK.
\]

Since $m_2(T)=2$, there exist two pseudo-dimers $(T'_1,T'_0)$ and $(T''_1,T''_0)$ such that
$T'_0\Ra T$ and $T''_0\Ra T$.  The shared edges $e'$ and $e''$ have length at least $1.8$.
Moreover, the angles are not good along $(T,e')$ and $(T,e'')$.  (See Definition \rif{def:good} and Lemma \rif{lemma:good}.)
The following lemma and the estimate 
\[
\aK \ge b(T) \ge \area(T) - 2\epso' > (\aK + 2\epso') - 2\epso' = \aK.
\]
complete the proof.
\end{proof}

\begin{lemma} Let $T$ be a nonobtuse $P$-triangle.  Suppose that two of its edges $e'$ and $e''$ have length at least $1.8$
and that the angles are not good along $(T,e')$ and $(T,e'')$.  Then $\area(T) > \aK + 2\epso'$.
\end{lemma}

\begin{proof}  If the third edge has length at least $1.63$, then the result follows:
\[
\area(T) \ge \area(1.8,1.8,1.63) > \aK + 2\epso''.
\]
We may assume without loss of generality that the third edge has length in the range $[2\kappa,1.63]$.
In terms of the pentagons and angles in Figure XX, we have
\[
\dd{A}{C}\ge 1.8,\quad \dd{B}{C}\ge 1.8,\quad \dd{A}{B}\le 1.63,\quad \alpha+\beta+\gamma=3\pi/5.
\]
The angle sum condition is only defined modulo $2\pi/5$, but we may fix $\gamma$ to lie in the
interval $[0,2\pi/5]$.
The assumption on angles gives $\alpha,\beta\le \pi/5$, so $\gamma\ge \pi/5$.
We now disregard the pentagon $C$.  We consider the minimization problem of minimizing $\dd{A}{B}$ such that
$\gamma\ge\pi/5$, assume that $\dd{A}{B}\le 1.63$.  Without loss of generality, we may assume that the
pentagons $A$ and $B$ are in contact, and that $B$ points to $A$.
We have the constraint
\[
\cos(\pi/5 - \gamma) =\sin(\gamma+3\pi/10) \le 1.63 - \kappa.
\]
This constraint expresses the fact the distance from $\c_B$ to the  receptor edge of pentagon $A$  can be at
most $1.63-\kappa$, by the triangle inequality.  The constraint implies that
$\gamma \ge \pi/5 + \arccos (1.63-\kappa) \approx 2\pi/5 - 0.02$.
The angle sum $\alpha+\beta+\gamma=3\pi/5$ is determined up to a multiple of $2\pi/5$.  The value
for $\gamma$ that we have obtained represents the jump by $2\pi/5$ to $\alpha+\beta+\gamma=\pi$.
This does not exist. (XX clean up.)
\end{proof}

\begin{lemma}\libel{lemma:T1-all} Let $T\rab T_0$. Assume that $T$ and $T_0$ are nonobtuse.
Then $m_1(T)=0$.  Thus, $n_1(T)=n_2(T)=m_1(T)=m_2(T)=0$ and $b(T)=\area(T)\le \aK$.
\end{lemma}

\begin{proof}
If $m_1(T)>0$, then there exists a pseudo-dimer $(T'_1,T'_0)$ such that $T=T'_0$.
An earlier lemma on pseudo-dimers (Lemma XX) gives that $b(T'_1)=\area(T'_1)\le \aK$.
Thus, $T'_1\rab T\rab T_0$.  This contradicts Lemma~\rif{lemma:sequence}.

The final statement is a summary of the preceding series of lemmas.
\end{proof}

Now we turn our attention to the target $T_0$ of an arrow $T\rab T_0$, where both
$T$ and $T_0$ are nonobtuse.  By Lemma \rif{lemma:singleton}, the shared edge
has length at least $1.72$.

\begin{lemma}\libel{lemma:n2-0} Let $T_0$ be a triangle 
 in a cluster $\C$ containing only nonobtuse triangles.  Then
$n_2(T_0)=0$.
\end{lemma}

\begin{proof}  If $n_2(T_0)>0$, then by the definition of $\N$,
there exists an obtuse triangle
$T'$ such that $T'\Ra T_0$ and 
$\area(T') \le \aK+  n_2(T',\N_+)\epso$.  For an obtuse triangle,
$T'$, we have $b(T') = \area(T') - n_2(T')\epso$, and $n_2(T',\N_+)=n_2(T')$.
Thus, $T'\rab T_0$, and the cluster $\C$ contains an obtuse triangle.
\end{proof}

\begin{lemma}\libel{lemma:m1-0}  Let $T\rab T_0$ where $T$ and $T_0$ are in a cluster $\C$
that contains only nonobtuse triangles.  Assume further that if
$\C=\{T_0,T\}$, then $(T,T_0)\not\in\PD$.
Then $m_1(T_0)=0$.
\end{lemma}

\begin{proof}  We assume that $m_1(T_0)>0$ for a contradiction.
By the previous lemma, we have $n_2(T_0)=0$.  By Lemma \rif{lemma:m1m2},
we have $m_2(T_0)=0$.  By Lemma \rif{lemma:n1m1}, we have $n_1(T_0)=0$.
Thus, $b(T_0) = \area(T_0) + \epso' m_1(T_0) = \area(T_0)+\epso'$.

By the definition of $\M$, there exists a unique $T_1$ such that $(T_1,T_0)\in \PD$.
In particular, by earlier results about pseudo-dimers $b(T_1)=\area(T_1)\le\aK$, so $T_1\rab T_0$.

We claim that if $T'\rab T_0$, we have $T'=T_1$.  This follows from the uniqueness of $T_1$.
In particular, $T=T_1$.
By the structure theorem for clusters, $\C = \{T_0,T_1\}$ and $(T,T_0)\in\PD$.
This contradicts the assumption of the lemma.
\end{proof}

\begin{lemma}\libel{lemma:4}
Let $\C$ be a cluster consisting of nonobtuse triangles.  Assume
that the cardinality of $\C$ is four. 
Then the strict main inequality holds for $\C$.
\end{lemma}

\begin{proof}  Let $\C = \{T_0\}\cup \{T_1^i \mid i=1,2,3\}$.  
We have $b(T^i_1) = \area(T^i_1)\le \aK$ by Lemma \rif{lemma:T1-all}.
By Lemmas \rif{lemma:n2-0} and \rif{lemma:m1-0}, we have $m_1(T_0)=n_2(T_0)=0$.

We claim $m_2(T_0)=0$.  Otherwise,
if $m_2(T_0) >0$, then  $(T,T^i_1)\in\PD$ for some $T$ and some $i$.
The arrow $T^i_1\rab T_0$ is inconsistent with Lemma \rif{lemma:pd-b}

We claim $n_1(T_0)=0$.  Otherwise, if $(T_0,T^i_1)\in\N$, then we get $n_2(T^i_1)>0$,
which is contrary to a recent fact.

Hence all the coefficients $n,m$ are zero on the cluster: $b(T^i_1)=\area(T^i_1)$ and
$b(T_0) = \area(T_0)$.
The result now follows from Lemma~\rif{calc:pent4}.
\end{proof}

\begin{lemma}\libel{lemma:3}
Let $\C$ be a cluster consisting of nonobtuse triangles.  Assume
that the cardinality of $\C$ is three. 
Then the strict main inequality holds for $\C$.
\end{lemma}

\begin{proof}  
Let $\C = \{T_0\}\cup \{T_1^i \mid i=1,2\}$.  
We have $b(T^i_1) = \area(T^i_1)\le \aK$ by Lemma \rif{lemma:T1-all}.
By Lemmas \rif{lemma:n2-0} and \rif{lemma:m1-0}, we have $m_1(T_0)=n_2(T_0)=0$.

We claim that $m_2(T_0)\le 1$.  Otherwise, by the definition of $\M$, there exists
$(T,T^i_1)\in\PD$ for some $T$ and some $i$.  The arrow $T^i_1\rab T_0$ is inconsistent
with Lemma \rif{lemma:pd-b}.
We have
\[
b(T_0) = \area(T_0) + \epso n_1(T_0) - \epso' m_2(T_0) \ge \area(T_0) - \epso'
\]
By Lemma~\rif{calc:pent3}, we have
\[
\sum_{T\in\C} b(T) \ge (\area(T_0) - \epso') + \sum_{T\ne T_0} \area(T) = -\epso'+ \sum_{T\in\C} \area(T) > 3\aK.
\]
This is the strict main inequality for $\C$.
\end{proof}

\begin{lemma}\libel{lemma:2} Let $\C$ be a cluster consisting of nonobtuse triangles.  Assume
that the cardinality of $\C$ is two.  If the cluster is not a dimer pair, 
then the strict main inequality holds for $\C$.
\end{lemma}

\begin{proof}  Let $\C = \{T_1,T_0\}$, with $T_1\rab T_0$. By Lemma \rif{lemma:pd-1},
$b(T_1)=\area(T_1)\le \aK$.  

We assume that $\C$ is not a dimer pair $(T_1,T_0)$.


We consider the case of a pseudo-dimer.
If $(T_1,T_0)\in\PD$, then $m_2(T_0)=n_2(T_0)=0$ (by Lemmas \rif{lemma:pd-m2} and \rif{lemma:pd-n2}).
Thus, by Lemma \rif{calc:pseudo-area}
\[
b(T_1)+b(T_0) \ge \area(T_1)+\area(T_0) + \epso' (n_1(T_0) + m_1(T_0)) > 2\aK - \epso'  + \epso' (n_1(T_0)+m_1(T_0)).
\]
The main inequality follows if we show that $n_1(T_0)>0$ or $m_1(T_0)>0$.
Assume for a contradiction that $n_1(T_0)=m_1(T_0)=0$.  Pick $T$ such that $T_0\Ra T$.
The condition $n_1(T_0)$ implies that $(T_0,T)\not\in\N$.  According to the definition of $\M$, 
we have $m_1(T_0)>0$ unless
there exists $T'_1\ne T_1$ such $(T'_1,T_0)\in\PD$.  This is impossible, because the cardinality of $\C$ is only
two.  This completes the case of a pseudo-dimer.

By Lemmas \rif{lemma:n2-0} and \rif{lemma:m1-0}, we have $m_1(T_0)=n_2(T_0)=0$.
Thus,
\begin{equation}\libel{eqn:2}
b(T_0) = \area(T_0) + \epso n_1(T_0) - \epso' m_2(T_0)\le \aK, \quad b(T_1) = \area(T_1) \le \aK.
\end{equation}

We may assume that  $\area(T_0)+\area(T_1)> 2\aK$.  Otherwise, $(T_1,T_0)$ is a dimer pair or 
a pseudo-dimer, and these cases have already been handled.  Combined with the Inequalities \rif{eqn:2},
this gives 
\[
2\aK \ge b(T_0) + b(T_1) > 2\aK + \epso n_1(T_0) - \epso' m_2(T_0).
\]
This implies that $n_1(T_0)=0$ and $m_2(T_0) > 0$.

We consider the case $m_2(T_0)= 3$.  In fact, Lemma~\rif{lemma:3m2}, we have
$b(T_1) + b(T_1) > a_0 + \aK + \epso = 2\aK$, which completes this case.

We consider the case $m_2(T_0)=2$.  In this case, there are two pseudo-dimer that share
a longest edge with $T_0$.  These edges have length at least $1.8$. The third edge
is shared with $T_1$ and has length at least $1.72$.
Then
\[
b(T_1)+b(T_0) \ge a_0 + (\area (1.8,1.8,1.72) - 2\epso') > 2\aK.
\]

Finally, we consider the case $m_2(T_0)=1$.  We are in the context covered
by Lemma \rif{calc:good}.  That lemma implies
\[
b(T_1) + b(T_0) = (\area(T_1) + \area(T_0)) - \epso' > (2\aK  +\epso') - \epso' = 2\aK.
\]
This completes the proof.
\end{proof}


We are ready to give a proof of the Kuperberg-Kuperberg conjecture. We repeat the statement
of the theorem from the introduction of the article.  

\begin{theorem}   No packing of congruent regular pentagons in the Euclidean
plane has density greater than that of
the Kuperberg-Kuperberg packing.   The Kuperberg-Kuperberg packing is the
unique periodic packing of congruent regular pentagons that attains optimal density.
\end{theorem}

We combine the proof with a proof of Theorem~\rif{main:conj}.

\begin{proof}
By the main inequality in Lemma \rif{lemma:main} applied to $a=\aK$, together with Remark~\rif{rem:equal}
it is enough to give a proof of Theorem~\rif{main:conj}.  Specifically, we show that
every $b$-cluster in every saturated packing is finite of cardinality at most $4$.  
This is Corollary \rif{lemma:card4}.
If $\C$ is a dimer pair, then $\C$ is the $K$-dimer in the Kuperberg-Kuperberg packing and the (weak)
main inequality holds for $\C$.  This is Theorem~\rif{thm:dimer}.  If $\C$ is not a dimer pair, then
we show that $\C$ satisfies the strict main inequality.  If $\C$ has an obtuse triangle, then this
is found in Lemma \rif{lemma:obtuse}.  If every triangle in $\C$ is nonobtuse, and if $\C$ has cardinality $4$, $3$,  $2$, or $1$,
this is found in Lemmas \rif{lemma:4}, \rif{lemma:3}, \rif{lemma:2}, \rif{lemma:singleton}.

This completes the proof of the main theorem.
\end{proof}



\section{Appendix on Explicit Coordinates}
\libel{sec:appendix}

Let $A$ and $B$ be pentagons in contact, with $B$ the pointer at
vertex $\v_B$ to the receptor pentagon $A$.  Label vertices
$(\u_A,\w_A,\u_B,\v_B,\w_B)$ of $A$ and $B$ as in
Figure~\rif{fig:theta}.  Let $x=\norm{\v_B}{\w_A}$ and $\beta =
\angle(\v_B,\u_B,\u_A)$.  We have $0\le x\le 2\sigma$ and $0\le
\beta\le 2\pi/5$.

\tikzfig{theta}{coordinates for a pair of pentagons in contact}{
[scale=1.2]
\pen{0}{0}{0};
\pen{1.616}{0.598}{168.54};
\draw[blue] (0,0) node[anchor=south,black] {$B$}
 -- (1.616,0.598) node[anchor=south,black] {$A$};
\draw (0,0) node[anchor=north,black] {$\c_B$};
\smalldot{0,0};
\draw (1.616,0.598) node[anchor=north,black] {$\c_A$};
\smalldot{1.616,0.598};
\draw (72:1) node[anchor=south] {$\w_B$};
\draw (0:1) node[anchor=west] {$\v_B$};
\draw (-72:1) node[anchor=north] {$\u_B$};
\smalldot{72:1};
\smalldot{0:1};
\smalldot{-72:1};
\smalldot{0.636,0.797};
\draw (0.636,0.797) node[anchor=west] {$\w_A$}
node[anchor=north west] {$x$};
\smalldot{1.124,-0.2727};
\draw (1.124,-0.2727) node[anchor=north west] {$\u_A$}
node[anchor=north east] {$\beta$};
}


Let $\bl = \bl(x,\beta) = \norm{\c_A}{\c_B}$, viewed as a function of
$x$ and $\beta$.  We omit the explicit formula for $\bl$, but it is
obtained by simple trigonometry.  Under the symmetry
$\u_A\leftrightarrow \w_A$, $\u_B\leftrightarrow\w_B$, we have the
symmetries $x\leftrightarrow 2\sigma-x$, $\beta\leftrightarrow
2\pi/5-\beta$ and
\[
\bl(x,\beta) = \bl(2\sigma-x,2\pi/5 - \beta).
\]

If we have coordinates on a $3C$-triangle that determine the variables
$(x,\beta)$ for each of the pairs $\{A,B\}$, $\{B,C\}$, and $\{A,C\}$
of pentagons, then we may use the function $\bl$ to calculate the edge
lengths and area of the $3C$-triangle.

The space of all $P$-triangles is six-dimensional, described by the
three edge lengths of the Delaunay triangle and the three rotation
angles of the pentagons at its vertices.  The space of $3C$-triangles
is three dimensional, obtained by imposing three contact constraints
between pairs of pentagons.

\subsection{$\Delta$-junction}

We describe a coordinate system on $3C$-triangles of $\Delta$-junction
type.  As indicated in Figure~\rif{fig:cord-delta}, we use coordinates
$(x_\alpha,\alpha,\beta)$, where $x_\alpha$ is a length and $\alpha$
and $\beta$ are each angles between lines through edges of pentagons
in contact.  We assume that $B$ points into $A$ and into $C$ and that
$A$ points into $C$. The length $x_\alpha$ is the (small) distance
between the nearly coincident vertices of pentagons $B$ and $C$.  The
coordinates satisfy the conditions $0\le\beta\le\alpha\le\pi/5$,
$\alpha+\beta\le \pi/5$, and $x_\alpha\in[0, 2\sigma -
\sigma/\kappa]$.  Starting from these coordinates, we define $\gamma$
by $\alpha+\beta+\gamma=\pi/5$, and angles of the triangle $\Delta$ by
\begin{align}\libel{eqn:abc}
\alpha+\alpha' &= 2\pi/5,\\
\beta+\beta' &= 2\pi/5,\nonumber\\
\gamma+\gamma' &= 2\pi/5.\nonumber
\end{align}
The edges $y_\alpha$, $y_\beta = 2\sigma$, and $y_\gamma$ of the
triangle $\Delta$ opposite the angles $\alpha'$, $\beta'$, $\gamma'$,
respectively are easily computed by the law of sines.  Define
$x_\beta$ by $x_\alpha+y_\gamma+x_\beta=2\sigma$, and $x_\gamma$ by
$y_\alpha+x_\gamma=2\sigma$.  The value $x_\beta$ is the distance
between the nearly coincident vertices of pentagons $A$ and $C$, and
$x_\gamma$ is the distance between the nearly coincident vertices of
pentagons $A$ and $B$.  The edges of the $3C$ Delaunay triangle have
lengths
\[
\bl(x_\alpha,\alpha'),\quad \bl(x_\beta,\beta'),\quad \bl(x_\gamma,\gamma').
\]

\tikzfig{cord-delta}{Coordinates for $\Delta$-types}
{
[scale=1.0]
\threepentnoD
{0.00}{0.00}{-5.16}%C
{0.99}{1.70}{235.38}%B
{1.98}{0.00}{183.43}; %A
\draw (0,0) node {$C$};
\draw (0.99,1.70) node {$B$};
\draw (1.98,0.0) node {$A$};
\draw (1.0,0.6) node {$\Delta$};
%\draw[blue] (0,0) -- (1.98,0);
\draw (-5.16:1) -- ++ (126 - 5.16 :2.5) node[anchor=south] {$\alpha$};
\draw (-5.16:1) -- ++ (- 180 + 126 - 5.16 :1.5) node[anchor=west] {$\beta$};
\draw ++ (1.98,0) ++ (183.43 - 72:1) -- ++ (54 + 3.43:1) node[anchor=south] {$\gamma$};
\draw (72-5.16:1) -- ++ (126 + 90 - 5.16:0.5);
\draw (0.99,1.70) ++ (235.38:1) node[anchor = south west] {$x_\alpha$} -- ++ (126+90 - 5.16:0.5);
}



\subsection{Pinwheel type}

We describe a coordinate system on $3C$-triangles of pinwheel type.
We assume that $C$ points into $B$, that $B$ points into $A$, and that
$A$ points into $C$.  As indicated in Figure~\rif{fig:cord-pinwheel},
we use coordinates $(\alpha,\beta,x_\gamma)$.  The angles $\alpha$ and
$\beta$ are angles between pentagon edges on touching pentagons.  The
value $x_\gamma$ is the distance between the pointer vertex of
pentagon $A$ and the pointer vertex of pentagon $B$.  The coordinates
satisfy constraints: $0\le\alpha$, $0\le\beta$, $\alpha+\beta\le
\pi/5$, and $0\le x_\gamma\le 2\sigma$.  Define $\gamma$ by
$\alpha+\beta+\gamma=\pi/5$.  The angles $\alpha'$, $\beta'$, and
$\gamma'$ of the inner background triangle $P$ of the pinwheel are
given by Equation~\rif{eqn:abc}.  The edge lengths $x_\alpha$,
$x_\beta$, $x_\gamma$ of the inner triangle $P$ are easily computed by
the law of sines.  The edges of the $3C$-triangle have lengths
\[
\bl(x_\alpha,\alpha),\quad \bl(x_\beta,\beta),\quad \bl(x_\gamma,\gamma).
\]

\tikzfig{cord-pinwheel}{Coordinates for pinwheel type}{
\begin{scope}[xshift=4cm,scale=1.2]
\threepentnoD{0.00}{0.00}{46.69}%A
{0.82}{1.53}{218.09}%C
{1.73}{0.00}{163.28};%B
\draw(0,0) node {$A$};
\draw(0.82,1.53) node {$C$};
\draw(1.73,0) node {$B$};
\draw(1.8,1.2) node {$\alpha$};
\draw(-0.2,1.0) node {$\beta$};
\draw(0.85,-0.7) node {$\gamma$};
\draw(0.9,0.5) node {$P$};
\draw(0.5,0.5) node {$x_\gamma$};
\smalldot{0.772,0.288}; %B pointer.
\smalldot{0.686,0.728}; %A pointer.
\end{scope}
}


\subsection{$L$-junction type}

We describe a coordinate system on $3C$-triangles of $L$-junction
type.  As indicated in Figure~\rif{fig:cord-L}, we use coordinates
$(\alpha,\beta,x_\alpha)$.  The angles $\alpha$ and $\beta$ are each
formed by edges of two pentagons in contact.  Let $x_\alpha$ be the
distance between the pointer vertex of $C$ to $A$ and the pointer
vertex of $B$ to $C$.  The coordinates satisfy relations:
$\alpha,\beta\in [0,2\pi/5]$, $\pi/5\le\alpha+\beta\le 3\pi/5$, and
$0\le x_\gamma\le 2\sigma$.  Define $\gamma$ by
$\alpha+\beta+\gamma=3\pi/5$.  The angles $\alpha'$, $\beta'$, and
$\gamma'$ of the inner $L$-shaped quadrilateral are given by
Equation~\rif{eqn:abc}.  The edge lengths of the $L$-shaped
quadrilateral are easily computed by triangulating the quadrilateral
into two triangles and applying the law of sines.  (Triangulate
$L$ by extending the line through the edge of $A$ containing the
pointer vertex of $C$ into $A$.)  This gives
$x_\beta$, the distance between the pointer vertex of $C$ to $B$ and
the inner vertex of $A$.  This gives $x_\gamma$, the distance between
the pointer vertex of $B$ and the inner vertex of pentagon $A$.
As before, the edges of the $3C$-triangle have lengths
\[
\bl(x_\alpha,\alpha),\quad \bl(x_\beta,\beta),\quad \bl(x_\gamma,\gamma).
\]

\tikzfig{cord-L}{Coordinates for $L$-junction type}{
\begin{scope}[xshift=8cm,scale=1.2]
\threepentnoD{0.00}{0.00}{80.86}
{0.97}{1.58}{232.21}
{1.82}{0.00}{223.24};
\draw (0,0) node {$C$};
\draw (0.97,1.58) node {$B$};
\draw (1.82,0) node {$A$};
\draw (2.0,1.2) node {$\gamma$};
\draw (-0.1,1.2) node {$\alpha$};
\draw (0.85,-0.8) node {$\beta$};
\draw (0.85,0.75) node {$L$};
\draw (0.55,0.4) node {$x_\alpha$};
%\smalldot{0.94,0.48};
%\smalldot{1.532,0.753};
\end{scope}
}


\subsection{$T$-junction type}

We describe a coordinate system on $3C$-triangles of $T$-junction
type.  As indicated in Figure~\rif{fig:cord-T}, we use coordinates
$(\alpha,\beta,x_\gamma)$.  The angles $\alpha$ and $\beta$ are each
formed by edges of two pentagons in contact.  The length $x_\gamma$ is
the distance between the pointer vertex of $B$ and the inner vertex of
$A$.  The coordinates satisfy: $\alpha,\beta\in[\pi/5,2\pi/5]$,
$3\pi/5\le \alpha+\beta\le 4\pi/5$, $0\le x_\gamma\le 2\sigma$.
Define $\gamma$ by $\alpha+\beta+\gamma=\pi$.  The angles $\alpha'$,
$\beta'$, and $\gamma'$ of the inner irregular $T$-shaped pentagon $P$
are given by Equation~\rif{eqn:abc}.  The edge lengths of the
$T$-shaped pentagon are easily computed by triangulating $P$ into
three triangles and applying the law of sines.  (Triangulate by
extending the edge of $P$ shared with $A$ that ends at the pointer
vertex of $A$ into $C$ and by extending the edge of $P$ shared with
$C$ that contains pointer vertex of $B$ into $C$.)  This gives
$x_\alpha$, the distance between the pointer vertex of $B$ to $C$ and
the inner vertex of $C$.  This gives $x_\beta$, the distance between
the pointer vertex of $A$ to $C$ and the inner vertex of $C$.  As
before, the edges of the $3C$-triangle have lengths
\[
\bl(x_\alpha,\alpha),\quad \bl(x_\beta,\beta),\quad \bl(x_\gamma,\gamma).
\]



\tikzfig{cord-T}{Coordinates for $T$-junction types}{
\begin{scope}[xshift=12cm,scale=1.2]
\threepentnoD{0.00}{0.00}{114.48} %A
{0.90}{1.59}{237.18} %B
{1.66}{0.00}{219.24}; %C
\draw (0,0) node {$A$};
\draw (0.9,1.59) node {$B$};
\draw (1.66,0) node {$C$};
\draw (0.9,0.9) node {$P$};
\draw (0,1.1) node {$\gamma$};
\draw (0.75,-0.9) node {$\beta$};
\draw (1.9,1.2) node {$\alpha$};
\draw (0.45,0.5) node {$x_\gamma$};
\smalldot{0.737,0.675};
\smalldot{0.358,0.75};
\end{scope}
}

\subsection{pin-$T$ junction type} We describe a coordinate system on
$3C$-triangles of pin-$T$ junction type.  As indicated in
Figure~\rif{fig:cord-pint}, we use coordinates $\alpha$, $\beta$, and
$x_\alpha$.  The angles $\alpha$ and $\beta$ are each formed by edges
of two pentagons in contact.  The length $x_\alpha$ is the distance
between the nearly coincident vertices of $B$ and $C$.
The coordinates satisfy $\pi/5 \le \alpha \le 2\pi/5$, $\pi/5\le
\beta \le 2\pi/5$, and $3\pi/5 \le \alpha+\beta$.
Lemma~\rif{lemma:0605} shows that $0\le x_\alpha\le 0.0605$.  Define
$\gamma$ by $\alpha+\beta+\gamma = \pi$.  The angles $\alpha'$,
$\beta'$, and $\gamma'$ of the inner irregular $T$-shaped pentagon are
given by Equation~\rif{eqn:abc}.  The edge lengths of the $T$-shaped
pentagon $P$ are easily computed by triangulating $P$ into three
triangles and applying the law of sines.  (Triangulate by extending
the two edges of $P$ that meet at the pointer vertex of $C$ into $A$.)
This gives $x_\beta$, the distance between the pointer vertex of $C$
to $A$ and the inner vertex of $A$.  This gives $x_\gamma$, the
distance between the pointer vertex of $A$ to $B$ and the pointer
vertex of $B$ into $C$.  As before, the edges of the $3C$-triangle
have lengths
\[
\bl(x_\alpha,\alpha),\quad \bl(x_\beta,\beta),\quad \bl(x_\gamma,\gamma).
\]

\tikzfig{cord-pint}{Coordinates for pin-$T$ junction types. Although
  it is difficult to tell from the figure, $A$ points into $B$, $B$
  into $C$, and $C$ into $A$.  The nearly horizontal edges of $A$ and $C$
  need not be parallel. The parameter $\beta'$ measures their
  incidence angle.  The region bounded by the three pentagons is a
  $T$-shaped pentagon, with stem between the nearly parallel
  edges of $A$ and $C$ and two arms along $B$.  The arm between $B$
  and $C$ is imperceptibly small.}{
\begin{scope}[xshift=4.5cm,yshift=1.5cm]
\threepentnoD{0.00}{0.00}{90} %C
{1.40}{-1.377}{0} % B
{-0.35}{-1.628}{-18.89}; % A
\draw (0,0) node {$C$};
\draw (1.4,-1.37) node {$B$};
\draw (-0.35,-1.62) node {$A$};
\draw (0.36,-2.4) node {$\gamma$};
\draw (-1.1,-0.5) node {$\beta$};
\draw (1.0,-0.3) node {$\alpha$};
\draw (0.4,-0.49) node {$x_\alpha$};
\draw (1.4 - 0.809,-1.377 - 0.5878) -- +(0,-0.75);
\end{scope}
}



\begin{lemma}\libel{lemma:0605}
  Let $T$ be a $3C$ triangle of type pin-$T$.  The coordinates
  $\alpha$, $\beta$, and $x_\alpha$ satisfy the relation
\[
x_\alpha \sin(2\pi/5) \le 2\sigma (\sin(\alpha+\pi/5) - \sin(\beta+\pi/5)).
\]
In particular, $x_\alpha \le 0.0605$.
\end{lemma}

\begin{proof} Let $\v_{AB}$ be the pointer vertex of $A$ to $B$, and
  let $\v_{BC}$ be the pointer vertex of $B$ to $C$.  Let $\v$ and
  $\v_{BC}$ be the endpoints of the edge of $B$ containing $\v_{AB}$.
  We represent $T$ as in Figure~\rif{fig:cord-pint}, with the lower edge of
  $C$ along the $x$-axis.  Since $\v_{AB}$ lies on the segment between
  $\v$ and $\v_{BC}$, The $y$-coordinate $y(\v_{AB})$ of $\v_{AB}$ is
  nonpositive and lies between the $y$-coordinates $y(\v)$ and
  $y(\v_{BC})$.  We have
\begin{align*}
y(\v) &= x_\alpha \sin(2\pi/5) - 2\sigma\sin(\alpha+\pi/5)\\
y(\v_{AB}) & = -x_\beta \sin(\beta') - 2\sigma\sin(\beta+\pi/5).
\end{align*}
Using $x_\beta \sin(\beta')\ge 0$ and $y(\v) \le y(\v_{AB})$, we
obtain the claimed inequality.

Recall that $\pi/5\le \beta \le 2\pi/5$.  In particular, we have
$\sin(\alpha+\pi/5) \le 1$ and $\sin(\beta+\pi/5)\ge \sin(2\pi/5)$.
This gives
\[
x_\alpha \le 2\sigma(1/\sin(2\pi/5) - 1) < 0.0605.
\]
\end{proof}


This completes our discussion of coordinates used for computations.

\section{Computer Calculations}

  The code for the computer-assisted proofs is written in Objective
  Caml.  There are several thousand lines of code, available for
  download from github \cite{Git}.  The computer calculations for this
  paper take many hours in total to run.  

To control
  rounding errors on the computer, we use an interval arithmetic
  package for OCaml by Alliot and Gotteland, which runs on the Linux
  operating system and Intel processors \cite{All}.  Intervals are represented as 
  pairs $(a,b)$ of floating point numbers, giving the lower $a$ and upper $b$ endpoints
of the interval. 



\subsection{meet-in-the-middle}

A common algorithmic technique for reducing the time complexity of an algorithms
through greater space complexity is called {\it meet-in-the-middle}.  This is very closely related to
the  {\it linear assembly algorithms} used in the solution to the Kepler problem \cite{XX}.
%% Algorithms Kepler.

A number of introductory examples of meet-in-the-middle (MITM) algorithms can be found
at the blog post \cite{XX}.
% Cosmin Negruseri, 10 aug 2012. Coding contest trick: Meet in the middle.
% www.infoarena.ro/blog/meet-in-the-middle.
A simple example from there is to find if there are four numbers in a given finite set $S$ of integers
that sum to zero, where repetitions of integers are allowed.  
If we calculate all possible sums of four integers, testing if each is zero,
then there are $n^4$ sums, where $n$ is the cardinality of $S$.  The MITM solution to
the problem  computes and stores (in a hashset) all sums $a+b$ of unordered pairs of elements from $S$.
We then search the hashset for a collision, meaning a sum $a+b$ that is the negative of another sum $c+b$:
$a+b = -(c+b)$.
Any such collision gives $a+b+c+d=0$.  The MITM solution involves the computation of
$n^2$ sums $a+b$, rather than $n^4$, for a substantial reduction in complexity.  
MITM techniques have numerous applications to
cryptography, and it is there that we first encountered the technique.  See for example,
\cite{XX}
% http://www.ma.huji.ac.il/~nkeller/CACM2013.pdf
which applies MITM to a general class of dissection problems, including the Rubik's cube.


We obtain computational bounds on the area of clusters of Delaunay triangles (or more accurately, $P$-triangles).
Each of our clusters will be assume to consist of one triangle (called the central triangle), flanked 
 by one, two, or three additional triangles along its edges.  We call the flanking triangles {\it peripheral triangles}.
The aim of the algorithm is to give lower bounds
on the sum of the areas of the triangles, subject to a collection of constraints.  Two types of constraints are
allowed: (1) constraints that can be expressed in terms of a single triangle, and (2) assembly constraints.
An assembly constraint states that the central $P$-triangle fits together with a flanking triangle.  
In more detail, the central triangle $T_0$ shared an edge and two pentagons $A$, $B$ with a flanking triangle $T_1$.
Associated with $T_0$ are parameters $d^0_{AB}$, $\theta^0_{ABC}$, $\theta^0_{BAC}$ giving the edge length of the
common edge with $T_1$, and the inclination angles of the pentagons $A$ and $B$ with respect to that common edge.
Similarly, associated with $T_1$ are parameters $d^1_{AB}$, $\theta^1_{ABC}$, $\theta^1_{BAC}$.  The assembly constraint
along the common edge is
\begin{equation}\libel{eqn:assembly}
d^0_{AB} = d^1_{AB},\quad \theta^0_{ABC} = -\theta^1_{ABC},\quad \theta^0_{BAC} = -\theta^1_{BAC}.
\end{equation}
The negative sign comes from the opposite orientations of the common edge with respect to $T_0$ and $T_1$.

Each $P$-triangle is a point in a six dimensional space
(described by three triangle edge lengths and three inclination angles of the pentagons
at its vertices).  A cluster consisting of one central $P$-triangle and $k$ flanking triangles is a point in a
space of dimension
$6 + 3k$.  If we cover the space with cubes of edge-length $\epsilon$, then there are order $(1/\epsilon)^{6+3k}$
cubes.  This is beyond our computational reach when $k>0$.

We can use MITM techniques to reduce to order $(1/\epsilon)^6$ cubes, and this puts the computations (barely) within the
reach of a laptop computer.  Specifically, we fix a edge size $\epsilon$ and cover the space of peripheral triangles
by cubes of size $\epsilon$, calculating area, edge lengths, inclination angles, and other relevant quantities
 (using interval arithmetic) over each cube.   

The idea of MITM is to place the data for each peripheral cube into a hash table, keyed by the  variables
$d^1_{AB}$, $\theta^1_{ABC}$, $\theta^1_{BAC}$ that are shared with the central triangle through Equation~\rif{eqn:assembly}.
We view Equation~\rif{eqn:assembly} as the analogue of the collision condition $(a+b) = -(c+d)$ in
the simple example of MITM given above.  Of course, the variables $d^1_{AB}$ etc. are represented as intervals
with floating point endpoints and cannot be used directly as keys to a hashtable.  Instead we cover these intervals
with a secondary finite set of intervals with rational endpoints, which are indexed by integers.  These integer indices
become the keys to the hash table.    For each key, the hash table records the smallest  area of a peripheral triangle with
that key.

Once the hash is created, we divide the space of central triangles into cubes and computes the relevant quantities
(edge lengths, triangle area, and inclination angles) over each cube using interval arithmetic.   Bounds on the peripheral
triangles are recovered from the hash table to get a lower bound on the sum of the areas of the triangles in the cluster.

The entire process is iterated for smaller and smaller $\epsilon$ until the desired bound on the areas of the triangles
in the cluster is obtained.   Each time $\epsilon$ is made smaller, only the peripheral cubes that were involved in a
key collision with a central cube are carried into the next iteration for subdivision into smaller cubes.  Only the cubes
with suitably small triangle area bounds are carried into the next iteration.  In practice, to achieve our bounds,
the smallest $\epsilon$ that
was required was approximately $0.00024$.

\subsection{Automatic Differentiation}

Recall that there are several ways to compute derivatives by computer, such as numerical approximation by
a difference quotient $f(x + \Delta x)/\Delta x$, symbolic differentiation (as in computer algebra systems), and
automatic differentiation.   In this project, we differentiate functions of a single variable using automatic differentiation.
The value of a function and its derivative are represented as a pair (f,f') of intervals, where $f$ is an interval bound
on the function, and $f'$ is an interval bound on the derivative of the function.    More complex expressions can be built
from simpler expressions by extending arithmetic operations $(+)$, $(-)$, $(*)$, $(/)$ to pairs.  For example,
\[
(f,f') + (g,g') = (f+g,f'+g'),\quad (f,f') * (g,g') = (f g,f g' + f' g),
\]
where the component-wise addition and multiplication operations are given by interval arithmetic.

Section~\rif{XX} describes the proof of the local minimality of the $K$-dimer.  We review that
argument here with an emphasis on automatic differentiation.
Automatic differentiation allows us to show that the $K$-dimer $D_K$ is the unique minimizer of area
in an explicit neighborhood of $D_K$.  In Section~\rif{XX}, we give a curve $\Gamma$ 
(in the space of dimers) with parameter $t\in\ring{R}$
that passes through the point $D_K$ at $t=0$.  

For any point $D$ in an explicit neighborhood of $D_K$ in the dimer space, that section describes a curve from $D$ to a point
on the curve $\Gamma$.  
Using automatic differentiation, we show that area decreases as we move from $D$ towards $\Gamma$.
Thus, the area minimizer, lies on $\Gamma$.  We can compute an upper bound $|t|\le M$.

By symmetry in the underlying geometry, 
it is clear that the area function has derivative zero along $\Gamma$ at
$t=0$. By taking a second derivative with automatic differentiation, we find that the second derivative of
the area function along this curve is positive
when  $|t|\le M$.   Thus, $D_K$ is the unique area minimizer on this curve within this explicit neighborhood.



      
      \bibliography{pentagon} 
%\bibliographystyle{plain}
      
      \bibliographystyle{plainnat}
%\bibliography{announce_refs}



%% shell:>makeindex index/Index
%% shell:>makeindex index/Notation
%% edit index/Notation.ind to put an \indexspace before Greek entries.

%\printindex{index/Index}{General index}
%\printindex{index/Notation}{Notation index}

%\noindent

\bigskip\noindent
This material is based upon work supported by the National Science
Foundation under
grant 0804189.



\smallskip
\newpage

%\end{thepreface}

\end{document}
