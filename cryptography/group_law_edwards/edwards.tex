
% May 11, 2016. Adding hyperbolic material.
\documentclass[18pt]{article}

\usepackage{graphicx}
\usepackage{amsthm}
\usepackage{amsfonts}
\usepackage{amsmath}
\usepackage{amscd}
\usepackage{amssymb}
\usepackage{alltt}
\usepackage{url}
\usepackage{ellipsis}
% 

\newtheorem{theorem}{Theorem}[section]
\newtheorem{lemma}[theorem]{Lemma}
\newtheorem{corollary}[theorem]{Corollary}
\newtheorem{definition}[theorem]{Definition}


\newcommand{\ring}[1]{\mathbb{#1}}
\newcommand{\op}[1]{\hbox{#1}}
\newcommand{\f}[1]{\frac{1}{#1}}
%\newcommand{\text}[1]{\hbox{#1}}

\title{Hyperbolic Addition on Edwards Curves}
\author{Thomas C. Hales}
\date{May 11, 2016}





\begin{document}

\maketitle

\centerline{\it I dedicate this article to Harold Edwards on his 80th birthday.}

\begin{abstract} We give an elementary proof and partial formal proof
of the group axioms for complete Edwards elliptic curves.  
The proof avoids concepts from algebraic geometry.  
Instead, the proof relies on the basic properties of rings and ring homomorphisms,
including the localization of rings.  
In particular,
we present an eleven-line elementary computational proof of associativity.
Following an approach introduced
by Friedl, the associativity law is expressed as a
polynomial identity over the integers that is constructed by Groebner basis methods. 
Edwards curves avoid the case-by-case analysis
that complicates previous proofs based on Groebner bases.
After the polynomial identity is found, 
 it can be verified directly and independently of the algorithm that
produces it.  The main polynomial identities in this article have been formally verified
in the HOL Light proof assistant.  
\end{abstract}

\parskip=\baselineskip

\newenvironment{blockquote}{%
  \par%
  \medskip%
  \baselineskip=0.7\baselineskip%
  \leftskip=2em\rightskip=2em%
  \noindent\ignorespaces}{%
  \par\medskip}


\section{Introduction}

The group law on elliptic curves is a fundamental part of cryptography.  In my undergraduate cryptography course this 
past semester (Spring 2016),
it was unpleasant for 
me to tell my students that the associative law for elliptic curve addition is too advanced for an undergraduate course.
It is clearly desirable to have a proof for an undergraduate course that is short and elementary.
Here, I present an eleven-line elementary computational proof of associativity (Lemma~\ref{lemma:assoc}).

There is another reason why we should care about having a simple proof of the group axioms for an elliptic curve.
The difficulty of the verification has been a impediment to the formalization of elliptic curve cryptography.  Th\'ery,
and more recently Bartzia and Strub, have
obtained formal proofs in the proof assistant 
Coq of the group axioms for an elliptic curve, 
but this project has a long history~\cite{thery2007primality}, \cite{thery2007proving},  \cite{bartzia2014formal}.
See also \cite{russinoffcomputationally}.

I confine my remarks to Edwards elliptic curves.  They have become popular in recent  years
in cryptography.

Proofs based on Groebner basis algorithms carry a particular appeal for those of us interested in formalization, because
many proof assistants support such proofs.
Buchberger's algorithm has previously been used
to prove the associativity of the group law for Weierstrass curves, but it encounters complications from the case-by-case
treatment of the group law.  As expected, Edwards elliptic curves avoid these complications.  A number of calculations here can be
viewed as reworkings of calculations found in Edwards, Bernstein, and Lange~\cite{edwards2007normal}, \cite{bernstein2008twisted}, \cite{bernstein2007faster}.

\section{The Circle}

It is a well-known fact that the unit circle $U$ in the complex plane $\ring{C}$ forms a group under
complex multiplication.  To be explicit, we write a complex number in the form 
$z = x+ i y$, where $x,y$ are real numbers and
$i^2 = -1$.  The multiplication of complex numbers $z_1 z_2$ is given by
\begin{equation}\label{eqn:cx}
z_1 z_2 = (x_1 + i y_1) (x_2 + i y_2) = (x_1 x_2 - y_1 y_2) + i (x_1 y_2 + y_1 x_2).
\end{equation}
Multiplication is clearly commutative: $z_1 z_2 = z_2 z_1$
The unit circle $U$ is the set of complex numbers of norm one: $|z|^2 := x^2 + y^2 = 1$.  
The identity element $1 \in \ring{C}$,
inverse $z^{-1}$, and product $z_1 z_2$ all belong to $U$,  provided $z,z_1,z_2\in U$.

\subsection{Hyperbolic addition}

We give an unusual interpretation of the group law on the unit circle that we call {\it hyperbolic addition}.
We now shift notation away from complex numbers and represent points in the plane by ordered pairs:
Let $z_1 = (x_1,y_1)$ and $z_2= (x_2 , y_2)$ 
be two points on the unit circle that we wish to multiply.  To avoid degenerate
cases, we assume that
the points $(-1,0)$, $z_1$ and $z_2$ in the plane are not collinear.  (Degeneracies will be removed later.)
We consider the family of hyperbolas in the plane
that pass through the point $z=(-1,0)$ and whose asymptotes are parallel to the coordinate axes.
The equation of such a hyperbola has the form
\begin{equation}\label{eqn:hyp}
x y + p (x+1) + q y = 0.
\end{equation}
We determine the parameters $p$ and $q$ by the condition that the hyperbola must pass through
the two points $z_1$ and $z_2$ on the unit circle as well.  
Under the non-collinearity assumption, this condition uniquely determines
$p$ and $q$.   The hyperbola meets the unit circle in one additional point $z_3 = (x_3,y_3)$,
for a total of four points of intersection $(-1,0)$, $z_1$, $z_2$, and $z_3$.


We have the following remarkable relationship among the three points $z_1$, $z_2$, and $z_3$ on the
intersection of the circle and hyperbola.

\begin{lemma}[hyperbolic addition on the circle]  
With $z_3$ constructed as above, with respect to the group law on the unit circle, we have
$z_1 z_2 z_3 = 1$.
\end{lemma}

The lemma is a special case of a much more general lemma (\ref{lemma:hyperbola}) that appears later in this article.
We wait until then to give the proof.  The proof relies on the explicit formula (\ref{eqn:cx}) for the group law.

If we write $z_3 = (x_3,y_3)$, the lemma implies that the product of $z_1$ and $z_2$ is $(x_3,-y_3)$.  
Rather than starting with the formula for the binary operation (\ref{eqn:cx}) and using it to prove that $(x_3,-y_3)$
is the product of $z_1$ and $z_2$, we can reverse the process.  That is, we can define a binary operation
$z_1 \oplus z_2 = (x_3,-y_3)$ by the circle and hyperbola construction, then calculate the explicit formula for the
additional point  $z_3 = (x_3,y_3)$ of intersection to obtain the explicit formula (\ref{eqn:cx}) for the group operation.

What happens when we replace the circle with a different curve?
We will use exactly the same hyperbola construction to define a binary operation on other curves.
We call this {\it hyperbolic addition} on a curve.   
We can ask whether hyperbolic addition ever gives a group structure on other curves?
(Hint: the answer is yes!)

\subsection{Deforming the circle}

Now we replace the unit circle with a more general algebraic curve, and ask whether hyperbolic addition can
make the curve into a group.  

We discuss some general features that we wish the curve to have.  
We will restrict ourselves to a curve $C$ that
pass through $(1,0)$, which we will take to be the candidate point for the identity element of the group.  
The hyperbolic addition construction also requires $(-1,0)$ to be a point on the curve.

We confine our discussion to a curve that that is the zero set of a polynomial of total degree  $4$.  Bezout's theorem states that
a curve of degree $4$ meets of hyperbola (or any conic) in $4\times 2 = 8$ points (counted with multiplicities,
over an algebraically closed field, including points at infinity).  For hyperbolic addition, we only care about
four intersection points of our curve and the hyperbola $(-1,0)$, $z_1$, $z_2$, and $z_3$.  The easiest
way to make the other four points of Bezout's theorem  go away is to push them off to infinity.  
The hyperbola contains two points at infinity: one along its horizontal asymptote and another along its
vertical asymptote.  We will
assume that the curve $C$ meets each of these two points at infinity with multiplicity two, four a total
of four intersection points at infinity.
This insures that $C$ meets the hyperbola in only four finite points, counted with multiplicity.

Finally, we will impose the symmetries $y\leftrightarrow -y$ and $x\leftrightarrow -x$ on the curve; that is, the polynomial defining the curve should contain only
even powers of $x$ and $y$.  The preceding conditions force the polynomial to have the form
$x^2 + c y^2 - 1 - d x^2 y^2$ (up to a nonzero constant factor) for some parameters $c$ and $d$.
This zero locus of this polynomial is called an {\it Edwards curve}.  The unit circle corresponds to
parameter values $c=1$ and $d=0$.

We define a binary operation on the Edwards curve by the hyperbolic addition law described above.
Let $(-1,0)$, $z_1 = (x_1,y_1)$ and $z_2=(x_2,y_2)$ be three points on an Edwards curve
 that are not collinear (to avoid degenerate cases).  We fit a hyperbola (\ref{eqn:hyp}) through these three points,
 and let $(x'_3,-y'_3)$ be the fourth point of intersection of the hyperbola with the curve.
 The following lemma gives an explicit formula for this fourth point.
 
 \begin{lemma}\label{lemma:hyp} In this construction, the coordinates are given explicitly by
 \[
 (x_3',y_3') = \left(\frac{x_1 x_2 - c y_1 y_2}{1 - d x_1 x_2 y_1 y_2},\frac{x_1 y_2 + y_1 x_2}{1+d x_1 x_2 y_1 y_2}\right)
 \]
 \end{lemma}

This lemma will be proved below (\ref{lemma:hyperbola}).
Until now, we have assumed the points $(-1,0)$, $z_1$, and $z_2$ are not collinear.  
Dropping the assumption of non-collinearity, we use the formula of Lemma~\ref{lemma:hyp} to define the hyperbolic sum
\[
 (x_1,y_1)\oplus (x_2,y_2) := (x_3',y_3').
\]
in all cases.  We prove below a completeness result (Lemma~\ref{lemma:complete}) showing that this formula works in
all cases (that is, the denominators are always nonzero) for suitable parameters $c$ and $d$.

Note that the unit circle belongs to this family of curves for parameter values $c=1$ and $d=0$.  In this case, the formulas in
the lemma reduce to the formula (\ref{eqn:cx}) for the complex multiplication.

In the next section, we prove quite generally that hyperbolic addition given explicitly by Lemma~\ref{lemma:hyp}
makes the Edwards curve into a group.



\section{Group Axioms}

Our primary aim is to prove the group axioms for hyperbolic addition on Edwards elliptic curves (Theorem~\ref{thm:group}).


We have tried to keep our presentation at an undergraduate level.
We shift away from a geometric language and work entirely algebraically over an arbitrary field.
We will assume a basic background in abstract algebra at the level of a first course (rings, fields, homomorphisms, and kernels).
We set things up in a way that all of the main identities to be proved are identities of polynomials with integer coefficients.

If $R$ is a ring (specifically, a ring of polynomials with integer coefficients), and if $\delta\in R$, then we write
$R[\f{\delta}]$ for the localization of $R$ with respect to the multiplicative set $S=\{1,\delta,\delta^2,\ldots\}$.  That is,
$R[\f{\delta}]$ is the ring of fractions with numerators in $R$ and denominators in $S$.  We will need the 
well-known fact that if $\phi:R\to A$
is a ring homomorphism that sends $\delta$ to a unit in $A$, then $\phi$ extends uniquely to a homomorphism
$R[\f{\delta}]\to A$ that maps a fraction $g/\delta^i$ to $\phi(g)\phi(\delta^i)^{-1}$.

We begin with an easy lemma.
\begin{lemma}[kernel property]  Suppose that an identity $g = r_1 e_1 + r_2 e_2 +\cdots + r_k e_k$ holds in a ring $R$.  If $\phi:R\to A$ is a ring homomorphism
such that $\phi(e_i) =0$ for all $i$, then $\phi(g)=0$.
\end{lemma}

\begin{proof}
\[
\phi(g) = \sum_{i=1}^k \phi(r_i) \phi(e_i) = 0.
\]
\end{proof}

We will use the following rings: $R_0 := \ring{Z}[c,d]$, that is, the ring of polynomials in $c$ and $d$ with integer coefficients;
and $R_n := R_0[x_1,y_1,\ldots,x_n,y_n]$, that is, the ring of polynomials in $c,d,x_1,\ldots,y_n$ with integer coefficients.  Let 
\begin{equation}
e(x,y) = x^2 + c y^2 -1 - d x^2 y^2 \in  R_0[x,y].
\end{equation}


For a given field $k$ and specializations $c,d\mapsto \bar c,\bar d \in k$, the zero set of $e(x,y)$ in $k^2$ is 
the set of points on the 
\emph{twisted affine Edwards elliptic curve}.  If $\bar c=1$, the curve is said to be \emph{untwisted}.
Except in incidental remarks such as this one, we avoid the  language of curves, working instead algebraically with rings.

We write $e_i = e(x_i,y_i)$ for the image of the polynomial in $R_j$, for $i\le j$, under $x\mapsto x_i$ and $y\mapsto y_i$.
Set
$\delta^\pm=\delta^{\pm} (x_1,y_1,x_2,y_2) = 1\pm d x_1 y_1 x_2 y_2$ and
\[
\delta=\delta(x_1,y_1,x_2,y_2) = 1 - d^2 x_1^2 y_1^2 x_2^2 y_2^2 =\delta^+(x_1,y_1,x_2,y_2)\delta^-(x_1,y_1,x_2,y_2)\in R_2.
\]
We write $\delta_{ij}$ for its image of $\delta$ under $(x_1,y_1,x_2,y_2)\mapsto (x_i,y_i,x_j,y_j)$.  In particular, $\delta=\delta_{12}$.

We define a pair of rational functions that we denote using the symbol $\oplus$:
\begin{equation}\label{eqn:add}
(x_1,y_1) \oplus (x_2,y_2) =  \left(\frac{x_1 x_2 - c y_1 y_2}{1 - d x_1 x_2 y_1 y_2},\frac{x_1 y_2 + y_1 x_2}{1+d x_1 x_2 y_1 y_2}\right) \in R_2[\f{\delta}]\times R_2[\f{\delta}].
\end{equation}
Commutivity is an obvious consequence of the symmetry $1\leftrightarrow 2$:
\[
(x_1,y_1) \oplus (x_2,y_2) = (x_2,y_2) \oplus (x_1,y_1).
\]
If $\phi:R_2[\f{\delta}]\to A$ is a ring homomorphism, we also write $(a_1,b_1)\oplus (a_2,b_2)\in A^2$ for the image
of $(x_1,y_1)\oplus (x_2,y_2)$, where $x_1,y_1,x_2,y_2 \mapsto^\phi a_1,b_1,a_2,b_2$.  We write $\bar e_i=e(a_i,b_i)\in A$ for the
image of $e_i$ under $\phi$.  Geometrically,
$\bar e_i=0$ asserts that $(a_i,b_i)$ is a point on the Edwards curve.
More generally, we often mark the image $\bar g=\phi(g)$ of an element with a bar accent.

There is an obvious identity element $(1,0)$, expressed as follows.  Under a homomorphism
$\phi:R_2[\f{\delta}]\to A$, mapping $x_1,y_1,x_2,y_2\mapsto a,b,1,0$,
we have $(a,b)\oplus(1,0) = (a,b)$.

\begin{lemma} [inverse] 
Under a homomorphism
$\phi:R_2[\f{\delta}]\to A$, with $x_1,y_1,x_2,y_2\mapsto a_1,b_1,a_1,-b_1$,
we have $(a_1,b_1)\oplus (a_1,-b_1) = (1,0)$, provided $\bar e_1=0$. 
\end{lemma}

\begin{proof}
This is elementary from the definitions of $\oplus$ and $\phi$.
\end{proof}



To develop informal intuition about the Edwards curve and the addition law, it can help to graph the
equation $e(x,y)=0$ over the real numbers with specializations $c\mapsto\bar c = 1$ and $d\mapsto\bar d\in\ring{R}$.  The graph has evident
symmetries $x\leftrightarrow -x$, $y\leftrightarrow -y$.  When $\bar d$ is small and negative, the curve is a symmetrical deformation of the circle, which corresponds to parameter values $(\bar c,\bar d) = (1,0)$.
The group addition law is also a deformation of the addition law of the addition law for the circle:
the numerators of Equation (\ref{eqn:add}) give the usual group law for the circle
 and the denominators are
the \emph{correction factors} needed to bring the sum of two points on the curve back onto the curve,
as the following lemma shows.



\begin{lemma}[closure under addition]
Let $\phi:R_2[\f{\delta}]\to A$ be a ring homomorphism with $x_1,y_1,x_2,y_2 \mapsto^\phi a_1,b_1,a_2,b_2$.
If $\bar e_1 = \bar e_2 = 0$ then ${\bar e}(a_3,b_3) = 0$, where $(a_3,b_3) = (a_1,b_1)\oplus (a_2,b_2)$.
\end{lemma}

\begin{proof} This proof will serve as a model of other proofs.
We write
\[
e(x_3',y_3') = \frac{g}{\delta^2},\quad \text{ where } (x_3',y_3')=(x_1,y_1) \oplus (x_2,y_2) ,\quad 
\]
for some polynomial $g \in R_2$.  It is enough to show that $\phi(g)=0$.
Polynomial division gives
\begin{equation}\label{eqn:closure}
g= r_1 e_1 + r_2 e_2,
\end{equation}
for some polynomials 
$r_i\in R_2$.  Concretely, the polynomials $r_i$ are obtained as the output of the one-line Mathematica command
\[
\op{PolynomialReduce}[g,\{e_1,e_2\},\{x_1,x_2,y_1,y_2\}].
\]
Thus, the result follows from the kernel property and (\ref{eqn:closure}); $\bar e_1 = \bar e_2 = 0$ implies $\phi(g)= 0$, giving ${\bar e}(a_3,b_3)=0$.
\end{proof}

\subsection{associativity}

This next step (associativity) is generally considered the hardest part of the verification of the group law on curves.
The polynomials $\delta^\pm$ appear as  denominators in the addition rule.  The polynomial denominators $\Delta^\pm$ that
appear
when we add twice are more involved.  Specifically, let $ (x_3',y_3')=(x_1,y_1) \oplus (x_2,y_2)$, 
 let $(x_1',y_1')=(x_2,y_2) \oplus (x_3,y_3) $, and set
\[
\Delta^{\pm} = \delta^\pm(x_3',y_3',x_3,y_3)\delta^\pm(x_1,y_1,x_1',y_1')\delta_{12}\delta_{23}\in R_3.
\]

\begin{lemma}[associativity] \label{lemma:assoc} Let $\phi:R_3[\f{\Delta^+\Delta^-}]\to A$ be a homomorphism sending $x_i,y_i\mapsto a_i,b_i$.
Assume $\bar e_1 = \bar e_ 2= \bar e_3 = 0$. Then 
\[
((a_1,b_1)\oplus (a_2,b_2)) \oplus (a_3,b_3)=
(a_1,b_1)\oplus ((a_2,b_2) \oplus (a_3,b_3)).
\]
\end{lemma}

\begin{proof}  The proof is almost identical to the previous proof.
We work in the ring $R_3[\f{\Delta^+\Delta^-}]$ and take the component-wise difference of the two sides, writing
\[
((x_1,y_1)\oplus (x_2,y_2)) \oplus (x_3,y_3)-
(x_1,y_1)\oplus ((x_2,y_2) \oplus (x_3,y_3)) = (\frac{g_1}{\Delta^+},\frac{g_2}{\Delta^-}),
\]
for some polynomials $g_1,g_2 \in R_3$.  It is enough to show that $\phi(g_1)=\phi(g_2)=0$. 
Polynomial division gives
\begin{equation}\label{eqn:assoc}
g_i = r_i^1 e_1 + r_i^2 e_2 + r_i^3 e_3,
\end{equation}
for some polynomials $r_i^j\in R_3$.  
Concretely, the polynomials $r_i^j$ are obtained as the output of the one-line Mathematica command
\begin{equation}\label{eqn:mathca}
\op{PolynomialReduce}[\{g_1,g_2\},\{e_1,e_2,e_3\},\{x_1,y_1,x_2,y_2,x_3,y_3\}].
\end{equation}
Thus, the result follows from the kernel property and (\ref{eqn:assoc}); $\bar e_1 = \bar e_2 = \bar e_3 = 0$ implies $\phi( g_1) = \phi( g_2) = 0$, giving the result.
\end{proof}

\subsection{completeness and group law}

The following result is traditionally called completeness, even though it has nothing to do with the notion of
complete varieties in algebraic geometry.  The meaning of the 
word \emph{complete} in the title and abstract  is that of the lemma: the affine plane contains all $k$-points, 
and denominators are nonzero at every $k$-point.  Because of this result, the addition law is free of the
case-by-case analysis that plagues other approaches.

\begin{lemma}[completeness] \label{lemma:complete} Let $k$ be a field, and let $\phi:R_2\to k$ be a homomorphism
such that $\bar d\in k\setminus k^{\times 2}$, the image of $d$, is not a nonzero square and such that 
$\bar c =\tau^2\in k$, the image of $c$, is a square.  
If $\bar e_1 = \bar e_2= 0$, then $\phi(\delta)\ne 0$.
\end{lemma}

\begin{proof}   We prove the lemma in the following contrapositive form:
assuming that $\phi(\delta) = 0$ together with the other assumptions, we show that $\bar d$ is a nonzero square.
Let $g = (1 - c d y_1^2 y_2 ^2) (1 - d y_1^2 x_2^2)$.   
We have
\begin{equation}\label{eqn:squares}
g = r_0 
\delta + r_1 e_1 + r_2 e_2, 
\end{equation}
where
\begin{equation}
r_0 = (1-d y_1^2),\quad r_1 = d^2 y_1^2 y_2^2 x_2^2,\quad r_2 = d y_1^2.
\end{equation}
The kernel property gives $\phi(g) = (1 - \tau^2 \bar d b_1^2 b_2^2)(1 - \bar d b_1^2 a_2^2) =0$.
We solve this equation for $\bar d$, which expresses it as the square of $1/(\tau b_1 b_2)$ or $1/(b_1 a_2)$.
\end{proof}

We are ready to state and prove the main result of this article.  The group law is expressed generally enough
to include the group law on the ellipse as a special case $\bar d = 0$.  Textbooks are emphatic%
\footnote{``A word of warning: \ldots Elliptic curves are not ellipses, and indeed, despite their somewhat unfortunate name, elliptic curves and ellipses have only the most tenuous connection with one another''~\cite{hoffstein2008introduction}.}
that elliptic curves are not ellipses, but here through hyperbolic addition, we give a unified treatment of both 
within a single family of curves with a uniformly defined group law.  



\begin{theorem}[group law]\label{thm:group} 
Let $k$ be a field, let $\bar c \in k$ be a square, and let $\bar d\in k\setminus k^{\times 2}$.
Let $\bar e(x,y) \in k[x,y]$ be the specialization of $e(x,y)$, obtained by $(c,d)\mapsto (\bar c,\bar d)$.
 Then $E= \{(a,b)\in k^2 \mid \bar e(a,b) = 0\}$ is an abelian group with binary operation $\oplus$.
\end{theorem}

\begin{proof} This follows directly from the earlier results.  For example, to check associativity of 
$(a_1,b_1)\oplus (a_2,b_2) \oplus (a_3,b_3)$, where $(a_i,b_i)\in E$, we define a homomorphism
$\phi:R_3\to k$ sending $(x_i,y_i)\mapsto (a_i,b_i)$ and $(c,d)\mapsto (\bar c,\bar d)$.   By a repeated use of the completeness lemma,
$\phi(\Delta^+\Delta^-)$ is nonzero and invertible in the field $k$.
The universal property of localization extends $\phi$ to a homomorphism $\phi:R_3[\f{\Delta^+\Delta^-}]\to k$.  
By the associativity lemma applied to $\phi$, we obtain the
associativity for these three (arbitrary) elements of $E$.  The other properties follow similarly from the lemmas on closure,
inverse, and completeness. 
\end{proof}

\section{Hyperbola lemma}

It is well-known that three points $(x_0,y_0)$, $(x_1,y_1)$, and $(x_2,y_2)$ in the plane are collinear if and only if
the following determinant is zero:
\[
\begin{vmatrix}
x_0 & y_0 & 1\\
x_1 & y_1 & 1\\
x_2 & y_2 & 1
\end{vmatrix}.
\]
When $(x_0,y_0) = (-1,0)$, the three points are collinear exactly when
$D=0$, where $D= (x_1+1) y_2 - (x_2+1) y_1$ is the determinant.  We treat $D$ as an element of $R_2$.
In this subsection, we will need a noncollinearity condition, which we impose by inverting $D$.

We now return to the family of hyperbolas that were introduced at the beginning of this article.
(Note that our proof of the group axioms in the previous section
does not depend on the hyperbolic interpretation of addition.)
The polynomial
\[
h(x,y) = x y + p (x+1) + q y \in R_0[p,q,x,y]
\]
with parameters $p,q$
represents the family of hyperbolas passing through $(-1,0)$ and having asymptotes parallel to the two coordinate axes.
We can solve the two linear equations in $p$ and $q$:
\[
h(x_1,y_1)=h(x_2,y_2)=0
\]
uniquely for $p$ and $q$ in the ring $R_2[\f{D}]$.  Identifying $p$ and $q$ with these ring elements, we now
take $h(x,y) \in R_2[\f{D}][x,y]$.  It represents the unique 
hyperbola passing through points and $(-1,0)$, $(x_1,y_1)$, and $(x_2,y_2)$.
 
\begin{lemma}[hyperbolic addition]\label{lemma:hyperbola}
Let $\phi:R_2[\f{D\delta}]\to A$ be a ring homomorphism sending $x_1,y_1,x_2,y_2\mapsto a_1,b_1,a_2,b_2$.
Let $(a_3,b_3) = (a_1,b_1)\oplus (a_2,b_2)$.  If $\bar e_1 = \bar e_2 = 0$, then $\bar h(a_3,-b_3) = 0$.
\end{lemma}

\begin{proof}
We work in the ring $R_2[\f{D\delta}]$ and write
\[
h(x_3',-y_3') = \frac{g}{D\delta},\quad \text{where } (x_3',y_3') = (x_1,y_1)\oplus (x_2,y_2)
\]
for some polynomial $g \in R_2$.  It is enough to show that $\phi(g)=0$. 
Polynomial division gives
\begin{equation}
g = r_1 e_1 + r_2 e_2,
\end{equation}
for some polynomials $r_i\in R_2$.
Concretely, the polynomials $r_i$ are obtained as the output of the one-line Mathematica command
\begin{equation}
\op{PolynomialReduce}[\{g\},\{e_1,e_2\},\{x_1,y_1,x_2,y_2\}].
\end{equation}
Thus, the result follows from the kernel property: $\bar e_1 = \bar e_2 = 0$ implies $\phi(g)  = 0$, as desired.
\end{proof}

As the reader has no doubt observed, all the major proofs in this article follow the same template, simply plugging
different polynomials into the same identity proving machine.

\section{Elliptic Curves}

In this section assume that the characteristic of the field is not $2$.

Starting with the equation $x^2 (1-d y^2) = (1 - c y^2)$ of
of the Edwards curve, we can multiply both sides by $(1- d y^2)$ to bring it into the form
\[
z^2 = (1 - d y^2) (1 - c y^2),
\]
where  $z = x (1 - d y^2)$.
This a Jacobi quartic.  It is an elliptic curve whenever the quartic polynomial in $y$ on the right-hand side
is separable.

Over an algebraically closed field, every elliptic curve is isomorphic to an Edwards curve.
Over arbitrary fields, some elliptic curves cannot
be represented as Edwards curves, but Edwards curves include a large fraction of 
isomorphism classes of elliptic curves over finite fields and are sufficient for many
cryptographic purposes~\cite{bernstein2008twisted}.  

Interpreted in this light, we have proved the group law for for a significant
portion of all elliptic curves.  Most writers sweat much more than we have to prove the group
law for elliptic curves.  In particular, as far as we know,
our $11$-line computational proof of the associativity of elliptic-curve addition is the
shortest proof on record.  Of course, we disqualify proofs that are incorrect or incomplete, but 
even most false proofs of associativity are longer than ours.

We mentioned B\'ezout's theorem, but it does not enter into the proof.  We mention Mathematica calculation,
but as the next section shows, the polynomial certificates from Mathematica can be independently verified.
By luck, we have even avoided the use of the Groebner basis algorithms, which is usually taken as the starting point
of elementary verifications of the elliptic curve group law.  For the proof of the group law,
we have used little more than polynomial division
and localization of polynomial rings.




\section{Formalization in HOL Light}

% We implicity assume the fact that the ring R[x1,...,xk] injects into the ring of functions on R^n.
We have formally verified the main polynomial identities Equations (\ref{eqn:closure}), (\ref{eqn:assoc}), (\ref{eqn:squares}) in
the proof assistant HOL Light using its tactic \verb!REAL_ARITH!, which checks polynomials identities
over the real numbers. The statements in HOL Light implicitly use an identification of the ring
$\ring{R}[x_1,\ldots,x_k]$ with a ring of real-valued functions on $\ring{R}^k$.
By the ring injection $\ring{Z}\subset \ring{R}$, these polynomial identities must then also hold over $\ring{Z}$.

Most of our software development took place in Mathematica, using the \verb!PolynomialReduce! function.  These
Mathematica calculations are very fast. For example, the associativity certificate (\ref{eqn:mathca})
takes about $0.12$ second to compute
on a 2.13 GHz processor.
Once the
Mathematica code was in final form, it took us less than 30 minutes of development time in HOL Light to copy the polynomial identities
over to the proof assistant and formally verify them.  All these polynomial identities combined 
can be formally verified in less than 2 seconds of CPU time
on a 2.13 GHz Intel Core 2 Duo processor. The most difficult formal verification is the associativity identity (\ref{eqn:assoc}), 
which takes
about 1.5 seconds.  The computer code is available at 
\url{www.github.com/flyspeck/publications-of-thomas-hales}.

These verification times are significantly faster than times reported in other projects.  In 2007, L. Th\'ery described
the formalization in Coq of elliptic curves in Weierstrass form~\cite{thery2007proving}:
\begin{blockquote}
This work has a long story.  Joe Hurd was the first to draw our attention to the possibility of formalising elliptic curves
inside a prover.  At that time, we had a go but were quickly convinced that a prover like Coq, which requires
a formal justification of every step of a proof, could not cope with the necessary computations.  Last september, we emailed
to John Harrison the example of the $x$ component of the generic case and to our great surprise he managed to prove
it in less than 3 minutes inside HOL Light with his integrated version of Buchberger algorithm.  So the
situation was not as hopeless as we thought.  Indeed, the proof presented here is checked by Coq, computations included,
in 1 minute 20 seconds.  This is a striking example of how crucial it is in a prover to be able to mix proof and computation.
\end{blockquote}

Friedl  was the first to carry out the computer algebra required for an elementary proof of the elliptic curve group axioms.  
He wrote in 1998 about the difficulty of these computations \cite{friedl}:
\begin{blockquote}
We will give a completely elementary proof, just using the above explicit definition of the group structure through formulas. It was always clear that such a proof exists, but it turns out that this direct proof is more difficult than one might have imagined initially. Many special cases have to be dealt with separately and some are non trivial. Furthermore it turns out that the explicit computations in the proof are very hard. The verification of some identities took several hours on a modern computer; this proof could not have been carried out before the 1980�s.
\end{blockquote}

In closing, we remark that we have not carried our formalization as far as some other teams.  In particular, we have not formally verified the group law (Theorem~\ref{thm:group}).  It would be desirable to do so.

\subsection{Acknowledgements}

Inspiration for this article comes from 
Bernstein and Lange's
wonderfully  gentle introduction to 
elliptic-curve cryptography at the 31st Chaos Communication Congress.
They use the group law on the circle to motivate the group law on Edwards elliptic curves.
Hyperbolic addition is introduced for Edwards elliptic curves in \cite{arene2011faster}.

% http://www.dagstuhl.de/en/program/calendar/semhp/?semnr=05021
I met Harold Edwards at a conference in Dagstuhl Germany in January 2005.  
The conference ``verification and constructive algebra'' combined verification (a big interest
of mine) with constructive algebra (a big interest of his).  All the meals were at the
on-site restaurant, and seats at the tables were assigned by a randomized algorithm.  By
chance, Edwards and I were assigned the same table the first day, and then again by
coincidence again on the second day.  By the third or fourth consecutive ``random'' assignment
at the same table, we doubted the integrity of the seating algorithm and took seat selection
into our own hands.  Thanks to faulty random-number generation, 
I learned much about Abel's work and about 19th century
constructive mathematics. His article on Edwards curves had not yet been written.

\bibliography{refs} 
\bibliographystyle{alpha}

\end{document}

% Friedl's elementary 
% http://math.rice.edu/~friedl/papers/AAELLIPTIC.PDF


\end{document}
