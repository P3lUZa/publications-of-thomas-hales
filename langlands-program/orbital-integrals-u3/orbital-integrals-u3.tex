\def\AA{{\bf A}}
\def\semi{\hbox{$\Omega\mathchar"3202\mkern -3mu\vrule height 4.5pt
    depth .1pt width 0.3pt Gal(E/F)$}}

\def\CC{{\bf{C}}}
\def\C{{\Gamma}}
\def\Ea{E_\a}
\def\Eb{E_\b}
\def\Eo{E_0}
\def\FF{{\bf{F}}}
\def\Fbar{\overline{F}}
\def\Gal{{Gal(\overline{F}/F)}}
\def\KOI#1#2{\Phi^{(T,\k)}_G(#1,#2)}
\def\Ni{{N_\infty}}
\def\Proof{\medbreak\noindent{\bf Proof.\enspace}} 
\def\QQ{{\bf{Q}}}
\def\Remark{\medbreak\noindent{\bf Remark. \enspace}}
\def\RR{{\bf{R}}}
\def\SOI#1#2{\Phi^{(T,st)}_H(#1,#2)}
\def\Tbold{{\bf T}}
\def\TmodG{{(T\backslash G)(F)}}
\def\ZZ{{\bf{Z}}}
\def\aXw{{1 + \alpha(X) w}}
\def\aX{\alpha(X)}
\def\a{{\alpha}}
\def\bXw{{1 - \beta (X) w}}
\def\bX{{\beta (X)}}
\def\b{{\beta}}
\def\cX{{\gamma (X)}}
\def\cog{{\s (g)g^{-1}}}
\def\coh#1{H^1(F,#1)}
\def\c{{\gamma}}
\def\dedef{=^{def}}
\def\dfrac#1#2{\displaystyle{ {{#1}\over{#2}}  } }
\def\d{{\delta}}
\def\endsect{\par\vskip .2 in }
\def\endtitle{}
\def\ep{{\epsilon}}
\def\frac#1#2{{ {#1}\over{#2}}}
\def\help{{\bf help}}
\def\he{{\eta_{E/F}}}
\def\i{{^{-1}}}
\def\k{{\kappa}}
\def\l{{\lambda}}
\def\nq{{\frac{n_\alpha n_\b}{n_\c}}}
\def\mult{\times}
\def\proof{\Proof}
\def\ref {{\bf{ [Ref] }}}
\def\sa{{\sigma_{\alpha}}}
\def\sect#1\endsect{\par\medskip{\bf\centerline{#1}}}
%\def\sox{\sigma_0 \sigma_{-}}
%\def\so{\sigma_0}
%\def\sx{\sigma_{-}}
\def\s{\sigma}
%\def\tag{}
\def\th{{\theta}}
\def\title#1\endtitle{\vfill {\bf\centerline{#1}} \vfill}
\def\w #1{w(#1)}
\def\wc{{w(\gamma)}}
\def\xa{{x(\alpha)}}
\def\xb{{x(\beta)}}
\def\xc{{x(\gamma)}}
\def\y#1{y(#1)}
\def\z #1{z(#1)}
\def\zquot{{\frac {z_1''}{z_3''}}}

%\input vanilla.sty
%\input format.tex
\magnification = 1300
\baselineskip = .4in  
\overfullrule 0pt
%  Cover Page 
\vfill
\title Orbital Integrals on $U(3)$ \endtitle
\centerline{Thomas C. Hales}
\vskip 1in
\centerline{The Institute for Advanced Study}
\vskip 1in
\centerline{Version - 25 -- 1 -- 1990}
\vfill
\eject
\pageno=1
\title Orbital Integrals on $U(3)$ 
\endtitle

\centerline{Thomas C. Hales}
\bigskip 

\par\noindent {\bf Section 1}\dotfill   Statement of Theorem
\par\noindent {\bf Section 2}\dotfill  Transfer Factors
\par\noindent {\bf Section 3}\dotfill  Shalika Germs
\par\noindent {\bf Section 4}\dotfill  Descent
\par\noindent {\bf Section 5}\dotfill  Unipotent Classes
\par\noindent {\bf Section 6}\dotfill  A Reduction
\par\noindent {\bf Section 7}\dotfill  Igusa Theory
\par\noindent {\bf Section 8}\dotfill  A Geometric Construction
\par\noindent {\bf Section 9}\dotfill  Regular Classes
\par\noindent {\bf Section 10}\dotfill  Actions of the Galois Group
\par\noindent {\bf Section 11}\dotfill  Subregular Classes
\par\noindent {\bf Section 12}\dotfill  Normalization of Measures
\par\noindent {\bf Section 13}\dotfill  Other Endoscopic Groups
\par\noindent {\bf References}\hfill

\bigskip 
\sect 1.	Statement of Theorem
\endsect

Let $G = U_E(3)$ denote a unitary group in three variables over a
nonarchimedean local  field $F$ of characteristic zero.  The group  $G$ is quasi-split and
splits over a quadratic extension $E$ of $F$.  To be precise we
assume that $G(F) = \{ g \in GL(3,E) | {}^t\overline{g}Jg = J\}$
where $J = \pmatrix {0&0&1\cr 0&1&0\cr 1&0&0} $.  If $x$ is a
matrix with coefficients in $E$, $\bar x$ denotes the matrix whose
coefficients are conjugated by $Gal(E/F)$.
The group of upper triangular matrices is a
Borel subgroup of $G$ defined over $F$.

Let $H$ be the group  $U_E(2) \times U_E(1)$ again quasi-split over
$F$ and split over the quadratic field extension $E$ of $F$.  Fix an
embedding $\iota: H \to G$ defined over $F$ whose image is the
subgroup $$\left\{\pmatrix {*&&*\cr &*&\cr *&&*\cr}\right\}.$$
$\iota$ is well defined up to an automorphism of $H$ over $F$.

Let $H'$ denote the image of $H$ in $G$.  We identify Cartan subgroups
$T$  in $H$ with Cartan subgroups $\iota(T)$ in $H'$  by this embedding.
We drop $\iota$ from the notation when the context is clear.

There is an induced map on cohomology:  $\iota_*: \coh H  \to
H^1(F,G)$.  There is exactly one non-trivial class
$a$ in $H^1(F,H)$ mapping to the trivial class in $ H^1(F,G)$; we
let $a'$ denote the image of $a$ in $H^1(F,H')$.  Let
$e$ (resp. $e'$) denote the trivial class in $\coh H$ (resp. $\coh {H'})$. 
Let $\Fbar$ denote the
algebraic closure of $F$.  Fix
$g_o \in G(\Fbar )$ such that $\sigma(g_o) g_o^{-1}$ for $\sigma \in
\Gal$ represents $a'$
in $H^1(F,H')$.  Then $H'' \dedef {H'}^{g_o}$ is a
reductive group over $F$ in $G$ which is an inner form of $H'$. 

Let $T \subset {H'}$ be a Cartan subgroup over $F$.  Note that in general
the quotient of the $F$-points $T(F)\backslash G(F)$ does not equal the
$F$-points of the quotient $(T\backslash G)(F)$.  The defining condition
of the latter is $\s(Tg)=Tg$ or $\cog \in T(\Fbar)$ for all $\s\in\Gal$.
If $Tg \in \TmodG$ then the cocycle $\cog$, $\s
\in \Gal$, gives a well-defined class in $H^1(F,T) $ and hence in $\coh
{{H'}}$.  Set $\kappa (\cog ) = +1 $ or $-1$  according as $\cog$
represents the trivial or non-trivial cocycle $e'$ or $a'$.  It is
a simple matter to see that 
$\kappa$ distinguishes between Cartan subgroups $T^g$ over $F$
conjugate by $G(F)$ into $H'$ from those $G(F)$-conjugate to one of
$H''$.

Fix invariant forms $\omega_G,\ \omega_H,\ \omega_T$ 
of maximal degree on $G$, $H$, and $T$ for all
Cartan subgroups $T$ of $H$.  Assume they are defined over $F$.  
Identifying $T$ with a Cartan subgroup
in $G$ by $\iota$ we obtain invariant forms $\omega_{T\backslash G}$
on $T\backslash G$ and $\omega_{T\backslash H}$ on $T\backslash H$
with associated measures $|\omega_{T\backslash G}|$ and
$|\omega_{T\backslash H}|$.

\def\KOI#1#2{\Phi^{(T,\kappa)}_G(#1,#2)}
\def\SOI#1#2{\Phi^{(T,st)}_H(#1,#2)}

Let $\KOI \c f$ for $f \in C_c^\infty(G)$ denote the integral $\int_{\TmodG}
\kappa(\cog)f(g\i\c g)|\omega_{T\backslash G}|$ for $\c $ regular in
$T$.  Similarly define $\SOI \c {f^H} = \int_{(T \backslash H)(F)}
f^H(h\i\c h)|\omega_{T\backslash H}|$ for $\c$ regular in $T$ and
$f^H \in C_c^\infty(H)$.  $\KOI \c f$ and $\SOI \c {f^H}$ are called
$\k$-orbital integrals and stable orbital integrals respectively.
Let $\gamma_1$, $\gamma_2$, $\gamma_3$ denote the eigenvalues of 
$\gamma$ chosen so that $\gamma_2$ is the projection of $\c$ 
from $H' = U_E(2)\times U_E(1)$ to $U_E(1)$.  
We consider the following functions
$$\eqalign {D_H(\c) &= \frac{|\c_1-\c_3|}{|\c_1\c_3|^{1/2}} \cr 
\Delta^*_G(\c)&=\Delta(\c)D_H(\c) \cr }$$
for $\c$ having regular image in $G$ ($G$-regular).
   Here $\Delta(\c)$, or rather 
$\Delta(\c,\iota(\c))$, is the transfer factor of Langlands 
and Shelstad [LS2].  It is defined in the next section.

The purpose of this paper is to sketch a proof of the following theorem
of Langlands and Shelstad.  [L2], [LS3].

\proclaim {Theorem  (Langlands-Shelstad) 1.1}.  For every $f \in C_c^\infty(G)$ 
there exists a function $f^H \in C_c^\infty(H)$
such that for all $T \subseteq H$ and $\c$ with $\iota(\c)$ 
regular in $G$ 
$$ D_H(\c)\SOI \c {f^H} = \Delta^*_G(\iota(\c))\KOI {\iota(\c)}
f$$

The outline of the proof is as follows.  In the next section the
transfer factor of Langlands and Shelstad is defined, and a few of
its elementary properties are described.  Section 4 discusses descent
which reduces the theorem to elements $\c$ in an arbitrarily small
neighborhood of the identity.  Near the identity element there is
an expansion, called the Shalika germ expansion, for orbital 
integrals.  We will see that 
our $\k$-orbital integrals possess a three term
expansion.  To prove the theorem we must match each term with a
corresponding term in the expansion of stable orbital integrals
on $H$.  Sections 7 and 8 describe a general method, Igusa theory,
of obtaining formulas for Shalika germs.  Section 9 matches the
first term of the expansion, the regular germ, with the corresponding
term on the endoscopic group.  Sections 10 and 11 show that one of
the remaining two terms of the Shalika germ expansion vanishes, while
the other matches the remaining term on $H$.  Section 12 relates
the normalization of measures to the identity of the Hecke
algebra.  Section 13 shows how to extend the results to other
endoscopic groups.

\sect 2.        Transfer Factors \endsect

The paper [LS2] of Langlands and Shelstad defines a factor 
$\Delta(\c) = \Delta(\c,\iota(\c))$ associated to strongly $G$-regular semisimple elements $\c$ in $H'$.  
This section gives an explicit formula for that factor.  
The factor $\Delta(\c)$ is defined as a product of 5 factors  $$\Delta_I,\quad \Delta_{II},\quad \Delta_{III_1},\quad \Delta_{III_2},\quad \Delta_{IV}$$
The product of the factors is canonically defined.  
The individual factors rely on several choices.   We remark 
that the factor $\Delta(\c)$ is actually a function $\Delta(\c_H,\c_G)$,
$\c_H\in H(F)$, $\c_G\in G(F)$ and that $\Delta(\c_H,\c_G)$ itself
is defined through a factor depending on four elements $(\Delta(\c_H,
\c_G,\bar\c_H,\bar\c_G)$, $\c_H,\bar\c_H \in H(F)$, $\c_G,\bar\c_G
\in G(F))$.  A pair $\bar\c_H,\bar\c_G$ is fixed, $\Delta(\bar\c_H,
\bar\c_G)$ is specified arbitrarily and $\Delta(\c_H,\c_G)$ is defined
as $$\Delta(\bar\c_H,\bar\c_G)\Delta(\c_H,\c_G;\bar\c_H,\bar\c_G).$$ 
What we write as $\Delta(\c)$ is the factor $\Delta_0(\c,\iota(\c))$
of [LS2] for an appropriate choice of $F$-splitting.


Let $\a=\c_1\c_2\i$, $\b=\c_2\c_3\i$ denote the positive simple roots of the diagonally embedded Cartan subgroup $\bf T$ in the upper triangular Borel subgroup $\bf B$.  
At times we will write the root $\a\b$ additively as $\a+\b$.  
Let $X^*({\bf T})$ denote the character group of $\bf T$.  
The Galois group of $E/F$ and the Weyl group $\Omega$ act on $X^*({\bf T})$.  
These actions may be combined to give an action of the semidirect product 
$\semi$ on $X^*({\bf T})$.  
We let $\omega$ denote the longest element $\omega=\sigma_\a\sigma_\b\sigma_\a$ in $\Omega$ 
($\s_\a,\s_\b$ simple reflections), and let $\tau$ denote the nontrivial element in $Gal(E/F)$.  
Let $\Theta$ denote the subgroup $\{1,\tau,\omega,\tau\omega\}$ of $\semi$.  
It is isomorphic to $\ZZ_2 \times \ZZ_2$.

If $T$ is a Cartan subgroup of $H'$ over $F$ split by a Galois extension $K$ of $F$ then by conjugating $T$ to $\bf T$ we identify the roots of $T$ with $\a,\b,\a\b,\a\i,\b\i,\a\i\b\i$ and we identify $X^*(T)$ with $X^*(\bf T)$.  
Transporting the action of $Gal(K/F)$ on $X^*(T)$ to $X^*(\bf T)$ one finds that there is a unique map $$Gal(K/F) \mapsto^{\phi_T}\Theta$$ with image $\C_T$ respecting 
 the transported action of $Gal(K/F)$ and $\Theta$ on $X^*(\bf T)$.  
Note also that we have a commutative diagram
$$\eqalign{Gal(K/F) &\to^{\phi_T}\Theta \subseteq \semi\cr
\searrow\phantom{F)} &\phantom{\semi}\swarrow\cr
  &Gal(E/F)}$$
We must distinguish three possibilities for $T$.  These are the only $T$ that are stably conjugate to a Cartan subgroup of $H'$ or $H''$:

\settabs 
\+ (iii)\quad&$[K:F]=4$\qquad&$\C_T=\{1,\tau,\omega,\omega\tau\}$
   \qquad&$Gal(K/F)=\{1,\s_\tau,\s_\omega,\s_{\omega\tau}\}$\cr
    %sample line
\+ (i)&$K=E$&$\C_T=\{1,\tau\}$&$Gal(E/F)=\{1,\s_\tau\}$\cr
\+ (ii)&$K=E$&$\C_T=\{1,\omega\tau\}$&$Gal(E/F)=\{1,\s_{\omega\tau}\}$\cr
\+ (iii)&$[K:F]=4$&$\C_T=\{1,\tau,\omega,\omega\tau\}$&$Gal(K/F)=\{1,\s_\tau,\s_\omega,\s_{\omega\tau}\}$\cr

In case $(iii)$, $E$ is the field fixed by $\s_\omega$, $E_\tau$ is the field fixed by $\s_\tau$, and $E_{\omega\tau}$ is the field fixed by $\s_{\omega\tau}$.

A certain selection of constants in $\overline{F}^\times$, known as $a$-data and a certain selection of characters on extensions of $F^\times$ known as $\chi$-data, are required for the definition of $\Delta_I$, $\Delta_{II}$, $\Delta_{III_1}$, $\Delta_{III_2}$.    
The following table indicates choices of $a$-data and $\chi$-data and the corresponding definitions of  factors.  
The reader may consult [LS2] for details.  We also select the element $h$ of the construction to lie in the derived group of $H'$.  If
$E''/E'$ is a quadratic extension of fields let $\eta_{E''/E'}:
E'^\times \to \{\pm 1\}$ denote the character of $E'^\times$ associated
to this extension by local class field theory.

\vfill\eject\hbox{}\vfill\eject %\input table0.tex

% table0.tex starts here
%\input format.tex
\nopagenumbers

\moveleft1.5in\vbox{\offinterlineskip\hrule
\halign{
&\vrule#&\strut\quad\hfil#\hfil\quad\cr
height4pt& \omit && \omit &&\omit && \omit &\cr
&\omit &&(i)&&(ii)&&(iii)&\cr
height4pt
&\omit&&\omit    &&\omit   && \omit &\cr
\noalign{\hrule}
height4pt
&\omit  &&\omit  &&\omit   && \omit &\cr
&$a$-data&&$a_\a=a_\b=a_{\a+\b} =1$&&$a_\a=-a_\b=a_{\a+\b}\in E^\times$&&$a_\a\in E^\times,\ \bar a_\a =-a_\a$&\cr
&        &&$a_{-\a}=a_{-\b}=a_{-\a-\b}=-1$&&$\bar a_\a=-a_\a$&&$a_{\a+\b}\in E_\tau,\ \s_\omega(a_{\a+\b}) = 
-a_{\a+\b}$&\cr
&        &&\omit&&$a_{-\a}=-a_{-\b}=a_{-\a-\b}=-a_{\a}$&&$a_{-*}=-a_{*},\ *=\a,\b,\a+\b$&\cr
&\omit   &&\omit&&\omit &&$a_\b =-a_\a$&\cr
height4pt &\omit &&\omit &&\omit && \omit &\cr
\noalign{\hrule}
height4pt &\omit &&\omit &&\omit && \omit &\cr
&$\chi$-data&&all trivial&&$\chi_\a=\chi_\b$ characters on $E^\times$&&$\chi_\a:K^\times \to \CC^\times$&\cr
&\omit&&\omit&&extending&&extending&\cr
&\omit&&\omit&&$\he:F^\times\to\{\pm1\}$&&$\eta_{K/E_{\omega\tau}}:E^\times_{\omega\tau}\to \{\pm1\}$&\cr
height4pt &\omit&&\omit &&\omit &&\omit &\cr
\noalign{\hrule}
height4pt &\omit&&\omit &&\omit &&\omit &\cr
&$\Delta_I(\c)$&&1&&1&&1&\cr
height4pt &\omit&&\omit &&\omit &&\omit &\cr
\noalign{\hrule}
height4pt &\omit&&\omit &&\omit &&\omit &\cr
&$\Delta_{II}(\c)$&&1&&$\chi_\a((\c_1\c_2\i-1)(\c_2\c_3\i-1))$&&$\chi_\a\left(
\displaystyle{\c_1\c_2\i -1 \over a_\a}
\right)$&\cr
height4pt &\omit&&\omit &&\omit &&\omit &\cr
\noalign{\hrule}
height4pt &\omit&&\omit && \omit &&\omit &\cr
&$\Delta_{III_1}(\c)$&&1&&1&&1&\cr
height4pt &\omit&&\omit&&\omit&&\omit&\cr
\noalign{\hrule}
height4pt &\omit&&\omit&&\omit&&\omit&\cr
&$\Delta_{III_2}(\c)$&&$\th_1(\c)$&&$\th_2(\c)$&&$\th_3(\c)$&\cr
height4pt &\omit&&\omit&&\omit&&\omit&\cr
\noalign{\hrule}
height4pt &\omit&&\omit&\omit&\omit&\omit&\omit&\cr
&$\Delta_{IV}(\c)$&&$D_{G/H}(\c)=|(\c_1\c_2\i-1)(\c_1\i\c_2-1)(\c_2\c_3\i-1)(\c_2\i\c_3-1)|^{1/2}$
\span\omit\span\omit\span\omit\span\omit&\cr
height4pt &\omit&&\omit&\omit&\omit&\omit&\omit&\cr
\noalign{\hrule}
height4pt &\omit&&\omit&&\omit&&\omit&\cr
&$\Delta(\c)$&&$\th_1(\c)D_{G/H}(\c)$&&$\th_2(\c)\chi_\a((\c_1\c_2\i-1)(\c_2\c_3\i-1))D_{G/H}(\c)$&&
$\th_3(\c)\chi_\a\left(\displaystyle {\c_1\c_2\i-1 \over a_\a}\right)D_{G/H}(\c)$&\cr
height4pt &\omit&&\omit&&\omit&&\omit&\cr
}\hrule}

% table0.tex ends here

$\th_1$, $\th_2$, $\th_3$ are continuous characters on the $F$-points
of the Cartan subgroups $T=C_G(\c)$.  We will see in section 4 that
it is not necessary to compute $\th_i$ to prove theorem 1.1.  since
for $\c$ sufficiently close to $1$ we have
$\th_i(\c)=1$.  

Here is a brief description of $\th_i$.  
To fix the characters $\th_1,\th_2,\th_3$ it is 
necessary to fix supplementary data, endoscopic data $(H,{\cal H},s,\xi)$
attached to the endoscopic group $H$.  Let $W_F$ denote the
Weil group of $F$ and $\hat G$, $\hat H$, $\hat T$ the connected
complex duals of $G$, $H$, and $T$.  Examples: $\widehat {U(n)} = GL(n,\CC)$, 
$\widehat{Sp(2n) } = SO(2n+1,\CC)$, $\widehat{SL(n)} = PSL(n,\CC)$;
the Weil group $W_{E/F}$, at a finite level, is a nonsplit extension
$$1 \to E^\times \to W_{E/F} \to Gal(E/F) \to 1$$
${\cal H}$ and ${}^LG$ are defined as semidirect products
$$1 \to \hat H \to {\cal H} \to W_F \to 1$$
$$1 \to \hat G \to {}^LG \to W_F \to 1$$
satisfying certain properties.  The element $s$ is semisimple in $\hat G$.
Finally, $\xi$ is an embedding $\xi: {\cal H} \to {}^LG$ which
restricts to an isomorphism of $\hat H$ and $C_{\hat G}(s)^0$.   This 
data is subject to a list of conditions [L1,LS2] and a notion of
equivalence.  The $\chi$-data and $\xi$ determine an element
$a\in H^1(W_F,\hat T)$.  This cohomology group pairs with $T(F)$ as
in [B], and determines a character $\th_i$ of $T(F)$.


It is also necessary to have a limit formula for $\Delta(\c)$.
Set $\c = \exp(\l X)$, $\l \in F^\times$.  Let $\a(X)$, $\b(X)$
be the simple roots evaluated on the element $X$ of the Lie algebra.

\proclaim {Lemma 2.1}.  $$\Delta_0(X) \dedef \lim_{\l\to 0} {\Delta(\c)
\over D_{G/H}(\c)} = \he(\a(X)\b(X))$$

\proof   This is immediate from the definitions of $\Delta(\c)$ in 
cases (i) and (ii).  In case (iii) 
$$\Delta_0(X) = \lim \eta_{K/E_{\omega\tau}}
(\displaystyle {\l\a(X)\over a_\a})$$ with $\displaystyle {\l\a(X)
\over a_\a}\in E_{\omega\tau}$.  By local class field theory this is
equal to $$\he\left(\displaystyle { {\l\a(X) \over a_\a} \tau \left(
{\l\a(X) \over a_\a}\right)}\right) = \he(\a(X)\b(X)).$$

\smallskip
Another limit of interest is

\proclaim{Lemma 2.2}.  For $\c$ sufficiently small $$\lim_{\c_1\to\c_3}
{\Delta(\c)\over D_{G/H}(\c)} =1$$

\proof  We may select $\c$ small so that $\th_i(\c)=1$.  
The lemma is immediate from the definitions in cases (i),(ii).
For case (iii) and $\c$ sufficiently small the previous lemma
may be used to approximate $\Delta(\c)$ by $\he(\a(X)\b(X))$, 
noting that $\a(X)\in E$ when $\c_1=\c_3$ so that $\a(X)\b(X)$ is
the norm of $\a(X)$.

\bigskip
Although this paper deals with $F$ a $p$-adic field, [LS3] also
defines transfer factors in the case of $F$ archimedean.  It is
enough to consider the case $F=\RR$ by restricting scalars for
complex groups.   Shelstad in [S] defines a transfer factor
$\Delta^{(\RR)}(\c_H,\c_G)$ for real groups and proves the
matching of orbital integrals.  In [LS4] Langlands and Shelstad
show that there is a constant $c$ such that $$c\Delta^{(\RR)}(\c_H,\c_G)=
\Delta(\c_H,\c_G)$$ for all $G$-regular $\c_H$ in $H(\RR)$.  The
matching of orbital integrals for $\Delta(\c_H,\c_G)$ then follows
by this work of Shelstad.

There is also a global product formula relating the transfer
factors at all places of a number field $k$ with the ring of
adeles $\AA$.  We take $H$ and $G$ to be over $k$.  We say that
$\c_H \in H(k)$ is an adelic image of $\c_G\in G(\AA)$ if for
every place $v$, $\c_H$ is an image of $\c_{G,v}$ in $G(F_v)$.

Using global Tate-Nakayama duality, Langlands and Shelstad 
define a global factor $\Delta_\AA(\c_H,\c_G)\in \CC^\times$,
$\c_H\in H(k)$, $\c_G\in G(\AA)$, which coincides with the
term $\k(\ep(D))$ of [L1].  It depends only on the stable
conjugacy class of $\c_H$, and the $G(\AA)$-conjugacy class
of $\c_G$.   We insert the superscript $(v)$ on local factors.
They prove the product formula:

  $$\Delta^{(v)}(\c_H,\c_{G,v}) =1 \hbox{ for almost all } v$$

  $$\prod_v\Delta^{(v)}(\c_H,\c_{G,v})=\Delta_\AA(\c_H,\c_G).$$



\sect 3.	Shalika Germs 
\endsect

As above let $G$ be a reductive group over a $p$-adic field of
characteristic zero, and let $T$ be a Cartan subgroup over $F$.
For every unipotent orbit $O$ in $G(F)$ fix an invariant measure $\mu_O$
on $O$.
Shalika [Sh] has shown that there
exist functions $\Gamma_O^T(\c)$ called germs defined on the regular
elements of $T(F)$ for all unipotent classes $O$ in $G(F)$ such that
for every $f \in C_c^\infty(G)$ there is a neighborhood $V_f$ of the
identity element in $T(F)$ in which the expansion
$$ \int_{T(F)\backslash G(F)} f(g\i\c g) |\omega_{T\backslash G}|
= \sum_O
\mu_O(f)\Gamma_O^T(\c)$$
holds for $\c$ regular in $V_f$.

Similarly for $\kappa$ and stable orbital integrals there are
expansions, called Shalika germ expansions,
 in sufficiently small neighborhoods of the identity:
$$ \Delta^*_G(\c) \KOI \c f = \sum_O \mu_O(f) \Gamma_O^{(T,\k)}(\c)$$ 
$$ D_H(\c) \SOI \c {f^H} = \sum_{O_H} \mu_{O_H}(f^H)
\Gamma_{O_H}^{(T,st)}(\c)$$
where $O_H$ runs over unipotent classes of $H(F)$.  The most difficult
part of theorem 1.1 is to obtain an expression for the Shalika germs
$\C_O^{(T,\k)},\C_{O_H}^{(T,st)}$.

\sect 4. Descent \endsect

In this section $G$ is any connected reductive group over $F$
and $H$ is an endoscopic group of $G$.  Orbital integrals
display complicated behavior near the identity element.  More
generally, near any fixed element $\c_0$, an orbital integral
behaves as an appropriate chosen orbital integral on the
centralizer $C_G(\c_0)$.  This section gives a precise form
to this assertion.  This section also indicates how Langlands
and Shelstad have used this descent property to reduce the
proof of theorem 1.1 to statements about Shalika germs near
the identity element. 


We recall of Rogawski's form of
Harish-Chandra descent 
[R1].  Let $M$ be a connected
reductive subgroup over $F$ 
of $G$ containing a Cartan subgroup of $G$.  Let
$\c_0 \in M(F)$ be such that $C_G(\c_0) \subseteq M$.  Let
$T_1,\ldots,T_j$ be the Cartan subgroups over $F$ of $M$ containing
$\c_0$ up to conjugacy by $M(F)$.
Let $\Omega$ be a compact open set in $G(F)$.  Then
there exists [R1 lemma1] a compact open set $K$ of $G(F)$ depending on $M$ and
$\Omega$ and there exist neighborhoods $V_i$ of $\c_0$ in $T_i(F)$
such that $\{ g \in G(F)| g\i V_ig \cap \Omega~\ne~\emptyset\} \subseteq M(F)K$
for $i= 1,\ldots,j$.  Fix invariant forms $\omega_G$, $\omega_M$,
$\omega_{T_i}$ on $G$, $M$, and $T_i$.  We obtain compatible forms
$\omega_{T_i\backslash G}$, $\omega_{T_i\backslash M}$, $\omega_G$
 and $\omega_M$ and corresponding measures 
 on $T_i(F)\backslash G(F)$, $T_i(F)\backslash M(F)$,
 $G(F)$, $M(F)$.  Further choosing $\a \in C_c^\infty(G)$ so that
$$\int_{M(F)} \a(mg) |\omega_M| = 
\cases {1& $g\in M(F)K$ \cr  0& otherwise,
\cr}$$ then for $f \in C_c^\infty(\Omega)$ and $\c_i \in V_i \cap
T_i^o(F)$ ($T_i^o(F)$ = the regular elements of $T_i(F)$) 
we have $$\int_{T_i(F)\backslash G(F)}
f(g\i \c_i g )|\omega_{T_i\backslash G}|
= \int_{T_i(F)\backslash M(F)} \l(f) (m\i \c_i m) 
|\omega_{T_i\backslash M}|,$$ $i=
1,\ldots,j$, where $\l : C_c^\infty(\Omega) \to C_c^\infty(M)$ is the linear
functional $$\l(f)(m) = \int_{G(F)} \a(g)f(g\i m g) |\omega_G|.$$  We note
that $\l$ depends on $\Omega,M,\c_0,K,\a$.

\bigskip
The rest of this section mentions a few results of [LS3].  Two points 
encumber the notation of this section.  First, for an arbitrary reductive
group $G$, the 
matching of orbital integrals is not between $G$ and an
endoscopic group $H$, but between $G$ and $\tilde H$ a 
central extension of $H$.  The transfer factor is then a function
$\Delta({\tilde\c}_H,\c_G)$ pairing certain elements of $\tilde H(F)$
and $G(F)$.  Near the identity element $\Delta({\tilde\c}_H,\c_G)$
descends to a factor $\Delta_{loc}(\c_H,\c_G)$ on $H$ and $G$.
The functions $f^{\tilde H}$ in $C_c^\infty(\tilde H)$ are to be
functions $C_c^\infty(\tilde H,\tilde \l)$ transforming by
a specified character $\tilde\l$ on the center of $\tilde H(F)$.  Second
we must use the relative factor $\Delta({\tilde\c}_H,\c_G;{\tilde{\bar
\c}}_H,\bar\c_G) = \displaystyle {\Delta({\tilde\c}_H,\c_G) \over \displaystyle\Delta({\tilde{\bar
\c}}_H,\bar\c_G)}$  mentioned at the beginning of section 2.

Start with a (possibly singular) semisimple element $\ep_H$ in $H(F)$
an {\it image\/} of $\ep_G$ in $G(F)$.  (For the general definition
of image see [LS3].  For $H=U_E(2)\times U_E(1)$, we say that
$\ep_H$ is an image of $\ep_G$ if it is conjugate to $\iota(\ep_H)$
or $\iota(\ep_H)^{g_0}$.)  The elements $\ep_H$ and $\ep_G$ have
connected centralizers $H_{\ep_H}$ and $G_{\ep_G}$ in $H$ and $G$.
We assume that $\ep_H$ is such that $H_{\ep_H}$ is quasisplit.
From the endoscopic data $(H,{\cal H},s,\xi)$ for $G$ they
define endoscopic data $(H_{\ep_H},{\cal H}_{\ep},s,\xi_\ep)$
for $G_{\ep_G}$.  Associated to this endoscopic data is a
transfer factor denoted $\Delta_\ep$.

We say that $(G,H)$ admits {\it local\/} $\Delta$-transfer at
the identity if for any $f\in C_c^\infty(G)$ there exists $f^H\in
C_c^\infty(H)$ such that $$\Phi^{st}(\c,f^H) = \sum_{\c_G}\Delta_{loc}
(\c_H,\c_G)\Phi(\c_G,f)$$ for strongly $G$-regular $\c_H$ near
1 in $H(F)$.

We say that $(G,H)$ admits $\Delta$-transfer if for any $f\in
C_c^\infty(G)$ there exists $f^{\tilde H}\in C_c^\infty(\tilde H,
\tilde \l)$ such that $$\Phi^{st}({\tilde\c}_H,f^{\tilde H}) =
\sum_{\c_G} \Delta({\tilde\c}_H,\c_G)\Phi(\c_G,f)\eqno(*)$$
for strongly $G$-regular ${\tilde\c}_H$ in $\tilde H(F)$.

\Remark  The sum runs over conjugacy classes of strongly regular
elements in $G(F)$.  All but finitely many terms are zero.  This
sum effectively replaces an integral over $T(F)\backslash G(F)$
by an integral over $(T\backslash G)(F)$.  
For $G=U_E(3)$ the right and left hand
sides of (*) coincide, up to the factor $D_H(\c)$, with the
right and left hand sides of the equation of theorem 1.1.

The main result of [LS3] is
\proclaim {Theorem 4.1}.  Suppose that all pairs $(G_{\ep_G},H_{\ep_H})$
admit local $\Delta_\ep$-transfer at the identity.  Then
$(G,H)$ admits $\Delta$-transfer.

Several remarks are in order.

1.  Three simplifications occur in local $\Delta$-transfer.
First $\Delta_{loc}$ is a function on $H(F)$ and $G(F)$ whereas
$\Delta$ in general is a function on the central extension $\tilde H$ and $G$.
Second the term $\Delta_{III_2}$, the term of the greatest complexity
in the definition of the transfer factor, is identically 1 near
the identity.  Third  the Shalika germ expansion holds near the
identity.  Thus theorem 4.1 allows us to use
the simplified transfer factor $\Delta_{loc}$, ignore
the central extension $\tilde H$ and to match terms of the
Shalika germ expansion term by term in proving theorem 1.1.
This is carried out in sections 6-11.

2.  The proof of the theorem 4.1 is elementary once it
is known that the transfer factors satisfy
$$\lim_{\displaystyle {{\tilde\c}_H,{\tilde{\bar\c}}_H\to{\tilde\ep}_H} \atop \displaystyle{\c_G,
\bar\c_G\to\ep_G} } {\Delta(\c_H,\c_G;\bar\c_H,\bar\c_G) \over
\Delta_\ep({\tilde\c}_H,\c_G;{\tilde{\bar\c}}_H,\c_G)} = 1\eqno(**)$$
However, the proof of (**) is extremely technical and occupies
the bulk of [LS3].

3. In the case of $G=U_E(3)$, 
one can check that if $G_{\ep_G}$ is
not conjugate to a subgroup of $H'$ or $H''$ and if $\ep_G$ is
conjugate to an element of $H'(F)$ or $H''(F)$, then $\ep_G$ is
central in $G$.
In the  situation where $G_{\ep_G}$ is conjugate to a subgroup of
$H'$ or $H''$, then $H_{\ep_H}$ is the quasisplit inner form of $G_{\ep_G}$.  Further, $G_{\ep_G}$ if not a Cartan subgroup must have a form
of $SL(2)$ as  derived group.  This situation, that of SL(2),
has been extensively studied in [LS1].  There it is shown that
local $\Delta_\ep$-matching holds for such $(G_{\ep_G},H_{\ep_H})$.

Thus in the case of $G=U_E(3)$ the only remaining case to be
studied is that of $\ep_G$ central in $G$, so that we may assume
$G_{\ep_G}=G,H_{\ep_H}=H$.  Thus theorem 4.1 becomes :  suppose
that $(G,H)$ admits {\it local\/} $\Delta$-transfer at the identity,
then $(G,H)$ has $\Delta$-transfer.  We can restate theorem 1.1
as 
\proclaim {Theorem 4.2}.  $(G,H)$ admits local $\Delta$-transfer
at the identity.

4.  Note that lemma 2.2 is a special case of (**), for if
$\ep_G = \iota(\c)$ with $\c=\ep_H$, $\c_1=\c_3$, then  $G_{\ep_G}=H'_{\ep_H}$,
$\Delta_\ep({\tilde\c}_H,\c_G,{\tilde{\bar\c}}_H,\c_G)$ is seen to be $1$, and by 2.2 (near the identity)
$$\lim_{\displaystyle {\tilde\c_H,\tilde{\bar\c}_H \to \tilde\ep_H} \atop \displaystyle{\c_G,\bar\c_G \to \ep_G} } \Delta(\c_H,
\c_G;\bar\c_H,\bar\c_G)=1.$$  Thus (**) holds in this special case.



\sect 5.   Unipotent Classes	
\endsect
It is known that the germ $\C_1^T$ associated to the identity
element $1 \in G(F)$ is a constant  independent of the elliptic Cartan
subgroup $T$ and is zero if $T$ is not elliptic [Ho,HC,R2].  
A simple general argument using this fact shows that $\Gamma_1^{(T,\k)}=0$ when $\k$ is non-trivial.  The argument is given in [LS3].  We turn to the other unipotent classes.

Over $\Fbar$ the group $G(\Fbar)$ has 3 unipotent conjugacy classes
corresponding to the various partitions of 3 : $3, \ 21, \ 1^3$.
These are the regular unipotent class, subregular unipotent class,
and the identity $\{1\}$.  Over $F$, the $\Fbar$ regular class
remains a single unipotent class but the subregular
class breaks into 2 classes.  This calculation can be easily
seen from conjugating the regular and subregular elements to  $$\pmatrix
{1&1&-\frac {1} 2  \cr &1&-1 \cr &&1 }\hbox { and } \pmatrix {1&0&ux \cr
&1&0 \cr &&1 } $$ where $u \in F^\times$ and $x \in E^\times$, $\bar x =
-x$.
Conjugation by an element $diag(a,b,c) \in U_E(3,F)$
only changes $u \in F$ by a norm from $E$.

Fix an invariant form $\omega_{sub}$ on the stable subregular class.
Then set $\mu_{sub}^+ = |\omega_{sub}|_{O_o} + |\omega_{sub}|_{O_1}$
and $\mu_{sub}^- = |\omega_{sub}|_{O_o} - |\omega_{sub}|_{O_1}$ where
$O_o$ and $O_1$ are the two subregular classes over $F$.
We shall write the unipotent measures as $\mu_{reg}$,
$\mu_{sub}^{\pm}$, and $\mu_1$ for the regular, subregular and identity
conjugacy classes.  We now write the germ expansion as 
$$\Delta^*_G(\c)\KOI \c f = \Gamma^{(T,\k)}_{reg}(\c)\mu_{reg}(f) + \sum_\pm
\Gamma^{(T,\k)}_{sub,\pm}(\c) \mu_{sub}^{\pm}(f).$$  
Write $O_{reg}$ for the regular class, and  $O_{sub}$ for the union of $O_0$
and $O_1$ in $G$.
The derived group of $H$ is isomorphic to $SL(2)$ so that over $\Fbar$
$H$ has conjugacy classes corresponding to the partitions $2,1^2$ of $2$.
These are the regular and subregular classes.  The $F$-points of these
classes will be denoted $O_{reg}^H$ and $O_{sub}^H$.  We also fix
invariant forms on these classes and associate invariant measures
$\mu_{reg}^{st}(f^H)$, $\mu_{sub}^{st}(f^H)$, $f^H\in C_c^\infty(H)$.
  The Shalika germ expansion on $H$ takes the form
$$D_H\Phi^{(T,st)}=\C_{reg}^{(T,st)}(\c)\mu_{reg}^{st}(f^H) +
\C_{sub}^{(T,st)}(\c)\mu_{sub}^{st}(f^H)$$
We will prove theorem 4.2 by showing that with appropriate
normalizations of measures 
$$\eqalign{
\C_{reg}^{(T,\k)}(\c) &= \C_{reg}^{(T,st)}(\c)\cr
\C_{sub,+}^{(T,\k)}(\c) &= \C_{sub}^{(T,st)}(\c)\cr
\C_{sub,-}^{(T,\k)}(\c) &= 0}$$
The functions on the left are associated to $G$.  Those on the right,
although the notation does not indicate it, are germs  on 
$H$.

Once these equalities are demonstrated we select $f\in C_c^\infty(G)$
and select any function $f^H\in C_c^\infty(H)$ such that
$$\eqalign{
\mu_{reg}(f) &= \mu_{reg}^{st}(f^H)\cr
\mu_{sub}^+(f) &= \mu_{sub}^{st}(f^H)}$$
Then $D_H\SOI \c {f^H} = \Delta_G^*\KOI \c f$ wherever the germ
expansion is valid.


\sect 6.      A	Reduction 
\endsect

Every $f \in C_c^\infty(G)$ may be written as $f = f_1 + f_2,\quad f_i \in
C_c^\infty(G)$, where $\mu_{reg}(f_1)=0$, $\mu_{sub}^\pm(f_2)=0$.
This allows us to break the proof of theorem 4.2 into two cases.  Section 9 proves theorem 4.2 for functions such that $\mu_{sub}^+(f)=\mu_{sub}^-(f)=0$.  This section gives a reduction for functions such that $\mu_{reg}(f)=0$.  
\proclaim {Lemma 6.1}.
Suppose that for each $(T,\k)$ associated to $H$
the germs $\C_{sub,\pm}^{(T,\k)}(\c)$ have the form $$
\C_{sub,\pm}^{(T,\k)}(\exp (\l^2 X)) = |\l|^2 m^{(T,\k)}_{\pm}(X) $$ where $m^{(T,\k)}_\pm$
satisfies 
$$m^{(T,\k)}_{\pm}(X+Z) = m^{(T,\k)}_{\pm}(X)$$ for $|\l| \le 1 $, for $X, \ X+Z $ sufficiently
small regular elements of $Lie \ H'$ and $Z $ belonging to the Lie
algebra of the center of $H'(F) \subseteq G(F)$.  Then theorem 4.2 holds for all
functions $f$ such that $\mu_{reg}(f) =0$.  

\proof  By [HC] there exists a neighborhood $V$ of the identity $0$ in $Lie\ 
G$ such that for all regular $X \in V \cap Lie\, G$ 
$$\C_{sub,\pm}^{(T,\k)}(\exp (\l^2X)) =
|\l|^2 \C_{sub,\pm}^{(T,\k)}(\exp (X))$$ for $|\l| \le 1$, $\l \in F^\times$.  Select
$z' = \exp(Z) \ne 1$ small enough in $Z(H'(F)) \setminus Z(G(F))$ so that there exists
a neighborhood $V_1$ of $0$ in $Lie \ G$ satisfying  $V_1, Z+V_1
\subseteq V$.  Then if $X$ is regular and lies in $V_1 \cap Lie\, H'$
$$\C_{sub,\pm}^{(T,\k)}(\exp (X)) = m^{(T,\k)}_{\pm}(X) = 
m^{(T,\k)}_{\pm}(X+Z) =
\C_{sub,\pm}^{(T,\k)}(z' \exp(X))$$ or replacing $X$ by $\l X$, $\C_{sub,\pm}^{(T,\k)}(\exp
(\l X)) = \C_{sub,\pm}^{(T,\k)}(z'\exp(\l X))$.   Selecting $f$ as in the
hypothesis of the lemma, $$\Delta^*_G(\exp(\l X))\KOI {\exp (\l X)} f =
\Delta^*_G(z' \exp(\l X))\KOI {z'\exp(\l X)} f.$$  Now $C_G(z') = H'$ so by
the results of section 4,
we see that this
integral is equal to a stable orbital integral $D_H (z'\exp(\l
X)) \SOI {z'\exp(\l X)} {f_1}$ (centered at $z_1$) for some $f_1
\in C_c^\infty(H)$.  Replacing $f_1$ by $f^H$ defined by
$f^H(x) = f_1(z_1 x)$ and using the fact that $D_H(z'\exp(\l X))
= D_H(\exp(\l X))$  we see that $f^H \in C_c^\infty(H)$ and that the
original $\k$-orbital integral is equal to $D_H(\c)\SOI \c
{f^H}$.  

\sect 7.	Igusa Theory
\endsect

The idea behind the explicit construction of germs is the following.  Fix a curve
$\Gamma(F) = \{\exp(\l X)\} \subseteq T(F)$.  One constructs a
variety $Y_1$ whose $F$-points form a compactification of $\C(F)
\times (T\backslash G)(F)$.  The orbital integral $\Delta^*_G(\c)
\KOI \c f$ is
replaced by an integral over the image of $\{ g\i exp(\l X)g | g \in
(T\backslash G)(F)\}$ in the variety $Y_1$.  For $\l \ne 0$ this
image is the set of $F$-points of a smooth hypersurface $C_\l$ of
$Y_1$.  For $\l = 0$, $C_0$ becomes a divisor
with normal crossings.  As $\l$ tends to zero $C_\l$ tends to $C_0$
and the integral over $C_\l$ tends to an integral over $C_0$. In
fact one obtains an expansion for $\Delta^*_G(\c)\KOI \c f$ as a sum of integrals
taken over the irreducible components of $C_0$ and their
intersections.  One then observes that this expansion coincides
with the Shalika germ expansion, thus obtaining an explicit integral
formula for the Shalika germs.  From this, the property of section 5
is easily verified.

In the next section we shall give local coordinate charts for a
variety $Y_1$ compactifying $\C \times T\backslash G$.  It is
verified in [L2] that this variety satisfies all of the technical
conditions of the theory needed to guarantee that the orbital
integral $\Delta^*_G(\c)\KOI \c f$ has an expansion in terms of integrals over the irreducible
components of $C_0$ and their intersections.  We shall take this
fact for granted.

To simplify matters we make one minor modification of the theorem of
Igusa giving  an expansion as set forth in [L2].  There it is required that any
irreducible component of $C_0$ with an $F$-rational point be defined
over $F$.  We shall   study the situation in which there
are two divisors whose intersection is defined over $F$ which are
not themselves defined over $F$.

Let $F$ be a p-adic field of characteristic zero.
Let $K/F$ be a degree $r$ cyclic extension with generator $\s$ of
the Galois group.  Let $NK$ denote the norms of elements of $K$ in
$F$.  Let $\eta_{K/F}:F^\times/NK^\times \to \CC^\times$ be a 
generator of the characters 
on this quotient.  
Let $|\ |$ be the absolute value on $F$
normalized by the Haar measure in the usual manner or a compatible
extension to $K$.  Let $$U = \{(\mu_1,\ldots,\mu_n)|\quad |\mu_i| \le
\epsilon_i,  \quad\mu = \mu_1 = \s(\mu_2) = \cdots = \s^{r-1}(\mu_r) \} \subseteq K^r \times F^{n-r}$$ where $
\epsilon_1 = \cdots = \epsilon_r$.  
Let $\lambda = \a_0\prod_1^n \mu_i^{a_i}$ where
$1=a_1 = \cdots = a_r$, $\a_0,\ \l \in F^\times$ and assume $|\a_0|$ and
$\eta_{K/F}(\a_0)$ are constant on
$U$.  Let $\omega = \prod_1^n
\mu_i^{b_i} \frac{d\mu_1}{\mu_1} \wedge \ldots \wedge
\frac{d\mu_n}{\mu_n}$ where $b = b_1 = \cdots = b_r$.  
Suppose that $b_i \ne
a_ib$ for $i > r$.  Let $f = \k_{r+1}(\mu_{r+1})\cdots \k_n(\mu_n)$ for
some unitary characters $\k_{r+1},\ldots,\k_n$ of $F^{\times}$. 
Then $f$ is a
function on $U$.  Write $N\mu = \mu_1\ldots\mu_r$. 
\proclaim {Lemma 7.1}.
In the above setting, $F(\l)=$
$$ \int_{U,\l = \a_0\mu_1\cdots\mu_n} f |\frac{\omega}{d\l}| =
\frac 1 r \sum_{j=0}^{r-1}|\l|^{b-1}\delta_{K/F}\eta^j_{K/F}(\l)F(j) + \sum_{\b
\ne b,\theta,k} \theta(\l)|\l|^{\b-1}m(\l)^kF_k(\theta,\b,f)$$
for $\l$ sufficiently small where 
$$\eqalign {\delta_{K/F} &= 
\int_{N\mu = 1} \frac{d\s(\mu)d\s^2(\mu)\cdots d\s^{r-1}(\mu)}
{|\s(\mu)\s^2(\mu)\cdots \s^{r-1}(\mu)|}\cr
F(j) &= \int_{U_0} \frac{|\omega_0|f}{\eta^j_{K/F}(\a_0\mu^{a_{r+1}}_{r+1}\cdots \mu_n^{a_n}) } \cr
U_0 &= \{(\mu_1,\ldots,\mu_n) \in U | \mu_1 = \cdots \mu_r = 0\}
\cr
\omega_0 &= \frac{\omega}{ \frac{d\mu_1\cdots d\mu_n}{\mu_1
\cdots \mu_n }|\l|^{b} } }$$  and where $F_r(\theta,\b,f)$ are
appropriate constants.


\proof  Denote the expansion on the right hand side in the lemma by $G(\l)$.  By the theory
of Mellin transforms it is enough to check that $$F_s = \int_{|\l|\le
\epsilon} F(\l)\theta\i(\l)|\l|^s d\l \quad 
\hbox{ and } \quad G_s = \int_{|\l|\le
\epsilon} G(\l)\theta\i(\l)|\l|^s d\l$$ have the same principal
parts at every complex $s$ for every unitary character $\theta$ of
$F^{\times}$.  We shall only treat the terms
$|\l|^{b-1}\eta^j_{K/F}(\l)$ as the form of the other terms is well
understood [L2,I].  The terms $|\lambda|^{b-1}\eta_{K/F}^j(\lambda)$ are
determined by the principal parts of $F_s$ and $G_s$ at $s=-b$.  Thus 
we shall only consider the principal parts
at $s = -b$.  Write $x \sim y$ if $x$ and $y$ are meromorphic
functions of $s$ having the same principal part at $s=-b$. 
It is easy to see that if $\theta(N\mu) \ne 1$ then both
Mellin transforms are analytic at $s=-b$.  So we take $\theta =
\eta_{K/F}^j$ for some $j$.  We are led to compare the two Mellin
transforms: 
\def\lamep{\int_{|\l|\le\epsilon}}
\def\la{\int_{|\l|\le\epsilon,U}}
$$\eqalign{F_s &\sim \la |\l|^s\eta_{K/F}^{-j}(\l)f|\omega| \cr
&= 
\la |\frac{(N\mu)^{s+b} d\mu_1
\ldots d\mu_r }{N\mu}| \ \times \cr &\cdot\int |\a_0|^s
\eta_{K/F}^{-j}(\a_0)\eta_{K/F}^{-ja_{r+1}}\k_{r+1}(\mu_{r+1})
|\mu_{r+1}|^{a_{r+1}s+b_{r+1}}
\frac{d\mu_{r+1}}{|\mu_{r+1}|} \cdots 
\eta_{K/F}^{-ja_n}\k_n(\mu_n)|\mu_n|^{a_ns+b_n} \frac {d\mu_n}{|\mu_n|}}$$ 
(which has
the same principal part without the constraint $|\l|\le\epsilon$)
and

$$G_s \sim \frac 1 r  \lamep |\l|^s\eta^{-j}(\l)|\l|^b
\frac{d\l}{|\l|}\eta^j(\l)\delta_{K/F}F(j) = \frac 1 r\lamep
|\l|^{s+b-1}d\l \delta_{K/F}F(j).$$  (The other terms of this Mellin
trasform are $\sim 0$).

Since $a_{r+i}s+b_{r+i} \ne 0$ at $s=-b$ the terms $\int
\eta_{K/F}^{-ja_{r+i}}\k_{r+i}(\mu_{r+i})|\mu_{r+i}|^{a_{r+i}s+b_{r+i}}
\frac{d\mu_{r+i}}{|\mu_{r+i}|}$ are regular at $s=-b$ and equal the
principal-value integral obtained by substituting $-b$ for $s$
($i>0$).  In
fact these analytic continuations serve as a definition of these
principal-value integrals.
The product of these integrals with $|\a_0|^s\eta_{K/F}^{-j}(\a_0)$ at
$s=-b$ is $F(j) = \int_{U_0} \eta^{-j}(\l)f|\omega_0|$.
Thus  $F_s \sim  F(j)\int_{|N\mu|\le\epsilon} |N\mu|^{s+b-1} d\mu_1
\ldots d\mu_r $.   Write $\l_0 = \mu_1\ldots\mu_r$ and substitute
for $\mu_1$ to obtain $$F(j)\int_{|\l_0|\le\epsilon} |N\mu|^{s+b-1} d\mu_1 \ldots d\mu_n
\sim F(j)\delta_{K/F}\int_{\l_0\in NK,\ |\l_0|\le\epsilon}|\l_0|^{s+b-1} d\l_0
$$   Now $$\int_{\l \in NK,|\l|\le\ep}|\l|^{s+b-1}d\l 
= \frac{1}{r} \int_{|\l|\le\ep} \sum_0^{r-1}\eta_{K/F}^j(\l)|\l|^{s+b-1}{d\l} 
\sim \frac{1}{r}
\lamep |\l|^{s+b-1} {d\l}.$$  
Making this substitution we obtain $F_s \sim G_s$.








\sect 8.	Langlands' variety of stars
\endsect
	Langlands constructs the variety $Y_1$ and gives explicit coordinate
	patches and coordinates on the variety.  We change notation from
	that used in [L2].  The roots  $\a$ and $\b$ of this paper are
         denoted
	$\a '$
	and $\a ''$ in [L2].  We introduce coordinates $\l$, $\xa$, $
	\xb$, $\xc$, $\z \a$, $\z \b$, $\w \c$ which are related to the
	coordinates of [L2] by $\xa = u$, $\xb = v$, $\xc = w$, $\z
	\a = x/b(\l)$, $\z \b = y/c(\l)$, $\w \c = \frac U {b(\l)c(\l)V} =
	\frac {d(\l)-Vc(\l)}{b(\l)c(\l)V}$.  Here $b = \frac
	{1-\a\i(s(\l))}{\l}, c = \frac{1-\b\i(s(\l))}{\l}, d =
	\frac{1-\a\i\b\i(s(\l))}{\l}$.  We set $b(0) = \aX$, $c(0) =
	\bX$, $d(0) = \cX$ so that $\aX + \bX = \cX$.  These are the positive roots of the Lie algebra evaluated on the tangent direction $X$.  In the following sections $\c$ always denotes the Lie algebra root $\a+\b$.  The coordinate $V$
	is defined by $V= \dfrac{-z_1''}{z_3''}$, where $\displaystyle{-\zquot}$ is a
	rational function to be defined and discussed in section 8.  Our definitions then imply
	that $$ 1 + \zquot = \frac {-\aX} \bX  \frac {(1-\bX \w \c)}{(1-\aX \w
	\c )}\eqno({8.1})$$
	whenever $\l = 0$ and both sides are defined.

	With this notation the coordinate relations (4.1)-(4.3) of [L2]
	become
	$$\l = \z \a \xa\eqno({8.2})$$
	$$\l = \z \b \xb$$
	$$\l \w \c = \xc \z \a \z \b$$

	Having made this conversion to new notation, the reader is free to
	forget the previous notation.  Only (7.1) and (7.2) need be
	retained.  We will often abbreviate $\w \c$ as $w$.

	We note that when $\xc \neq 0$ , then $\z \a = \frac {\xb \w
	\c }{\xc}$, $\z \b = \frac {\xa \w \c}{\xc}$, and $\l =
	\frac {\xa \xb \w \c}{\xc}$.  
Then $\xc \neq 0$, $\l = 0$ defines three  irreducible divisors $\Ea$, $\Eb$, and
$\Eo$ with normal crossings defined respectively by $\xa =0$, $\xb =
0$, $\w \c = 0$.   In other words $C_0$ breaks into three
irreducible components on this patch.  These divisors are called $E_1''$, $E_1'$, and $E_4$ in
[L2].   

Actually the construction of [L2] leads to nine divisors not just
these three.   Fortunately, none of the other divisors contribute to the
Shalika germ expansion of a $\kappa$-orbital integral.   Two of
those 
divisors $E_3$ and $E_5$ have support lying over the identity element.  We
have shown above that the Shalika germ associated to the identity element
is zero.  Hence these divisors do not contribute to the germ expansion.
Three other divisors $E_2^1$, $E_2^2$, $E_2^3$ have support lying over the
subregular unipotent classes.   It is shown in [L2] however that their
contribution to the asymptotic expansion is by terms of the form
$\theta(\l)|\l|^2$, $\th$ a character of finite order of $F^{\mult}$.  Since there are no such terms in our expansion (see proof of lemma 5), they
too make no contribution.  Finally there is a divisor $E_6$.  It is not actually a 
divisor on $Y_1$ but on the variety obtained by blowing up along the
intersection of $\Ea$ and $\Eb$.   Lemma 7.1 allows us to 
work directly with $Y_1$ -not the blown up variety.  Thus the divisor
$E_6$ is not present in our construction.

We will  confine our remarks to a single patch.  The complement of
the union of patches isomorphic to this one lies in a union of
irreducible divisors excluded in the previous paragraph, and we
leave it as an exercise to see that the constructions carried out on
this patch extend to the isomorphic patches.

\sect 9. Regular Classes 
\endsect

This section gives an explicit formula for the
function $\kappa(\cog)$ in terms of the transfer factor and the
coordinates on $Y_1$.  This formula is then used to show that theorem 4.2 holds for functions satisfying $\mu_{sub}^+(f)=\mu_{sub}^-(f)=0$.

Let $X$ be the tangent direction of the curve $\Gamma$ at the identity.
Let $\bf{T}$ be the diagonal subgroup of $G$.  Let $\pi_2 : {H'}
\to U_E(1)$
be the projection onto the (2,2) matrix coefficient.  Let $\he$ be the
nontrivial character on $\coh{U_E(1)}$ or the nontrivial character of 
$F^{\mult}$ 
modulo the norms of $E^{\mult}$ - depending on the context.
As in section 2 let 
$$\Delta_0(X) = \lim_{\l\to 0} \left[ {{\Delta(\exp(\l X))} \over
{D_{G/H}(\exp(\l X))}}\right] $$
Let $\Delta_{E_0} = \kappa(\cog)|_{\Eo}$.  We will see that this limit
exists on an open set of $\Eo$ and depends on $E_0$ only through
the tangent
direction $X$.  Define $w' \dedef w/(1+\aX w)$, $w$ as in section 7.

\proclaim {Proposition \ 9.1}. 
$$\Delta_0(X)\Delta_{E_0}(X) = 1$$
and
$$\eqalign{
m_\k=^{def}\Delta_0(X)\k(\cog)|_{\l=0}
 &= \Delta_0(X)\he(\pi_2(\cog))|_{\l=0} \cr
 &= \he((1+\aX w)(1- \bX w)) 
 \cr
 &= \he(\frac {1 - \cX w'}{- w'^2})}$$

The first part of proposition 9.1 implies the matching of regular 
germs.  The equation states that the transfer factor $(\Delta_0(X))$
times the regular germ of a $\k$-orbital integral $(\Delta_{E_0}(X))$
equals the regular germ of a stable orbital integral $(1)$.

The identification of $\Delta_{E_0}$ as the regular germ occurs as 
follows.  The divisor $E_0$ is the only divisor which contributes
to the regular germ.  The integral over $E_0$ is equal to $\Delta_{E_0}(X)$
times the integral of $f$ over $O_{reg}$ : $\mu_{reg}(f)$.

Similarly one considers a variety $Y_1^H$ which gives the germs of $H$.  
There is one divisor which contributes to the regular germ.  The 
analysis proceeds as in the case of $G$ except that with stable
orbital integrals the term $\k(\cog)|_{E_0}$ does not arise.
The regular germ for a stable orbital integral is found to be $1$
(using the normalizations of measures arising in Igusa theory).


The action of $\Gal$ on $Y_1$ depends on $T$.  The action is obtained by twisting an action independent of $T$ by a biregular action of $\C_T\subseteq \Theta = \{1,\omega,\tau,\omega\tau\}$ on $Y_1$.  
We note here for reference the action of $\Theta$  on the
coordinates $w$ and $w'$ at $F$-points of our variety.  These
relations will be proved in section 10. 


\bigskip

%\input table1.tex
% table 1 starts here

\moveright 1.5in\vbox{\offinterlineskip\hrule
\halign{
&\vrule#&\strut\quad\hfil#\hfil\quad\cr
height2pt& \omit && \omit &&\omit && \omit &\cr
&$x$&&$\tau(x)$&&$\tau\omega(x)$&&$\omega(x)$&\cr
height2pt
&\omit&&\omit    &&\omit   && \omit &\cr
\noalign{\hrule}
height2pt
&\omit  &&\omit  &&\omit   && \omit &\cr
&$w'$&&$\dfrac{-w'}{1-\c(X)w'}$&&${-w'}$&&$\dfrac{w'}{1-\c(X)w'}$&\cr
&       &&       &&        &&        &\cr
&$\a(X)$&&$\b(X)$&&$-\a(X)$&&$-\b(X)$&\cr
&$\b(X)$&&$\a(X)$&&$-\b(X)$&&$-\a(X)$&\cr
&$\c(X)$&&$\c(X)$&&$-\c(X)$&&$-\c(X)$&\cr
%&8000 B.C.&&5,000,000&\cr
height2pt
&\omit  &&\omit  && \omit  && \omit & \cr}
\hrule}

\bigskip
\moveright 1.3in\vbox{\offinterlineskip\hrule
\halign{
&\vrule#&\strut\quad\hfil#\hfil\quad\cr
height2pt&\omit &&\omit &&\omit &&\omit & \cr
&$x$&&$\tau(x)$&&$\tau\omega(x)$&&$\omega(x)$&\cr
height2pt
&\omit &&\omit &&\omit &&\omit &\cr
\noalign{\hrule}
&\omit&&\omit &&\omit &&\omit &\cr
&$w$&&$-w$&&$-w$&&$w$&\cr
&$1+\a(X)w$&&$1-\b(X)w$&&$1+\a(X)w$&&$1-\b(X)w$&\cr
&$1-\b(X)w$&&$1+\a(X)w$&&$1-\b(X)w$&&$1+\a(X)w$&\cr
height2pt
&\omit &&\omit && \omit &&\omit &\cr}
\hrule}

% table 1 ends here

\vskip .2 in
We also note that $\aXw = 1/(1-\aX w')$ and $\bXw = \dfrac{1-\cX
w'}{1-\aX w'}$.

\proof  Let $\bf{B}$ be the group of upper triangular matrices.
Select $h $ in  the derived group of $H'$ such that $T^h = \bf{T}$.  
We may assume that $h\in H'_{der}(K)$ where $K$ is the field splitting
$T$ in section 2.
From the definition
of $\kappa$ in section 1 we see that $\kappa(\cog)$ must equal $ \he(\pi_2(\cog))$.  Writing
$g$ in a certain open set of $G$ as $g = htn\nu$ with $t \in T$, $n \in
\bf{N}$ and $\nu$ in the unipotent radical $\bf{N}_{\infty}$ of the Borel subgroup
opposite $\bf{B}$ through $\bf{T}$ we find that $\cog$ defines
the same cohomology class in $H^1(F,T)$ as $\s (h) \s (n) \s (\nu) \nu\i n\i h\i$.
Set $\omega_\s = h\i \s(h)$.  It is an element of the normalizer of
$\bf{T}$ in $H'_{der}$ depending on $\s \in Gal(K/F)$.  The image of $\omega_\s$ in $\Omega$ coincides with the element in $\{1,\omega\}\subseteq \Omega$ obtained by the map $$Gal(K/F) \to \C_T\subseteq\semi\to\Omega$$  The final map is projection on the first factor.  The (2,2) matrix
element is not changed by conjugating by $\s(h)$; thus $\kappa(\cog)
= \he(\pi_2(t_\s))$ where $t_\s = \s (n) \s (\nu) \nu \i n \i \omega_\s
\in T^{\s(h)} = \bf{T}$.  By its form, $t_\s$ is then uniquely
determined by the equation ${\bf{N}}_\infty {\bf{N}} t_\s \ni n\i \omega_\s$.  For those
elements $\s\  (\in \{1,\s_\tau\}\subseteq Gal(K/F))$ mapping by $\omega_\s$ to the trivial element of the Weyl group of $H'$
it follows that $t_\s = \omega_\s$ lies in the derived group of $H'$ and
so $\pi_2(t_\sigma) = 1$.   If, on
the other hand, $\s\  (\in \{\s_\omega,\s_{\tau\omega}\}\subseteq
Gal(K/F))$ maps to a non-trivial element in the Weyl group of $H'$ 
then writing 

$$ n\i = {\pmatrix {1&{-n_\a}&{-n_\c + n_\a n_\b}\cr
			     0&1&{-n_\b}\cr
			     0&0&1\cr  }}    $$

\noindent we see that $t_\s$ is equal to $diag(x(-n_\c +n_\a
n_\b),\dfrac{-n_\c}{-n_\c + n_\a n_\b}, -\dfrac{1}{x n_\c})$ where
$x$ is a constant 	depending on $\omega_\s$.  So 
$\pi(t_\s)\i = (1-\dfrac{n_\a n_\b}{n_\c})$.  

\def\Xa{{X_{-\a}}}
\def\Xb{{X_{-\b}}}
Next we relate $\displaystyle\nq$ to the coordinate $w(\c)$ on
the variety.  To do so we recall the definition of $\displaystyle
\zquot$.
For every element of the Weyl group $\omega$ define $n_\omega \in
\Ni$ by $(\c^g,B_+^{\hat\omega g}) = (b,{\bf B}^{n_\omega})^{\nu'}$ where
$\omega = \hat\omega^h$, $B_+^h = \bf {B}$,  and $g = htn\nu$ in a Zariski open set of $G$.  
Setting $\omega = 1$ we find $\nu = \nu'$.  Further
$B_+^{\hat\omega g} = {\bf B}^{\omega n \nu}$ so that $n_\omega n\i \in
{\bf B}\omega$.  Define elements $z_i',z_i''$ by 

$$n_\omega =
\cases {\exp(z_1'\Xa)&$\omega = \s_\a$\cr
       \exp(z_1''\Xb) n_{\s_\a}& $\omega = \s_\b \s_\a $\cr
       \exp(z_2'\Xa) n_{\s_\b \s_\a}& $\omega = \s_\a \s_\b \s_\a $\cr
       \cdots & $\cdots $\cr
       \exp(z_3''\Xb) n_{\omega'}& $\omega = \s_\b \omega'; \ \omega'
       = \s_\a \s_\b \s_\a \s_\b \s_\a$\cr }
$$
Here $$\exp(X_{-\a}) = \pmatrix {1&&\cr 1&1& \cr &&1 \cr} \hbox{ and }
\exp(X_{-\b}) = \pmatrix {1&&\cr &1& \cr &1&1 \cr}$$

The condition $n_\omega = 1$ for $\omega = (\s_\a \s_\b)^3$ by a
$3\times3$ matrix calculation implies
$z'_1 + z'_2 + z'_3 =0$, $z''_1 +z''_2 +z''_3 = 0$, $z'_1z_3'' =
z_2'z_2''$.  The condition $n_\omega n\i \in \bf {B}_+ \omega$ for
$\omega = \s_\a, \s_\b \s_\a, \s_\b$ implies by a $3\times3$ matrix
calculation $z'_1 = 1/n_\a$, $z''_1
= n_\a/n_\c$, $z''_3 = -1/n_\b$ so that by (8.1) $$1-\frac{n_\a n_\b}{n_\c} =
 1 + \frac{z''_1}{z''_3} = \frac{-\aX}{\bX}\frac{\bXw}{\aXw}$$  

Thus we see that the cocycle $\pi_2(t_\s)$ $(\l=0)$ is given by
$$1,\s_\tau \to 1, \quad\quad \s_\omega,\s_{\tau\omega} \to
\frac {-\b(X)(1+\a(X)w)} {\a(X)(1-\b(X)w)}$$
in $H^1(F,U_E(1))$.   Set $t= (\frac {\b(X)} {1-\b(X)w}) \in
U_E(1,K)$.  Then $\pi_2(t_\s)\s(t)t\i$ is given by
$$1,\s_\omega \to 1, \quad\quad \s_\tau,\s_{\tau\omega}\to
\frac {(1+\a(X)w)(1-\b(X)w)} {\a(X)\b(X)} \in F^\times$$
Thus $\he(\pi_2(t_\s))$ is equal to $$\he(\frac {(1+\a(X)w)
(1-\b(X)w)} {\a(X)\b(X)})$$
Inserting the norm $-w^2$ we find the equivalent expressions
$$\he(\frac {(1+\a(X)w)(1-\b(X)w)} {\a(X)\b(X)(-w^2)}) = \he(\frac
{1-\c(X)w} {-\a(X)\b(X)w'^2})$$
The divisor $E_0$ is defined by $w=0$.  Thus $$\Delta_{E_0}(X) =
\he(\frac {(1+\a(X)w)(1-\b(X)w)} {\a(X)\b(X)})|_{w=0} =
\he(\frac 1 {\a(X)\b(X)})$$  Also $\Delta_0(X) = \he(\a(X)\b(X))$ so that
$\Delta_{E_0}(X)\Delta_0(X) = 1$.  This completes the proof.



\sect 10.  Actions of the  Galois Group
\endsect

We reproduce here for reference the conditions in [L2] on the
coordinates for a point in $Y_1$ to be in $Y_1(F)$.  They are
determined by the requirement that the embedding $\C_0 \times
T\backslash G \to Y_1$ be defined over $F$.  Once the action on
coordinates is determined, the condition for rationality will
be $\s_\rho(x) = \rho(x)$ for $\rho \in \C_T$ and $x$ a coordinate
of $Y_1$.  To calculate these
conditions for rationality it is enough to do so for the generators
$\tau$ and
$\omega$.

Begin with $\s_\tau$, $\C_T = \{1,\s_t\}$.
  The condition for rationality of
$(b,B^{n_\omega})^\nu$ is determined by $b = J{}^t\overline{(b)}\i J$ and
$n_\omega\nu = J{}^t\overline{(n_\omega\nu)}\i J$, where $\bar{(x)}$ denotes
conjugation coefficient-wise.  In other words $\s_\tau$ twists
the ordinary conjugation by the algebraic action $\tau : \ b \to
J{}^tb\i J$, $\tau: n_\omega \nu \to J{}^t(n_\omega\nu)\i J$.
Writing $b = t \pmatrix {1 & \xa &\xc
\cr
&1&\xb \cr
&&1 \cr}$, we see that $\tau(\xa) = -\xb$, $\tau(\xc) =
-\xc(1-\frac{\xa\xb}{\xc})$.  On $E_\a \cup E_\b$, $\xa\xb = 0$ so
that on this set $\tau(\xc) = -\xc$.  From (8.2) it follows that on
$E_\a \cup E_\b$ that $\tau(\wc) = \tau(\frac{\l\xc}{\xa\xb}) =
\frac{-\l\xc}{\xa\xb} = -\wc$.  This action of $\tau$ is independent
of $T$.


When $\s = \s_\omega$ then $\omega = \sa\s_\b\sa$ where $\sa$ (resp.
$\s_\b$) is the simple reflection associated to the root $\a$ (resp.
$\b$).  The action of $\omega$ factors by $\sa\s_\b\sa$.
The action of ${\s_\a}$ on $x(\a)$, $x(\b)$, $x(\c)$ is
determined, according to [L2], by the condition $$\pmatrix
{1&&\cr z&1&\cr &&1\cr}
\pmatrix {t_1&t_1\xa&t_3\xc\cr &t_2&t_2\xb\cr &&t_3\cr}
\pmatrix {1&&\cr -z&1&\cr &&1 \cr}  = 
\pmatrix {t_1'&&\cr &t_2'&\cr &&t_3'\cr}
\pmatrix {1&\xa'&\xc'\cr &1&\xb'\cr &&1\cr}$$
Here $\xa' = {\s_\a}(\xa)$, etc.  By equating coefficients of these
matrices one finds first that $z = \dfrac {\aX \l} {\xa}$ and then
that $\dfrac {\xa'} {\xa} =1$, $\dfrac {\xb'} {\xb} =(1+\a(X)w)$, $\dfrac
{\xc'} {\xc} = 1$ when $\l =0$.
So ${\s_\a}(w) = {\s_\a}(\dfrac {\l x(\c)}{\xa \xb}) =
\dfrac {\l \xc} {\xa \xb (\aXw)} = \dfrac w {\aXw}$ when $\l =0$.

Similarly the calculation of the action of ${\s_\b}$ on
$x(\a)$, $x(\b)$, $\xc$ is identical except that 
$\pmatrix {1&&\cr z&1&\cr &&1 \cr}$ 
is replaced by 
$\pmatrix {1&&\cr &1&\cr &z'&1\cr}$.  Equating coefficients of the $3\times
3$ matrices one obtains on $\l = 0$: $z' = \dfrac {\l\bX}{\xb}$, $\dfrac {\xa'} {\xa} = \aXw$,
$\dfrac {\xb'} {\xb} = 1$, $\dfrac {\xc'}{\xc} = 1$.  Then
similarly, ${\s_\b}(w) = \dfrac w {\bXw}$.   Finally,
$\s_\a\s_\b\s_\a(w) = \s_\a\s_\b(\dfrac {w}{\aXw}) = \s_\a(\dfrac w
{\aXw}) = w$.  This leads immediately to the table of section 9.

\sect 11.   Subregular Classes 
\endsect


By the action of the Galois group of the splitting field of  $T$  calculated
in section 10
$$
\s_\omega[(1 - \beta(X)w)x(\alpha)] = (1 - \beta(X)w)x(\alpha) \in E
$$
\centerline {and}
$$
\sigma_\tau[(1 - \beta(X)w)x(\alpha)] = - (1 - \alpha(X)w)x(\beta).
$$
So
$$
\lambda = {\frac{x(\alpha)x(\beta)w(\gamma)}{x(\gamma)}} = 
- {\frac{[(1-\beta(X)w)x(\alpha)]\sigma_\tau[(1-\beta(X)w)x(\alpha)]w}{(1-\beta(X)w)(1+\alpha(X)w)x(\gamma)}}.
$$
Consequently
$$
\eta_{E/F}(\lambda) = \eta_{E/F}\left( {\frac{-w}{(1-\beta(X)w)(1+\alpha(X)w)x(\gamma)}}\right) .
$$
By the theory of Igusa outlined in section 7 (with  $K = E, r = 2$)
we obtain the formula

${\displaystyle \sum_{\pm}} \Gamma^{(T,\k)}_{sub,\pm} (\gamma) 
\mu^{\pm}_{sub}(f) =$
$$
\frac{\delta_{E/F}}{2} |\lambda| \int_{E_\alpha \cap E_\beta} m_\k f |\omega_0| 
+ \frac{\delta_{E/F}}{2} |\lambda| \eta_{E/F}(\lambda) \int_{E_\alpha\cap E_\beta}
m_\k {\frac{f|\omega_0|}{\eta_{E/F}(\lambda)}} . \eqno(*)
$$

\noindent
Now $|\omega_0| = {\frac{d\omega}{|w|^2}} \mu_{sub}^{+}$,
and  $m_\k$  is the factor of proposition 9.1.
The first term of $(*)$ is
$$
\frac{\delta_{E/F}}{2} |\lambda| \int_{{\bf  P}^1=E_\alpha\cap E_\beta(u)}
\eta_{E/F} \left( {\frac{1-\gamma(X)w'}{-w'^2}}\right) {\frac{dw'}{|w'|^2}}\
\mu^{+}_{sub}(f) .
$$
The second term is
$$
\frac{\delta_{E/F}}{2} |\lambda| \eta_{E/F}(\lambda) \mu^{-}_{sub}(f) \int_{{\bf  P}^1}
\eta_{E/F} \left(\frac{(1+\alpha(X)w)(1-\beta(X)w)}{\lambda x(\c)}\right)
{\frac{dw}{|w|^2}}
$$
$$
= \frac{\delta_{E/F}}{2} |\lambda| \eta_{E/F}(\lambda)\mu_{sub}^-(f) \int_{{\bf  P}^1}
{\frac{\eta_{E/F}(- w)dw}{|w|^2}} = 0.
$$
\noindent Note that $\eta_{E/F}(x(\gamma))$ has been pulled in front of
the integral over $w$.  The factor $\eta_{E/F}(x(\gamma))$ is constant
on each $F$-class and gives rise to the signed measure
$\mu_{sub}^-$.
This integral vanishes because for any non-trivial quasi-character $\theta
: F^\times \to \CC^\times$:
$$ \eqalign{
\int_{{\bf  P}^1} {\frac{\theta(w)dw}{|w|}} & =^{(w\to \alpha w)}
\int_{{\bf  P}^1} \theta (\alpha w) {\frac{d(\alpha w)}{|\alpha w|}} \cr
& = {{\theta(\alpha)}} \int_{{\bf  P}^1} \theta(w) 
{\frac{dw}{|w|}} .  } $$

\noindent So that  $\theta(\alpha)\ne 1$  implies
${\displaystyle\int_{{\bf  P}^1}} \theta(w)  {\frac{dw}{|w|}} = 0$.	
A more rigorous treatment of this integral is given in [LS1].  
Thus we have that the subregular germ is 
$$
\Gamma_{sub,+}^{(T,\k)} (\gamma) \mu^+_{sub}(f) = \frac{\delta_{E/F}}
{2}|\lambda| \int_{\bf P^1}
\eta_{E/F} \left( {\frac{1-\gamma(X)w'}{- w'^2}} \right) 
{\frac{dw'}{|w'|^2}} \mu_{sub}^+ (f).
$$

This clearly depends only on  $X$  through  $\gamma(X)$.  
By lemma 6.1, this completes the
proofs of theorems 4.2 and 1.1.

\vskip .2 in
To check the constants explicitly that are involved in the transfer consider the
an\-iso\-tropic Cartan subgroup $T$ split by a quadratic extension  $E$  with Galois group $\{ 1,\sigma_{\omega\tau}\}$.
Then 
$$
\gamma(X), w' \in E,\quad \sigma_{\omega\tau}(\gamma(X)) = - \gamma(X),\quad  \sigma_{\omega\tau}(w') = -w'.
$$
Then  $- w'^2 \in NE$  and  $u = \gamma(X)w'$  is a variable over 
$F$ so that
$$\eqalign{
\Gamma_+ &= {\frac{|\lambda|}{2}} \delta_{E/F} \int \eta_{E/F}
\left( {\frac{1-\gamma(X)w'}{-w'^2}} \right) {\frac{dw'}{|w'|^2}} = \cr  &\quad \quad
{\frac{|\lambda\gamma(X)|}{2}} \int_{\sigma_{\omega\tau}(x)x = 1} 
{\frac{dx}{|x|}} 
\int_{{\bf  P}^1} \eta_{E/F}(1-u) {\frac{du}{|u|^2}}.
}$$
It is also shown in [LS1] that the integral 
${\displaystyle\int_{{\bf  P}^1}} {\frac{du}{|u|^2}}$ vanishes.  Also  
$ {\frac{\eta_{E/F}(1-u)+1}{2}} = 0$  if and only if  $1-u =\sigma_{\omega\tau}(a_0)a_0 \in NE$  we may write (setting $xa_0=a$)
$$
\Gamma_+ /|\gamma(X)\lambda| = \int_{\sigma_{\omega\tau}(x)x=1} \frac{dx}{|x|}\int 
{\frac{\eta_{E/F}(1-u)+1}{2}} {\frac{du}{|u|^2}} = \int_{\sigma_{\omega\tau}(a)a = 1-u} 
{\frac{da}{|a|}} \int_{{\bf  P}^1} {\frac{du}{|u|^2}} .
$$

\noindent
Set  $b = a/(1-u) \in E$  (so  $u = 1 - a/b$).
$$
\Gamma_+\left/ |\gamma(X)\lambda|\right. = \int_{\sigma_{\omega\tau}(a)=b^{-1}} 
{\frac{da}{|a|}} \int {\frac{d(a/b)}{|1-a/b|^2}} = \int_{Q(F)}
{\frac{da\ db}{|a-b|^2}}
$$
where  $Q(F)$  is the form of  ${\bf  P}^1 \times {\bf  P}^1$  defined by the
rationality condition  $\sigma_{\omega\tau}(a,b) = (b^{-1}, a^{-1})$ on
inhomogenious coordinates.

The integral $\int_{Q(F)} {da db \over |a-b|^2}$ arises in the paper [LS1] in connection with the germs of $SL(2)$.  
In that paper it is proved, using Igusa theory, that $$|\c(X)\lambda|\int_{Q(F)} \frac {dadb} {|a-b|^2}$$ is equal to the germ in $SL(2)$ associated to the subregular conjugacy class and the torus split by $E$.  Now $\c(X)$
denotes the simple positive root of $SL(2)$.
The derived group of the quasi-split group $H'$ is isomorphic to $SL(2)$ so that this is also equal to the germ associated to the subregular class on $H'$.   
(We have used the result, proved by Harish-Chandra [HC], that germs are independent of the center in a suitable sense).  Set
$$u_\ep(x,z) = \pmatrix { 1&x&\ep z - \frac {x\bar x} 2 \cr
&1&-\bar x \cr &&1 \cr}\in U_E(3,F)$$  where $x \in E$, $z\in F$, $\bar \ep = -\ep$.
Also set $\ell_\ep(x,z) = {}^tu_{\ep}(x,z)$.
By comparing the normalization of measures used in [LS1] with the normalization of measures used here, we obtain the following explicit form of the matching of orbital integrals near the identity:
\proclaim {Theorem 11.1}.  Suppose that  $f,f^H$  are chosen so that
$$
\mu_{reg}(f) = \mu^H_{reg}(f),\quad \mu_{sub}^+
(f) = \mu^H_{sub}(f^H) \eqno(1)
$$

$$
\lim_{\gamma\to 1} \Delta^*_G(\c)\Phi_G^{(T,\kappa)} (\gamma ,f) = \mu_{reg} (f) \eqno(2)$$
$$
\lim_{\gamma\to 1} D_H(\c)\Phi_H^{(T,st)}(\gamma ,f^H) 
= \mu^H_{reg}(f^H).
\eqno(3) 
$$

Suppose also that measures are 
normalized by the invariant forms
$$
\eqalign{
\mu_{reg}&:\   dzdxdsdt {\hbox{coordinates}}: \quad
         \ell_\ep(t,s)u_\ep(x,z)\ell_\ep(t,s)\i \cr
\mu^+_{sub}&:\  zdzdtds   {\hbox{coordinates}}:  \quad
          \ell_\ep(t,s)u_\ep(0,z)\ell_\ep(t,s)\i \cr
\mu^H_{reg}&:   dudx   {\hbox {coordinates}}: \quad
{\pmatrix {1 & 0 \cr  x & 1}} 
{\pmatrix {1 & u \cr  0 & 1 }}
{\pmatrix {1 & 0 \cr  -x & 1 }} \cr 
\mu^H_{sub}  (f) &= f(1).  \quad
}$$
Then $$\Delta^*_G(\c)\KOI \c f = D_H(\c)\SOI \c
{f^H}$$ in a sufficiently small neighborhood
of the identity.



\sect 12. Normalization of Measures and the Hecke algebra
\endsect
The results of this section assume that $E/F$ is an unramified
extension.
If $M$ is a reductive group over the integers $O_F$ of $F$ set 
$$[M] = {\#M(\FF_q) \over q^{dim\ M}}.$$  
For instance $$[GL(n)] = ( 1- \frac 1 {q^n})(1 - \frac  1 {q^{n-1}})
\cdots (1-\frac 1 q)$$ and $$[U_E(3)] = (1+\frac 1 {q^3})
(1-\frac 1 {q^2})(1+\frac 1 q)$$  Here $q$ is the cardinality of the residue field $\FF_q$ of $O_F$.

Let $1_G$ denote the normalized characteristic function $1_G = \frac {char\ K} {[G]}$, $K = U_E(3,F) \cap GL(3,O_F)$.  
Similarly put $1_H = \frac {char K'} {[H]}$, 
$K' = H' \cap GL(3,O_F)$.  Normalize $\omega_{T\backslash G}$ 
and $\omega_{T\backslash H}$ by
$$\eqalign{
\lim_{\c\to 1} \Delta_G^*(\c)\KOI \c f &= \mu_{reg}(f)\cr
\lim_{\c\to 1} \Delta_H^*(\c)\SOI \c {f^H} &=\mu_{reg}^{st}(f^H)
}$$
This section proves
\proclaim {Theorem 12.1}.  $$\Delta_G^*\KOI \c {1_G} = D_H\SOI \c {1_H}$$
for $\c$ sufficiently small and $G$-regular.

This is a weak form of the fundamental lemma. 
In the case of $SL(n)$ the assumption that $\c$ is close to the identity can be removed and consequently for $SL(n)$ the matching of germs implies the fundamental lemma for the identity of the Hecke algebra {[H3]}.

This theorem gives a precise statement of how the measures arising in Igusa theory relate to the maximal compact subgroup.  The theorem is actually a statement about germs and is proved by computing the integrals $\mu_{reg}(f),\ \mu_{sub}^\pm,\ldots$.  

As a corollary we obtain the alternate form of theorem 12 which holds true for any normalization of measures $\mu_{reg},\mu_{sub},\ldots$.

\proclaim {Corollary 12.2}.  Suppose that 
$$ {{\mu_{reg}(f)} \over {\mu_{reg}(1_G)}} = {{\mu_{reg}^{st}(f^H)} \over {\mu_{reg}^{st}(1_H)}}$$

$${ {{\mu_{sub}^+(f)} \over {\mu_{sub}^+(1_G)}} = {{\mu_{sub}^{st}(f^H)} \over \mu_{sub}^{st}(1_H)}}$$
{\it Also suppose that the measures of the orbital integrals are normalized so as to satisfy:}
$$ \lim_{\c\to 1} { \Delta_G^*\KOI \c {1_G} \over {\mu_{reg}(1_G)}} =
   \lim_{\c\to 1} {D_H\SOI \c {1_H} \over {\mu_{reg}^{st}(1_H)}} = 1$$
{\it Then $\Delta_G^*\KOI \c f = D_H\SOI \c {f^H}$ in  a sufficiently small neighborhood of $1$.}

\proof (of corollary).  Take germ expansions of both sides in theorem 12.1.

\proof (of theorem).  By Rao [Ra]
for $f\in C_c^\infty(G)$, $f^H \in C_c^\infty(H')$,

$$\eqalign {
f &\to \int_{K,E,F} f(k\i u_\ep(x,z)k)dkdxdz = \mu_{reg}^{Rao}(f)\cr
f &\to \int_{K,F} f(k\i u_\ep(0,z)k)dkdz = \mu_{sub}^{Rao}(f)\cr 
f^H &\to \int_{K',F} f^H(k\i u_\ep(0,z)k)dkdz = \mu_{reg}^{Rao,st}(f^H)\cr 
f^H &\to \int_{K'} f^H(1)dk = \mu_{sub}^{Rao,st}(f^H)\cr 
}$$
are invariant measures on $O_{reg}$
, $O_{sub}$, $O_{reg}^H$, $O_{sub}^H$ respectively and so are equal to $\mu_{reg}$, $\mu_{sub}^+$, $\mu_{reg}^{st}$, $\mu_{sub}^{st}$ resp.  
up to scalars.   
Let $\chi$ be the characteristic function of 
$$\{\ell_\ep(t,s)u_\ep(x,z)\ell_\ep(t,s)\i | t,x \in O_E; s,z \in O_F\}$$
on $G$, and let $\chi'$ be the characteristic function of 
$$\{\ell_\ep(0,s)u_\ep(0,z)\ell_\ep(0,s)\i | z,s \in F \}$$ on $H'$.
If $\chi(k\i u_\ep(x,z)k)\ne 0$ with $x\ne 0$ then $x\in O_E$, $y\in O_F$
and $k = k_bk_\ell$, $k_b \in K \cap {\bf B}$, $k_\ell \in K \cap {\bf N}_\infty$.  Similarly if $\chi'(k'\i u_\ep(0,z)k') \ne 0$ with $z \ne 0$ then
$z \in O_F$ and $k' = k'_b k'_\ell$, with $k_b \in K' \cap {\bf B}$,
$k_\ell \in K' \cap {\bf N_\infty}$.  We use normalizations of measures
that assign volume one to the set of elements in $K,K'$ that are
congruent to the identity modulo a uniformizing element.

Then 
$$\eqalign{
\mu_{reg}(1_G) &= {\mu_{reg}(\chi) \over \mu_{reg}^{Rao}(\chi)}
\mu_{reg}^{Rao}(1_G)\cr
                 &= {\int_{O_F,O_F,O_E,O_E} dsdzdxdt \over
                        \int_{K\cap {\bf B}\cdot K\cap {\bf N}_\infty}
                     dk \int_{O_F,O_E}dzdx }
                       {\int_K dk \int_{O_F,O_E} dzdx \over [G]} \cr
             &= {1 \over (q+1)(q^2-1)q^6}{[G] q^{dim\ G} \over [G]}
              = {1 \over (1 + \frac 1 q)^2(1- \frac 1 q)} =
                {1 \over [{\bf T}]}
}$$

$$\eqalign{
\mu_{sub}(1_G) &= {\mu_{sub}(\chi) \over \mu_{sub}^{Rao}(\chi)} 
\mu_{sub}^{Rao}(1_G) \cr
&= {\int_{O_F} zdz \over \int_{K\cap{\bf B}\cdot K\cap {\bf N}_\infty}
             dk \int_{O_F} zdz} {\int_K dk \int zdz \over [G]} =
                {1 \over (1+\frac 1 q)} \mu_{reg}(1_G)
}$$

$$\eqalign{
\mu_{reg}^{st}(1_H) &= {\mu_{reg}(\chi') \over \mu_{reg}^{Rao}(\chi')}
                        \mu_{reg}^{Rao,st}(1_H) \cr
 &= {\int_{O_F,O_F} dzds \over \int_{K'\cap {\bf B}\cdot K'\cap {\bf N_\infty} } dk \int_{O_F} dz } {\int_{K'}dk \int_{O_F}dz \over [H]} \cr
&= {1 \over (q+1)(q^2-1)q^2} {[H]q^{dim \ H} \over [H]} =
   {1 \over (1 + \frac 1 q )^2 (1 - \frac 1 q)} = {1 \over [{\bf T}]}
}$$

$$\eqalign{
\mu_{sub}^{st}(1_H) &= {\mu_{sub}^{st}(\chi') \over \mu_{sub}^{Rao}(\chi') } \mu_{sub}^{Rao,st}(1_H) \cr
&= {1 \over \int_{K'} dk}{[H]q^{dim \ H} \over [H]} \cr
&= {q^{dim \ H} \over (q^2-1)(q^2+q)(q+1)} = {1 \over (1+\frac 1 q)}
\mu_{reg}^{st}(1_H)
}$$

So $$\Delta_G^*\KOI \c {1_G} = {1 \over [{\bf T}]} + {1 \over
[H]}\C_{{}_+} = D_H\SOI \c {1_H}$$ for $\c$ small.


\sect 13. Other endoscopic groups
\endsect

The endoscopic groups $H$ of $U_E(3)$ are found among the endoscopic
groups $H$ of $SU_E(3)$ (allowing for minor modifications of the
centers of $H$). 

Upon extension of scalars  to $E$ the endoscopic groups of $SU_E(3)$ become
endoscopic groups of $SL(3)/E$.  It is known that these are either a
split Levi factor of $SL(3)$ or a torus split by a cubic cyclic
extension.  Descending back down to $SU_E(3)$ it is not difficult to
check that the endoscopic groups must be one of the following:

\item {1.}  $SU_E(3)$ itself. (This group is quasi-split and has no
inner forms).
\item {2.}  the subgroup $\left\{\pmatrix {*&&*\cr &*&\cr *&&* }\right\}$
\item {3.}  a Cartan subgroup   of $SU(3)$ contained in a 
Borel subgroup over $F$
\item {4.}  a Cartan subgroup of $SU(3)$ split by an $S_3$ extension of
$F$ which becomes cyclic over $E$.
\vskip .2 in

Cases (1) and (3) correspond to  Levi factors of $SL(3)$ over $E$.  The
transfer of (3) is easily dealt with using arguments of [HC].

Case (2), at least for $G=U_E(3)$ has been dealt with fully above.
The argument for $G=SU_E(3)$ is identical, and is easily deduced
from the above.

Case (4).  Note that $U_E(3)$ is isomorphic to $GL(3)$ over $E$.
Consequently this class of endoscopic groups does not arise for
$U_E(3)$ (every endoscopic group of $GL(3)$ is a split Levi factor).
Similarly $G_{adj}$ does not have any endoscopic Cartan subgroups of
this type.

Proving the transfer to a Cartan subgroup is equivalent to showing
that the germs $\C^{(T,\k)}_O(\c) = 0$ vanish when $O$ is not
regular.  For then the orbital integral becomes a locally constant
function of $\c \in T(F)$.

In our situation $G=SU_E(3)$ there is a simple symmetry which forces
$\C^{(T,\k)}_{sub,\pm}=0$ and which gives the result.

\proclaim {Lemma 13.1}.  If $h \in G(\Fbar)$ and $\s(h)h\i \in H^1(F,Z_G)$
then $$\C_{O^h}^{(T,\k)} = \C_O^{(T,\k)}\k(\s(h)h\i)$$

\proof  Compare orbital integrals of $f$ and $f_h$ where $f_h(x) =
f(x^h)$.  See [H2].  

For $g \in G_{adj}(\Fbar)$ let $\hat g \in G(\Fbar)$ denote a lift.
Let $\hat G_{adj}(F)$ denote the inverse image of $G_{adj}(F)$ in
$G(\Fbar)$.   
\def \hG {{\hat G_{adj}(F)}}
Then $\hG = \{h \in G(\Fbar) | \s(h)h\i \in Z_G\}$.   If $O$ is
subregular then $O^h = O$ for $h \in \hG$ as can be seen by
conjugating $\pmatrix {1&0&ux\cr &1&0\cr &&1}$ by $diag(a,b,c)
\in \hG$ as in section 4.  Thus we are reduced to finding $h \in
\hG$ with $\k(\s(h)h\i)\ne 1$.  

There is an isomorphism over $F$:  $(T\backslash G)(F) \simeq
(T_{adj}\backslash G_{adj})(F)$.    Then there is a well defined
function $m_{\k'}$ on $(T_{adj}\backslash G_{adj})(F)$ given by
$m_{\k'}(T_{adj}g) = \k(\s(\hat g)\hat g\i)$.  If $\k(\s(h)h\i) = 1$
for all $h \in \hG$ then $m_{\k'}$ factors through $\k' :
Z^1(T_{adj}) \to \CC$, $m_{\k'}(T_{adj}g) = \k(\s(\hat g)\hat g\i)$.
$\k'$ is trivial on coboundaries so descends to $H^1(F,T_{adj})$.
But then $(\k_{adj},T_{adj})$ defines an endoscopic group of
$G_{adj}$ falling into case (4) above,
contradicting the remarks above.

\vfill
\eject



%\font\ly=lasy10
%\ly
%\NoBlackBoxes \nologo \printoptions \document
\def\kkey#1{\par\noindent  [{\bf {#1}}]}
\def\by{}
\def\paper#1{{\it #1}}
\def\yr{}
\def\vol{}
\def\book{}
\def\pages{}
\def\endref{}
\def\jour{}
\def\publ{}
\def\paperinfo{}

%\newpage
%\Refs
\sect References \endsect
\bigskip

\everypar={\hangafter1\hangindent .5in}


\kkey {H1}  T.C. Hales, {\it  The Subregular Germ of Orbital
Integrals},  thesis, Princeton University, (1986). 

\kkey {H2} T.C. Hales, {\it Shalika Germs on $GSp(4)$}, 
preprint. 

\kkey {H3} T.C. Hales, {\it Unipotent Representations and Unipotent
Orbits on $SL(n)$}, preprint.

\kkey {HC} \by Harish-Chandra, {\it Admissable Invariant
Distributions on Reductive p-adic groups\/},  Queens' Papers in
Pure and Applied Math {\bf 48} \yr (1978), \pages 281-347. \endref

\kkey {Ho} \by R. Howe, {\it The Fourier Transform and Germs of
Characters\/}, Math. Ann. {\bf 208} \yr (1974), \pages 305-322. \endref

\kkey {I} \by J.I. Igusa, ``Lectures on forms of higher degree''
Tata Institute of Fundamental Research, Bombay \yr (1978).
\endref

\kkey {L1} \by R.P. Langlands, \book ``Les debuts d'une formule des
traces stable'', \publ Publications math. de l'Univ. de Paris VII \vol
{\bf 132}, \yr (1983). \endref

\kkey {L2} \by R.P. Langlands, \paper {\it Orbital Integrals on Forms of
$SL(3)$, I}, \jour American Journal of Mathematics, {\bf  105}, (1983)
\pages 465-506. \endref

\kkey {LS1} \by R.P. Langlands and D. Shelstad, \paper {\it On Principal
Values on P-adic Manifolds}, \jour Lie Group Representations II,
Lecture Notes in Math, \vol {\bf 1041} \publ Springer-Verlag, Berlin \yr
(1984). \endref

\kkey {LS2} \by R.P. Langlands and D. Shelstad, \paper {\it On the
Definition of the Transfer Factors}, \jour Math. Ann. \vol {\bf 278}, \pages
219-271, \yr (1987). \endref

\kkey {LS3} \by R.P. Langlands and D. Shelstad, \paper {\it Orbital
Integrals on Forms of $SL(3)$, II}, \paperinfo preprint. \endref

\kkey {LS4} \by R.P. Langlands and D. Shelstad, \paper {\it Descent for
Transfer Factors}, \paperinfo preprint. \endref

\kkey {Ra}  \by R.R. Rao, {\it Orbital integrals in reductive groups}, Ann. of Math. {\bf 96},
(1972).

\kkey {R1} \by J. Rogawski, \paper {\it Applications of the Building to
Orbital Integrals}, \paperinfo thesis, Princeton University \yr (1980).

\kkey {R2} \by J. Rogawski, \paper {\it An Application of the Building to
Orbital Integrals} \jour Compositio Mathematica \vol {\bf 42.3}, \yr (1981),
\pages 417-423. \endref

\kkey {Sh} \by J. Shalika, \paper {\it A theorem on semi-simple p-adic
groups}, \jour Annals of Math, \vol {\bf 95} \yr (1972), \pages 226-242. \endref

\kkey {S} \by D. Shelstad, \paper {\it L-indistinguishability for real groups}, Math. Ann,
{\bf 259}, 385-430, {(1982)}.
\bye
