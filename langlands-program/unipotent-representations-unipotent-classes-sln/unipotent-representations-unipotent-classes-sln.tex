% What:      Paper "Unipotent Representations and Unipotent Classes in SL(n)"
% Author:    Thomas C. Hales
% Where:     to appear in the AJM
% Affil:     IAS School of Mathematics, 1989-90 Member
% Current Affil: University of Chicago
% Typist:    Cindy Ericksen
% Filename:  sln.tex
% TeXType:   AMS without amsppt (Used pre-compiled format)
% Date:      October 19, 1989 (draft produced)
% Revision:  003
% Last Revision: July 23, 1992
% Retypeset March 4, 2015.

% Issues 2015. 
% Labeled equations with tags have been
% removed. References need updating.
% enumerates and itemizes.

% The following changes were made in preparing the manuscript for AJM publishers Dec 92.
% Dec 21, 1992 removed words "but unmistakable" from first page
%              added punctuation to 6.3.
%              inserted subscript {O} to first sum of the proof of Prop 1.4
%              replaced -1 by \sqrt{-1} in 2.5
%              added subscript 1 to first W in 2.5
%              replaced t(s) by t(s,r') in 2.5 and reworded slightly
%              changed reference to 2.1 to "beginning of the section"
%              reworded the definition of t in the proof of 2.6 
%              \xi removed in subscript of K in c4 term in the proof of 3.12
%
%

\documentclass{amsart}

\usepackage{amssymb}
\usepackage{amsthm}

\begin{document}

\title{Unipotent Representations and Unipotent Classes in $\text{SL}(n)$}
\author{Thomas C. Hales}
\address{University of Chicago}
\date{Revised July 23, 1992; Printed December 21, 1992}


%\loadmsbm   
\parindent=8mm 
\font\smc=cmcsc10
%\font\title=cmr17
%\font\name=cmr12
%\font\bigmath=cmmi10 scaled\magstep2    % used on title page
%\magnification=1200 
%\hsize=5.21in       % 6.25 true in 
%\hoffset=.166in     % 0.20 true in 
%\vsize=7.08in       % 8.50 true in 
%\voffset=.166in     % 0.20 true in 
%\parskip=\medskipamount 

%
% Definitions needed when not using amsppt:
% =========================================
%
\def\proclaim #1.{\medbreak\noindent 
  {\smc#1.\enspace}\sl} 
\def\finishproclaim {\par\rm 
  \ifdim\lastskip<\medskipamount\removelastskip 
  \penalty55\medskip\fi} 
\def\pproclaim #1:{\medbreak\noindent 
  {\bf#1.\enspace}\rm}   % : changed to . 7/20/92
\def\finishpproclaim{\par\rm 
  \ifdim\lastskip<\medskipamount\removelastskip 
  \penalty55\medskip\fi} 
\def\vqedbox{%
    \vbox{\hrule height .4pt%
          \hbox{\vrule width .4pt%
                      \hskip .15em%
                      \vbox{\vskip .61em}%
                      \hskip .15em%
                      \vrule width .4pt%
                }
          \hrule height .4pt%
          }}
% This needs to be corrected so that there is no extra space, and it is right justified
% even if it starts a new line.
\def\qed{\hbox{}\nobreak\hfill\hbox{\vqedbox}}
%
%  Definitions specific to this paper: NONE
%  ========================================
%

%\document 
\baselineskip=14pt
%\LimitsOnInts      % affects displayed int's only
%


% This is the beginning of the title page
%\footline={\hfill}
%\null
%\vskip .75in


%\newpage
%\footline={\hss\tenrm\folio\hss}
%\pageno=-1

\begin{abstract}
Each unipotent conjugacy class of $ SL(n,F)$ (where $F$ is a $p$-adic
  field of characteristic zero) defines an invariant distribution on $
  C_c^{\infty} \bigl( SL(n,F) \bigr) $ and, by restriction, a linear
  functional on the Iwahori Hecke algebra.  Each tempered
  representation of $ SL(n,F) $ with an Iwahori fixed vector defines
  an invariant distribution on $ C_c^{\infty} \bigl( SL(n,F) \bigr) $
  and, by a truncation to strongly compact elements, a linear
  functional on the Iwahori Hecke algebra.  We show that the vector
  spaces spanned by these two different classes of functionals are
  equal.  This gives a precise form for a bijection between certain
  combinations of unipotent classes in $ SL(n,F) $ and compact traces
  of representations with Iwahori fixed vectors of $ SL(n,F) $.

As a consequence, we deduce the existence of a uniform Shalika germ
  expansion holding on all strongly compact elements for functions in
  the Iwahori Hecke algebra.  The uniform germ expansion is used to
  make a detailed study of orbital integrals on $ SL(n,F) $.  It is
  shown that the existence of an abstract matching of smooth functions
  $ f \mapsto f^H $ between $ SL(n,F) $ and its endoscopic groups
  implies the {\it fundamental lemma} for the unit element of the
  spherical Hecke algebra.
\end{abstract}

\maketitle

To achieve the stabilization of the trace formula for
  a reductive group [{\bf 18}], there are two related  
  problems that must first be solved.
One is the matching of the
    $ \kappa $-orbital 
  integrals of smooth functions on a reductive group $G$ (over a 
    $p$-adic field of characteristic zero)
  with stable orbital integrals on its endoscopic
  groups $H$.
  Igusa theory clarifies the nature of this first problem
  [{\bf 11}], [{\bf 12}], [{\bf 13}],  [{\bf 19}],
  [{\bf 22}], [{\bf 23}].
The second problem is the {\it fundamental lemma}, which 
  asserts the compatibility of the matching of smooth functions
  with the Hecke algebras on $G$ and $H$.  
  Buildings have been used to study this problem
  [{\bf 25}].
The statements of the problems are closely related,
and the main results of this paper give a method
of reducing one problem to the other.
 If the first problem (matching smooth functions)
  is solved for $SL(n)$, then the second problem (the fundamental lemma
  for the Hecke algebras) follows as a consequence,
  at least for the identity of the algebra.
The solutions of the two problems are linked by making
  a detailed study of the unipotent representations and 
  the unipotent classes of 
    $SL(n)$.
  Representations with Iwahori fixed vectors on
  $SL(n,F)$ are called ``unipotent'' in this paper to emphasize
  the close relationship between unipotent conjugacy classes and
  these representations.

Not long after this paper was written,
Waldspurger refined, reformulated and greatly extended the results of this paper. 
This paper made a modest contribution to this
subsequent work of his.
He eliminated the hypothesis that functions on $SL(n)$ have matching smooth
functions, by making clever use of a strengthened version of Kazhdan's lemma.
The opinion of the experts is still divided, will the solution of the fundamental
lemma for other reductive groups come from the matching of smooth functions, 
or will it come from even stronger
versions of Kazhdan's lemma.  This much is clear: the fundamental lemma
no longer needs to be viewed as an isolated combinatorial problem in buildings. 
Howe's conjecture, uniform germ expansions, homogeneity of germs, compact traces, unipotent orbits,
and Kazhdan's density theorem give us a framework through which the fundamental
lemma may be understood.  

In the final portion of the introduction, I would like to state a
theorem that has considerable interest beyond its usefulness for the
fundamental lemma.   This theorem will be used in Section 1 to produce
a {\it uniform germ expansion}.   The theorem shows that the study of compact traces
and the study of unipotent orbital integrals are in some sense
equivalent questions.
The endoscopic groups considered in this paper are the standard endoscopic groups
of $SL(n,F)$, where $F$ is a $p$-adic field of characteristic zero.  
However, simple arguments (Lemma 3.1) allow us to work instead with the group
$G_1$, consisting of elements of $GL(n,F)$, whose
determinant has a valuation which is a multiple of $n$.  

To state the theorem we need the following notation.
Let ${\mathcal H}_1'$ denote the linear dual to the Iwahori Hecke algebra 
${\mathcal H}_1$  on $G_1$.  Let $\Pi^{\text{t}}$ denote the set of irreducible
tempered representations of $G_1$. Let $\Omega$ denote the set
of strongly compact elements in $G_1$.  More concretely, $\Omega$
is the set of elements $GL(n,F)$-conjugate to $GL(n,O_F)$, where
$O_F$ denotes the ring of integers in $F$.  
If $1_\Omega$ is the characteristic function
of $\Omega$, then the compact trace, denoted 
$\text{trace}_\Omega\pi(f) $, is defined to be $\text{trace}
(1_\Omega f)$, for $f\in C_c^\infty(G_1)$.  
This is a trace on {\it strongly} compact elements, but for simplicity, it
will just be called the compact trace.
If ${\mathcal O}$ is a unipotent
conjugacy class, then we let $\mu_{\mathcal O}$ denote the invariant distribution
supported on ${\mathcal O}$.  We also write $\Phi(\gamma,f)$ for the
orbital integral of $f\in C_c^\infty(G)$, over the conjugacy class
containing the strongly regular
semisimple element $\gamma\in G_1$.  
We let {\sl val} denote the normalized valuation on $F^\times$, and let $|\cdot|$
denote the normalized absolute value on $F^\times$.  The notation $|S|$ will also be
used for the cardinality of a set $S$.  The context should make the meaning of
$|\cdot|$ clear.
See the beginning of Section 2, and
Section 3.4, for additional notation.

\bigskip
\proclaim Theorem 1.  
  The following subspaces of 
  $ {\mathcal H}_1' $ are equal.
\begin{itemize}
\item{{\rm (a)}}
  The image in 
    $ {\mathcal H}_1' $
  of the space of invariant distributions on
    $ C_c^{\infty}  (G_1) $
  supported on the set of unipotent elements in $G_1$,
\item{{\rm (b)}}
  The span of 
    $ \{ \mu_{\mathcal O} \mid \mathcal O \text{ unipotent}\} $,
\item{{\rm (c)}}
  The span of 
    $ \{ \text{trace}_{\,\Omega}\,\pi \mid
         \pi \in \Pi^{\text{t}} \} $,
\item{{\rm (d)}} 
  The span of
    $ \{ \text{trace}_{\,\Omega}\,\pi \mid
         \pi \in \Pi^{\text{t}} $, 
    $ \pi $
    has an Iwahori fixed vector $\}$,
\item{{\rm (e)}}
  The space spanned by orbital integrals:
    $ f \mapsto \Phi (\gamma, f) $, for
    $ f \in {\mathcal H}_1 $,
  as $\gamma$ runs over the strongly regular semisimple
  elements of 
  $ \Omega $.
\end{itemize}
\finishproclaim

\medskip
\noindent
The theorem is proved in the following steps.
Denote the vector spaces described in (a)$,\dots,$(e) by
  $ V_A,\ldots,V_E$.
We have obvious inclusions
  $ V_B \subseteq V_A$ and $V_D \subseteq V_C $.
From Section 1, using arguments valid for an arbitrary
  reductive group, we obtain
  $ V_A \subseteq V_E \subseteq V_D = V_C $.
In Section 2 we prove that the dimension of $V_D$ is less than or
equal to a constant, which we will call $g(n)$.
In Section 3 the proof is completed by proving that the dimension
of $V_B$ is at least $g(n)$.
This last inequality is the most difficult step of the proof.

Also, it is shown below that the number of unipotent conjugacy classes of
$G_1$ is 
    $ g(n) $.
    This number is also
  equal to the cardinality of a basis of $GL(n,O_F)$-characters
  of irreducible admissible representations with an Iwahori fixed vector.
It would be interesting to have a direct geometric interpretation of 
  this fact.

By identifying these vector spaces in Theorem 1, the existence of
  a uniform germ expansion is proved (Section 1).
This uniform germ expansion is used to link the matching of 
  smooth functions to the fundamental lemma.

The strategy of the proof was developed during several conversations
  with D.~Ramakrishnan.
It is a pleasure to thank him for the important role he played in
  the development of the ideas of this paper.
The papers of Langlands--Shelstad~[{\bf 20}] and Waldspurger~[{\bf 31}]
  also greatly influenced this work.
I would also like to thank F.~Shahidi for his assistance, and D.~Joyner 
  for some helpful comments.

\section{{{\bf 1@.  Uniform germ expansions.}}}


\medskip
\noindent
In this first section, we let $G$ be the $F$-points of a 
  connected linear reductive group over $F$. Also, let $K$ 
  be a maximal
  compact subgroup containing an Iwahori subgroup $B$, and
let
   $ \Omega $
denote the set of strongly compact elements.  In this paper, the term
{\it strongly compact element} means an element lying in the
intersection of the set of compact elements $\Omega_c$ with
the intersections of the kernels of all unramified quasicharacters of $G$.
Let 
  $ {\mathcal H} (B,G) $
denote the Hecke algebra of functions on $G$ bi-invariant
  by $B$.
Otherwise we retain the notation of the introduction.

\proclaim Theorem {1.1.} {\rm (Kazhdan)}.
Let
  $ f$ belong to $C_c^{\infty} (G) $.
If \, {\rm trace}
  $ \pi (f)$ is zero 
for all irreducible tempered representations, then the orbital integral
  $ \Phi (\gamma,f) $ is zero
for all regular semisimple elements~$ \gamma $.
\finishproclaim

Let
  $ I (\Omega) $
denote the space of invariant distributions of
  $ C_c^{\infty} (G) $
supported on $\Omega$.
Note that 
  $ \text{trace}_{\,\Omega} \pi $ belongs to $I(\Omega) $.

\proclaim Theorem {1.2.} {\rm (Clozel)}.
  $ I (\Omega) \mid_{{\mathcal H}(B,G)} $
is finite dimensional.
\finishproclaim

\proclaim Corollary {1.3}.
There is a finite set
  $ \Pi_* $ (depending on $\Omega$)
of irreducible tempered representations and functions
  $ a_{\pi} (\gamma) $,
for 
  $ \pi \in \Pi_* $,
defined on regular semisimple elements
  $ \gamma $ 
in
  $ \Omega $,
such that
%
$$
  \Phi (\gamma, f) 
=
   \sum_{\pi \in \, \Pi_*}
  a_{\pi} (\gamma) 
  \text{trace}_\Omega{\pi} (f),
$$
%
for all 
regular semisimple elements $\gamma$ in 
  $ \Omega $
and all 
  $ f \in {\mathcal H} (B,G) $.
\finishproclaim

\pproclaim Proof:
Pick $\Pi_*$ to be a finite set 
  $ \{\pi_1, \dots , \pi_r \} $
  of representations
such that the set
  $\{ \text{trace}_{\,\Omega} \, \pi_i \}$
forms a basis for the span of
  $ \{ \text{trace}_{\,\Omega} \, \pi \mid
    \pi \,\, \text{tempered irreducible} \} $
on 
  $ {\mathcal H} (B,G) $.
If
  $ \text{trace}_{\,\Omega} \, \pi_i (f) = 0 $, for all $\pi_i\in\Pi_*$,
then for all irreducible tempered representations $\pi$ we have
  $ \text{trace}_{\,\Omega} \, \pi (f) = 0 $.
Hence,
  $ \text{trace} \, \pi (1_{\Omega} f) = 0 $,
and
  $ \Phi (\gamma, 1_{\Omega} f) = 
    \Phi (\gamma, f) = 0 $.
Thus, by linear algebra,
  $ \Phi (\gamma,f) = 
    \sum_{\pi \in \,\Pi_*} 
    a_{\pi} (\gamma) \text{trace}_{\,\Omega}\pi(f) $
for constants
  $ a_{\pi} $
depending on
  $ \gamma$, a regular
  semisimple element of $\Omega$.
\qed
\finishpproclaim

Let 
  $ \{ \mathcal O \} $
denote the set of unipotent conjugacy classes of $G$.
Let 
  $ \mathcal M $
denote the space spanned by distributions
  $ \mu_{\mathcal O} $ 
on
  $ {\mathcal H} (B,G) $,
  for
  $ \mathcal O \in \{ \mathcal O \} $.
Let 
  $ {\mathcal M}_* $
denote a basis of
  ${\mathcal M}$
consisting of distributions of the form
  $ \mu_{\mathcal O} $.
Let
  $\text{tr}_{\,\Omega}\Pi_*$
be the span of 
  $ \{ \text{trace}_{\,\Omega} \pi \mid \pi
     \in \Pi_* \} $.

\proclaim Proposition {1.4}.
If 
  $ \dim {\mathcal M} \geq \dim \text{tr}_{\,\Omega}\Pi_*$,
then
  $ {\mathcal M} = \text{tr}_{\,\Omega}\Pi_*$.
\finishproclaim

\pproclaim Proof:
It is enough to show that ${\mathcal M}$ is contained in $\text{tr}_{\,\Omega}\Pi_*$.
We have the Shalika germ expansion
$$\Phi(\gamma,f) = \sum_{\{{\mathcal O}\}}\Gamma_{\mathcal O}(\gamma)\mu_{\mathcal O}(f),
	\quad \text { for } f\in {\mathcal H}(B,G),$$
valid for regular semisimple elements in some neighborhood 
$\Omega'$ of $1$ [{\bf 27}].   Since the Shalika germs
$\Gamma_{\mathcal O}(\gamma)$ are linearly independent [{\bf 14}],
each unipotent orbital integral $\mu_{\mathcal O}$ may be written
as a finite linear combination of integrals $\Phi(\gamma,f)$,
for appropriate regular semisimple elements $\gamma$.  Each
orbital integral $\Phi(\gamma,f)$, in turn, may be written
as a linear combination of compact traces, by Corollary 1.3.
The result follows.
\qed
\finishpproclaim

\proclaim Theorem {1.5}.
If 
   $ \dim {\mathcal M} \geq \dim \text{tr}_{\,\Omega}\Pi_*$,
then there exist functions
  $ \Gamma_{\mathcal O} (\gamma) $, for ${\mathcal O}\in{\mathcal M}_*$,  defined on the
  regular semisimple elements of
  $ \Omega $,
such that, for all
  $ f \in {\mathcal H} (B,G) $
and all such 
  $ \gamma  $, we have the equation
%
$$
  \Phi (\gamma,f) =
  \sum_{{\mathcal O}\in{\mathcal M}_*}
  \Gamma_{\mathcal O}
  (\gamma) \mu_{\mathcal O} (f).
$$
%
\finishproclaim
\noindent
We call such an expansion a {\it uniform germ expansion}.

\pproclaim Proof:
Expand each
  $ \text{trace}_{\,\Omega}\pi $, with  
  $ \pi \in \Pi_* $,
as a linear combination of
  $ \mu_{\mathcal O} \in {\mathcal M}_* $,
and substitute these linear combinations into the expansion
%
$$
  \Phi(\gamma,f) =
  \sum_{\pi\in\Pi_*}
  a_{\pi}(\gamma) \,\, \text{trace}_{\,\Omega} \pi(f).
$$
%
\qed
\finishpproclaim

\proclaim Lemma {1.6}.
If
  $ \text{trace}_{\,\Omega}\pi (f) $ is nonzero, 
for some 
  $ f \in \mathcal H (B, G) $,
then
  $ \pi $
has an Iwahori fixed vector.
\finishproclaim

\pproclaim Proof:
We first prove the analogous result for $\Omega_c$, the set of 
compact elements, instead of for $\Omega$.
Waldspurger's proof for 
  $ GL(n) $
carries over directly to the general reductive group.
We recall the argument.
There are three ingredients to the proof:
a formula expressing
  $ \text{trace}_{\,\Omega} \pi $
as a combination of ordinary traces, 
the observation that the functions 
  $ \bar f^P $
are nearly in the Iwahori Hecke algebra on $M$,
and the fact that if a Jacquet module of
  $ \pi $
has an Iwahori fixed vector then so does 
  $ \pi $.

If
  $ P = MN $
is a parabolic subgroup of $G$ and
  $ f$ belongs to $C_c^{\infty} (G) $, then
define 
  $ \bar f^P \in C_c^{\infty} (M) $
by
%
\begin{align*}
%\align
  \bar f^P (m) &=
    \delta_P^{1/2} (m)
    \int_{K,N}
    f (kmnk^{-1}) dkdn, \\
\vspace{4pt}
  \delta_P(m) &=
     \Bigl|
       \det
       \bigl(
          Ad\,m \bigm|_{\text{Lie}\,N}
       \bigr)
     \Bigr|,
     \quad\text { for }
     m \in M.
%\endalign
\end{align*}
%
Let $A_0$ be the split component of a minimal parabolic subgroup
  $P_0$.  Assume that $ B, K$ and $P_0 $ are compatibly chosen so that
  $ M \cap B = B_M $ is an Iwahori subgroup of $M$.  Assume $K$ is
  adapted to $ A_0 $ as in [{\bf 7}].

Let $W (M)$ denote the Weyl group of $M$.
Let $S$ be the set of shortest representatives (by length)
of the cosets 
  $ W/W(M) $.
We may take them to be represented by fixed elements in $K$.
Then there is a double coset decomposition
  $ K = 
    \coprod_{s \in S} 
      B s K_P =
    \coprod_{s \in S }
      B s K_M K_N $,
where 
  $ K_M =
    K \cap M $,
  $ K_N = K \cap N $ and
  $ K_P = K \cap P $.
Refine the double coset decomposition to one of the form
$K=\coprod_{S,T_S} B s\,t K_N$, where $T_S$ is a finite
subset of $K_M$.   

We define an action
of $M$ on $C_c^\infty(M)$ by
  $ (Ad\,m \cdot f) (m_0) = f (m^{-1} m_0 m) $.
Following the notation introduced by Waldspurger, we let 
  $ {\mathcal H}^i (B_M, M) $
denote the vector space spanned by the functions
  $ Ad\,m \cdot f $, for
  $ f \in {\mathcal H} (B_M, M) $ and
  $ m \in M $.
Next we will prove that
  $ \bar f^P \in {\mathcal H}^i (B_M, M) $.
We ignore the factor
  $ \delta_P^{1/2} $, because
it is bi-invariant by 
  $ B_M $
and its conjugates.
The integral
$$\int_{K,N} f(kmnk^{-1}) dk\,dn$$
may be broken into a finite sum of integrals over the double cosets
$BstK_N$ in $K$.   If $t\in K_M$, and if $s\in S$, then the double coset 
$BstK_N$ is right invariant by $t^{-1}B_Mt$, because $Bs$ is right
invariant by $B_M$  [{\bf 4}; prop\. 1.1.2] 
and because both $t$ and $t^{-1}B_Mt$ normalize $K_N$.
Hence, the contribution to $\bar f^P$ from the double coset $BstK_N$
is a $t^{-1}B_Mt$-bi-invariant function.  Such a contribution
lies in ${\mathcal H}^i(B_M,B)$.  Hence $\bar f^P\in {\mathcal H}^i(B_M,M)$.
The same argument establishes the following lemma.

\proclaim Lemma {1.7}. $\bar f^{P_1}$ belongs to ${\mathcal H}^i(B_M,M)$, for
every parabolic subgroup $P_1$ containing $P$.
\finishproclaim

To continue with Lemma 1.6, we recall
a theorem of Clozel and Waldspurger [{\bf 8}]
%
\begin{equation}\label{eqn:1.8}
  \text{trace}_{\,\Omega_c} \, \pi (f) =
  \sum_{P \in {\mathcal P}}
  (-1)^{a_P-a_G}
  \big\langle
    \text{trace}\, \delta_P^{-1/2} 
    \pi_N, 
    \hat \chi_N
    \bar f^P 
  \big\rangle_M.
%XX \tag{1.8}
\end{equation}
%
Here
  $ {\mathcal P} $
is a set of representatives, modulo conjugation by $G$, for
the parabolic subgroups of $G$.
The unnormalized Jacquet module of
  $ \pi $ is denoted here by $\pi_N$ 
[{\bf 4}]. The Jacquet module
is an admissible representation of $M$.
The constants
  $ a_P $
and
  $ a_G $
do not concern us.
The pairing
  $ \langle , \rangle _M $
is the usual pairing of characters of $M$,
viewed as distributions, with locally constant compactly
  supported functions on $M$.
  The function
  $ \hat {\chi}_N $
is a function on $M$ factoring through the Harish-Chandra
  map $H$ from $M$ to the
real Lie algebra of the split center of $M$.
The Harish-Chandra map $H$ and hence 
  $ \hat {\chi}_N $ 
are bi-invariant by every compact subgroup of $M$.
Thus
  $ \hat {\chi}_N \bar f^P$ belongs to  $\mathcal H^i(B_M, M) $.

Consider a function
  $ f \in {\mathcal H} (B,G) $.
Suppose that 
  $ \pi $
does not have an Iwahori fixed vector.
If 
  $ \pi $
does not have an Iwahori fixed vector, then neither does
  $ \pi_N $.
(This statement is an immediate consequence of [{\bf 5};prop 2.4].)
From
  $ \hat {\chi}_N \bar f^P \in {\mathcal H}^i
    (B_M,M) $,
it follows that
  $ \langle
      \text{trace} \, \delta_P^{-1/2}
      \pi_N,
      \hat \chi_N
      \bar f^P
    \rangle_M$ %\allowmathbreak$ 
is zero.
Thus, by the theorem of Clozel-Waldspurger,
$
  \text{trace}_{\,\Omega_c}\, \pi (f)
$ is zero.
%
This completes the proof of the analogue Lemma 1.6 for $\Omega_c$.
Now to prove the result for $\Omega$, use the additional fact
that if $\text{trace}_{\,\Omega_c}\,\pi\otimes\chi(f)$ is zero for all
unramified quasicharacters $\chi$ of $G$, then 
$\text{trace}_{\,\Omega}\,\pi(f)$ is also zero.
\qed
\finishpproclaim

\section{{\bf 2@.  Some representations of $G_1$.}}

\medskip\noindent
We now return to the groups $ G=GL(n,F) $
and 
  $G_1=\{g\in G\mid \text{val\,}\det g\in n{\Bbb Z}\}$.
In this section we study tempered, irreducible admissible
  representations of $G_1$ with an Iwahori fixed vector.  Since our applications 
  will truncate these representations on $\Omega$, we
  may assume that the central character is trivial.
Call 
  $ \Pi_1$
this set of representations of $G_1$
  (listed up to equivalence).
Every such representation of $G_1$ is a constituent
  of the restriction of a representation of $G$ with 
  the same properties (tempered, irreducible, trivial central character,
   possessing an Iwahori fixed vector).
Notice that the groups $G$ and $G_1$ have the same center:
  $Z(G_1) = Z(G)$.
First we consider such representations of 
$G$, and then we consider their restrictions to $G_1$.

We will use the following notation.  
Let $\Lambda_n$ be
  the set of ordered partitions of $n$.  Such a partition is a tuple
  $ \lambda = (\lambda_1, \dots , \lambda_{\ell})$ satisfying
    $\lambda_1 \geq \lambda_2 \geq \dots \geq\lambda_{\ell} > 0 $.
    The letter $\ell$ will always be used for the number of
    parts of the partition.
  If 
  $ \lambda \in \Lambda_n$, then define  $gcd (\lambda)$ to be the
  greatest common divisor of the parts
  $ \lambda_i$ of $\lambda$.
  If $\lambda\in \Lambda_n$, then let $\lambda^t\in\Lambda_n$ 
  denote the dual partition (or transpose), defined in the usual way.
We let $g(n)$ denote the constant
  $ \sum\limits_{\lambda \in \Lambda_n}
  gcd (\lambda) $.
%
Let 
$\varpi$ be
  a uniformizing element.  Let $K$ denote the maximal
  compact subgroup $GL(n,O_F)$.  Let
$B$ be the standard Iwahori subgroup.  Explicitly, $B$ is the group
  $ \{
      (x_{ij}) \in K \mid
       x_{ij} \in (\varpi), \,\,
    \text{ if }  i > j
    \} $.
Let $P_0$ denote the 
    Borel subgroup of upper triangular matrices
      $ \{ (x_{ij}) \in G \mid x_{ij} = 0 $,\text { if }
      $ i > j \} $.
%
For $\lambda\in \Lambda_n$, let $P_\lambda$ be the
    standard parabolic subgroup.  It satisfies
      $ P_0 \subseteq P_{\lambda} \subseteq G $.
%
We have a Levi decomposition
$P_{\lambda}=M_{\lambda}N_{\lambda}$,
and
      $ M_{\lambda} $
    is isomorphic to 
      $ GL(\lambda_1, F)\times \dots \times 
        GL(\lambda_{\ell}, F) $.
%
Let $\bar N_\lambda$ denote the transpose of the group
$N_\lambda$ in $G$.  It is the unipotent radical of a
parabolic subgroup opposite $P_\lambda$ containing $M_\lambda$.



The classification of the given representations of $ GL (n,F) $
is well known [{\bf 2}], [{\bf 32}].  For other split reductive
groups see [{\bf 17}].  Let $\sigma_r$ denote the Steinberg
representation of $GL(r,F)$.  For a partition 
$\lambda=(\lambda_1,\ldots,\lambda_\ell)\in\Lambda_n$, let
$\sigma_\lambda$ be the representation $\sigma_{\lambda_1}\otimes
\cdots\otimes\sigma_{\lambda_\ell}$ of $M_\lambda  = 
GL(\lambda_1,F)\times \cdots\times GL(\lambda_\ell,F)$.  For
$s\in {\Bbb C}^\times$, let $\nu_s$ be the quasicharacter of
$GL(r,F)$ given by $\nu_s(g) = s^a$, if the valuation of the determinant
of $g$ is $a$.  For $s=(s_1,\ldots,s_\ell)\in {\Bbb C}^{\times\,\ell}$, let
$\nu_s$ denote the quasicharacter $\nu_{s_1}\otimes\cdots\otimes
\nu_{s_\ell}$ on $M_\lambda$.  Set $\sigma_{\lambda,s}=\sigma_\lambda
\otimes \nu_s$.  Then the irreducible tempered representations 
in question are the induced representations 
$\pi_{\lambda,s} = \text{Ind}\,\sigma_{\lambda,s}$ (unitary
induction), with $s\in {\Bbb C}_1^\ell$, where ${\Bbb C_1}= \{z\in {\Bbb C}\mid
\ |z|=1\}$.

Add ``$1$'' to notation to indicate quantities on 
  $ G_1 : M_1 = M \cap G_1$,
  $ \sigma_1 = \sigma \big|_{M_1} $,
  $ \pi_1 = \pi \big|_{G_1} $, and so forth.
Let $A_\lambda$ denote the center of $M_\lambda$.
Set
  $ W_1 (\lambda,s) =
    \{ w \in N_G(A_{\lambda}) \mid
       w \cdot \sigma_{\lambda,s,1} \approx
       \sigma_{\lambda,s,1} 
    \} $,
and set
  $ W(\lambda,s) =
   \{ w \in N_G(A_{\lambda}) \mid
       w \cdot \sigma_{\lambda,s} \approx
       \sigma_{\lambda,s} 
    \} \subseteq W_1(\lambda,s) $.
Here
  $ w \cdot \sigma (m)$ is the representation  $ \sigma(w^{-1} mw) $,
  and ``$\approx$'' denotes equivalence.

\proclaim Lemma {2.1}.
For
  $ w \in W_1(\lambda,s) $,
there is a uniquely defined character
  $ \omega_{w,0}: F^{\times} \to {\Bbb C}^{\times} $
such that
  $ w \cdot \sigma_{\lambda,s} $ is equivalent to
   $\sigma_{\lambda,s} \otimes
   \omega_{w,0} \cdot \det $.
The map
  $ w \mapsto \omega_{w,0} $
is a group homomorphism from 
  $ W_1(\lambda,s) $
to the group of characters on 
  $ F^{\times} $.
The kernel of the homomorphism is
  $ W(\lambda,s) $.
\finishproclaim

\pproclaim Proof: This result is contained in {\rm [{\bf 9}], [{\bf 10}] and [{\bf 26}]}.
Note that
  $ \omega_{w,0} $
is trivial on
  $ M_1 $.
%
\qed
\finishpproclaim

For any representation
  $ \pi $,
let
  $ V(\pi)$
denote its underlying vector space.
Fix a complex number $t$, and let
  $ \text{Ind}_t \, \sigma_{\lambda,s} $
denote the representation obtained by right action on
functions from
  $ GL(n,F) $
to 
  $ V(\sigma_{\lambda,s}) $
such that%
$$
  f(mng) =
  \delta_P^{t+\frac12} (m)
  \sigma_{\lambda,s} (m) f(g),
  \quad\text { for }
  m \in M_{\lambda} \text{ and }n \in N_{\lambda}.
$$
%
For 
  $ w \in W(s,\lambda) $,
set
  $ N_w = N_\lambda \cap w \bar N_\lambda w^{-1} $.
For 
  $ f \in  V
    (\text{Ind}_t \, \sigma_{\lambda,s}) $,
define the operator
  $ A_{t,\lambda,s} (w) $
by
%
$$
  \bigl(
    A_{t,\lambda,s} (w) f
  \bigr)
  (g) =
  \int_{N_w}
  f (w^{-1} ng) dn, 
  \quad 
  \text{for }g \in GL(n,F).
$$
%
When the real part of $t$ is sufficiently large,
this integral is absolutely convergent.  The integral has a meromorphic
continuation to the complex $t$-plane.
There is a meromorphic function of $t$ such that, when
  the operator $ A_{t,\lambda,s} (w) $
is multiplied by this scalar function, the resulting operator does not have
a pole at
  $ t = 0 $. 
This construction yields a normalized intertwining operator
  $ {\mathcal A}_{\lambda,s} (w) $,
which intertwines
  $ \text{Ind} \, \sigma_{\lambda,s} $
and
  $ \text{Ind} \, w \cdot \sigma_{\lambda, s}  $.

If 
  $ w_1 $ and 
  $ w_2$ belong to $W_1 (\lambda,s) $,
and if
  $ \omega_{w_1,0} =
    \omega_{w_2,0} $,
then we have
  $ w_1 \cdot 
    \sigma_{\lambda,s} \approx
    \sigma_{\lambda,s} \otimes
    \omega_{w_1,0} =
    \sigma_{\lambda,s} \otimes
    \omega_{w_2,0} \approx
    w_2 \cdot 
    \sigma_{\lambda,s} $.
Thus, there is an operator 
  $ \phi : 
    V
    (w_1 \cdot \sigma_{\lambda,s})
    \to
    V(w_2 \cdot \sigma_{\lambda,s}) $
intertwining these irreducible representations.
The intertwining operators depend only on the image of $w$
in 
  $ W_1(s,\lambda) /
    W (s,\lambda) $
in the following sense.

\proclaim Lemma {2.2}.
The operators
  $ \phi {\mathcal A}_{\lambda,s} (w_1) $
and
  $ {\mathcal A}_{\lambda,s} (w_2) $
are proportional.
\finishproclaim

\pproclaim Proof:
They both intertwine
  $ \text{Ind}\, \sigma_{\lambda,s} $
and
  $ \text{Ind}\, w_2 \cdot \sigma_{\lambda,s} $.
Both representations are irreducible.
Use Schur's lemma.
\qed
\finishpproclaim

Now we restrict representations to
  $ G_1 $.
We may identify
  $ \pi_{\lambda,s,1} $
and
  $ \text{Ind}_{P_1}^{G_1} \,
     \sigma_{\lambda,s,1} $.
For
  $ w \in W_1 (\lambda,s) $,
we may fix
  $ \phi_w $
intertwining
  $ w \cdot \sigma_{\lambda,s,1} $
and
  $ \sigma_{\lambda,s,1} $.
By a theorem of Harish-Chandra [{\bf 29}],
the operators
  $ \phi_w {\mathcal A}_{\lambda,s} (w) $
span the commuting algebra of 
  $ \pi_{\lambda,s,1} $.
This theorem, expressed in terms of distributions,
implies the following proposition.

\proclaim Proposition {2.3}.
If 
  $ \pi $
is a constituent of
  $ \pi_{\lambda,s,1} =
    \text{{\rm Ind}}_{P_1}^{G_1} \,
    \sigma_{\lambda,s,1} $,
then 
  $ \text{{\rm trace}}\,\pi $
lies in the space spanned by
%
$$
  \text{{\rm trace}} 
  \left(
    \phi_w 
    {\mathcal A}_{\lambda,s} (w)
    \pi_{\lambda,s,1}
  \right), \qquad\text{ for }
   w \in W_1(\lambda,s).
$$
%
\finishproclaim

By Lemma 2.2,
it is enough to let $w$ range over coset representatives in
  $ W_1(\lambda,s)  $
of
  $ W_1(\lambda,s) / W (\lambda,s) $.
A different choice of 
  $ \phi_w $
changes 
  $ \text{trace}
    \bigl(
       \phi_w
       {\mathcal A}_{\lambda,s} (w) 
       \pi_{\lambda,s,1} 
    \bigr) $
by a scalar.
We fix the isomorphisms
  $ \phi_w $
and drop them from the notation.

\proclaim Lemma {2.4}.
If
  $ h $ belongs to $G$,
  and if $ f$ belongs to $C_c^{\infty} (G_1) $, then
%
$$
  \text{{\rm trace}}
  \bigl(
    {\mathcal A}_{\lambda,s} (w)
    \pi_{\lambda,s,1}
    (Ad\,h \cdot f)
  \bigr) 
=
  \omega_{w,0} 
  (\det h) \,
  \text{{\rm trace}}
  \bigl(
    {\mathcal A}_{\lambda,s} (w)
    \pi_{\lambda,s,1}
    (f)
  \bigr) .
$$
%
\finishproclaim

\pproclaim Proof: Clearly,
%
\begin{align*}
  \pi_{\lambda,s,1} (Ad\,h \cdot f)
&=
  \pi_{\lambda,s}(h)
  \pi_{\lambda,s,1} (f) 
  \pi_{\lambda,s} (h^{-1}) ,
\\
  {\mathcal A}_{\lambda,s} (w)
  \pi_{\lambda,s} (h) 
&=
  \omega_{w,0} (\det h)
  \pi_{\lambda,s} (h) 
  {\mathcal A}_{\lambda,s} (w).
\end{align*}
%
The result follows.
\qed
\finishpproclaim

\noindent
\hbox to 8mm{{\bf 2.5}\hfil}
Fix a partition $\lambda =(\lambda_1,\ldots,\lambda_\ell)\in \Lambda_n$. We
define an element $$s(\lambda)=(s(\lambda)_1,\ldots,s(\lambda)_\ell)\in
{\Bbb C}_1^\ell$$ as follows.  Let $r_i$ be the cardinality of the
set $\{j\mid \lambda_j = i\}$, and let $r=r(\lambda)$ be the greatest
common divisor of the integers $r_1,r_2,\ldots$.  Notice that $r(\lambda^t) = \text{gcd}(\lambda)$.
Let $\zeta=\exp(2\pi\,\sqrt{-1}/r)$.  Set $s(\lambda)_i = \zeta^i$.

Now fix an element
  $s \in {\Bbb C}^\ell_1$.
The characters 
  $ \omega_{w,0} $, for
  $w \in W_1(\lambda,s) $,
are unramified characters
  $ \omega_{w,0}: F^{\times} \to {\Bbb C}^{\times} $
whose order $r'$ divides $r$.
The representation $\pi_{\lambda,s}$ is equivalent to the representation
$\pi_{\lambda,s'}$, if $s'$ is obtained from $s$ by interchanging
$s_i$ and $s_j$, when $\lambda_i=\lambda_j$.  Call parameters $s$ and $s'$
equivalent if they are related by a sequence of such permutations.
There exist a primitive $r'$\,th root of unity $\zeta'$ and a
tuple $t=(t_1,\ldots,t_{\ell/r'})\in {\Bbb C}^{\ell/r'}_1$ such that, after
replacing $s$ with an equivalent parameter, we have
$s_i = t_a\zeta^{\prime\,b}$, for all  $i,a,b$  satisfying 
	$$\quad i = 1+(a-1)r'+b,\ \ 1\le a\le \ell/r',\ \  0\le b\le r'-1.$$
Write $t = t(s,r')$ to indicate the dependence of $t$ on $s$ and $r'$.  As a particular
example, for each $r'$ dividing $r$, we find that a parameter equivalent to $s(\lambda)$ has the
form just described.
We may assume that $w$ acts by a permutation $\epsilon$ of order $r'$,
that the permutation carries the $i$th factor $GL(\lambda_i)$ of $M_\lambda$
to the following factor $GL(\lambda_{i+1})$ if $i$ is not a multiple of $r'$, and finally
that the permutation carries the $i$th factor $GL(\lambda_i)$ of $M_\lambda$
to the factor $GL(\lambda_{i-r'+1})$ if $i$ is a multiple of $r'$.
The conditions on $r(\lambda)$ and $r'$ insure that factors permuted by the
permutation $\epsilon$ have the same rank.

\proclaim Proposition {2.6}.
For
  $ f \in C_c^{\infty} (G_1) $ and
  $ w \in W (\lambda,s) $,
pick 
  $ w' \in W (\lambda,s(\lambda)) $
such that
  $ \omega_{w,0}$ and $\omega_{w',0} $
  are equal
as characters of 
  $F^{\times} $.
Then
%
$$
  \text{{\rm trace}} \, 
    \bigl(
      {\mathcal A}_{\lambda,s} (w)
      \pi_{\lambda,s,1} 
      ( 1_{\Omega} f)
    \bigr) 
    \quad
    \text{is proportional to}\quad
  \text{{\rm trace}} \, 
    \bigl(
      {\mathcal A}_{\lambda,s(\lambda)} (w')
      \pi_{\lambda,s(\lambda),1} 
      ( 1_{\Omega} f)
    \bigr). 
$$
%
\finishproclaim

\proclaim Theorem {2.7}.
The dimension of the span of
  $ \{ \text{{\rm trace}}_{\, \Omega} \pi \} $
on
  $ {\mathcal H} (B, G_1) $
is at most
  $ g(n) $.
\finishproclaim

\pproclaim Proof of Theorem 2.7:
By the results of Section 1, we may assume
  $ \pi $
has an Iwahori fixed vector. 
For a fixed 
  $ \lambda $
the distributions are combinations of
the linear functionals of Proposition 2.6
%
$$
  f \mapsto \text{ trace }
  \bigl(
    {\mathcal A}_{\lambda,s(\lambda)} (w)
    \pi_{\lambda,s(\lambda),1} (1_{\Omega} f)
  \bigr),
$$
%
with
  $ w $
ranging over representatives of 
  $ W_1 (\lambda,s(\lambda)) /
    W   (\lambda,s(\lambda)) $.
Thus, the dimension of the span is at most
%
$$
  \sum_{\lambda \in \Lambda_n}
  r (\lambda) =
  \sum_{\lambda \in \Lambda_n}
  gcd (\lambda) = g(n).
$$
%
\qed
\finishpproclaim

\pproclaim Proof of Proposition 2.6:
By going to equivalent data we may assume that 
  $s$ and 
  $ s(\lambda) $
have the form given in 2.5.
By replacing $w$ with a different representative in
  $ W_1(\lambda,s) $,
we may assume it is given by the permutation matrix $\epsilon$
described above.
Similarly, we may assume that
  $ w = w' $.

Write
  $ 1_{\Omega} f =
   \sum Ad \, h_i \cdot f_i $,
where
  $ \text {supp} (f_i)$ is contained in $K$
and
  $ h_i$ belongs to $GL(n,F)$.
Suppose that the proposition holds for functions supported in $K$.
Then, for any 
  $ f \in C_c^{\infty}(G) $,
%
$$
  \text{trace}
  \bigl(
    {\mathcal A}_{\lambda,s} (w)
    \pi_{\lambda,s,1}
    (1_{\Omega} f)
  \bigr)
=
  \sum
  \omega_{w,0}
  (\det h_i) \,
  \text{trace}
  \bigl(
    {\mathcal A}_{\lambda,s} (w)
    \pi_{\lambda,s,1}
    (f_i)
  \bigr)
$$
%
is proportional to
%
$$
  \sum  \omega_{w,0}
  ( \det h_i)
  \text{trace}
  \bigl(
    {\mathcal A}_{\lambda,s(\lambda)} (w)
    \pi_{\lambda,s(\lambda),1}
    (f_i)
  \bigr)
=  
  \text{trace}
  \bigl(
    {\mathcal A}_{\lambda,s(\lambda)} (w)
    \pi_{\lambda,s(\lambda),1}
    (1_{\Omega}f)
  \bigr).
$$
%
Thus, we may assume that the support of $f$ lies in $K$.

The restriction $V$ of 
  $V(\pi_{\lambda,s,1})$
to functions on $K$ is independent of $s$.
It is
%
$$
  V =
  \{
    f \in C_c^{\infty} (K) \mid
    f (k_m k_n k ) =
    \sigma_{\lambda} (k_m) f (k)
  \}, \ \text{ for } k_m\in M\cap K
  \ \text{and } k_n\in N\cap K.
$$
%
For
  $ f \in V $,
let 
  $ f_s $
denote its unique extension to
  $ V(\pi_{\lambda,s,1} ) $.
Let
  $ \xi_i = r'\lambda_{r'i} $,
for
  $i\in \{1,\dots,\ell'\}$ (where $\ell'=\ell/r'$),
and let
  $ \xi $
be the partition 
  $ ( \xi_1, \dots, \xi_{\ell'}) $.
Notice that
  $ P_{\xi} \supseteq P_{\lambda} $,
  $ M_{\xi} \supseteq M_{\lambda} $ and
  $ N_{\xi} \subseteq N_{\lambda} $.
By construction,
  $ w$ belongs to $M_{\xi} $.
  The subgroup $N_w$ of $N_\lambda$ is defined in
  at the beginning of the section.  It satisfies the condition
%
$$
  w^{-1} N_w w = 
  w^{-1} N_\lambda w \cap \bar N_\lambda \subseteq 
  P_{\xi} \cap \bar N_\lambda \subseteq
  M_{\xi}.
$$
%
So
  $ N_w $ is a subset of $ w M_{\xi} w^{-1} = M_{\xi} $.
Let 
  $ {}^{\circ}\! M_{\xi} = 
    \{ m \in M_{\xi} \bigm|
       |\chi(m)| = 1 $,
for every character
  $ \chi $
of 
  $ M_{\xi} \} $.
Since
  $ N_w $
is unipotent,
  $ N_w$ is contained in  
    ${}^{\circ} \!M_{\xi} $.
Also,
  $ w \in 
    {}^{\circ} \!M_{\xi} $,
so
  $ w^{-1} N_w$ is contained in 
    ${}^{\circ} \!M_{\xi} $.
For
  $ w^{-1} n \in G $,
write
  $ w^{-1} n = m n_1 k_1 $
according to the decomposition
  $ G = M_{\lambda} N_{\lambda} K $.
Since
  $ w^{-1} n$ belongs to  ${}^{\circ} \!M_{\xi} $,
we may also assume
  $m$, $n_1$ and $k_1$ belong to
    ${}^{\circ} \!M_{\xi} $.
This decomposition is not unique, but our manipulations
  will be well defined.  We find that
%
$$
  f_s (w^{-1}nk) =
  f_s (m n_1 k_1 k) =
  \sigma_{\lambda,s}(m) f (k_1 k).
$$
%
Set $t_i = t(s(\lambda),r')_i/t(s,r')_i$, and
consider the element $t\in {\Bbb C}^\ell_1$ of the form
$$t = (t_1,\ldots,t_1,\ldots ,t_{\ell'},\ldots,t_{\ell'}),$$ where
each coordinate $t_i\in{\Bbb C}_1$ is repeated $r'$ times.
If $m=(m_1,\ldots,m_\ell)\in M_\lambda \cap {}^{\circ}\!M_\xi$,
then
\begin{align*}
 \nu_t (m) &=
   t_1^{%
      \operatorname{val}
      (\det m_1 \det m_2 \dots \det m_{r'}) } \;
   t_2^{%
      \operatorname{val}
      (\det m_{r'+1} \dots) }
    \dots
    t_{\ell'}^{%
      \operatorname{val}(\dots \det m_\ell)}
\\
\vspace{4pt}
  &=
   t_1^0
   t_2^0 \dots = 1.
\end{align*}
%
Thus, we find that
%
$$
  \nu_s (m) =
  \nu_s (m)
  \nu_t(m) =
  \nu_{s(\lambda)} (m), \; \text{ and } \;
  \sigma_{\lambda,s} (m) =
  \sigma_{\lambda,s(\lambda)} (m).
$$
%
Thus,
  $ f_s (w^{-1} nk) =
    f_{s(\lambda)} (w^{-1} nk) $.
Hence, 
  $\mathcal A_{s, \lambda} (w) =
   \mathcal A_{s(\lambda), \lambda} (w) $.
%
\qed\finishpproclaim


\section{{\bf 3@.  Linear independence of unipotent classes.}}

\medskip
\noindent
In this section we prove that the dimension of the span of
%
$$
  \{ \mu_{\mathcal O} \mid
     {\mathcal O} \text{ unipotent } \} 
  \quad \text{on} \quad
  {\mathcal H} (B, G_1)
$$
%
is equal to 
  $ g(n) $.  Our justification for working with $G_1$ rather
  than $SL(n,F)$ comes from the following lemma.
  For any distribution $F_1$ and any $g\in G$, let $F^g_1$ be the distribution
  $F^g_1(f) = F_1(\text{Ad}\,g\cdot f)$ (see Section 2).

\proclaim Lemma {3.1}.
If
  $ {F}_1$ and ${F}_2 $
are distributions on 
  $ SL(n,F) $
such that
  $ {F}_1^g = 
    {F}_2 $
with $g\in G_1$, then
  $ F_1$ and $F_2$ 
are equal to each other on
the Iwahori Hecke algebra of $SL(n,F)$.
In particular, the distributions supported on two different conjugacy
classes in $SL(n,F)$ are equal up to a scalar, if the classes are
conjugate in $G_1$.
\finishproclaim

\pproclaim Proof:
Let $B_s$ denote $B\cap SL(n,F)$.
Take
  $f$ to be the characteristic function of 
  a double coset of $B_s$ in $SL(n,F)$.
  Each coset of $SL(n,F)$ in $G_1$ is represented by a diagonal element
  $g$ stabilizing the double coset.
Then
  $ Ad \, g \cdot f = f $,
so that
  $ F_2 (f) =
    F_1^g (f) =
    F_1 (Ad \, g \cdot f) =
    F_1 (f) $.
\qed
\finishpproclaim

Each $GL(n,F)$-conjugacy class
  $ {\mathcal O} $
breaks into finitely many 
  $ G_1 $-classes, the
  number of classes in a stable conjugacy class
  being equal to the cardinality of
%
$$
  \bigl\{
    \text{val det}(g):
    g \in G
  \bigr\}
  \big/
  \bigl\{
    \text{val det}(g):
    g \in C_{G} (u)
  \bigr\},
$$
%
where $u$ is an element of the conjugacy class.
If 
  $ {\mathcal O}_{\lambda} $
is a conjugacy class in $GL(n,F)$, and if the class is 
the Richardson class of the
  parabolic subgroup 
  $ P_{\lambda} $
(with Levi factor
  $ GL (\lambda_1) \times \dots \times GL(\lambda_{\ell}) $),
then the elementary divisors of an element in
  $ {\mathcal O}_{\lambda} $
are given by the partition $\lambda^t$ dual to $\lambda$.  
If $u$ belongs to ${\mathcal O}_\lambda$, then it is easy to check that
  $ \det C_G (u) = F^{\times\,r(\lambda)} $,
with
  $ r(\lambda) $
defined as in the previous section.
Then
%
$$
  \mid 
      {\Bbb Z} / \text{val}(F^{\times\,r(\lambda)})
     \mid =
  \mid 
      {\Bbb Z} / r(\lambda) {\Bbb Z}
      \mid
     = r(\lambda) .
$$
%

It is convenient to combine the $G_1$-unipotent classes in 
  the same $GL(n,F)$-class as follows.
Fix
  $ \lambda \in \Lambda_n $,
and let
  $ \epsilon $
be an unramified character of $F^\times$ 
of finite order
  $ r_{\epsilon} \in {\Bbb N}$.
We assume
  $ r_{\epsilon} $
divides
  $ r(\lambda) $.
Fix a $G$-invariant measure $dg$
on the $G$-conjugacy class
  $ {\mathcal O}_{\lambda} $.
Then set
%
$$
  \mu_{\lambda}^{\epsilon} (f) =
  \int_{ C_G (u) \backslash GL(n,F) }
  \epsilon \, (\det g) f
  (g^{-1} ug)dg.
$$
%
Selecting a different element
 $ u \in {\mathcal O}_{\lambda} $
will alter
  $ \mu_{\lambda}^{\epsilon} $
by an
  $ r_{\epsilon}$-root of unity.
The assumption that
  $ r_{\epsilon} $
divides
  $ r(\lambda) $
is needed to insure that
  $ \mu_{\lambda}^{\epsilon} $
is nonzero. 
This divisibility condition will be used throughout the paper.  A
distribution such as $\mu^\epsilon_\lambda$ will be called
$\epsilon$-invariant, because of its obvious transformation
property under $GL(n,F)$.
Each
  $ \mu_{\lambda}^{\epsilon} $
is a linear combination of invariant distributions
  on the $G_1$-classes in
  $ {\mathcal O}_{\lambda} $.
Conversely each invariant distribution on a unipotent class in $G_1$ 
may be expressed as a linear combination of the distributions
  $ \mu_{\lambda}^{\epsilon} $.
As 
   $ \epsilon $
varies over unramified characters, the number of
  distributions on
  $ {\mathcal H}(B, G)  $ associated with ${\mathcal O}_\lambda$ is $r(\lambda)$.

Notice that our integrals are now in 
  $ GL (n,F) $
instead of
  $ SL(n,F)  $ or $G_1$.
For the rest of the paper, we will work in the group
  $ G = GL (n,F) $.

\proclaim Theorem {3.2}.
The distributions
  $ \bigl\{
      \mu_{\lambda}^{\epsilon} 
    \bigr\}_{\lambda \in 
             \Lambda_n ,
             r_{\epsilon}
            |r(\lambda) } $
are linearly independent on \linebreak
  $ {\mathcal H}(B, G_1) $.
\finishproclaim

The proof of this theorem will occupy the remainder of this section.
We note that this will complete the proof of Theorem 1.

\proclaim Lemma {3.3}.
Let $\{F_i\}$ be a collection of distributions on ${\mathcal H}(B,G_1)$.
For each $i$,
suppose that there is a character $\epsilon_i$ of
$GL(n,F)$ whose order divides $n$ and that $F_i$ is $\epsilon_i$-invariant.  For each
such character $\epsilon$, let $\{F_i\}_\epsilon$ denote the
set $\{F_i\mid \epsilon_i=\epsilon\}$.  Suppose that
$\{F_i\}_\epsilon$ is a set of linearly independent distributions,
for every character $\epsilon$.  Then $\{F_i\}$ is also a set
of linearly
independent distributions.
\finishproclaim

\pproclaim Proof:
Let 
  $ w_{\text{x}} $
be the
  $ n \times n $
permutation matrix with
%
$$
  (w_{\text{x}})_{i, j} =
  \begin{cases}
    1, \quad &\text{if } i + 1 = j, \quad \text{ (indices read modulo $n$)} \\
    0, &\text{otherwise}.
  \end{cases}
$$
%
Then
  $ (w_{\text{x}})^{-1} g_{ij} (w_{\text{x}}) =
     h_{ij} $
with
  $ h_{ij} = g_{i-1, j-1} $
(indices read modulo $n$).

Let
  $ p_0 =
    w_{\text{x}} \,
    \text{diag}
    (\varpi,1, 1, \dots , 1, 1)
    \in GL(n,F) $.
    Then $p_0$ belongs to the normalizer of $B$.
Also
  $ p_0^n =
    (\varpi, \dots , \varpi) 
    \in Z (G) $.
Thus, we obtain an action of
  $ {\Bbb Z} / n {\Bbb Z} $
on $B$, which extends to a linear action $R$ of
  $ {\Bbb Z} / n {\Bbb Z} $
on
  $ {\mathcal H} (B, G_1) $.
Explicitly, $R(i)$ sends the characteristic
function of $BwB$ to the characteristic function of
  ${B w^{p_0^i} B} $.

Set
  $ R_{\epsilon} =
    \frac1n\sum_{i = 1}^n 
    \epsilon (\det p_0^i) R(i) $.
Then $R_\epsilon$ projects the span of $\{F_i\}$ onto the
span of $\{F_i\}_\epsilon$.  Similarly, $R_\epsilon$ projects
a linear relation between the elements of $\{F_i\}$ to a linear relation
between the elements of $\{F_i\}_\epsilon$.  The result
follows easily.
\qed\finishpproclaim

\bigskip
\noindent
\hbox to 8mm{{\bf 3.4.}\hfil} 
We will need the following notation.
Fix a divisor $r$ of $n$.  Fix an unramified character $\epsilon:F^\times\to{\Bbb C}^\times$
whose order $r_\epsilon$ is $r$. 
Set $ rm = n $.
Let $E$ be the unramified extension of $F$ of degree $r$.
Assume that $e$ is a root of unity that generates $E$\ (see [{\bf 6}]),
and that a prime-to-$p$ power of $e$ is $1$.  There is
an embedding
  $ j : E^{\times} \hookrightarrow GL(r,F) $
determined by the basis
  $ v_i = e^i $, 
and by the condition
  $ j(y) v_i = y e^i$, for $y \in E^{\times}$
  and $ i \in\{ 0, \dots , n-1\}$.
Let $H$ denote the group $GL(m,E)$.  The group $H$ will
be identified with the centralizer of the image of $E^\times$
under the ``diagonal'' embedding
%% XX check this comes out ok.
\[
E^\times \longrightarrow GL(r,F)  \longrightarrow
    %\underbrace{m} 
%\to{%
     \underbrace{GL(r,F)\times\dots\times GL(r,F)}_\text{m}\subset GL(n,F).
\]

For $\lambda\in\Lambda_n$, we let $K_\lambda$ denote
the standard parahoric subgroup of $G$.  It satisfies
$B\subseteq K_\lambda\subseteq K$.  Let $\text{char}_\lambda$
denote the characteristic function of $K_\lambda$.
Fix a Haar measure $dg$ on $G$.  Then set
$$f_{\lambda}
     = \frac
           { \text{char}_{\lambda} }
           { \text{vol}(K_{\lambda}, dg) }.$$
  In particular, 
     $ f_{(n)} $ and $f^H_{(m)}$ denote the normalized
     characteristic functions of $K$ and $K\cap H$.  I
     hope that this will not create any confusion later,
     when $f$ and $f^H$ will denote arbitrary 
     smooth functions on $G$ and $H$ with matching
     orbital integrals.
   Define $\pmb\tau_r:\Lambda_m\to\Lambda_{mr}$,
  and $\pmb\sigma_r:\Lambda_m\to\Lambda_{mr}$ by
%% XX check this.
\begin{align*}
    \pmb\tau_r
      (\lambda_1 \dots , \lambda_{\ell} ) &=
      (r \lambda_1, \; r \lambda_2, \dots , r \lambda_{\ell} ), \;\;\;
\\
  \pmb\sigma_r
     (\lambda_1 \dots , \lambda_{\ell} ) &=
  ( \,
%    \undersetbrace{r} \to{%
        \underbrace{\lambda_1, \dots, \lambda_1}_\text{r} , \;
%    \undersetbrace{r} \to{%
        \underbrace{\lambda_2, \dots, \lambda_2}_\text{r} , \dots
%    \undersetbrace{r} \to{%
        \underbrace{\lambda_{\ell}, \dots, \lambda_{\ell}}_\text{r} \, 
  ).
\end{align*}
%
The subscript $r$ will often be dropped.

If $ x$ belongs to $G$, then let $ \mathcal O_x $
    denote the conjugacy class of $x$.
%
Let ${\mathcal O}_{\lambda}$ denote the Richardson class of 
      $ P_{\lambda} $.
%
  For example,  $ {\mathcal O}_{(1^n)}$ is the 
regular unipotent class, and $ {\mathcal O}_{(n)} = \{1\} $.


If $X$ is a set of 
  $ n \times n $
matrices with coefficients in $F$, then write
  $ \tilde{X} \subseteq GL_n ({\Bbb F}_q) $
for the set obtained by reduction of the coefficients of 
  $ X \cap K $
modulo the maximal ideal $(\varpi)$
and let
  $ K_X $ denote the intersection
  $ X \cap K $.
Write
  $ \tilde{x} $
for the reduction of
  an element $ x \in K$
modulo $(\varpi)$.
Let
  $ \text{Mat}_{ab} (F) $
denote the set of 
  $ a \times b $
matrices with  $F$-coefficients, and let
  $ \text{Mat}_a (F) = \text{Mat}_{aa} (F) $.

\proclaim Lemma {3.5}.
Suppose that $\lambda$ and $\xi$ belong to $\Lambda_n$.
Then
  $ \mu_{\lambda}^{\epsilon}
    (\text{{\rm char}}_{\xi})$ is zero
unless
  $ \xi = \pmb\tau_r (\xi') $
for some partition
  $ \xi' \in \Lambda_m $.
\finishproclaim

\pproclaim Proof:
Consider the parabolic subgroup
  $P_{\lambda} = M_{\lambda} N_{\lambda} $.
A function 
  $ \beta_{\epsilon} (n) $ on
  $ N_{\lambda} \cap {\mathcal O}  _{\lambda} $ is
determined up to a constant by the condition
  $ \beta_{\epsilon} (n^m) =
    \epsilon \, (\det m) 
    \beta_{\epsilon} (n)$, for
  $ m \in M_{\lambda} $.
A precise formula for 
  $ \beta_{\epsilon} $
will be given in Section 3.6.
There is a formula for unipotent orbital integrals
given in [{\bf 15}], which will be referred to
in this paper as {\it Howe's normalization} of measures.
For some choice of nonzero constants 
  $ c_{\lambda} $,
depending on measures, we have
%
$$
  \mu_{\lambda}^{\epsilon}
  (\text{char}_{\xi}) =
  c_{\lambda}
  \int_{K,N_{\lambda}} 
  \beta_{\epsilon} (n) 
  \text{char}_{\xi} (n^k) dkdn.
$$

For a unipotent element $u$ in the reduction $\tilde G$ of $G$ modulo $(\varpi)$,
let
  $ N_{\lambda} [u]$ denote the set  
    $\{ n \in N_{\lambda} \cap K \mid \tilde n = u \} $
and let 
  $ a_{\lambda} (u) = \int_{N_{\lambda}[u]}
    \beta_{\epsilon} (n) dn $.
The integral defining $\mu^\epsilon_\lambda(\text{char}_\xi)$ breaks
into a sum
%
$$
  \mu_{\lambda}^{\epsilon} (\text{char}_{\xi}) =
  c_{\lambda}
  \sum_{\substack{
     \text{unipotent} \\
     u \in \tilde G}}
  a_{\lambda} (u)
  \int_K
  \bigl[
     \text{char}
     \tilde P_{\xi}
  \bigr]
  (u^{\tilde k}) dk.
$$
%
Furthermore, if $K(1)$ is defined to be the congruence subgroup
$\{k\in K\mid \tilde k=1\}$, then we have the elementary formula
%
$$
  \int_K
  \bigl[
    \text{char}\,\tilde P_\xi
  \bigr]
  ( u^{\tilde k})
  dk =
  \text{vol}
  \bigl(
    K(1), dk
  \bigr)
 \mid 
  \{
    g \in \tilde G \mid u^g \in \tilde P_{\xi}
  \}\mid.
$$

If
  $ \xi = (\xi_1, \xi_2, \dots ,\xi_{\ell} ) \in\Lambda_n$ is an
  ordered partition,
and if
  $ \theta: \{ 1, \dots , \ell \} \to
    \{ 1, \dots , \ell \} $
is a permutation, then we use the nonstandard convention of denoting by
  $ P_{\xi, \theta} $
the parabolic subgroup associated with the sequence
  $ (\xi_{\theta(1)},
     \xi_{\theta(2)}, \dots ,
     \xi_{\theta(\ell)} ) $.
Similar conventions apply to 
  $ K_{\xi,\theta} $.

The cardinality of
  $ \{ g \in \tilde G \mid u^g \in \tilde P_{\xi,\theta} \} $
is known to be independent of the order
  $ \theta $
on the factors for all orderings
  $ \theta $ (see [{\bf 3}]).
The lemma now follows if we find 
  $ g \in G $
such that
  $ K_{\xi}^g  = K_{\xi,\theta} $
and
  $ \epsilon \, (\det g) \neq 1 $.
For then 
%
$
   \mu_{\lambda}^{\epsilon} (\text{char}\,K_{\xi,\theta}) $
equals
    $ \epsilon \,(\det g) \mu_{\lambda}^{\epsilon} 
      (\text{char}\, K_{\xi})$, and this forces
    $\mu_{\lambda}^{\epsilon} (\text{char}\, K_{\xi})$ to be zero.
Let
  $ w_{\text{x}} $ be the permutation matrix defined in 
  the proof of Lemma 3.3.
Set
%% XX check this converted properly
  $ p_1 = w_{\text{x}}^{\xi_1} \; \text{diag} 
     ( 
      \overbrace{\vphantom{1}
         \varpi,\dots , \varpi,}^{\xi_1} \,1,\dots,1) $.
Then
  $ p_1 K_{\xi} p_1^{-1} $
is the parahoric with ``Levi factor''
  %
$$
  GL(\xi_2) \times \dots \times 
  GL(\xi_n) \times
  GL(\xi_1),
$$
%
and the determinant of
  $ p_1 $ is
  $\pm\varpi^{\xi_1} $.
In a similar manner, we form
  $ p_i $,
  for $ i = 1, \dots , \ell $.
This gives the parahoric subgroups
  $ p_i \dots p_1 \;
    K_{\xi}
   p_1^{-1} \dots p_i^{-1} $
with corresponding factor
%
$$
  GL(\xi_{i+1})
  \times \dots \times
  GL(\xi_n) \times
  GL(\xi_1)
  \times \dots \times GL(\xi_i).
$$
%
Moreover, the determinant of 
  $ p_i \dots p_1 $ is
  $ \pm\varpi^{\xi_1 + \dots + \xi_i} $.
Suppose that $\xi$ does not have the form $\pmb\tau_r(\xi')$, for any $\xi'\in \Lambda_m$.
Then, for some $i$, the integer
  $ r$ does not divide $\xi_1 + \dots + \xi_i $,
and so
  $ \epsilon \, (\det p_i \dots p_1) \neq 1 $.
Then set
  $ g^{-1} = p_i \dots p_1 $.
%
\qed
\finishpproclaim

\bigskip
\noindent
\hbox to 8mm{{\bf 3.6}\hfil}
Next we specify a normalization of the distributions
$ \mu_{\lambda}^{\epsilon} $,
which we will denote
  $ \mu_{\lambda}^{\epsilon, \text{alg}} $
(for algebraic).
Let
  $d n $
(resp. $d \bar n $)
denote the Haar measure on 
  $ N_{\lambda} $
(resp. $\bar N_{\lambda} $)
assigning mass $1$ to 
  $ N_{\lambda} \cap K $
(resp. $ \bar N_{\lambda} \cap K $).


For every $i$ define
  $ x_i : N_{\lambda} \to 
    \text{Mat}_{\lambda_i, \lambda_{i+1}} (F) $
to be the projection onto the $i$th superdiagonal
  $ \lambda_i \times \lambda_{i+1} $
block of 
  $ N_{\lambda} $.

\noindent
{\bf Example}.
If 
  $ \lambda = (3, 3, 2, 2, 2, 2) $,
then
there are five superdiagonal blocks $x_1,\ldots,x_5$ of shapes
$3\times 3$, $3\times 2$, $2\times 2$, $2\times 2$ and $2\times 2$.
There is a well-defined determinant on all but the second block.

\noindent
Set
  $ \beta_{\epsilon} (n) =  
    \prod_{i=1}^{\ell-1} \epsilon^{-i} 
    \bigl(
       \det x_i (n)
    \bigr) $, for $n\in N_\lambda\cap{\mathcal O}_\lambda$.
Notice that $ \epsilon^i = 1 $,
whenever
  $ x_i $
is not square
  $ (\lambda_i \neq \lambda_{i+1} ) $, provided that
  $r_\epsilon$ divides $r(\lambda)$.
Consequently, the product giving
  $ \beta_{\epsilon} (n) $
is well defined.
Also, if
  $ m =
     (m_1, \dots, m_{\ell}) \in M_{\lambda} $,
then
%
%XX check formatting
\begin{align*}
\beta_{\epsilon} (n^m) &=
    \prod_{i=1}^{\ell-1} 
    \epsilon^{-i}
    \bigl(
      \det 
         m_i^{-1} 
            x_i (n)
            m_{i+1}
     \bigr)
\\
\vspace{4pt}
&=
    \prod_{i=1}^{\ell} 
    \epsilon^i
    (\det m_i)
    \prod_{i=1}^{\ell} 
    \epsilon^{1-i} 
    (\det m_i) 
    \beta_{\epsilon} (n)
\\
\vspace{4pt}
&=
   \epsilon
   (\det m)
   \beta_{\epsilon} (n).
\end{align*}
%
   We have used the identity $\epsilon^{\ell} = 1 $.
\noindent
We now give the definition.
%
$$
  \mu_{\lambda}^{\epsilon,\text{alg}}
  (f) =
  \int_{N_{\lambda},\bar N_{\lambda}}
  \beta_{\epsilon} (n) 
  f (n^{\bar n}) d nd \bar n.
$$
%
The function $\beta_{\epsilon} (n) $ is undefined on 
a set of measure zero.
Similarly, for $ \lambda' \in \Lambda_m $,
we define the integral
%
$$
  \mu_{\lambda'}^{\text{st}, \text{alg}}
  (f^H) =
  \int_{ N_{\lambda'}, \bar N_{\lambda'} }
  f^H(n^{\bar n}) d n d \bar n, \quad \text{ for }f^H\in C_c^\infty(H),
$$
%
where
  $ P_{\lambda'} = 
    M_{\lambda'} 
    N_{\lambda'} 
    \subseteq
    H$, \ 
    $\int_{N_{\lambda'}\cap K_H}
    d n = 1 $, and so forth.

\proclaim Lemma {3.7}.
  $ \int
    \beta_{\epsilon} (n) f 
    (n^{\bar n}) 
    d n d \bar n $
is an 
  $ \epsilon$-invariant integral.
\finishproclaim

\pproclaim  Proof:
We have already checked correct transformation of
  $ \beta_{\epsilon} (n) $.
It remains to be seen that
  $ d n d \bar n$
is an invariant measure on
  $ {\mathcal O}_{\lambda} $.
The measure $dn\,d\bar n$ is clearly invariant
by $M_\lambda$ and $\bar N_\lambda$.  The group
$G$ is generated by $M_\lambda$, $\bar N_\lambda$ and
the root spaces of the simple positive roots contained
in $N_\lambda$.  The invariance of $dn\,d\bar n$
by such a root space is an elementary calculation,
which reduces quickly to a calculation inside a group of rank $1$.
\qed
\finishpproclaim

For 
  $ \zeta \in \Bbb C $,
set
  $ \phi_j (q,\zeta) =
    ( 1 - \zeta q^{-j})
    ( 1 - \zeta q^{-j+1}) \dots
    ( 1 -  {\zeta}q^{-1}) $.
For any partition
  $ \lambda \in \Lambda_n $,
set
  $ \phi_{\lambda} (q, \zeta) = 
    \prod_{i=1}^{\ell} \phi_{\lambda_i}
    (q, \zeta) $.

\proclaim Lemma {3.8}.
Let
  $ \epsilon_0 $
be an unramified quasicharacter of $F^\times$ with
  $ \epsilon_0 (\varpi) = \zeta $.  Let
  $ d x $ be
an additive Haar measure on 
  $ \text{{\rm Mat}}_j (O_F) $
of mass $1$.
Then
%
$$
  \int_{ \text{{\rm Mat}}_j (O_F)}
  \epsilon_0 (\det x) d x =
  \frac
     {\phi_j (q, 1)}
     {\phi_j (q, \zeta)}.
$$
%
\finishproclaim

\noindent
The right-hand side is the principal value of the
  left-hand side  whenever
  $ \phi_j (q,\zeta) \neq 0 $.

\pproclaim Proof:
Expand the right-hand side, for small
  $ \zeta  $, as
%
$$
%  \undersetbrace \to {%
    \underbrace{\Bigl(
       \frac {q^j -1}{q^j}
    \Bigr)
    \Bigl(
      1 +
      \frac {\zeta}{q^j} +
      \frac {\zeta^2}{q^{2j}} + \dots
    \Bigr)
  }_{\text{Factor $1$}} \;\;
  \underbrace %\to{%
    {\Bigl(
       \frac {q^j -q}{q^j}
    \Bigr)
    \Bigl(
      1 +
      \frac {\zeta}{q^{j-1}} +
      \frac {\zeta^2}{q^{2(j-1)}} + \dots
    \Bigr)
  }_{\text{Factor $2$}}
\dots
    \underbrace %\to {%
       {\Bigl( 
          \hskip .5in
       \Bigr).
    }_{\text{Factor $j$}}
$$
%
The result will follow immediately from the correct interpretation
of each term.
Factor $i$ is the contribution of the $i$th column of the matrix
  $x$, given that the previous
  $ i - 1 $ 
columns have already been selected.
For any vectors
  $ v_1, \dots , v_i \in O_F^j $,
let
  $ \text{val}_i (v_1, \dots , v_i) $
be the largest value of 
  $ i' \in {\Bbb Z} $
such that an 
  $ i \times i $
minor in the 
  $ j \times i $ 
matrix
  $ [v_1, \dots , v_i ] $
has valuation $i'$.  
Then the volume (with respect to additive Haar measure
  of mass  $1$ on
  $ O_F^j $) of
%
  $$\{ v_{i+1} \in O_F^j \mid
   \text{val}_{i+1} (v_1 , \dots, v_{i+1}) -
    \text{val}_i (v_1 , \dots, v_i) = \ell\}$$
is a constant independent of 
  $ v_1, \dots , v_i \in O_F^j $.
This volume is equal to
  $ (q^j-1)/q^{(j-i)\ell + j}$.
This is the coefficient of 
  $ \zeta^{\ell} $
in the 
  $ (i+1)$st factor above.
The result follows.
\qed\finishpproclaim
 
\proclaim Lemma {3.9}.
If
   $\pmb\sigma \, (\lambda') = \lambda $,
then
  $ \mu_{\lambda}^{\epsilon, \text{{\rm alg}} }
      (f_{(n)}) =
    \mu_{\lambda'}^{\text{{\rm st,alg}}}
      (f^H_{(m)})$.
\finishproclaim

\pproclaim Proof:
We compute both integrals.
It is more convenient to work with a different normalization.
By Howe [{\bf 15}],
%
$$
  \mu_{\lambda}^{\epsilon,\text{x}}
  (f) =
  \int_{ K, N_{\lambda} }
  \beta_{\epsilon} (n) 
  f (k^{-1}nk) dkdn, \ \text{ for }
  f \in 
  C_c^{\infty} (G),
$$
%
is an $\epsilon$-invariant distribution, and
%
$$
  \mu_{\lambda'}^{\text{st,x}} (f^H) =
  \int_{K_H,N_{\lambda'}}
  f^H
  (k^{-1}nk)
  dkdn 
$$
%
is an invariant distribution.
Thus, they equal
  $ \mu_{\lambda}^{\epsilon, \text{alg}}$ and
    $\mu_{\lambda'}^{\text{st,alg}} $
up to scalars.
Assume that $dk$ on $K$ is normalized to give $K$ mass $1$, and
that $dk$ on
  $ K_H \subseteq H $
is normalized to give 
  $ K_H $
mass $1$.
(These are separate measures--one on $K$ and one on
  $ K_H $
--but the same notation is used for both.)

To compute these scalars, we consider the characteristic function $\chi$
 of 
%
$$
   \{
      n^{\bar n} \mid
      n \in N_{\lambda} \cap K, \;
      \bar n \in \bar N_{\lambda} \cap K 
    \}.
$$
%
Then 
  $ \mu_{\lambda}^{\epsilon,\text{alg}}
    (\chi) =
    \int_{ N_{\lambda}\cap K }
    \beta_{\epsilon} (n) d n $.

If
  $ \chi (n^k) = 1 $ and $n\in {\mathcal O}_\lambda\cap N_\lambda$,
then one easily checks that
  $ n$ belongs to $N_{\lambda} \cap K $,
and
  $k$ belongs to  $(P_{\lambda} \cap K)
          (\bar N_{\lambda} \cap K) $.
Then 
%
$$
  \mu_{\lambda}^{\epsilon,\text{x}} (\chi) =
  \int_{N_{\lambda}\cap K}
  \beta_{\epsilon} (n) d n
  \int_{ (P_{\lambda}\cap K) 
         (\bar N_{\lambda}\cap K) }
  dk =
  \mu_{\lambda}^{\epsilon,\text{alg}} (\chi) \,
  \frac
    { \phi_\lambda(q,1)} {\phi_n(q,1)}.
$$
Similarly, one finds
%
$$
  \mu_{\lambda'}^{\text{st,x}} =
  \mu_{\lambda'}^{\text{st,alg}} \,
  \frac {\phi_{\lambda'}(q^r,1)} {\phi_m(q^r,1)}.
$$
Now we evaluate
  $ \mu_{\lambda}^{\text{st,alg}} (f_{(n)})$. 
%
\begin{align*}
 \mu_{\lambda}^{\epsilon,\text{alg}} (f_{(n)})
&=
  \frac {\phi_n(q,1)}{\phi_\lambda(q,1)}
  \frac
     {  \mu_{\lambda}^{\epsilon,\text{x}} (\text{char}_{(n)})  }
     {  \phi_n (q,1)  },  \\
\vspace{4pt}
  \mu_{\lambda}^{\epsilon,\text{x}} (\text{char}_{(n)})
&=
  \int_K dk 
  \int_{ N_{\lambda} \cap K }
  \beta_{\epsilon} (n)  d n =
  \prod_{i=1}^{\ell} 
  \int
  \epsilon^{-i}
  (\det x_i)
  d x_i.
\end{align*}
%
Here 
  $ d x_i $
is a measure of mass $1$ on
 $ \text{Mat}_{\lambda_i} (O_F) $. By Lemma 3.8,
%
\begin{align*}
%\spreadlines{7pt}
  \prod_{i=1}^{\ell}
  \int
  \epsilon^{-i} (\det x_i)
  d x_i 
&=
  \prod_{i=1}^{\ell}
  \frac
     { \phi_{\lambda_i} (q,1)  }
     { \phi_{\lambda_i} (q,\zeta^{-i})  } 
= 
  \frac
     { \phi_{\lambda} (q,1)  }
     { \phi_{\lambda'} (q^r,1)  }. 
\\
\vspace{8pt}
  \mu_{\lambda}^{\epsilon,\text{alg} }
  (f_{(n)})
&=
  \frac{1} {\phi_{\lambda'}(q^r,1)}. 
\end{align*}
Similarly,
$$
  \mu_{\lambda}^{\text{st,alg}}
  (f_{(m)}^H)
=
  \frac 
  { \phi_m(q^r,1) } {\phi_{\lambda'}(q^r,1)
        \text{vol}(K_H, dh)   } \;\;
   \int_{  N_{\lambda'}\cap K_H }  
   d n
=
   \frac 1
       { \phi_{\lambda'} (q^r,1)   }.
$$
\qed\finishpproclaim
\bigskip

\proclaim Proposition {3.10}.
If
  $ \xi', \lambda' \in \Lambda_m $,
  $ \pmb\tau_r (\xi') = \xi $, and
  $ \pmb\sigma_r (\lambda') = \lambda$,
then
  $ \mu_{\lambda}^{\epsilon,\text{alg}}
    (f_{\xi}) = \mu_{\lambda'}^{\text{st,alg}}
    (f_{\xi'}^H) $.
\finishproclaim

\proclaim Corollary {3.11}.
The distributions 
  $ 
      \mu_{\pmb\sigma (\lambda')}^{\epsilon}
    $, for $\lambda'\in\Lambda_m$,
are linearly independent on
  $ {\mathcal H} (B, G_1) $.
\finishproclaim

\pproclaim Proof of Corollary 3.11:
If
  $ 0 = 
    \sum
    a_{\lambda'}
    \mu_{\pmb\sigma(\lambda')}^{\epsilon,\text{alg}}
    (f_{\xi}) =
    \sum
    a_{\lambda'}
    \mu_{\lambda'}^{\text{st,alg}}
    (f_{\xi'}^H) $,
then
  $ a_{\lambda'} = 0 $,
because 
  $ \mu_{\lambda'}^{\text{st,alg}} $,
  for $ \lambda' \in \Lambda_m $,
are known to be linearly independent on the span of
  $ f_{\xi'}^H $, for
  $ \xi' \in \Lambda_m $
(see Waldspurger [{\bf 31}]).
\qed\finishpproclaim

\pproclaim Proof of Proposition 3.10:
If
  $ r_{\epsilon} = r = n $, and 
  $ \xi = \pmb\tau (\xi') $, then
  $ \xi = (n) $
and
  $ \xi' = (1) $.
The proposition in this case follows from
Lemma 3.9.
Thus, we assume 
  $ r_{\epsilon} = r < n $ and 
  $ m > 1 $.
To avoid trivialities, also assume
  $ r_{\epsilon} = r \neq 1 $.

By induction we may assume that the proposition is true for
  all linear groups of smaller rank, in particular for all of
  the proper Levi factors of 
  $ G = GL (n) $.
The induction is started on 
  $ GL(p') $, for 
a prime $p'$.  In this case, either
  $ r_{\epsilon} = 1 $,
in which case the proposition is a tautology, or
  $ r_{\epsilon} = p' = r = n $,
a case the previous paragraph treats.
We note that the corollary is the only remaining step of the
  proof of Theorem 1, and that this theorem implies the existence
  of a uniform germ expansion. 
Thus, we may assume by induction that a uniform germ expansion holds
  on all of the proper Levi subgroups of
  $G$.

Suppose it is shown that
  $ \mu_{\lambda}^{\epsilon,\text{alg}} (f_{\xi}) =
    \text{c}_{\lambda}
    \mu_{\lambda'}^{\text{st,alg}}
    (f_{\xi'}^H) $
for some scalar 
  $ \text{c}_{\lambda} $
independent of
  $ \xi' \in \Lambda_m $ but depending
  possibly on $\lambda$.
Then, setting
  $ \xi'= (m) $,
  $ \xi = (n) $
and using Lemma 3.9, we find that
  $ \mu_{\lambda}^{\epsilon,\text{alg}}
    (f_{\xi}) = 
    \mu_{\lambda'}^{\text{st,alg}} 
    (f_{\xi'}^H) $.
So it is enough to establish this weaker identity.

Fix $\lambda'$, write $\lambda''=\pmb\tau(\lambda')$, and
write $P=P_{\lambda''}=MN$.  Fix $\xi=\pmb\tau(\xi')$.  In order to
prove the equality $\mu_\lambda^{\epsilon,\text{alg}}(f_\xi) = c
\mu^{\text{st,alg}}_{\lambda'}(f^H_{\xi'})$, we show that for an
appropriate element $z$, both
sides equal $\bar f^P_\xi(z)$, up to a constant.
In the following
lemmas the constants $c_i$ will be independent of $\xi$, but
will depend possibly on $\lambda$.  Lemmas 3.12, 3.13 and 3.14 tacitly assume
that the residual characteristic is sufficiently large.  
This is harmless, because Proposition 3.10 is an equality between
two rational functions of $q$.  Now
fix $z$ to be a regular element of finite order in $H\cap K$, 
whose centralizer is $H$,
and whose order is prime to $p$.

\proclaim{Lemma 3.12}.  There is a nonzero constant $c_0$ such that
$\mu_{\lambda'}^{\text{st,alg}}(f^H_{\xi'}) = c_0 \bar f_\xi^P(z)$.
\finishproclaim

\pproclaim{Proof}:  Set $N'=N\cap H$, and $K'=K\cap H$.  The unipotent
class of $H$ dense in $N'$ is ${\mathcal O}_{\lambda'}$.
Use Howe's normalization
of measures to write 
$$
\mu^{\text{st,alg}}_{\lambda'}(f^H_{\xi'}) = c_1\int_{N',K'} 
	f_{\xi'}^H (k^{\prime\, -1}n'k')dn'\,dk' = c_1\int_{N',K'}
	f_{\xi'}^H (k^{\prime\,-1}zn'k')dn'\,dk'.$$
%
Since $H\cap K_\xi=K'_{\xi'}$, the integral may be replaced with
\begin{align*}
c_2\int_{N',K'}f_\xi(k^{\prime-1}zn'k')dn'dk' &=
 c_3\int_{N',K',K_\xi} f_\xi(k^{-1}k^{\prime-1}zn'k'k) dn'dk'dk\\&=
 c_4\int_{N',K} f_\xi(k^{-1}zn'k)dn'dk.
\end{align*}
By Kazhdan's lemma, as given in [{\bf 31}], 
the integral over $K'$ and $K_\xi$ may be replaced  with
an integral over $K$.  

Kazhdan's lemma will be used several more times in this paper, and
we make a digression to explain it in more detail.
We say that an element in $GL(n,O_F)$ is {\it topologically unipotent} if
$\lim_{i\to\infty}\gamma^{q^i} = 1$.  We say that an element in $GL(n,O_F)$
is {\it absolutely semisimple} if $\gamma^c=1$, where $c = (q^n-1)\cdots (q-1)$.
Any semisimple element in $K$ has a topological Jordan decomposition
$\gamma=\gamma_s\gamma_u$ into commuting absolutely semisimple and topologically
unipotent elements $\gamma_s$ and $\gamma_u$.  Kazhdan's lemma states
that if $z$ is an absolutely semisimple element in $E^\times\cap K_\xi$
(with the embedding of $E^\times$ given in Section 3.4), then $z^g\in K_\xi$
if and only if $g\in C_G(z)K_\xi$.  For a proof see [{\bf 31}].

The integral that results from the application of Kazhdan's lemma is nearly $\bar f_\xi^P(z)$,
except that, in the integral defining $\bar f_\xi^P(z)$, integration
extends over $N$, not merely over $N'=N\cap H$.  In view of the support
of $f_\xi$, we must show that the integral over $N'$ and $K$ remains unchanged
if we integrate over $N\cap K$ instead of $N\cap K\cap H$.  This
is accomplished if we show the equality of two sets $X_0$ and $X_0'$:
\begin{align*}
   	X'_0 &= \{(zn)^k \mid k\in K,\ n\in N'\cap K\},\\
	X_0 &= \{ (zn)^k \mid k\in K, \ n\in N\cap K\}.
\end{align*}
%
To see that $X_0'\subseteq X_0$, it is enough to observe that
$N'\subseteq N$.  

Now we show $X_0\subseteq X_0'$.  For this, it is enough to show
that the map $$\varphi:(N'\cap K)\times (N\cap K)\to z(N\cap K),\qquad
      \varphi(n',n) = (zn')^n$$
is onto.  This calculation is best understood by considering the
example $m=2$, and $\lambda'=(1,1)\in \Lambda_2$.
We write a matrix in $\text{Mat}_{2r}(F)$ in the form
$\begin{pmatrix}  a&b\\c&d\end{pmatrix}$, with $a,b,c,d\in\text{Mat}_r(F)$.
The $2\times2$ matrix calculation
%
$$\begin{pmatrix}  1&-X\\0&1\end{pmatrix}
  \begin{pmatrix}  z&y\\0&z\end{pmatrix}
  \begin{pmatrix}  1&X\\0&1\end{pmatrix} =
  \begin{pmatrix}  z& zX-Xz+y\\0&z\end{pmatrix},$$
 %
  with 
  $n=\begin{pmatrix} 1&X\\0&1\end{pmatrix}$ and 
  $zn'=\begin{pmatrix}  z&y\\0&z\end{pmatrix}$,
%
shows that the following lemma is needed for $\varphi$ to be onto.

\proclaim Lemma {3.13}.
Let $E$ be the unramified extension of $F$ of degree $r$.
Select $e\in E$, as in Section 3.4.  Select an embedding
  $ j : E \hookrightarrow \operatorname{Mat}_r (F) $ defined
  over $O_F$,
as in\/ {\rm 3.4}.
Set $ j(e) = z$.
Then the $F$-linear map
  $ L :
    E \oplus 
    \bigl(
      \operatorname{Mat}_r (F) \big/ j (E)
    \bigr)
    \to
    \operatorname{Mat}_r (F) $
given by 
  $L (y,X) =z X - X z + j (y) $
sends
  $ O_E \oplus \operatorname{Mat}_r (O_F) $
{\rm onto}
  $ \operatorname{Mat}_r (O_F) $.
\finishproclaim

\pproclaim Proof:
The linear map
  $L$ is surjective by dimension counts, since
  $ \ker (L) = 
    \bigl(
       0, j (E)
    \bigr) $.
The matrix of $L$ has entries in
  $ O_F $.
Thus, the claim follows if we show the determinant of $L$ is a unit.
Fix a basis defined over $E$ such that 
  $ z = (z_1,\ldots,z_r) $ 
is diagonal.
Then, for
  $ X= (x_{ij}) \in \operatorname{Mat}_r (F) $,
  we find $ L(0,x_{ij}) = (z_i - z_j) x_{ij} $.
By the hypotheses of Section 3.4 on $e$, we have
  $ |z_i - z_j| = 1 $.
So 
  $ \det (L) $
is a unit.
\qed
\finishpproclaim

Even when $m$ is not $2$, repeated application of this lemma
proves that the map $\varphi$ is onto.  This completes the
proof of Lemma 3.12.
\qed
\finishpproclaim

The following lemma, combined with Lemma 3.12, completes the proof
of Proposition 3.10.  Let $\lambda'$, $\lambda''$, $\xi'$, $\xi$, $P$,
$M$ and $N$ retain their meaning from Lemma 3.12.

\proclaim{Lemma 3.14}.  
$$\mu_\lambda^{\epsilon\text{,alg}}(f_\xi) = c_5 \bar f^P_\xi(z),$$
for some constant $c_5$.
\finishproclaim

\pproclaim{Proof}:  In addition to the parabolic subgroup
$P=P_{\lambda''}=MN$ already introduced, we make use of the parabolic subgroup
$P_{\lambda_0}=M_0N_0\subseteq P$, where $\lambda_0$ is the
partition $(r,\ldots, r)\in \Lambda_n$.  The maximally split
Cartan subgroup $T$ of $H$ is isomorphic to $(\text{Res}_{E/F}{\Bbb G}_m)^m$,
and $T$ may be identified with a Cartan subgroup in $M_0$ containing $z$.
Here, as in Lemma 3.12, $z$ is a prime-to-$p$ root of unity, considered
as a central element of $H$ in $G$.  Note that $M_0$ may be identified
with the centralizer of an appropriate element 
$(x_1,\ldots,x_m)\in F^{\times\,m}\subseteq
E^{\times\,m} = T(F)$.  Set $K_0 = K\cap M_0$.

The support of $f_\xi$ is contained in $K$.  If $(z^mn)^k$ lies in $K$,
with $m\in M_0$, $n\in N$, and $k\in K$, then $z^mn\in K$.  It also follows
 that $z^m\in K_0$.  This implies, by the version of Kazhdan's lemma
 described in the proof of Lemma 3.12,
 that $m$ belongs to $T(F)K_0$ (since $T$ is the centralizer of $z$ in $M_0$).
 Notice that $\det T(F)$ consists of norms from $E^\times$, so that
 $\epsilon$ is trivial on the determinant of any $m\in M_0$ such that $z^m\in K_0$.
 We conclude then from the definition of $\bar f_\xi^P$ that
 $$\epsilon(\det m)\bar f_\xi^P(z^m) = 
  \begin{cases} 
	\bar f^P_\xi(z), & \text{ if }m\in T(F)K_0,\\
	0, & \text{ otherwise. }\end{cases}
	$$
Thus $\bar f_\xi^P(z)$ is a constant independent of $\xi$ times
$$\int_{T\backslash M_0} \epsilon(\det m) \bar f_\xi^P(z^m) dm.$$
By the induction hypothesis made above, there exists a
uniform germ expansion for $M_0$, for functions in ${\mathcal H}(B_{M_0},M_0)$,
and more generally for functions (such as $\bar f_\xi^P$) in ${\mathcal H}^i
(B_{M_0},M_0)$ (see Lemma 1.7).  
Note that $M_0$ is a proper Levi subgroup of
$GL(n,F)$ when $\epsilon\ne 1$.  

In the germ expansion of this integral, all
$\epsilon$-invariant orbital integrals vanish except for the one
corresponding to the regular unipotent  class.  To see this, note that $M_0$
is a product of factors $GL(r,F)$ and that $\epsilon$ has order
$r=r_\epsilon$.  In the discussion following Lemma 3.1, it was observed
that an $\epsilon$-invariant unipotent orbital integral, 
associated with a partition
$\lambda\in\Lambda_r$, is zero unless $r_\epsilon$ divides $r(\lambda)$.
The only partition of $r$ with this property is $\lambda=(1,\ldots,1)$,
and this corresponds to the regular unipotent class of $GL(r,F)$.
We may choose to take the germ expansion about a diagonal central element $z_0$
of $M_0$, whose centralizer is $M_0$.  Assume that
a prime-to-$p$ power of $z_0$ is 1.  Again, the germ expansion has a single term.

Using the definition of $\bar f_\xi^P(z)$ and Howe's normalization of
the regular unipotent orbital integral on $M_0$, we find that
$\bar f_\xi^P(z)$ is a constant times
$$\int_{U,K_0} \beta^0_\epsilon(u)\bar f_\xi^P((z_0u)^{k_0}) du\,dk_0 =
	c_6\int_{U,N,K} \beta_\epsilon^0(u) f_\xi((z_0un)^k) du\,dn\,dk.$$
Here $U$ is the radical of the upper-triangular Borel subgroup
in $M_0$, and $\beta_\epsilon^0$ is the equivalent of $\beta_\epsilon$,
for the group $M_0$.

We must introduce a third parabolic subgroup $P_{\pmb\sigma(\lambda')} =
P_\sigma=M_\sigma N_\sigma$ (in addition to the parabolic subgroups
$P$ and $P_{\lambda_0}$ already in use).  Note that the
projection of $U$ to a factor $GL(r\lambda'_i,F)$ of $M$ has
dense orbit ${\mathcal O}_{\pmb\sigma(\lambda'_i)}$.  This projection
of $U$ to $GL(r\lambda'_i,F)$ is conjugate under $GL(r\lambda'_i,F)$ to
a subset of the projection of $M\cap N_\sigma$ to $GL(r\lambda'_i,F)$.  
In fact,
there exists a permutation matrix $x\in M\cap K$ such that $U^x$ is contained
in $M\cap N_\sigma$.  Set
$\hat N_\sigma = U^xN\subseteq N_\sigma$.  One can check that
an open subset of $\hat N_\sigma$ belongs to ${\mathcal O}_\lambda = {\mathcal O}_{\pmb\sigma(\lambda')}$
and that $\beta^0_\epsilon(u) = \beta(n)$, if $n$ is the element
$u^xn_1\in U^xN \cap
{\mathcal O}_{\lambda} = \hat N_\sigma\cap {\mathcal O}_\lambda$.

To complete the proof, we must show that the value of the integral
$$\int_{\hat N_{\sigma},K} \beta_\epsilon(n) f_\xi((z_0^xn)^k) dn\,dk$$
is changed only by a constant, if $\hat N_\sigma$ is replaced with
$N_\sigma$.  The argument is similar to the argument used in the
proof of Lemma 3.12.  It relies on the equality of the two sets:
$$z_0^x(N_\sigma\cap K)\ \text{ and } \{(z_0^xn)^k\mid n\in \hat N_\sigma\cap K,
\quad \text{and } k\in N_\sigma\cap K\}.$$
After $\hat N_\sigma$ is replaced by $N_\sigma$, Lemma 1.7 allows us to drop
$z_0^x$ from the integral.  We omit the remaining details, which are routine.
\qed\finishpproclaim

\section{{\bf 4@.  An interpretation of descent and homogeneity.}}

\medskip
\noindent
The work of Waldspurger gives an interpretation of the identity
  $$ \mu_{\pmb\sigma(\lambda')}^{\epsilon,\text{alg}} 
    (f_{\pmb\tau(\xi')}) =
    \mu_{\lambda'}^{\text{st,alg}}(f_{\xi'}^{H}) $$
in terms of the Hall algebra.
Zelevinsky defines three bases
  $ x_{\lambda} $,
  $ y_{\lambda} $, and
  $ z_{\lambda} $,
  for $\lambda \in \cup_n \Lambda_n$,
of the PSH-algebra $R\otimes_{\Bbb Z} {\Bbb Q}$.
There is an inner product on the PSH-algebra depending on $q$. It comes
from an identification of the PSH-algebra and the Hall algebra.
We let
  $ \langle \;,\; \rangle_q $
denote this inner product.
There exists an endomorphism
  $ \pmb\tau $
of 
  $ R \otimes_{\Bbb Z} {\Bbb Q} $
given on the basis
  $ z_{\lambda} $
by
  $ \pmb\tau (z_{\lambda}) =
    z_{\pmb\tau_r (\lambda) } $.
See [{\bf 31}] and [{\bf 33}] for details.


\proclaim Proposition {4.1}.
%
\begin{alignat*}{2}
  \mu_{\pmb\tau(\lambda')}^{\epsilon, \text{{\rm alg}}} (f_{\xi})
&=
  \langle
    x_{\xi}, \pmb\tau (y_{\lambda'}) 
  \rangle_q, \quad
&&\text{{\rm if}} \quad
  \lambda' \in \Lambda_m \text{\rm \ and }
  \xi \in \Lambda_n.
\\
  \mu_{\lambda'}^{\text{{\rm st,alg}}} (f^H_{\xi'})
&=
  \langle
    x_{\xi'}, y_{\lambda'} 
  \rangle_{q^r}, \quad
&&\text{{\rm if}} \quad
  \lambda', \xi' \in \Lambda_m.
\end{alignat*}
%
\finishproclaim

\pproclaim Proof:
  This follows easily from the results of
  [{\bf 31}] combined with Proposition 3.10, adjusting
for different normalizations of measures.
\qed
\finishproclaim

Next we consider the homogeneity of germs on topologically 
  unipotent elements.
Write
  $ \gamma = 1 + t^2 X  \in \operatorname{Mat}_n (O_F) $
and
  $ \gamma_0 = 1 + X \in \operatorname{Mat}_n (O_F) $.
Write $\Gamma_\lambda^\epsilon(\gamma)$ for the Shalika germ
corresponding to the unipotent distribution $\mu_\lambda^{\epsilon,\text{alg}}$.

\proclaim Proposition {4.2}.
Assume that 
  $ \gamma =1+t^2X$
and
  $ \gamma_0 =1+X$
are topologically unipotent elements in $K$ and that 
  $ t \in O_F $.
Then the germs satisfy
  $ \Gamma_{\lambda}^{\epsilon} (\gamma) =
    \Gamma_{\lambda}^{\epsilon} (\gamma_0) 
    |t|^{2n (\lambda)} $,
for appropriate integers
  $ n(\lambda) $.
\finishproclaim

\pproclaim Proof:
The proof of  [{\bf 31}] goes through with the following
  modification.
Since we include a factor
  $ \epsilon $, 
we replace $t$ with $t^2$ in both the statement of the 
  proposition and in the proof.
It is then easily verified that
  $ \beta_{\epsilon} (1 + t^2 N) = 
    \beta_{\epsilon} (1 + N) $, where
  $ 1 + N $ is
an element of the Richardson class of 
  $ O_{\lambda} $.
The character 
  $ \epsilon $
does not enter otherwise into the proof.
\qed
\finishpproclaim

\section{{\bf 5@.  Transfer factors.}}

\medskip
\noindent
We are concerned with the transfer factors only for strongly compact
  elements in
  $ GL (n) $.
This permits a simplification of the general results of
  Langlands and Shelstad on transfer factors [{\bf 20}], [{\bf 21}].

Suppose that $f$ and $g$ are monic polynomials with roots
  $ x_1, \dots , x_m $ 
and
  $ y_1, \dots , y_{m'} $,
respectively.
The resultant
  $ R(f,g) $
of $f$ and $g$ is the product
  $ \prod_{i,j}^{m,m'} (x_i-y_j) $.
If
  $ \gamma_1, \gamma_2 $
are elements of 
  $ \operatorname{Mat}_{\ell} $, then
let
  $ R (\gamma_1, \gamma_2) =
    R (f_{\gamma_1}, f_{\gamma_2} ) $,
where 
  $ f_{\gamma} $
is the characteristic polynomial of 
  $ \gamma $.
Let 
  $ E / F $ 
be an unramified extension of $F$ of degree $r$.
For
  $ \gamma \in \operatorname{Mat}_m (O_E) $,
define
%
$$  
  \Delta_{E/F}^{m,1} (\gamma) =
    \Bigl|
      \prod_{\substack{
         \sigma \neq \tau \\
         \sigma, \tau \in \text{Gal} (E/F)
       }}
       R 
       \bigl(
          \sigma (\gamma), \tau (\gamma)
       \bigr)
    \Bigr|^{\frac12} .
$$
%
If $r$ is odd, then define
  $ \Delta_{E/F}^{m,2} (\gamma) = 1 $.
If $r$ is even,  then let
  $ \sigma_+ $
denote the element of order $2$ in 
  $ \text{Gal} (E/F) $.
Let
  $ \eta_E $
denote the unramified character of
  $ E^{\times} $ 
of order $2$.
Define
  $$ \Delta_{E/F}^{m,2}
    (\gamma) =
    \eta_E
    \Bigl(
      R 
      \bigl(
        \gamma,\sigma_+ (\gamma) 
      \bigr)
     \Bigr) .$$
Then set
  $ \Delta_{E/F}^m (\gamma) = 
    \Delta_{E/F}^{m, 1} (\gamma) \,
    \Delta_{E/F}^{m,2} (\gamma) $.
It is a continuous function on
  $ GL(m, O_E) $.
Extend
  $ | \cdot | $
to extensions of $F$ as needed.

\bigskip
\proclaim Proposition {5.1} (Homogeneity).
If
  $ \gamma = 1 + t^2 X $,
  $ \gamma_0 = 1 + X $,
  $ t \in O_F $, and 
  $ X \in \operatorname{Mat}_m (O_E) $,
then
  $ \Delta_{E/F}^m (\gamma) =
     |t|^{m^2 (r^2-r) } 
    \Delta_{E/F}^m (\gamma_0) $.
\finishproclaim

\pproclaim Proof:
$  R 
  \bigl(
    \sigma (\gamma),\tau (\gamma) 
  \bigr)
=
  R 
  \bigl( 
    \sigma (t^2 X), \tau (t^2 X)
  \bigr) 
=
  t^{2m^2} R
  \bigl(
    \sigma (\gamma_0), \tau (\gamma_0)
  \bigr)
$.
%
The result follows easily.
\qed
\finishpproclaim

\bigskip
\proclaim Proposition {5.2} (Central Elements).
Select 
  $ e \in O_E $, as in Section 3.4.
Set
  $ z = 1 + e t\in O_E$, with $t\in O_F$.
Consider $z$ as a central element in 
  $ GL(m, O_E) $
by the diagonal embedding.
Then 
%
$$
  \Delta_{E/F}^m (z) =
  |t|^{ m^2 (r^2-r)/2 } \;
  \epsilon^{ m (r^2-r)/2 }(t),
$$
%
where
  $ \epsilon: F^{\times} \to {\Bbb C} $
is any unramified character 
of order $r$.
\finishproclaim

\pproclaim Proof:
  $\sigma (z)- \tau (z) = t (\sigma(e)-\tau(e))$,
  and $\sigma(e)-\tau(e)$ is a unit (for $\sigma\ne\tau$).
Hence,
$R(\sigma(z),\tau(z))$ and $t^{m^2}$ are equal, up
to a unit.
%
It is now clear that
  $ \Delta_{E/F}^{m,1} (z) =
    | t|^{ m^2 (r^2-r)/2 } $.
If $r$ is odd, then
  $ \epsilon^{(r^2-r)/2} = 1 $
and the result follows.
If $r$ is even, then
  $ \epsilon^{(r^2-r)/2} $
is the restriction to
  $ F^{\times} $ 
of
  $ \eta_E $.
Then
  $ \Delta_{E/F}^{m,2} (z) =
    \eta_E^{m^2} (t) =
    \epsilon^{ m (r^2-r)/2 } (t) $.
\qed
\finishpproclaim 



\bigskip
\proclaim Proposition {5.3} (Compatibility).
Suppose that the residual characteristic is sufficiently large.
If $ \gamma $ is regular and sufficiently small in
  $ H$, then 
$\Delta^m_{E/F}(\gamma)$ coincides with the transfer factor of Langlands
and Shelstad (up to a scalar independent of $\gamma$).
\finishproclaim

\pproclaim Proof:
The proof of this lemma is somewhat technical, and relies on the paper [{\bf 21}].
We adopt the notation of [{\bf 21}] and [{\bf 22}] without further notice, 
and examine each of the terms defining the transfer factor.
%
Pick
  $ \gamma = \gamma_H $
in $H$, and let
  $ \gamma_G $
denote its image in $G$ with respect to our fixed embedding.
Then we clearly have
%
$$
  \Delta_{E/F}^{m,1} (\gamma) =
  \Delta_{\text{IV}}
  (\gamma_H, \gamma_G),\quad \text{ for }
  \gamma \in GL_m (O_E).
$$
%
The factor
  $ \Delta_2 (\gamma_H, \gamma_G) $
is trivial in a neighborhood of the identity. 
The transfer factor 
  $ \Delta_1 (\gamma_H, \gamma_G) $ is identically $1$
on elements
  $ (\gamma_H, \gamma_G ) $
such that
  $ \gamma_G $ 
lies in the 
  $G_1$-orbit of $H$.
More generally,
%
$$
  \Delta_1
  (\gamma_H, \gamma_G^g) =
  \epsilon \,(\det g) \Delta_1 (\gamma_H, \gamma_G),
$$
%
for $ \gamma_G $ in the $GL(n,F)$-orbit of $ \gamma= \gamma_H $.
We select $a$-data so that every 
  $ a_{\lambda} $
is a unit in an extension of $E$.
When $r$ is odd, we take the
  $ \chi $-data to be trivial so that
  $ \Delta_{\text{II}}
    (\gamma_H, \gamma_G) $
is trivial.
When $r$ is even,
we shall choose
  $ \chi $-data so that
%
$$
  \Delta_{\text{II}} (\gamma_H, \gamma_G) =
  \Delta_{E/F}^{m,2} (\gamma) .
$$
%
We say that roots $\alpha$ and $\beta$ of $G$ belong
to the same {\it block} if there is an element $w$ of
the absolute Weyl group $(S_m\times\ldots\times S_m)$
of $H$ with $w\cdot \alpha=\beta$. The Galois group
$\text{Gal}(E/F)$ permutes the blocks.  
Suppose that the equation $\tau(B_0) = -B_0$ holds for
some $\tau\in \text{Gal}(E/F)$, and some block $B_0$ of roots
that are not roots of $H$.  Then $r$ is even,
and $\tau$ equals the element $\sigma_+$ of order 2 in $\text{Gal}(E/F)$.
Furthermore, the $\text{Gal}(E/F)$-orbit of $B_0$ is uniquely determined.
Fix such a block $B_0$, when $r$ is even.

Let $\lambda$ be a root in $B_0$.  The fixed field $F_+$ of $\Gamma_+$
contains $E$.  Similarly, the fixed field $F_\pm\subseteq F_+$
of $\Gamma_\pm$ contains $E_\pm$, the fixed field of $\sigma_+$ in 
$E$.  Let ${\mathcal B}_0$ denote the set of $\text{Gal}(\overline F/F)$-orbits
of roots in $B_0$. If $\lambda$ is a root in the block $B_0$, then we take
%
  $ \chi_{\lambda}  = \eta_E \circ \text{Norm}_{\, F_+/E} $, where
  $ \eta_E $ is
the unramified character of order 2 on $E^\times$. Otherwise, if
$\lambda$ is not in the $\text{Gal}(\overline F/F)$-orbit of $B_0$, then
set $\chi_\lambda=1$.
It is straightforward to verify that
  $ \chi_{\lambda} $
gives
  $ \chi $-data.
Then, by definition,
%
\begin{align*}
%\spreadlines{10pt}
  \Delta_{\text{II}} (\gamma_H, \gamma_G)
  &=
  \prod_{\alpha\in{\mathcal B}_0}
  \chi_{\alpha}
  \Bigl(
    \frac
      { \alpha (\gamma)-1 }
      { a_{\alpha} }
  \Bigr)
=
  \prod_{\alpha\in{\mathcal B}_0}
  \prod_{\sigma \in \text{Gal}(F_+/E)}
  \eta_E
  \Bigl(
    \sigma
    \Bigl(
      \frac
        { \alpha (\gamma)-1 }
        { a_{\alpha} }
    \Bigr)
  \Bigr) 
\\
\vspace{10pt}
&=
  \prod_{\alpha\in B_0}
  \eta_E
  \Bigl(
    \frac
      { \alpha (\gamma)-1 }
      { a_{\alpha} }
  \Bigr) =
  \eta_E
  \Bigl(
    R \bigl( \gamma, \sigma_+ (\gamma) \bigr)
  \Bigr).
\end{align*}
%
This completes the proof that
  $ \Delta_{E/F}^{m,2} (\gamma) =
    \Delta_{\text{II}}
    (\gamma_H, \gamma_G) $.

By definition,
  $ \Delta_{\text{I}} (\gamma_H, \gamma_G) = 
    \big\langle
      \lambda (T_{sc}), s_T
    \big\rangle $.
We must check that this is a constant independent of
  $ \gamma_H,
    \gamma_G \in  H \subseteq G $.
We have shown that 
    $\Delta_I
     (\gamma_H, \gamma_G) ^{-1}
    \Delta
    (\gamma_H, \gamma_G) $ and
  $ \Delta_{E/F}^m (\gamma)$  
    are equal, up to a scalar.
So
%
\medskip
\begin{equation}\label{eqn:star}
%\spreadlines{8pt}
  \frac
     { \Delta_{E/F}^m (\gamma_1) }
     { \Delta_{E/F}^m (\gamma_2) }
  =
  \frac
     { \Delta_{\text{I}} (\gamma_H^2, \gamma_G^2) }
     { \Delta_{\text{I}} (\gamma^{1}_H, \gamma^{1}_G) }
  \Delta
  ( \gamma^{1}_H, \gamma^{1}_G; \gamma_H^2, \gamma_G^2)
  =
  \frac
    { \langle \lambda (T_{sc}^2), s_T \rangle }
    { \langle \lambda (T^1_{sc}), s_T \rangle }
  \Delta
  ( \gamma^{1}_H, \gamma^{1}_G ; \gamma_H^2, \gamma_G^2) .
\tag{$*$}
\end{equation}
\medskip
%
We consider the limit as the elements $\gamma_1$, $\gamma_2$,
  $\gamma^{1}_H$, $\gamma^{1}_G$, $\gamma_H^2$, and $\gamma_G^2$ 
tend
to $z$.
%
By descent,
  $ \Delta
    (\gamma^{1}_H, \gamma^{1}_G ; \gamma_H^2, \gamma_G^2)
    =
    \Delta_{\epsilon} 
    (\gamma^{1}_H, \gamma^{1}_G ; \gamma_H^2, \gamma_G^2) $,
for
  $ \gamma^{1}_H, \gamma^{1}_G , \gamma_H^2, \gamma_G^2 $
near $z$ [{\bf 22}].
%
But 
  $ \Delta_{\epsilon}
    (\gamma^{1}_H, \gamma^{1}_G ; \gamma_H^2, \gamma_G^2) $
is the transfer factor for the pair
  $ (G_{\epsilon_G},  H_{\epsilon_H} ) =
     (H, H) $,
and so must be identically $1$.
The left-hand side of ($*$) clearly tends to $1$ as
  $ \gamma_1$ and $\gamma_2$ tend to  $z$.
Every Cartan subgroup in
  $ H $
contains 
  $ Z(H) $
and hence $z$.
So we conclude that
  $ \Delta_{\text{I}}
     (\gamma_H, \gamma_G ) =
    \langle 
      \lambda(T_{sc}), s_T 
    \rangle $
is independent of the Cartan subgroup $T$ in
  $ H $.
\qed\finishpproclaim

Suppose that $\gamma$ belongs to a Levi factor $M$.  Let
$H_M$ be the endoscopic group associated with $M$ by descent [{\bf 22}].
Let $\Delta_{M,E/F}$ denote the transfer factor defined
for $M$ and $H_M$.  In general, $M$ is a product of general
linear groups, so that its transfer factor may be defined
as the product of the transfer factors for its individual linear
factors. Let $D_{G/M}(\gamma)$ and $D_{H/H_M}(\gamma)$
denote the usual discriminant factors.  For instance,
$D_{G/M}(\gamma)$ is equal to $D_G(\gamma)/D_M(\gamma)$, where
$D_G(\gamma)$ is, for instance, given by
$$D_G(\gamma) = \mid \Pi_\alpha (\alpha(\gamma) - 1)\mid^{1/2},$$
with the product taken over roots $\alpha$ in $G$. 

\bigskip
\proclaim Proposition {5.4} (Levi Factors).
$$D_{H/H_M}(\gamma)\Delta^m_{E/F}(\gamma) = D_{G/M}(\gamma) 
\Delta_{M,E/F}(\gamma).$$
\finishproclaim

\pproclaim Proof:  This is verified without difficulty by making use of the
identification of $\Delta^{m,1}_{E/F}$ with $\Delta_{\text{IV}}$ and 
of $\Delta^{m,2}_{E/F}$ with $\Delta_{\text{II}}$ proved in Proposition
5.3.  Note that a root of $G_1$, but not of $M_1$, is asymmetric,
and such a root does not contribute to $\Delta_{\text{II}}$
(see [{\bf 21},3.3.A]).
\qed
\finishpproclaim

\bigskip
\subsection{5.5. {\smc (Elliptic Descent).}}\nopagebreak
\medskip
\noindent
In this subsection we assume that $\gamma$ is elliptic.
Let
  $ \gamma = \gamma_s \gamma_u \in H \subseteq G $ be
  the topological Jordan decomposition of the element $\gamma$ (see Section 3.12).
Let
  $ G_s = C_G ( \gamma_s) $, and let $H_s =C_H(\gamma_s)$.
Since $\gamma$ is elliptic, $H_s$ and $G_s$ must have the form
\begin{align*}
  G_s &= GL(n_1,F_1),\quad \text{where } n_1[F_1:F] = n,\\
  H_s &= GL(m_1,E_1),\quad \text{where } m_1[E_1:E] = m.
\end{align*}
%
Furthermore, $E_1$ is an unramified extension of $F_1$.  Since
$\gamma_s$ is absolutely semisimple, the eigenvalues of $\gamma_s$
are roots of unity; it follows that $F_1$ is an unramified
extension of $F$.

\proclaim Theorem {5.5}.
%
$$
  \Delta_{E/F}^m (\gamma) = \Delta_{E_1/F_1}^{m_1}(\gamma).$$
%
\finishproclaim

\pproclaim Proof:
The only roots present in the product defining $\Delta^{m,1}_{E/F}(\gamma)$,
but absent in the product defining $\Delta^{m_1,1}_{E_1/F_1}$, are those
for which $\alpha(\gamma_s)$ is not 1.  But such roots satisfy the condition
$|\alpha(\gamma)-1| = 1$.  Each root of $H_s$ occurs $[F_1:F]$
times in the product for $\Delta^{m,1}_{E/F}$.  But the different
normalizations of absolute values compensates for this:
$$\mid\cdot\mid_{F_1} = \mid\cdot\mid_{F}^{[F_1:F]}.$$
Thus, $\Delta^{m,1}_{E/F}(\gamma) = \Delta^{m_1,1}_{E_1/F_1}(\gamma)$.

Now we consider the factor $\Delta^{m,2}$.  As in the treatment
of the factor $\Delta^{m,1}$, we may ignore the roots not
in $G_s$, since $\eta_E$ (like $\mid\cdot\mid$) is an unramified
character.  Suppose there is a root $\lambda$ of $G_s$ (but not of $H_s$)
and $\tau\in \text{Gal}(\overline F/F)$, 
such that $\tau(\lambda) = -\lambda$.  Clearly
we must have $\tau\in\text{Gal}(\overline F_1/F_1)$.  The rest
of this proof assumes some familiarity with the proof of Lemma 5.3.
As in the proof of that lemma,
the equation $\tau(\lambda) =-\lambda$ has no solutions
unless $[E_1:F_1]$ is even.  If $[E_1:F_1]$ is odd (or more generally if
$\lambda$ is asymmetric), then we may take the $\chi$-data to be trivial.
Then, $\Delta^{m,2}_{E/F}(\gamma) = \Delta^{m_1,2}_{E_1/F_1}(\gamma) = 1$.

Now assume that $[E_1:F_1]$ is even.  Let $v$ be the valuation of
the product of the factors $(\alpha(\gamma)-1)$, taken over the
roots in the $\text{Gal}(E/F)$-orbit of a block $B_0$.  Then by the
proof of Lemma 5.3, $\Delta^{m,2}_{E/F}(\gamma) = (-1)^{v/[E:F]}$.
Similarly, $\Delta^{m_1,2}_{E_1/F_1}(\gamma) = (-1)^{v_1/[E_1:F_1]}$,
where $v_1$ is the valuation of the product of the corresponding
factors in the $\text{Gal}(E_1/F_1)$-orbit of a corresponding
block in $GL(n_1,F_1)$.  Each root in $GL(n_1,F_1)$ is counted
$[F_1:F]$ times in $GL(n,F)$, so that $v=v_1[F_1:F]$.
Hence,
$$\frac v {[E:F]} = \frac {v_1[F_1:F]}{[E:F]} =
   \frac {v_1[F_1:E\cap F_1]}{[E:E\cap F_1]} =
   \frac {v_1[F_1:E\cap F_1]}{[E_1:F_1]}.$$
Note that $[E_1:F_1]=[E:E\cap F_1]$  is even, and the unramified
extensions $E$ and $F_1$ of $E\cap F_1$ are linearly disjoint.
It follows that $[F_1:E\cap F_1]$ is odd.  Hence,
$(-1)^{v/[E:F]} = (-1)^{v_1/[E_1:F_1]}$.
\qed
\finishpproclaim

\subsection{{\bf 6@.  Local compatibility of unit in Hecke algebra.}}

\medskip
\noindent

For the rest of this paper, we let 
  $ \Phi_G^{F'\!/F}(\gamma,f)$
denote the orbital integral
%
$$
  \Phi_G^{F'\!/F}
  (\gamma,f) =
  \int_{ C_G(\gamma) \backslash G }
  \epsilon_{F'\!/F}
  (\det g) f (g^{-1} \gamma g) dg,
$$
%
where
  $ \epsilon_{F'\!/F} $
is an unramified character of $F^\times$ of degree
  $ [F':F] $.
In particular
  $ \Phi_{G}^{F'\!/F}$
is a stable orbital integral, if $F'=F$.


\medskip
\noindent
\proclaim Theorem {6.1}.
The equality
%
$$
  \Delta_{E/F}^m 
  (\gamma) \Phi_G^{E/F}
  (\gamma, f_{(n)}) =
  \Phi_H^{E/E} (\gamma, f^H_{(m)} )
$$
%
holds for regular
  $ \gamma $
in a sufficiently small neighborhood of the identity provided
  that the abstract matching
  $ f \mapsto f^H $
of smooth functions holds between
  $ \epsilon $-invariant integrals on
  $ GL(n,F) $
and stable orbital integrals on 
  $ GL(m,E) $.
\finishproclaim

The proof will occupy the entire section.  Lemma 6.4 shows that a limit formula
exists for certain orbital integrals.  Lemma 6.2 shows that if this limit formula has
a certain form, then Theorem 6.1 holds.  An induction argument proves that this
limit formula holds, provided that it holds for a particular conjugacy class in $G$.
In this special case, an explicit expression for this limit
formula is then obtained from Igusa theory.  This application of Igusa theory 
is closely related to the application to the regular unipotent germs described
by Langlands and Shelstad.  

Assume that $z$ is an element in $G$ with $C_G(z) = H$.  For $\lambda'\in\Lambda_m$,
let $f^H\mapsto \mu_{z,\lambda'}^{\text{st,alg}}(f^H)$ be the linear
functional on $C_c^\infty(H)$ given by 
	$\mu_{z,\lambda'}^{\text{st,alg}}(f^H)=
\mu_{\lambda'}^{\text{st,alg}}(f^H_z)$, where $f^H_z(x) = f^H(zx)$.
Let $\Gamma^{\text{st,alg}}_{z,\lambda'}(\gamma)$ be the stable germ
on $H$, for the conjugacy class in $H$ whose semisimple part is
conjugate to $z$ and whose unipotent part is conjugate
to ${\mathcal O}_{\lambda'}$.  Assume that the germ is normalized
so that
$$D_H(z\gamma)\Phi^{E/E}_H(z\gamma,f^H) = \sum \Gamma^{\text{st,alg}}_{z,\lambda'}
(\gamma)\mu_{z,\lambda'}^{\text{st,alg}}(f^H).$$
Here $D_H$ denotes the usual discriminant factor (see Section 5.4).
Define linear functionals $I^{\epsilon}_{z,\lambda'}$ on 
$C_c^\infty(G)$ by the condition
$$D_H(z\gamma)\Delta^m_{E/F}(z\gamma)\Phi^{E/F}_G(z\gamma,f) = 
	\sum\Gamma^{\text{st,alg}}_{z,\lambda'}(\gamma)\Delta^{m,2}_{E/F}(z)
	I^{\epsilon}_{z,\lambda'}(f),$$
for $\gamma$ strongly $G$-regular and sufficiently small in $H$.
Such linear functionals $I^{\epsilon}_{z,\lambda'}$ exist
and are unique when $C_G(z) = H$, by the theory of descent of
orbital integrals.  (See, for instance [{\bf 25}].)
We suppose throughout this section that $z_0,z_1,\ldots$ is a sequence
of elements in $H$, with $C_G(z_i) = H$, which tend to the identity
element $1\in H$.  

\proclaim{Lemma 6.2}.  
Suppose that, for all $f\in C_c^\infty (G)$, there exists
an integer $i_f$ for which
$$\Delta^{m,2}_{E/F}(z_i)I^{\epsilon}_{z_i,\lambda'}(f)
	= \mu^{\epsilon,\text{alg}}_{\pmb\sigma(\lambda')}(f),\ \ \text{ for }
	i>i_f.\hss$$
Then Theorem 6.1 holds.
\finishproclaim

\pproclaim{Proof}:  Suppose that $f^H$ exists and that $z\in \{z_i\}$.
Then, for $\gamma$ small and $i>i_f$, we have
\begin{align*}
D_H(z\gamma)\Delta^m_{E/F}(z\gamma)\Phi^{E/F}_G(z\gamma,f) =
	D_H(z\gamma)\Phi^{E/E}_H(z\gamma,f^H) &= 
 	\sum\Gamma^{\text{st,alg}}_{z,\lambda'}(\gamma)\mu^{\text{st,alg}}_
	{z,\lambda'}(f^H)\\
	\vspace{8pt}
	= \sum\Gamma^{\text{st,alg}}_{z,\lambda'}(\gamma)
	\Delta^{m,2}_{E/F}(z)
	I^{\epsilon}_{z,\lambda'}(f) &= \sum\Gamma^{\text{st,alg}}_
	{z,\lambda'}(\gamma)\mu^{\epsilon,\text{alg}}_{\pmb\sigma(\lambda')}
	(f).
\end{align*}
By the linear independence of germs, we have  $\mu^{\epsilon,\text{alg}}_%
	{\pmb\sigma(\lambda')}(f)
	=\mu^{\text{st,alg}}_{z,\lambda'}(f^H)$, for $z=z_i$ and $i>i_f$.
Let $z$ tend to $1$, to obtain $\mu^{\epsilon,{\text{alg}}}_{\pmb\sigma(\lambda')}
(f) = \mu^{\text{st,alg}}_{\lambda'}(f^H)$.  If we let $f=f_{(n)}$,
then, by Proposition 3.10,
$$\mu^{\epsilon,{\text{alg}}}_{\pmb\sigma(\lambda')}(f_{(n)}) =
  \mu^{\text{st,alg}}_{\lambda'}(f^H_{(m)}) = 
  \mu^{\text{st,alg}}_{\lambda'}(f^H),$$
for all partitions $\lambda'\in \Lambda_m$.  Hence, we may
take $f^H = f^H_{(m)}$, the normalized unit of the Hecke algebra.
\qed
\finishproclaim

The group $H$ contains the conjugacy class with semisimple part $z$ and
unipotent part ${\mathcal O}_{\lambda'}$.  Write ${\mathcal O}_{z,\lambda'}$ for
the conjugacy class in $G$ that meets this class of $H\subset G$.

\proclaim{Lemma 6.3}.  $I^{\epsilon}_{z,\lambda'}$ is an
$\epsilon$-invariant distribution on the conjugacy class ${\mathcal O}_{\lambda'}$.
\finishproclaim

\pproclaim{Proof}:  The $\epsilon$-invariance results from the
$\epsilon$-invariance of the integral $\Phi^{E/F}_G(\gamma,\cdot)$
and the uniqueness of the germ expansion.  From the explicit description
of descent given in [{\bf 13}], we see that $I^{\epsilon}_{z,\lambda}$
is supported on ${\mathcal O}_{z,\lambda'}$.
\qed\finishpproclaim

\proclaim{Lemma 6.4}.  There exists a function $\rho^\epsilon_{\lambda'}(z_i)$
on $\{z_i\}$ with the property that,
for $f\in C_c^\infty(G)$ and for $i>i_f$, we have
$$I^{\epsilon}_{z_i,\lambda'}(f) = 
	\rho^\epsilon_{\lambda'}(z_i)\mu^{\epsilon,\text{alg}}_{\pmb
	\sigma(\lambda')}(f).$$
\finishproclaim

\pproclaim{Proof}:   Set $P=P_{\pmb\tau(\lambda')} = MN$.  Let $C = C_M(z)$,
and let $dm$ be an invariant measure on $C\backslash M$.  It is easy to
check that
$$f\mapsto I_{z,\lambda'}(f) = \int_{C\backslash M}\epsilon(\det m )\bar f^P(z^m)dm$$
is an $\epsilon$-invariant distribution supported on the conjugacy
class ${\mathcal O}_{z,\lambda'}$. It is enough to prove the
result for $I_{z,\lambda'}$, instead of $I^{\epsilon}_{z,\lambda'}$.

Expand the orbital integral $\int_{C\backslash M}\epsilon(\det m) \bar f^P(z^m) dm$
by its Shalika germ expansion.  (Although $z$ is not regular, 
a Shalika germ expansion exists by {[\bf 30]}.)
The sum runs over unipotent classes in the closure of the conjugacy 
classes of $z_i$ in $M$.  Over the field extension $E$ splitting $C$,
the group $C$ becomes isomorphic to a Levi subgroup $M_\eta$ of a parabolic
subgroup in $M$, a parabolic subgroup we denote by $P_\eta = M_\eta N_\eta$.
By the Iwasawa decomposition in 
  $ M $,
%
$$
  z^g =
  z^{pk}=
  z^{nk} =
 (zn')^k \in 
  (z N_{\eta})^k,
 $$
  for $g \in M$, 
  $p \in P_{\eta}$, $n, \, n' \in N_{\eta}$, and
  $k \in K \cap M$.
Thus, a class obtained in the limit as $z$ tends to $1$ must be conjugate to a unipotent element in
  $ N_{\eta} $.
Work momentarily with a single factor of $M$.
If $ M$
consists of a single factor, then the dense class in $ N_{\eta} $ 
is 
  $ {\mathcal O}_{\eta} $, where $ \eta = ( m, m,\dots ,  m) $.
So the sum in the Shalika germ expansion
runs over only those classes whose partition is dominated 
  by $ (m, m, \dots , m) $
in the
  natural partial order on partitions.

Next we use the fact that both sides are
  $ \epsilon $-invariant.
  By the discussion in Section 3.1, the
only
  $ \epsilon $-invariant 
integrals
on unipotent classes are on classes
  $ \mathcal O_{\eta'} $,
where $r=r_\epsilon$ divides 
  $ r(\eta') $.
This, together with the condition that
  $ \eta $
dominates
  $ \eta' $,
forces the sum to collapse to a single term.
We call the $ \epsilon $-invariant integral over this class
  $ \mu_1^{\epsilon} $, and denote the germ by $ \rho_1 (z) $.  
  Thus $I_{z,\lambda'}(f) = \rho_1(z)\mu_1^\epsilon(\bar f^P)$.

It is easy to see, by expanding
by the definition of
  $ \bar f^{P} $,
that
  $ \mu_1^{\epsilon} 
    \left(
      \bar f^{P} 
    \right) $
is an integral over the conjugacy class of ${\mathcal O}_{\pmb\sigma(\lambda')}$
(compare the end of 3.14).
Since the left-hand side is $\epsilon$-invariant, the right-hand
side is as well.  This completes the proof.
\qed\finishpproclaim
\bigskip


It is now clear what the strategy of the proof will be.  Theorem 6.1 will follow from
the identity $\Delta^{m,2}_{E/F}(z)\rho^\epsilon_{\lambda'}(z)=1$.  This identity will
be established in two steps.  First, an induction argument on the rank of $G$  will be used to
reduce us to the case $\lambda'=(m)$.   Second, a simple form of Igusa theory
will be used to compute the germ $\rho^\epsilon_{(m)}(z)$ explicitly.

To prepare for the induction argument, we must specify a normalization
of measures.
Let $P=MN$ be a standard parabolic subgroup, and as usual let $\bar N$
denote the transpose of $N$.
Let $dm$, $dn$  and $d\bar n$ be Haar measures on $M$, $N$ and $\bar N$.
A Haar measure $dg$ on $G$ is given by $dg=dm\,dn\,d\bar n$, where $g = (mn)^{\bar n}$.
Suppose that $f$ belongs to $C_c^\infty(G)$.  Define $f^M\in C_c^\infty(M)$ by
$$f^M(m) = \int_{N,\bar N} f((mn)^{\bar n}) dn\,d\bar n.$$
Similarly, define compatible Haar measures $dm$, $dn$, $d\bar n$ and $dh = dm\,dn\,d\bar n$ on
$H_M$, $N\cap H$, $\bar N\cap H$ and $H$,
where $h=(mn)^{\bar n}$.  The notation for coordinates $m$, $n$ and $\bar n$
is the same for
different choices of Levi subgroups $M$, 
and the same for $H$ as for $G$.
I hope this does not create any confusion.
Here, as in Section 5.4, $H_M$ is a Levi factor of $H$, viewed as
an endoscopic group of $M$.
For $f^H\in C_c^\infty(H)$, define
$$f^{H_M}(m) = \int_{N\cap H,\bar N\cap H} f^H((mn)^{\bar n}) dn\,d\bar n.$$


Until now, notation such as $dn$ has referred to a measure.  But in
this paragraph, it will denote the corresponding differential form.
This shift in convention allows us to work over the field extension
$E$.  A form defined over $F$ will determine a measure on the
$F$-points of the appropriate variety.
Set $P_\sigma = P_{\pmb\sigma(\lambda')} = M_\sigma N_\sigma$, and set 
$\bar N_\sigma= \bar N_{\pmb\sigma(\lambda')}$, for 
$\lambda'\in \Lambda_m$.   
Over $E$, we may assume that $z$ lies in the center of $M_\sigma$.
A Zariski open subset of ${\mathcal O}_{z,\lambda'}$ and a Zariski open subset
of $(zn)^{\bar n}$ coincide, for $n\in N_\sigma$ and $\bar n\in \bar N_\sigma$.
Then $dn\,d\bar n$ is an invariant form on ${\mathcal O}_{z,\lambda'}$.
Define $f\mapsto I^{\epsilon,\text{alg}}_{z,\lambda'}(f)$ as $I^{\epsilon}_{z,\lambda'}$,
but replacing the positive measure implicit in the definition of $I^{\epsilon}_{z,\lambda'}$
with the positive measure associated to $dn\,d\bar n$.  
Thus $I^{\epsilon,\text{alg}}_{z,\lambda'}$
and $I^\epsilon_{z,\lambda'}$ are equal, up to positive scalar.
As a short calculation will show, the differential form $dn\,d\bar n$ 
is defined over $F$
(although the coordinates
$n$ and $\bar n$ are not defined over $F$)
and gives a measure on the $F$-conjugacy class ${\mathcal O}_{z,\lambda'}$.


All of the Levi factors of $GL(n)$ are products of linear groups.  This
allows us to extend the constructions of the previous sections to Levi factors.
In the next set of equations, we apply descent to a Levi factor and take
a germ expansion.  Using what I hope is an obvious extension of notation,
we have by Proposition 5.4 and descent
\begin{align}
  D_{H}(z\gamma)\Delta^m_{E/F}(z\gamma)\Phi^{E/F}_G(z\gamma,f)
  &=
    D_{H_M}(z\gamma)\Delta_{M,E/F}(z\gamma)\Phi^{E/F}_M(z\gamma,f^M) \\
\vspace{4pt}
  &=
    \sum_{\eta'} \Gamma^{\text{st,alg}}_{H_M,z,\eta'}(\gamma)
                 I^{\epsilon}_{M,z,\eta'}(f^M)
  \Delta_{M,E/F}^{2}(z),\label{eqn:6.5} \\ %%\tag{6.5}
%
D_{H}(z\gamma)\Phi^{E/E}_H(z\gamma,f^H)
  &=
     D_{H_M}(z\gamma)\Phi^{E/E}_{H_M}(z\gamma,f^{H_M}) \\
\vspace{4pt}
    &=
     \sum_{\eta'}\Gamma^{\text{st,alg}}_{H_M,z,\eta'}(\gamma)
                 \mu^{\text{st,alg}}_{H_M,z,\eta'}(f^{H_M}). \label{eqn:6.6}
  %XX \tag{6.6}
\end{align}
Here $\eta'$ indexes unipotent conjugacy classes of the endoscopic group $H_M$,
obtained by descent, of the Levi subgroup $M$.  If
$H_M = M_{\lambda'} = GL(\lambda'_1,E)\times\cdots\times GL(\lambda'_\ell,E)$, then
$\eta' = (\eta^1,\ldots,\eta^\ell)\in
   \Lambda_{\lambda'_1}\times\cdots\times \Lambda_{\lambda'_\ell}$.

There are two hypotheses that we will prove inductively.  They
are
\begin{equation}\label{eqn:tag1}
I^{\epsilon,\text{alg}}_{z,\lambda'} =
I^{\epsilon}_{z,\lambda'},%\tag{$1$}
\end{equation}
and
\begin{equation}\label{eqn:tag2}
\Delta^{m,2}_{E/F}(z)I^{\epsilon,\text{alg}}_{z,\lambda'} =
  \mu^{\epsilon,\text{alg}}_{\pmb\sigma(\lambda')}. % XX \tag{$2$}
\end{equation}
(Compare Lemma 6.2.)
Now we fix $\lambda'\in\Lambda_m$ and set $M=M_{\lambda''}$ and  $H_M = M_{\lambda'}$,
where $\lambda'' = \pmb\tau(\lambda')$.  We assume that $\lambda'\ne(m)$, or
equivalently that $M\ne G$.  By induction we may assume (using what I hope
is an obvious extension of notation) that
\begin{equation}\label{eqn:tag1'}
I^{\epsilon,\text{alg}}_{M,z,\eta'} =
I^{\epsilon}_{M,z,\eta'}, %\tag{$1'$}
\end{equation}
and
\begin{equation}\label{eqn:tag2'}
\Delta^{2}_{M,E/F}(z)I^\epsilon_{M,z,\eta'} =
\mu^{\epsilon,\text{alg}}_{M,\eta},%\tag{$2'$}
\end{equation}
where $\eta =(\pmb\sigma_r(\eta^1),\ldots,\pmb\sigma_r(\eta^\ell))\in
   \Lambda_{r\lambda'_1}\times\cdots\times \Lambda_{r\lambda'_\ell}$.

Given $\eta' = (\eta^1,\ldots,\eta^\ell)$, let $\eta''\in\Lambda_m$ be the
partition obtained by combining all of the parts $(\eta^i)_j$ of all the
partitions $\eta^i\in\Lambda_{\lambda'_i}$.  In particular, if each
$\eta^i = (\lambda'_i)\in\Lambda_{\lambda'_i}$ is a partition with one part,
then $\eta''=\lambda'$.  Moreover, this is the only way for $\eta''$ to
equal the partition $\lambda'$.  Write $\eta'_0$ for this sequence of partitions.
The sequence of partitions $\eta_0'$ corresponds to the trivial unipotent class $\{1\}$ in $H_M$.
Set 
$\eta_0 = (\pmb\sigma(\eta^1_0),\ldots,\pmb\sigma(\eta_0^\ell))\in
  \Lambda_{r\lambda'_1}\times\cdots\times \Lambda_{r\lambda'_\ell}$.
It is easy to check, by using our explicit normalization of measures, that
\begin{align}\label{eqn:6.7}
  \mu^{\text{st,alg}}_{H_M,z,\eta'}(f^{H_M}) &= \mu^{\text{st,alg}}_{z,\eta''}(f^H),\\
  I^{\epsilon,\text{alg}}_{M,z,\eta_0'}(f^M) &= I^{\epsilon,\text{alg}}_{z,\lambda'}(f),\\
         %%\tag{6.7}\\
  \mu^{\epsilon,\text{alg}}_{M,\eta_0}(f^M)
     &= \mu^{\epsilon,\text{alg}}_{\sigma(\lambda')}(f).
\end{align}
The first of these equations, together with Equation 6.6, implies that
$$\Gamma^{\text{st,alg}}_{H_M,z,\eta'_0} (\gamma)
         = \Gamma^{\text{st,alg}}_{z,\lambda'}(\gamma).$$
This equation, Equation 6.5, the linear independence of germs on $H_M$, and the
definition of $I^\epsilon_{z,\lambda'}$ give:
\begin{equation}\label{eqn:6.8}
I^\epsilon_{M,z,\eta'_0}(f^M)\Delta^{2}_{M,E/F}(z)
      = I^\epsilon_{z,\lambda'}(f)\Delta^{m,2}_{E/F}(z).%\tag{6.8}
\end{equation}
%
By Proposition 5.4, we find $\Delta^{2}_{M,E/F}(z) = \Delta^{m,2}_{E/F}(z)$.
By the induction hypothesis ($1'$) and Equation 6.7, 
the integral $I^\epsilon_{M,z,\eta'_0}(f^M)$ may be replaced
with $I^{\epsilon,\text{alg}}_{M,z,\eta'_0}(f^M)$,
and hence with $I^{\epsilon,\text{alg}}_{z,\lambda'}(f)$.
Hence, Equation 6.8 implies that ($1$) holds.  By the second induction hypothesis ($2'$), and by 
Equation 6.7,
the left-hand side of Equation 6.8 may be replaced with
$\mu^{\epsilon,\text{alg}}_{\pmb\sigma(\lambda')}(f)$.  Hence, (2) holds.
This completes the reduction of the identities (1) and (2) to the case $\lambda'=(m)$.
\bigskip

When $\lambda'=(m)$, it is a simple comparison of normalizations of measures that leads to the
identity (1).  For instance, $\mu^{\text{st,alg}}_{z,(m)}(f^H) =
f^H(z)$. Over $E$ we may identify $H$ and $M_\sigma$, and
descent to $H$ or $M_\sigma$ is obtained by integrating $dn\,d\bar n$ over
$N_\sigma$ and $\bar N_\sigma$.  This, by definition, is the measure used
in the definition of $I^{\epsilon,\text{alg}}_{z,(m)}$.  This gives the result.

\bigskip
In the remainder of this section, we will prove (2) for $\lambda'=(m)$ by applying a simple
form of Igusa theory.  The construction is extremely close to one used by
Langlands and Shelstad to compute the regular germ [{\bf 19}],[{\bf 21}],
[{\bf 28}].  Since they explain their
argument and method carefully, we will omit many of the details.  It will be a simple
matter to work out the leading term of the expansion of $I^\epsilon_{z,\lambda'}$,
once the leading term of the corresponding stable orbital integral has been
calculated.

Fix the parabolic subgroup $P_0 = P_{(m,\ldots,m)} = M_0N_0$.  Let ${\mathcal P}$
denote the variety of parabolic subgroups conjugate to $P_0$, and let
$P$ denote a generic element of $\mathcal P$.  The
symmetric group on $r$ letters $S_r$ acts naturally on $M_0$, permuting the
parabolic subgroups having Levi subgroup $M_0$.  Let $T$ be a torus defined over $F$
in $G$ of rank $r$ that is conjugate over $\overline F$ to the center of $M_0$.
Identify $T$ with the center of $M_0$.
We
construct a diagram similar to that of [{\bf 12}].  (Also see [{\bf 19}].)
This gives maps
\begin{align*} \phi:T'\times C_G(T)\backslash G &\to G\times {\mathcal P}^{r!},
     \quad \phi(\gamma,g) = (\gamma^g,(P_0(W))^g), \\
%
      \pi_1: G\times {\mathcal P}^{r!} &\to G,\quad
      \pi_1(x,(P(W))) = x,\\
%
      \xi: X &\to T,\quad
      \xi(\gamma^g,(P_0(W))^g) = \gamma.
\end{align*}
The variety $X$ is the
closure of the image of $\phi$, and $W$ indexes the factors $\mathcal P$
in ${\mathcal P}^{r!}$.  
Here $T'$ denotes the maximally regular elements of $T$.
By a maximally regular element
in $G$ we mean an element
$x$ such that $\dim C_G(x) =\dim C_G(T)$.

We now proceed as in Langlands and Shelstad using $T$ and $\mathcal P$, rather
than a Cartan subgroup and the variety of Borel subgroups.  With this small
difference, their argument goes through almost without any modifications.
Let $Y_\Gamma$ be the variety constructed from
a maximally regular curve $\Gamma\subset T$ instead of $T$.  In general,
$\pi_1(Y_\Gamma)$ does not contain regular elements, but the image of $\Gamma'\times C_G(T)\backslash G$
in $G$ consists of elements that are maximally regular.  On the set of maximally regular
semisimple elements (that is, on the image of $\phi$), 
the map $\pi_1$ is an $r!$-to-$1$ map.  The Richardson class
of $P_0$ is also in the image of $\pi_1$, and the elements of this class are 
maximally regular.  Such a
unipotent element is contained in a unique parabolic subgroup of type
$(m,\ldots,m)$, so that $\pi_1$ becomes bijective when restricted to this
conjugacy class.  By the regularity of $\Gamma$, the only maximally regular
classes in $\pi_1(Y_\Gamma)$ are semisimple or unipotent.  The only such unipotent class is
the Richardson class just described.  The map from $X$ to the partial
Springer variety [{\bf 19}] obtained by projecting to the first parabolic subgroup,
$$(x,(P(W))) \mapsto (x,P(W_+)),$$
is one-to-one on maximally regular elements (see [{\bf 21},5]).
We use the twisted $F$-structure
on $X$ determined by $T$. This twisted structure is defined in [{\bf 19}].  We use this bijection
to define a twisted $F$-structure on the partial Springer variety
$$\{(x,P)\mid x\in P\in{\mathcal P}, \quad x \in Z(M)N_P,\quad x\text{ is maximally regular }\}.$$
Here $N_P$ is the radical of $N$, and $Z(M)$ is the center of $M$.
This variety is smooth.  With respect to this
twisted structure, the maps $\phi$, $\pi_1$ and $\xi$ are defined over $F$.
As in the calculation of Langlands and Shelstad, there
is a single irreducible divisor supported on maximally regular elements.
In fact, in local coordinates
on an open set,
points have the form $(\gamma n,P_0)^{\bar n}$, for $n\in N_0$, $\bar n\in \bar N_0$,
and $\gamma\in T'\subset Z(M)$.
%
The divisor over $1\in\Gamma\subset T$ is obtained when $\gamma=1$, and is
isomorphic to the Richardson class of $P_0$.  It is now clear that, with
a compatible
normalization of measures, the germ for the leading term of the expansion
of stable orbital integrals on ${\mathcal O}_{z,(m)}$ is
identically $1$.  Compare [{\bf 28}].

This stable  germ is identically $1$ for the normalization of measures
arising naturally in Igusa theory.  This differs from the normalization
implicit in $\Phi$
by the discriminant [{\bf 19},2.12].  
Referring to the definition of $I^\epsilon_{z,\lambda'}$, we
see that the discriminant $\Delta^{m,1}_{E/F}(z)$ is built into
its normalization.
We conclude that
$\rho^{\text{st}}_{(m)}(z)$ is equal to 1,
for $z$ sufficiently close to 1.

We now turn to the $\epsilon$-germ.
We must prove that $\Delta^{m,2}_{E/F}(z)\rho^\epsilon_{(m)}(z) = 1$, or equivalently that
$$\Delta^{m,2}_{E/F}(z)\rho^\epsilon_{(m)}(z) = \rho^{\text{st}}_{(m)}(z).$$
This has a simple geometric interpretation.  We will find a unipotent $u$
in the class of $(m,\ldots,m)=\pmb\sigma(m)$, with $\beta_\epsilon(u) = 1$, and a small
neighborhood $U$ of $u$ such that, if $z^g$ lies in $U$, then
\begin{equation}\label{eqn:6.9}
\epsilon(\det g)\Delta^{m,2}_{E/F}(z) = 1.%\tag{6.9}
\end{equation}
Then, for functions supported on $U$, the $\epsilon$-germ $\rho^\epsilon_{(m)}(z)$
is $\epsilon(\det g)$ (or $\Delta^{m,2}_{E/F}(z)^{-1}$) times the stable germ.
The result follows.

Small neighborhoods of $u$ are obtained by taking a transversal slice $S_u$
at $u$ to the unipotent orbit of $u$ and then translating $S_u$ by a small
amount along the orbit.
Translation of $g$ by an element of $G$ in a small neighborhood
of $1$ will leave $\epsilon(\det g)$ unchanged.
Therefore, if we know that Equation 6.9 holds for all $x$ in $S_u\cap {\mathcal O}_{z,(m)}$,
then we will know that Equation 6.9 holds for all elements
in a neighborhood of $u$.  In such a transversal slice $S_u$, there is only
one element of ${\mathcal O}_{z,(m)}$ near $u$.  This follows from the
Igusa theory described above.  In fact, $\pi_1$ is an $r!$-to-$1$ map on 
points over ${\mathcal O}_{z,(m)}$
and a one-to-one map on the class of $u$.  In $X$ there are $r!$ points
 over ${\mathcal O}_{z,(m)}$ near $u$
in a given slice.
Explicitly, these points are given by
$(\gamma^wu,P_0)$ with $w\in S_r$, $\gamma^w\in T'\subset Z(M_0)$ and $u\in N_0$.
Hence in the image there is only one element of ${\mathcal O}_{z,(m)}$
near $u$ in a given slice.

The next lemma calculates $\epsilon(\det g)$ for such an element.  This lemma
will complete the proof of Theorem 6.1.

% end

\proclaim{Lemma 6.10}.  Fix $u\in {\mathcal O_{\pmb\sigma(\lambda')}}\cap 
N_{\pmb\sigma(\lambda')}$.  Assume that $\beta_\epsilon(u) = 1$.
There exist an analytic curve $\Gamma(t)$ in $G_1$, a $G_1$-conjugate
$u_1$ of $u$, a $G$-conjugate $z(t)= \Gamma(t)^{g(t)}$ of $\Gamma(t)$, and a
neighborhood $U$ of $u_1$ in $G$,
with the following properties:
\begin{enumerate}
\item{1.}  $\Gamma(0) =u_1$.
\item{2.}  For $t\ne 0$, $z(t)$ is an element in $H$ satisfying
                  $C_G(z(t)) = H$.
\item{3.}  If $t\in F$ is sufficiently small but nonzero,
       then $\Gamma(t)\in U$ and 
       $$\epsilon(\det g(t))\Delta^{m,2}_{E/F}(z(t))=1.$$
\end{enumerate}
\finishproclaim

\pproclaim{Proof}:  We construct $\Gamma(t)$ explicitly.  Identify
$\text{Mat}_n(F)$ with $\text{Mat}_m(R)$, where $R = \text{Mat}_r(F)$
and $rm=n$.  Write $D(t_1,\ldots,t_r)$ for the matrix in $R$ of the
form
$$\begin{pmatrix} 1& t_1& 0 & . & . \\
	   0& 1  & t_2&. &.  \\
	   .& .  & .  &. &t_{r-1}\\
	   t_r&0 & . & 0 &1\end{pmatrix}.$$
Set $h(t) = \text{diag}(t^{r-1},\ldots,t,1)\in R$.  Then
$$h(t)^{-1}D(t,t,\ldots,t,tx)h(t) = D(1,1,\ldots,1,t^rx).$$  This
matrix tends to the unipotent matrix
  $d = D(1,1,\ldots,1,0)$, as $t$ tends to $0$.  For an
appropriate choice of embedding $E\hookrightarrow \text{Mat}_r(F)$, 
and appropriate
choice of $x$, we may assume that $d_t=D(t,t,\ldots,t,tx)$ lies in $E$,
and is regular in $\text{Mat}_r(F)$, if $t\ne 0$.

Consider the element $z(t) \in \text{Mat}_m(R)$ given by
$z(t) = \text{diag}(d_t,\ldots,d_t)$.
Consider also the element $g(t) \in \text{Mat}_m(R)$ given by
$$g(t)^{-1} = \text{diag}(h(t),\ldots,h(t)),\ \text{($m$ copies)}.$$
Then define $\Gamma(t)$ by
  $\Gamma(t)^{g(t)}= z(t)$.  It follows easily that
$u_1 = \Gamma(0)$ has the same form as $z(t)$, but with $d_t$ replaced
with $d$.  From this explicit description it follows that $u_1\in
{\mathcal O}_{\pmb\sigma(\lambda')}$ and $\beta_\epsilon(u_1^y)=1$, for
some $y\in G_1$.
%
Clearly, $z(t) = \text{diag}(d_t,\ldots,d_t) \in \text{Mat}_m(E) \subseteq
\text{Mat}_m(R)$.  Also $\det g(t) = t^{mr(r-1)/2}$, so, by Lemma 5.2,
$$\epsilon(\det g(t)) = \epsilon(t^{mr(r-1)/2}) = 
\Delta^{m,2}_{E/F}(z(t)).$$
Since these factors are $\pm1$, the result follows.
\qed
\finishpproclaim

\subsection{{\bf 7@.  Unit of Hecke algebra.}}

\medskip
\noindent
Now we state and prove the main theorem of this paper.
We say that local matching occurs for
  $ (F, n ) $
if the following condition holds
  $ LM(F,n) $: 
For every
  $ f \in C_c^{\infty}
    \bigl(
      GL(n, F)
    \bigr) $
and every divisor $r$ of $n$
there is a function
  $ f^H $
on
  $ C_c^{\infty}
    \bigl(
        GL ( n / r, E )
    \bigr) $
(where $E$ is the unramified extension of degree
  $ r $)
such that
%
$$
  \Delta_{E/F}^{n/r}
  (\gamma) \,
  \Phi_{GL(n, F)}^{E/F} \,
  (\gamma, f) =
  \Phi_{GL(n/r, E)}^{E/E} \,
  (\gamma, f^H)
$$
%
for all strongly $G$-regular semisimple elements
  $ \gamma $
in a neighborhood of the identity in
  $ GL (n/r, E) $.
The neighborhood is allowed to depend on $f$.

\proclaim Theorem {7.1}.
Suppose that for every
  $ n_0 \leq n $
and unramified extensions $F_0$ of $F$ of degree at most $n$ that
local matching
  $ LM(F_0, n_0 ) $
holds.
Then
%
$$
  \Delta_{E/F}^{m}
  (\gamma) \,
  \Phi_{GL(n, F)}^{E/F} \,
  (\gamma, f_{(n)}) =
  \Phi_{GL(m, E)}^{E/E} \,
  (\gamma, f^H_{(m)})
$$
%
for all
  $ \gamma $
in
  $ GL(m, O_E )$.
\finishproclaim

\pproclaim Proof:
We argue by
induction on $n$.
First, suppose that $\gamma\in GL(m,O_E)$ lies in a proper Levi
subgroup $H_M=M_{\lambda'}$ of $H$.  We may identify $M_{\lambda'}$ with
an endoscopic group of the Levi subgroup $M = M_{\pmb\tau(\lambda')}$.
Compatibly normalized, the orbital integrals satisfy  the descent conditions
$$D_{G/M}(\gamma)\Phi^{E/F}_G(\gamma,f_G) = \Phi^{E/F}_{M}(\gamma,f_M),\quad
  D_{H/H_M}(\gamma)\Phi^{E/E}_H(\gamma,f_H) = \Phi^{E/E}_{H_M}(\gamma,
  f_{H_M}).$$
We have shifted notation slightly to let $f_A$ denote the normalized unit
of the Hecke algebra in any reductive group $A$.  Also $D_*(\gamma)$ is the
usual discriminant factor introduced in Section 5.
%
The induction hypothesis and Lemma 5.4 show that we have
\begin{align*}
D_{H/H_M}(\gamma)\Delta^m_{E/F}(\gamma)\Phi^{E/F}_G(\gamma,f_G) &=
	\Delta_{M,E/F}(\gamma) D_{G/M}(\gamma)\Phi^{E/F}_G(\gamma,f_G)\\
	&= \Delta_{M,E/F}(\gamma) \Phi^{E/F}_M(\gamma,f_M)\\
	&=  \Phi^{E/E}_{H_M}(\gamma,f_{H_M}) \\
	&= D_{H/H_M}(\gamma)\Phi^{E/E}_H(\gamma,f_H).
\end{align*}

Now assume that $\gamma\in GL(m,O_E)$ is elliptic.
Let
  $ \gamma = \gamma_s \gamma_u $
(topological Jordan decomposition).
If
  $ \gamma^g \in K $,
then, by Kazhdan's lemma,
  $g$ belongs to $G_s K$, where (as in Section 5.5)
%
$$
  G_s = C_G(\gamma_s) 
  \overset \sim \to \to
  GL (n_1, F_1).
$$
Similarly, if $\gamma^h\in K\cap H$, with $h\in H$, then $h$ belongs
to $H_s(K\cap H)$.
%
This, together with Theorem 5.5, gives the identities
%
\begin{align}\label{eqn:star2}
 \Delta_{E/F}^m \,
  (\gamma) \,
  \Phi_G^{E/F} \,
  (\gamma, f_G ) &=
  \Delta_{E_1/F_1}^{m_1} \,
  (\gamma) \,
  \Phi_{G_s}^{E_1/F_1} \,
  (\gamma, f_{G_s}),\\ 
  \vspace{4pt}
  \Phi^{E/E}_H(\gamma,f_H) &= \Phi^{E_1/E_1}_{H_s}(\gamma_,f_{H_s}).
%\tag{$*$}
\end{align}
%
To justify the appearance of the superscript
  $ E_1 / F_1 $
to
  $ \Phi $
in the rightmost term of ($*$), we argue as follows.
We have   $ \epsilon \, (\det g) =
    \epsilon (\text{Norm}_{F_1/F}\det_1
    g$),
where $g$ belongs to $GL(n_1,F_1)$, $\det_1$ denotes the
determinant on $GL(n_1,F_1)$, and $\det g$ is the determinant
of $g$ viewed as an element of $G$.
Extend
  $ \epsilon $
to the unramified character of degree
  $ r_{\epsilon} = r $
on 
  $ F_1 $.
Set $\epsilon_1 = 
   \epsilon \circ \text{Norm}_{F_1/F} =
    \epsilon^{\,[F_1:F]}$.
The order of
  $ \epsilon_1 $
is 
  $ [E: E\cap F_1] = [E_1 : F_1 ] $.
Thus, the character
  $ \epsilon_1 $ 
on 
  $ F_1 $
is associated with the extension 
  $ E_1 $ of
  $ F_1 $.


Consider the case
  $ F_1 \neq F $.
Then we have reduced to the situation of a smaller $n$.
By induction,
%
$$
  \Delta_{E_1/F_1}^{m_1} (\gamma) \,
  \Phi_{G_s}^{E_1/F_1} (\gamma, f_{G_s} ) =
  \Phi_{H_s}^{E_1/E_1}  (\gamma, f_{H_s} ) .
$$
%
This and the identities $(*)$ prove Theorem 7.1 in this case.
%

Finally, if
  $ F_1 = F $,
then the absolutely semisimple part
  $ \gamma_s $
of
  $ \gamma $
is central in
  $ GL(n,F) $.
Hence,
%
\begin{align}\label{eqn:double-star}
%\spreadlines{4pt}
 \Delta_{E/F}^m \,
  (\gamma) \,
  \Phi_G^{E/F} \,
  (\gamma, f_G)
&=
  \Delta_{E/F}^m \,
  (\gamma_u) \,
  \Phi_G^{E/F} \,
  (\gamma_u, f_G )
\\
\vspace{4pt}
&=
  \Phi_H^{E/E} \,
  (\gamma_u, f_H ) 
%\tag{$**$}
\\
&=
  \Phi_H^{E/E}
  (\gamma, f_H).
\end{align}
%
The next to last step ($**$) requires some explanation.
First, the matching of smooth functions
  $ f \mapsto f^H $
implies the matching 
of the unit
  $ f_G \mapsto f_H$
near the identity (Theorem 6.1).
Also, both the transfer factor and the germs on both sides are
  homogeneous by Proposition 4.2.
Extending by homogeneity, we see finally that
  ($**$) holds for all topologically unipotent elements.
This completes the proof.
\qed
\finishpproclaim

% References
\newpage
\noindent
\centerline{ {\smc References} }
\bigskip
\parindent=11mm
\baselineskip=12pt

%
\begin{enumerate}
\item{[{\bf 1}]}
  D. Barbasch and A. Moy,
  ``A Unitary Criterion for $p$-adic Groups,'' Invent. Math\. {\bf 98}
  (1989), no. 1, pp\.~19@--37.
%
\item{[{\bf 2}]}
  J. Bernstein and A.V. Zelevinski,
  ``Induced representations of reductive $p$-adic groups, I,''
  Ann\. Sci\. E.N.S. {\bf 10}
  pp\.~441@--472 (1977).
%
\item{[{\bf 3}]}
  R. W. Carter,
  {\sl Finite Groups of Lie Type: Conjugacy Classes and Complex
     Characters},
  John Wiley and Sons, 1985.
%
\item{[{\bf 4}]}
  W. Casselman,
  ``Introduction to the Theory of Admissible Representations
     of $p$-adic Reductive Groups,'' preprint.
%
\item{[{\bf 5}]}
  W. Casselman,
  ``The unramified principal series of $p$-adic groups I@. 
    The spherical function,'' Compositio-- Math\. {\bf 40}
    (1980), no. 3, pp\.~387@--406.
%
\item{[{\bf 6}]}
  J.W.H. Cassels,
  {\sl Local fields}, London Math\. Soc\. {\bf 3}, Cambridge
  University Press, 1986.
%
\item{[{\bf 7}]}
  L. Clozel,
  ``Orbital integrals on $p$-adic groups:  A proof of 
     the Howe conjecture,''  Ann\. of Math\. {\bf 129} (1989),
     no. 2, pp\.~237@--251.
%
\item{[{\bf 8}]}
  L. Clozel,
  ``The Fundamental Lemma for Stable Base Change,''
  Duke Math\. Jour\. {\bf 61} (1990), no. 1, pp\.~255@--302.
%
\item{[{\bf 9}]}
  S. S. Gelbart, and A. W. Knapp,
  ``Irreducible constituents of principal series of $SL_n(k)$,''
  Duke Math\. Jour\. {\bf 48} pp\.~313@--326 (1981).
%
\item{[{\bf 10}]}
  S. S. Gelbart, and A. W. Knapp,
  ``$L$-indistinguishability and $R$-groups for the
     special linear group,''
  Advances in Math\. {\bf 43} pp\.~101@--121 (1982).
%
\item{[{\bf 11}]}
  T. C. Hales,
  ``The Subregular Germ of Orbital Integrals,''
  thesis, Princeton University, (1986).
%
\item{[{\bf 12}]}
  T. C. Hales,\enspace%
  ``Shalika Germs on $GSp(4)$,''
  {\sl Orbites Uni\-po\-tentes et Repr\'e\-sen\-ta\-tions}, 
  Ast\'erisque {\bf 171-172} (1989).
%
\item{[{\bf 13}]}
  T. C. Hales,
  ``Orbital Integrals on $U(3)$,''
  {\sl The Zeta Function of Picard Modular Surfaces},
  Les Publications CRM, eds\. R\. P\. Langlands and
  D\. Ramakrishnan, (1992).
%
\item{[{\bf 14}]}
  Harish-Chandra\hbox{,\kern-.15em``}Admissible invariant distributions on reductive $p$-adic 
    groups,''
  Queen's Papers in Pure and Applied Math\. No\. {\bf 48}, pp\.~281@--347
  (1978).
%
\item{[{\bf 15}]}
  R. Howe,
  ``The Fourier Transform and Germs of Characters (Case of $GL_n$ 
    over a $p$-adic field),''
  Math\. Ann\. {\bf 208}, pp\.~305@--322 (1974).
%
\item{[{\bf 16}]}
  D. Kazhdan,
  ``On Lifting,'' {\sl Lie Group Representations, II},
  Lecture Notes in Math\. {\bf 1041} (1984).
%
\item{[{\bf 17}]}
  D. Kazhdan and G. Lusztig,
  ``Proof of the Deligne--Langlands conjecture for Hecke algebras,''
  Invent\. Math\. {\bf 87} (1987), no. 1, pp\.~153@--215.
%
\item{[{\bf 18}]}
  R. P. Langlands,
  {\sl Les d\'ebuts d'une formule des traces stables},
  Pub\. math\. de~l'univ\. de~Paris~VII, {\bf 13}.
%
\item{[{\bf 19}]}
  R. P. Langlands,
  ``Orbital Integrals on Forms of $SL(3)$, I,''
  Amer\. Jour\. Math\., {\bf 105} pp\. 465@--506
  (1983).
%
\item{[{\bf 20}]}
  R. P. Langlands and D. Shelstad,
  ``On Principal Values on $p$-adic Manifolds,''
  {\sl Lie Group Representations~II}, Lecture Notes in Math\.
  {\bf 1041}, Springer-Verlag, Berlin (1984).
%
\item{[{\bf 21}]}
  R. P. Langlands and D. Shelstad,
  ``On the Definition of Transfer Factors,''
  Math\. Ann\. {\bf 278}, pp\.~219@--217 (1987).
%
\item{[{\bf 22}]}
  R. P. Langlands and D. Shelstad,
  ``Descent for Transfer Factors,''
  {\sl Grothendieck Festschrift}, Vol II,
  Progress in Math\. {\bf 87} (1990), pp\.~485@--563.
%
\item{[{\bf 23}]}
  R. P. Langlands and D. Shelstad,
  ``Orbital Integrals on Forms of $SL(3)$, II,''
  Canad\. J\. Math\. {\bf 41} (1989), no. 3, pp.~480@--507.
%  
\item{[{\bf 24}]}
  R. Ranga Rao,
  ``Orbital Integrals in reductive groups,''
  Annals of Mathematics {\bf 96} (1972).
%
\item{[{\bf 25}]}
  J. Rogawski,
  ``Applications of the Building to Orbital Integrals'',
  thesis, Princeton University, 1980.
%
\item{[{\bf 26}]}
  F. Shahidi,
  ``Some Results on $L$-indistinguishability for $SL(r)$,''
  Canadian Journal of Mathematics, Vol\.~35, No\.~6,
  pp\.~1075@--1109 (1983).
%
\item{[{\bf 27}]}
  T. A. Shalika,
  ``A theorem on semi-simple $p$-adic groups,''
  Annals of Mathematics {\bf 95} (1972).
%
\item{[{\bf 28}]}
  D. Shelstad,
  ``A Formula for Regular Unipotent Germs,''
  {\sl Orbites Unipotentes et Repr\'esentations},
  Ast\'erisque {\bf 171-172}, (1989).
%
\item{[{\bf 29}]}
  A. J. Silberger,
  {\sl Introduction to Harmonic Analysis on Reductive $p$-adic Groups},
  Princeton University Press, 1979.
%
\item{[{\bf 30}]}
  M-F. Vign\'eras,
  ``Caract\'erisation des int\'egrales orbitales sur un
     groupe r\'eductif $p$-adique,''
     J\. Fac\. Sci\. Univ\. Tokyo,
     Sect\. 1A,
     {\bf 28} (1981), no. 3, pp\.~945@--961.
%
\item{[{\bf 31}]}
  J-L. Waldspurger,
  ``Sur les germes de Shalika pour les groupes lin\'eaires,''
  Math\. Ann\. {\bf 284} (1989), no. 2, pp\.~199@--221.
%
\item{[{\bf 32}]}
  A. V. Zelevinsky,
  ``Induced representations of reductive $p$-adic groups II@.
     On irreducible representations of $GL_n$,''
  Ann\. Sci\., E.N.S.~{\bf 13}
  pp\.~165@--210 (1980).
%
\item{[{\bf 33}]}
  A. V. Zelevinsky,
  {\sl Representations of Finite Classical Groups},
  SLN~869, Springer-Verlag, 1981.
\end{enumerate}

\end{document}