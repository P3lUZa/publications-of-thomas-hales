
% spherical Hecke algebra and motivic integration
% Tex file started June 28, 2016
% 
% 

%Authors: Jorge Cely, Thomas Hales

\newcommand{\XX}[1]{{\it  [To do: #1]}}
\newcommand{\ring}[1]{\mathbb{#1}}
\newcommand{\g}[1]{\langle{#1}\rangle}
\def\op#1{{\operatorname{#1}}}
\def\inv{\op{inv}}
\def\dom{P^+}
\def\Q{{\ring{Q}}}
\def\card{\op{card}}

\def\C{\mathcal C}
\def\N{\mathcal N}
\def\H{\mathcal H}
\def\M{\mathcal M}
\def\T{\mathcal T}

\def\n{{\mathfrak n}}


\def\libel#1{{\text{\sc [#1]~}}\label{#1}}
\def\rif#1{(\ref{#1}-{\text{\sc #1})}}



\section{Introduction}\libel{XX} % dummy label

In this article, we use motivic integration to describe the spherical
Hecke algebra, its Satake transform, and inverse Satake transform.
We use the transfer principle of motivic integration to deduce the fundamental
lemma for the spherical Hecke algebra in sufficiently large positive characteristic.

Let $F$ be a $p$-adic field.  Specifically, 
by $p$-adic field we mean a finite extension of on of the fields $\ring{Q}_p$ or $\ring{F}_p((t))$.
Let $G$ be an unramified reductive group and $H$ an unramified endoscopic group of $G$, both defined over $F$.
Let $\H(G)$ and $\H(H)$ be the spherical Hecke algebras on $G$ and $H$.
Associated with  a morphism $\xi:{}^LH\to {}^LG$ of $L$-groups, there is a homomorphism
$b_\xi:\H(G)\to \H(H)$, obtained by composing three maps: the Satake transformation of $\H(G)$,
the pullback under $\xi$, and the inverse Satake transformation to $\H(H)$.

The spherical Hecke algebra $\H(G)=\H(G,K)$ of functions that are bi-invariant with respect
to a given hyperspecial subgroup $K$ has a linear basis given by characteristic functions $f_\lambda$
of double cosets $K\varpi^\lambda K$.  Here $\varpi$ is a fixed uniformizing element and $\lambda$
runs over the cocharacters in a positive Weyl chamber $P^+$.

A primary aim of this article is to study the function $B_\xi:P^+\times H(F)\to \ring{C}$, given by
$(\lambda,h)\mapsto b_\xi(f_\lambda)(h)$.   We show that $B_\xi$ can be lifted to a constructible motivic
function.   In particular, $B_\xi$ admits a field-independent description.  

As an application, we prove a transfer
principle for the fundamental lemma for the  spherical Hecke algebra.
This asserts that for $p$-adic
fields of sufficiently large residual characteristic, the fundamental lemma for the  spherical Hecke
algebra holds for one field of given residue field if and only if it holds for all fields of the given residue field.
In particular, the fundamental lemma for the spherical Hecke algebra in positive characteristic follows from the fundamental lemma 
for the spherical Hecke algebra in characteristic
zero, which is known.

This application to the fundamental lemma was the original motivation for our work.  Our results overlap with
those of Bouthier, who proves the fundamental lemma for the spherical Hecke algebra in positive
characteristic under the restrictions that the group $G$ is semisimple and simply connected, 
and the endoscopic group is split \cite[Theorem~0.2]{bouthier}.  We obtain a proof of the fundamental lemma for the
spherical Hecke algebra in positive characteristic without the restrictions on the group and endoscopic group.
but unlike Bouthier, we are unable to be explicit in our assumption
on the characteristic of the field.  Lemaire, Moeglin, and Waldspurger propose that the method of close fields might be used to transfer the
fundamental lemma for the spherical Hecke algebra from characteristic zero to positive characteristic, but as far as we know, 
this has not been
carried out~\cite[\S1.3]{LMW}.

The construction of $B_\xi$ involves the Langlands dual ${}^LG$, which is a complex reductive group.
Our constructibility result for $B_\xi$ follows from the  Presburger constructibilty of various
functions on lattices in the dual:  Macdonald's formula, weight multiplicity formulas, the inverse of the weight multiplicity matrix,
the Plancherel measure,
and the Kato-Lusztig formula for the inverse Satake transform.  
When $G$ and $H$ are split, we can take ${}^LG = \hat G$ and ${}^LH=\hat H$ to be connected.  In this case,
formulas of the desired form were previously known for Macdonald, Plancherel, and Kato-Lusztig.  In this article, we
generalize these formulas to disconnected complex reductive groups.


One novelty of this work is that we show how to extend the theory of motivic integration to the Langlands
dual group, by encoding representation-theoretic data of the complex dual group as Presburger constructible functions on the character lattice. These
Presburger functions can then be recombined with constructible functions on the $p$-adic group.  A second innovation is
to encode the entire Hecke algebra into a single constructible function $B$.  This makes it possible to invoke the the transfer
principle of motivic integration a single time, rather than once for each function in the Hecke algebra.  (Invoking the transfer principle
an infinite number of times could potentially leave us with nothing, because we lose finitely many primes with each invocation.)
Also, for the first time, motivic integration has been used to analyze simple cases of abelian reciprocity in local class field theory.

A framework for studying the spherical Hecke algebra through motivic integration is provided by Cely's thesis \cite{cely}.
This article builds on that work.  We thank Julia Gordon, who served on Cely's thesis committee and who provided valuable suggestions.
%It was Bill Casselman who pointed us in the direction of the Kato-Lusztig formula through his preprint \cite{cass-symmetric}.  We thank
%him for many helpful comments.


\section{Motivic Integration}

This section reviews the theory of motivic integration as developed by Cluckers and
Loeser~\cite{cluckers2008constructible}.  No new theorems are presented here.

\subsection{The Denef-Pas language}

The Denef-Pas language is a three-sorted first-order formal language in the sense of model theory.  
Its intended structures are triples $(F,k_F,\ring{Z})$, 
where $F$ is a discrete valued field, 
$k_F$ is the residue field of $F$, 
and the value group of $F$ is the ring of integers $\ring{Z}$ . 
We call the three sorts $VF$ (the valued-field sort), $RF$ (the residue-field sort), and $\ring{Z}$ (the value-group sort).

In general, a first-order formal language is specified by sets of relation symbols and function symbols, including constant symbols.  
The Denef-Pas language has the following relation and function symbols.  
The valued-field sort $VF$ has the symbols of the first-order language of rings $(0,1,+,\times)$.  
The residue field sort also has the symbols of the first-order language of rings.  
The value-group sort is the Presburger language of of an ordered additive group with symbols $(0,+,\le)$.  
Each sort is assumed to have an equality symbol.  
In addition, there are two function symbols $\op{ord}:VF\to\ring{Z}$ (interpreted as the valuation on the valued-field) and $\op{ac}:VF\to RF$ 
(interpreted as the angular component map).  
For the structure $(K((t)),K,\ring{Z})$, where $K((t))$ is the field of formal Laurent series, 
the intended interpretation of $\op{ac}$ 
is the function $\sum_{i\ge N} a_i t^i\mapsto a_N$ 
that returns the first nonzero coefficient of the Laurent series (and sending $0\in K((t))$ to $0$).

First-order languages are intended in the usual sense, with formulas built from logical connectives $(\land)$, $(\to)$, $(\lor)$, $\neg$, equality
$(=)$, variables of the three sorts, function symbols, relation symbols, existential quantifiers of each sort, and universal quantifiers of each sort.


Following the terminology of \cite{gordon}, we call  a {\it fixed choice} any set-theoretic data that does not depend in any way on the Denef-Pas
language, its variables, nor on the structures of $VF$ and $RF$.   Examples of fixed-choices that appear in this paper are Weyl groups, abstract groups,
representations of split reductive groups over $\ring{Q}$, root systems,

\subsection{motivic integration}


Let $\op{Field}_\Q$ be the category of fields of characteristic zero.  

Cluckers and Loeser have used the Denef-Pas language to define various categories.  In particular, there
is a category  $\op{Def}_\Q$ of definable subassignments, given as follows.
For each $(m,n,r)\in\ring{N}^3$, let $h[m,n,r]$ be the functor from $\op{Field}_\Q$ to the category of sets that assigns
to each field $K$, the set $h[m,n,r](K)=K((t))^m\times K^n\times \ring{Z}^r$.  A subassignment of this functor is by
definition, a subset $S(K) \subseteq h[m,n,r](K)$ for each $K\in\op{Field}_\Q$.  
A definable subassignment $S$ is  a subassignment for which there exists a formula $\phi$ in the Denef-Pas language such that for each $K\in\op{Field}_\Q$, 
the set of solutions of $\phi$ in $h[m,n,r](K)$ is $S(K)$.
The definable subassignments are the objects of the category $\op{Def}_\Q$.  
A morphism $\phi:X\to Y$ is a definable subassignment 
\[
\phi\subseteq X\times Y\subseteq h[m,n,r]\times h[m',n',r'] = h[m+m',n+n',r+r']
\]
that is the graph of a function $X(K)\to Y(K)$ for each $K\in \op{Field}_\Q$.

A free parameter refers to a collection of free variables of the same sort in a formula in the Denef-Pas language, ranging over a definable
subassignment.  A bound parameter is similar, except that the variables are all bound by a contiguous block of existential or
a contiguous block of universal quantifiers.

For each definable subassignment $X\in \op{Def}_\Q$, Cluckers and Loeser have defined a ring $C(X)$ of 
{\it constructible motivic functions}.  The construction of this ring is a major undertaking.  Space does not permit us
to give the details of this construction.     The elements of this ring are called constructible motivic functions.  Although
they behave in many ways as functions on $X$,  the elements of the ring are not literal functions in the set-theoretic
sense of a relation.

If $\phi:X\to Y$ is a morphism of definable subassignments, there is a pullback of functions $\phi^*:C(Y)\to C(X)$.  The
pullback $\phi^*$ is a ring homomorphism and is functorial for composition of morphisms: $(\phi\psi)^* = \psi^* \phi^*$.

If $X\to S$ is a morphism of definable subassignments, there is  a subgroup $I_S C(X)$ of $S$-integrable constructible motivic functions.
The intuitive interpretation of an $S$-integrable function $f$ is a function such that the integral over each fiber of $X\to S$ 
is convergent with respect to the canonical motivic measure.  For a morphism $\phi: X\to S$, there is a 
homomorphism of groups $\phi_!:I_SC(X)\to C(S)$ that is called integration over fibers.  It behaves functorially with
respect to composition of morphisms: $(\phi\psi)_! = \phi_!\psi_!$.
In this article, we will always deal with bounded constructible functions.  Such functions are always integrable by \cite{XX}.
Thus, we will not need to deal with integrability issues.

\subsection{Presburger constructible functions}

The ring of constructible functions is a tensor product $P(X) \otimes Q(X)$.  (This is not quite correct; $C(X)$ is the graded
algebra associated to a filtration on the tensor product.)  In terms of the three sorts of the Denef-Pas language,
data related to the value-group sort $\ring{Z}$ is encoded in $P(X)$ and data related to the residue field sort $RF$ is encoded
in $Q(X)$.  The left-hand side $P(X)$ is a ring of {\it Presburger
constructible functions.}  Every Presburger constructible function $f$ gives a constructible motivic function $f\otimes 1$.

Much of what we do in this article is related to constructible functions on integer lattices.  For this, we work with Presburger
constructible functions rather than the entire ring of constructible motivic functions.

\subsection{volume forms}

\XX{describe this.}

\subsection{$p$-adic specialization}

Let $\C$ denote the set of $p$-adic fields.
Let $\C_N\subseteq \C$ denote the subset of fields whose residue characteristic is at least $p\ge N\in\ring{N}$.

In general, we only care about what occurs in fields in $\C_N$ for $N$ arbitrarily large.
To make this precise, suppose that we have for some $N$, a function $X$ with domain $\C_N$.
Then by restriction of domain $\C_i$ to $\C_{j}$, for $N\le i\le j$, we may take the filtered colimit of $X_i=X|_{\C_i}$.
Two functions $X$, $X'$ have the same filtered colimit if they are equal in $\C_i$ for some sufficiently large $i$.

Let $X$ be a definable subassignment of $h[m,n,r]$, and let $f$ be a constructible motivic function on $X$.   There exists
an $N$ such that for all $F\in \C_N$, there are specializations
\[
X(F)\subseteq F^m\times k_F^n\times \ring{Z}^r,  \quad f_F: X(F) \to\ring{C},
\]
We only care about the  filtered colimits of $X$ and $f$.

We warn the reader of a notational overload; we write $X(K)$ or $X(F)$ as $K$ and $F$ range over two quite different
sets of fields.  We use the different symbols $K$ and $F$  to disambiguate the context.
When $K\in \op{Field}_\Q$, the valued field is $K((t))$ and the residue field is $K$; but when $F$ is a $p$-adic field, $F$
is the valued field and its residue field is denoted $k_F$.  We also use $K$ for both a hyperspecial subgroup and for $K\in\op{Field}_\Q$.

The specializations have various expected properties.
If $\phi:X\to Y$ is a morphism of definable subassignments, then we have functions $\phi_F:X(F)\to Y(F)$.
When $f$ is $Y$-integrable on $X$,  integration over fibers $\phi_!(f)$ specializes to integration over fibers in
 $p$-adic fields $F\in \C_N$ (for some $N$).


The functions $f_F:X(F)\to\ring{C}$ that come from constructible motivic functions  $f\in C(X)$ have a special form
\[
q.
\]
\XX{insert formula.}  
We call (the filtered colimits of) these functions:  $q$-constructible functions.  
We  call the specialization of a  definable subassignment a definable set.  
There is an element $\ring{L}$, called the Lefschetz motive, in the ring of constructible motivic functions that specializes to $q_F$
for every $p$-adic field $F$.

We warn the reader that
very different constructible motivic functions can yield the same $q$-constructible function.  For example,
let $[S]\in C(*)$ be the isomorphism class in the residue sort of the set of nonzero squares, considered as a 
constructible motivic functions on a point.  Similarly, let $[N]$ be the class of the set of nonsquares.  Then, under specialization
to $p$-adic fields, the
two functions are equal:
$[S](F) = [N](F) = (q_F-1)/2$, for $F\in\C_1$. However, $[S]$ and $[N]$ are not at all the same constructible motivic function. Indeed,  their values on 
algebraically closed residue fields $K$ are not equal: $[N](K)$ is the emptyset and $[S](K) = K^\times$ is not.
If a constructible motivic function specializes to a $q$-constructible function that is identically zero, then we call
it a null function.

The theory of motivic integration specializes to $q$-constructible functions. To integrate a $q$-constructible function $f$, we lift it to 
a contructible motivic function, use Cluckers-Loeser integration there, then take its specialization again.
Two different lifts differ by a constructible motivic function whose integral specializes to zero. Thus, this
is well-defined.

\section{Definable reductive groups}

We consider definable reductive groups in the sense of \cite{cluckers2011transfer}, \cite{gordon}.
In this work we restrict to unramified reductive groups (quasi-split and split over an unramified extension).

In the definable context, the construction of reductive groups are defined over a definable subassignment
$Z$ called the cocycle space.  In the case of unramified reductive groups we can take this to be a
a subassignment $Z\subseteq h[m,0,0]$.  The coordinates of $Z$ are the coefficients of an irreducible monic 
polynomial defining a degree $r$ unramified extension of $F$.  A field extension $E/VF$ of degree $k$ is identified
with $VF^k = VF[x]/(p)$, where $p$ is a degree $k$ irreducible monic polynomial.

Split reductive groups are defined as definable subassignments in terms of explicit representations of those groups,
identifying them with closed subgroups of $GL(n,F)$.  Quasi-split reductive groups that split over an unramified degree $k$
extension (parameterized by a cocycle space $Z$) are defined in terms of explicit representations of those groups
in $GL(n,E)$, where the unramified degree $k$ extension $E/VF$ is defined by coefficients of the characteristic polynomial in $Z$.

If $G$ is an unramified reductive group, we may construct a hyperspecial subgroup $K$ as a definable subassignment of $G$.

A quasi-split reductive group $G$ carries an invariant differential form $\omega$ of top degree, which is described in the
context of definable subassignments in \cite{gordon}.
We may integrate functions $f\in C_q(G)$ with respect to the invariant measure $|\omega|$.
All integration in this article is assumed to be carried out with respect to invariant measures.  We write, for example,
$\vartheta_!^\inv(f) = \vartheta_!(f|\omega|)$ for the invariant integral of a constructible integrable function $f\in C_q(G)$ with respect to the
morphism $\vartheta:G\to\{*\}$ to a point.

\section{Spherical Hecke algebra}

Let $G$ be a definable unramified reductive group over a cocycle space $Z$.  Let $A$ be a maximal split torus in $G$.  Suppose that
it has dimension $r$.  We identify its lattice of cocharacters $X_*(A)$ with $\ring{Z}^r$ by a choice of free generators of $X_*(A)$.
This allows us to treat $X_*(A)$ as a definable subassignement of $h[0,0,r] = \ring{Z}^r$.

There is a perfect pairing $\langle \cdot,\cdot\rangle:X^*(A)\times X_*(A) \to \ring{Z}$.
For each $\lambda\in X_*(A)$, there is a definable subassignement $A_\lambda \subseteq A$ given by the formula
\[
\{ a \in A \mid \op{ord}(\phi(a)) = \langle \phi,\lambda \rangle \}.
\]
There is a definable subassignment of $X_*(A)\times A$
given by pairs $(\lambda,a)$ satisfying the same formula.

Let $P^+\subseteq X_*(A)$ be the set of cocharacters in the positive Weyl chamber.

\begin{lemma} $P^+$ is a definable subset (of $\ring{Z}^r$).
\end{lemma}

\begin{proof} $P^+$ is defined by linear inequalities, which can be expressed in the Presburger language.
\end{proof}

We prove the existence of a Cartan decomposition for $G$.

\begin{lemma} There is a definable subassignment of $P^+\times G$ given by the formula
\[
D_G = \{(\lambda,g)\in P^+\times G \mid g \in K A_\lambda K \}.
\]
The fiber $D_G(\lambda)$ over each $\lambda\in P^+$ is definable.  Moreover,
$D_G(\lambda)\cap D_G(\lambda') = \emptyset$, for $\lambda\ne \lambda'$.
\end{lemma}


We note that Bruhat-Tits gives the Cartan decomposition over general discrete valued fields.

\begin{remark}   $D_G$ captures the entire spherical Hecke algebra as a single
definable subassignment.  In applications, it will be important to work with a single subassignment
rather than an infinite basis of the spherical Hecke algebra.
\end{remark}

We define the spherical Hecke function to be the characteristic function of $D_G$, viewed as a $q$-constructible function.

Let $G$ be an unramified reductive group and let ${}^LG = \hat G\rtimes \langle \sigma\rangle$ be its Langlands dual, where
$\sigma$ acts on $\hat G$ as the action of the Frobenius on the root datum.
Let $Y^* = X^*(\hat T)^\sigma$, and let $W^\sigma$
is the $\sigma$-fixed subgroup of $W$.   

We have  the group algebra $\ring{C}[Y^*]$ of $Y^*$, with basis elements
$e^\lambda$ corresponding to group elements $\lambda\in Y^*$.
We have a basis of $W^\sigma$-invariant functions in $\ring{C}[Y^*]$ given by
\[
m_\mu = \sum_{\lambda\in W^\sigma(\mu)} e^\lambda, \quad \mu\in P^+,
\]
where $W^\sigma(\mu)$ is the orbit of $\mu\in\ring{C}[Y^*]$ under $W^\sigma$.

The Satake transform $f\mapsto \hat f$
is an isomorphism  $\H(G)\to\ring{C}[Y^*]^{W^\sigma}$.
We have a change of basis $\hat f_\lambda = \sum_\mu s_{\lambda,\mu} m_\mu$, for some coefficients $s_{\lambda,\mu}$.

Let us lift the Satake transform to the $q$-constructible setting.  The Satake transform involves
a term $q^{\langle\rho,\mu\rangle}$, where $\rho = \frac{1}{2} \sum_{\alpha > 0} \alpha$.   Constructible motivic functions
can only contain integral powers of $q$, which is not always the case here.  However, \cite{XX} notes
that the theory of constructible motivic functions can be extended to allow half-integers.  We extend
the theory in that way without further comment.\footnote{Even though intermediate steps such as the 
Satake transform are not integral in $q$, 
I suspect that the square roots can always be
eliminated from the definitions in such a way that the statement of the fundamental lemma itself only
involves integral powers of $q$.  In particular,
in every case I have computed (and that includes many cases), $\rho_G - \rho_H$ is an integral sum of roots.}

\begin{lemma}\label{lemma:satake} There is a $q$-constructible function $s$ on $P^+\times P^+$ that
specializes to the function $(\lambda,\mu)\mapsto s_{\lambda,\mu}$,
\end{lemma}

\begin{proof} 
The coefficients $s$ are given by a integral of a $q$-constructible function on $P^+\times G$:
\[
(\lambda,\mu)\mapsto s_{\lambda,\mu}=\frac{q^{\langle \rho,\mu\rangle}}{\op{vol}(A_0)} \int_{A_\mu} \int_N D_G(\lambda,t n) dn dt.
\]
Here $N$ is the unipotent radical of a Borel subgroup $B$ containing maximally split Cartan subgroup
$T$.  The subgroup $A_0 = T\cap K$ is a maximal compact subgroup of $T$.  Its volume $\op{vol}(A_0)$ specializes
to a polynomial in $q$ that can be written as a product of cyclotomic polynomials.  Adjusting $\op{vol}(A_0)$ by a 
null function, we may assume that $\op{vol}(A_0)$ is the specialization of a product of cyclotomic polynomials in $\ring{L}$.
Cyclotomic polynomials in $\ring{L}$ are invertible constructible motivic functions.  Thus, the denominator is
permissible.  Integration is motivic integration with respect to invariant volume forms on $N$ and $A_0$.
The pushforward under a definable morphism (integration over fibers) carries $q$-constructible functions
to $q$-constructible functions.
Therefore the coefficients $s$ give a $q$-constructible function.

The $q$-constructible function is given explicitly by Macdonald's formula \cite{casselman1980unramified}.
\end{proof}


\section{Constructibility in complex groups}

In this section, we check that some functions related to the finite dimensional representations of complex reductive groups
are Presburger constructible functions on the appropriate integer lattices.  In this section, constructible means Presburger
constructible.



\subsection{twisted formulas}

Let ${}^LG = \hat G \rtimes \langle\sigma\rangle$ be the Langlands dual of an unramified group $G$.
It is a semidirect product of a connected complex reductive group $\hat G$ and a finite cyclic group
generated by an outer automorphism $\sigma$ of $\hat G$.  The automorphism $\sigma$ is assumed
to preserve a pinning of $\hat G$; that is a Borel subgroup $\hat B$, Cartan $\hat T\le \hat B$, and
set of simple positive roots.  

We consider irreducible characters of ${}^LG$, restricted
to the component $\hat G \rtimes \sigma$.  We write $Y^*$ for the set of $\sigma$-fixed
elements of $X^*(\hat T)$.  Note that we have identifications
\[
Y^* = X^*(\hat T)^\sigma = X_*(T)^\sigma = X_*(A).
\]
Using the standard additive notation for a multiplicative group, 
we write $(1-\sigma)\hat T = \{ t\sigma(t)^{-1}\mid t\in\hat T\}$, and let $\hat T_\sigma = \hat T/((1-\sigma)\hat T)$.
If $\lambda$ is a $\sigma$-fixed highest weight in $X^*(\hat T)$, then it is trivial on $(1-\sigma)\hat T$,
giving a morphism $Y^*\to X^*(\hat T_\sigma)$.  This is an isomorphism of lattices.  We identify
these lattices by this isomorphism.

Let $\dom$ be the set of dominant weights in $Y^*$, which we identify with set $P^+$ of cocharacters in the positive chamber $X_*(A)$.
If $\lambda$ is a dominant weight in  $X^*(\hat T_\sigma)$, we may construct an irreducible representation
of $\hat G$ with highest weight $\lambda\in X^*(\hat T)$.  There is an extension of the representation to ${}^LG$ in
which $\sigma$ fixes a vector in the highest weight space.  Let $\tau_\lambda$ be the
character of the restriction of the representation to $\hat G\rtimes \sigma$.  This restriction depends only
on the conjugacy class of $g\rtimes\sigma$, that is, the $\sigma$-conjugacy class of $g\in \hat G$.
Each $\sigma$-conjugacy class depends only on its image in $\hat T_\sigma$.  The weight space of $\mu\in X^*(\hat T)$ is
fixed by $\sigma$ exactly when $\mu\in Y^*$. We can therefore
consider $\tau_\lambda\in \ring{C}[Y^*]$.

Let $\hat T$ be a complex torus with character group $X^*(\hat T)$.  
By fixing a basis, we may identify $X^*$ with
with $\ring{Z}^r$ for some $r$.  

\begin{remark}\label{rem:matrix}
For purposes of constructibility, we consider $\ring{Z}^r$ and also $Y^*$ as 
definable subassignments $h[0,0,r]$. When dealing with $\ring{Z}^r$, integrals over fibers
in the sense of motivic integration are discrete sums.  For example, if $(\lambda,\mu)\to a_{\lambda,\mu}$
and $(\mu,\nu)\to b_{\mu,\nu}$ are constructible functions of integer parameters $(\lambda,\mu,\nu)\in L\times M\times N$,
then we may intepret the matrix product $(\lambda,\nu)\to \sum_{\mu} a_{\lambda,\mu} b_{\mu,\nu}$ as a
fiber integral as follows.  We pull $a_{\lambda,\mu}$ and $b_{\mu,\nu}$ both back to $L\times M\times N$, multiply
them as constructible functions on $L\times M\times N$, then integrate (sum) over the fibers of the projection morphism
$L\times M\times N\to L\times N$.
\end{remark}

%We write $e^\lambda$ for
%elements of basis of the group algebra $\ring{C}[X^*]$, with $\lambda\in X^*$.

We fix a root system $R \subseteq X^*(\hat T)$.  We consider this a fixed choice.
Let $\hat G$ be a complex reductive group with Cartan subgroup $\hat T$ and
root system $M$.
Let $R^+$ be the set of positive roots with respect to some order.



\subsection{weight multiplicities}

We have a partition function given by
\[
\prod_{\alpha\in R_1^+}\frac{1}{ (1-e^{-\alpha})} = \sum_\mu p(\mu) e^{-\mu},
\]
where the product runs over $\alpha\in R_1^+$. 
The root system $R_1$ is defined in \cite[\S3.12]{chriss}.
Let $R_1^+$ be the set of positive roots in  that root system.

\begin{lemma}\label{lemma:partition}
The partition function $p$ is a Presburger constructible function of $\mu$.
\end{lemma}

\begin{proof} Let $k$ be the number of positive roots in $R_1^+$.
Fix an enumeration $\alpha_1,\ldots,\alpha_k$ of the positive roots.
Expanding the series for $p(\mu)$, we observe that
the partition function $p$ is the fiber integral (sum) of the characteristic function of $\ring{N}^k$
for the morphism
\[
\ring{N}^k\to X^*(T),\quad (n_1,\ldots,n_d) \mapsto n_1\alpha_1+\cdots+n_k\alpha_k.
\]
\end{proof}

\begin{lemma}
$\dom$ is a definable subset of $Y^*$.
\end{lemma}

\begin{proof}
$\dom$ is given by a set of linear inequalities that can be expressed in the Presburger language.
\end{proof}


Let $m_{\lambda,\mu}\in \ring{N}$ be the multiplicity of the weight $\mu\in X^*(T)$
in the irreducible representation with highest weight $\lambda$.


\begin{lemma}  The weight multiplicity function $m_{\lambda,\mu}$  is a Presburger constructible function.
\end{lemma}

\begin{proof} 
The twisted Weyl character formula is
\[
\tau_\lambda = \sum_{w\in W^\sigma} \frac{(-1)^{\ell(w)} e^{w (\lambda+\rho)-\rho}}{ \prod_{\alpha\in R_1^+} (1 - e^{-\alpha})},
\]
where $\rho = \frac{1}{2} \sum_{\alpha\in R_1^+} \alpha$ and $\ell(w)$ is the length of $w$ with respect to the reflection group $W^\sigma$.
This is the version of the formula found in \cite{chriss}, \cite{haines2016dualities}.



Since constructible functions form a ring, it is enough to check the constructibility of a summand for
fixed $w$.  The sign $(-1)^{\ell(w)}$ is a fixed choice.  The contribution of $w$ to the multiplicity $m_{\lambda,\nu}$ is the
pull back of the constructible function $(-1)^{\ell(w)} p(\mu)$ under the definable morphism
\[
\dom\times X^*\to X^*,\quad (\lambda,\nu)\mapsto \mu = w(\lambda+\rho)-(\rho+\nu).
\]
The result follows.
\end{proof}



\subsection{inverting weight multiplicities}

In this section, we follow  van Leeuwen's algorithm to invert the weight multiplicity
formula~\cite{vanleeuwen}.  This algorithm yields an explicit formula for the inverse and 
implies constructibility (Theorem \ref{lemma:van-leeuwen}).
For type $A_n$, van Leeuwen's formula agrees with the inverse of the Kostka
matrix described in \cite{duan}.


We have two bases of $\ring{C}[Y^*]^{W^\sigma}$, given by $\{m_\lambda\}$ and $\{\tau_\lambda\}$, indexed
by $\lambda\in \dom$.  
The change of basis matrix expressing $\chi_\lambda$ in terms of $m_\mu$ is the weight multiplicity
matrix $m_{\lambda,\mu}$, which is constructible by an earlier lemma. 

We define a dot operator $w\bullet \mu = w(\mu+\rho)-\rho$, for $w\in W^\sigma$ and $\mu\in Y^*$.
We define the alt-symmetrizer operator by
\[
J:\ring{C}[Y^*]\to \ring{C}[Y^*],\quad J(f) = \sum_{w\in W^\sigma} (-1)^{\ell(w)} w(f e^\rho) e^{-\rho}.
\]
It has the properties $J(f) = f J(1)$ if $f\in \ring{C}[X^*]^W$ and $J(\tau_\lambda) = J(e^\lambda)$.


We have definable set $\Lambda_0\subseteq Y^*$ of characters $\mu$ such that $\mu$ is fixed by some
reflection in $W^\sigma$.  For each $w\in W^\sigma$, we have a definable set 
\[
\Lambda_w = \{\lambda\in X^*\setminus \Lambda_0\mid w\bullet \lambda \in\dom\}
\]
These sets partition $Y^*$, so that each $\lambda\in Y^*$ belongs to a unique $\Lambda_x$, for $x\in W^\sigma\cup\{0\}$.

We define a desymmetrizer operator $L$ by
\[
L(e^\mu) = \begin{cases}
0,& \mu\in \Lambda_0;\\
(-1)^{\ell{w}} e^{w\bullet \mu},& \mu\in\Lambda_w.
\end{cases}
\]
We extend $L$ linearly to $\ring{C}[Y^*]$.
We have $L(\tau_\lambda) = e^\lambda$, for $\mu\in\dom$.  This means that
for any $f = \sum_\lambda c_\lambda \tau_\lambda \in \ring{C}[Y^*]$, the coefficient $c_\lambda$
is the coefficient of $e^\lambda$ in $L(f)$.

For each subgroup $H\le W^\sigma$, we have a definable set
$M_H \subseteq Y^*$ given by the set of $\mu\in Y^*$ such that
the stabilizer of $\mu$ in $W$ is $H$.  
The sets $M_H$ partition $Y^*$. 
The basis $m_\mu$ is given
as 
\[
m_\mu = \sum_{w\in W/H} e^{w \mu},
\]
for $\mu\in M_H$.

We have definable sets 
\[
E_{H,w,w'} = \{(\mu,\lambda)\in M_H\times \dom \mid w\mu\in\Lambda_{w'},\quad w'\bullet(w\mu)=\lambda\}.
\]
indexed by subgroups $H\le W$, cosets $w\in W/H$, and elements $w'\in W$.

We have a change of basis matrix $n_{\mu,\lambda}$ 
\[
m_\mu = \sum_{\mu,\lambda} n_{\mu,\lambda} \tau_\lambda.
\]

\begin{theorem}\label{lemma:van-leeuwen} $n_{\mu,\lambda}$ is a Presburger constructible function on $\dom\times\dom$.
\end{theorem}

\begin{proof} This is a consequence of van Leeuwen's algorithm. \XX{clean this up.}

It is enough to show that the restriction of $n$ to each of the definable sets
$M_H\times \dom$ is constructible.  Van Leeuwen's formula is that $n$ restricted
to $M_H\times \dom$ is 
\[
\sum_{(w,w')\in (W/H)\times W} (-1)^{\ell(w)} \op{char}\, E_{H,w,w'}.
\]
Since $E_{H,w,w'}$ is definable, its characteristic function is constructible.
A finite sum of constructible functions is constructible.
\end{proof}

\subsection{branching formulas}

We do not need the results in this section, but while we are on the topic of constructibility,
we point out the constructibility of branching multiplicities follows from the Kostant
multiplicity formulas.  For example, we have the following corollary of the branching
multiplicity formula \cite[Theorem ~8.2.1]{goodman}.

\begin{lemma} Let $H\le G$ be complex reductive groups with Lie algebras ${\mathfrak h}\subseteq {\mathfrak g}$.
Fix maximal tori $T_H\le T_G$ with Lie algebras $t_h\subseteq t_g$.  Assume that there is an element
$X_0\in t_h$ such that $\langle \alpha,X_0\rangle>0$ for every positive root of ${\mathfrak g}$.
Let $\dom_G$ and $\dom_H$ be the sets of dominant weights in $G$ and $H$.  Let $m(\lambda,\mu)$
be the multiplicity of the irreducible $\mathfrak h$-module with highest weight $\mu$ in the irreducible
$\mathfrak g$-module with highest weight $\lambda$.  Then $m(\lambda,\mu)$ is a Presburger constructible
function on $\dom_G\times\dom_H$.
\end{lemma}

\begin{proof}  Kostant's formula expresses each branching multiplicity as a finite sum of partition
functions.  Each partition function is rational.  
Thus, the argument used in the proof of Lemma~\ref{lemma:partition} applies.
\end{proof}

Explicit formulas for branching multiplicities are typical of what Presburger constructible functions look like.
Typically branching formulas look like products of linear factors depending on cases that can be
described by linear inequalities on parameters $\lambda$ and $\mu$.

\XX{give an example.}

We will not pursue the topic, but we can similarly investigate the constructibility of the function giving the
multiplicities of $\tau_\mu$ in $\op{Sym}^k \tau_\lambda$, and related operations on characters.




\section{A twisted Kato-Lusztig formula}

We need the Presburger constructibility of the change of basis matrix (the Kato-Lusztig formula \cite{kato1982spherical} in the twisted case) for 
$\tau_\lambda$ in terms of the basis of $\ring{C}[Y^*]^{W^\sigma}$ provided by the Satake transform.  This is based on the Plancherel formula
for spherical functions \cite{macdonaldspherical}.

\begin{conjecture}\label{conj:kato}  Let $G$ be an unramified connected reductive group with $L$-group ${}^LG = \hat G \rtimes \langle \sigma\rangle$.
Define $t_{\lambda,\mu}$ by
\begin{equation}\label{eqn:kato}
\tau_\lambda = \sum t_{\lambda,\mu} \hat f_\lambda.
\end{equation}
Then the function $(\lambda,\mu)\mapsto t_{\lambda,\mu}$ is represented by a Presburger constructible function on $\dom\times \dom$.
\end{conjecture}

The conjecture holds when $G$ is split by Kato's formula that expresses the coefficients $t_{\lambda,\mu}$ as shifted values of the partition function.
We briefly recall Kato's proof of his formula in the untwisted case: ${}^LG=\hat G$ (that is, $G$ is split).  
We will see that Presburger constructibility is a consequence
of some very generic features of Plancherel, Macdonald's formula, and the Weyl character formula.

Let $\hat S_1$ be the maximal compact subgroup of the complex torus $\hat S = \hat T/(1-\sigma \hat T)$.   
We continue to use the notation we have introduced for the twisted case, even though we restrict for a moment to the untwisted case.
In the untwisted case, $T=A$ is split, and
\[
\hat S  = \hat T = X_*(\hat T)\otimes \ring{C}^\times  = \op{Hom}(X_*(T),\ring{C}^\times).
\]
A function $f\in \ring{C}[Y^*]$ can be considered a continuous function on $\hat S$
by $s\mapsto e^\lambda(s) = s(\lambda)$, for $\lambda\in X_*(T)$.
Let $ds$ be the invariant measure on $\hat S_1$, normalized to have volume $1$.  
Then $e^\mu(\bar s) = e^{-\mu}(s)$ and we have an orthonormal basis,
\begin{equation}\label{eqn:on}
\int_{\hat S_1} e^\lambda e^{-\mu} ds = \delta_{\lambda,\mu}.
\end{equation}
There is a Weyl-group invariant Plancherel measure $dm(s)$ on $\hat S_1$.
We write 
\[
(f,g) = \int_{\hat S_1} f(s) \bar{g} {(s)} dm(s), 
\]
for $f,g\in \ring{C}[Y^*]$.
By Macdonald,
\begin{equation}\label{eqn:ip}
(\hat f_\lambda,\hat f_\mu) = \delta_{\lambda,\mu} c_\lambda,
\end{equation}
for explicit constructible invertible factors $c_\lambda$.

If $f,g\in \ring{C}[Y^*]$, and if $f$ is Weyl group invariant, then we may push an average over to $f$ to obtain
\begin{equation}\label{eqn:inv}
(f,\frac{1}{|W|}\sum_{w\in W} w(g)) = (f,g).
\end{equation}
By Equation (\ref{eqn:kato}) and orthogonality (\ref{eqn:ip}), we expand with respect to an
orthogonal basis:
\[
t_{\lambda,\mu}  = (\tau_\lambda,\hat f_\mu)/c_\mu.
\]
We obtain a formula for $t_{\lambda,\mu}$ by computing the right-hand side using explicit formulas
for the Weyl character $\tau_\lambda$, Macdonald's formula for $\hat f_\mu$, and  Plancherel $dm(s)$.
Macdonald's formula is a sum over $w\in W$, which we may reduce to a single term $w=1$ by (\ref{eqn:inv}).
These are all rational functions.  We formally expand the rational functions as infinite series in the basis $e^\nu$ and
use orthogonality (\ref{eqn:on}) to get $t_{\lambda,\mu}$ as the coefficient of $1=e^0$ in the expansion.
The Presburger constructibility is essentially equivalent to the rationality of the product of the Weyl, Macdonald (at $w=1$), and Plancherel formulas.


\XX{work with Casselman here.}


\section{The constructibility of $B$}

Let ${}^LG$ be the Langlands dual of an unramified reductive group $G$.  Let ${}^LH$ be the dual of an
unramified endoscopic group $H$.  We assume that both $H$ and and $G$ are given in the category of definable subassignments
over a cocycle space $Z$.  We can assume that the cocycle space $Z$ is the same for $H$ and $G$.
As part of the cocycle space data $Z$, we assume we are given a quasi-Frobenius element $\op{qFrob}$ that defines an
automorphism of the root datum $(X^*,X_*,R,R^\vee)$.

There exists a homomorphism ${}^LH\to {}^LG$ that factors through a semidirect product with a finite group $\langle\sigma\rangle$:
\[
\xi:\hat H \rtimes \langle \sigma\rangle \to \hat G \rtimes \langle \sigma\rangle.
\]
We fix such a homomorphism $\xi$.  

Working $p$-adically, Langlands gives describes a homomorphism $b = b_\xi$
from the spherical Hecke algebra of $G$ to the spherical Hecke algebra of $H$.
If $f$ belongs to the spherical Hecek algebra of $G$, its Satake transform belongs
to $\ring{C}[\hat T_\sigma]^{W^\sigma}$, where $W^\sigma$ is the set of fixed points in $W$ under $\sigma$.
The morphism $\xi$ gives a restriction map
\[
\ring{C}[\hat T_\sigma]^{W^\sigma} \to \ring{C}[\hat T_{H,\sigma}]^{W_H^\sigma}
\]
to the corresponding ring for $H$.  The inverse Satake transform of the image of $f$ in
\[
\ring{C}[\hat T_{H,\sigma}]^{W_H^\sigma}
\]
is $b_\xi(f)$.

The following is our main result.

\begin{theorem}
Let $G$ be an unramified connected reductive group with unramified endoscopic group $H$, both considered as definable
subassignments over a cocycle space $Z$.  Fix an $L$-embedding $\xi:{}^LH\to {}^LG$ that factors through a finite
cyclic group $\langle \sigma\rangle$; that is, $\xi:\hat H\rtimes \langle \sigma\rangle \to \hat G \rtimes \langle \sigma\rangle$.
Assume Kato-Lusztig constructibility (Conjecture \ref{conj:kato}) for the complex group $\hat H$ with automorphism $\sigma$.
Then
there is a $q$-constructible function $B$ on $P^+_G\times H$ and a natural number $N$ with the following specializations:
\[
B(\lambda,h)_F = b_\xi(f_{F,\lambda})(h),\quad \text{for } h\in H(F),
\]
for all $p$-adic fields in $F\in C_N$.  
\end{theorem}

Recall that for each $F$, we let  $f_{F,\lambda}$ denote the characteristic function
of the double coset $K\varpi_F^\lambda K$ in the unramified reductive group $G$ over $F$.

The the Kato-Lusztig constructibility assumption is known when $H$ is split ($\sigma=1$).
In that case, we obtain our theorem without assumption.

The theorem implies that the homomorphism $b_\xi$ has a uniformity as the $p$-adic field
varies, and as $\lambda$ varies.  It suggests the existence of a sort of ``Macdonald-Kato-Lusztig''
formula for $b_\xi$.
Stated slightly differently, there is a single constructible function that unites
all of the functions $b_\xi (f^G_\lambda)$ on the endoscopic group, as $\lambda$ varies.

\begin{proof}
We have done most of the work already for this theorem.
By Lemma~\ref{lemma:satake}, the change of basis $D_G(\lambda,\cdot)$ to the basis $\mu_\lambda$
is constructible.  By an easy observation, the restriction of $\mu^G_\lambda$ is expressed in terms of the basis $\mu^H_\mu$ through
constructible coefficients $d_{\lambda,\mu}$:
\[
\mu^G_\lambda| = \sum_{\mu} d_{\lambda,\mu} \mu^H_\mu.
\]
By Theorem~\ref{lemma:van-leeuwen}, the change of basis from $\mu^H_\lambda$ to $\tau^H_\mu$ is constructible.
By the Kato-Lusztig constructibility assumption, the change of basis from 
$\tau^H_\lambda$ to $\hat f^H_\mu$ or $D_H(\lambda,\cdot)$ is constructible.  

Composition of change of basis is given by matrix multiplication of the transformation matrices.
This matrix multiplication is constructible by Remark \ref{rem:matrix}.  Thus the transformation $b_\xi$
expresses each $D_G(\lambda,\cdot)$ as a linear combination with constructible coefficients
of the $D_H(\mu,\cdot)$.   In other words, it is a linear combination of some constructible functions
$d(\lambda,\mu)$, indexed by $\lambda$ and summed over $\mu\in X_*(A_H)$.  This sum
is a fiber integral of the constructible function $d$ on $X_*(A_G)\times X_*(A_H)$ for the
projection to $X_*(A_H)$.  It is therefore a constructible function.

In fact, the support of $d$ lies in a definable set $E$ such that the fibers of $E\to X_*(A_H)$
are finite, so there are no integrability issues.
\end{proof}

\begin{remark}  We have stated $q$-constructibility results in terms of the limiting behavior on
$p$-adic fields $\C_N$ for $N$ large.  However, the formulas we obtain for $B$ should in fact hold for all
$p$-adic fields.
\end{remark}

\section{transfer principle for the fundamental lemma for the spherical Hecke algebra}


\subsection{transfer principle}

We review the transfer principle from \cite{cluckers2010constructible}.

\subsection{enumerated Galois groups}

We deal with field extensions and Galois groups in the way described in \cite{gordon} and \cite{cluckers2011transfer}.
We let $\Sigma$ be an abstract group with fixed enumeration $1=\sigma_1,\ldots,\sigma_n$ of its elements.  We
assume a fixed short exact sequence
\[
1\to \Sigma^t\to\Sigma\to\Sigma^{unr}\to 1,
\]
with $\Sigma^t$ and $\Sigma^{unr}$ both cyclic.
The group $\Sigma$ plays the role of a Galois group with inertia subgroup $\Sigma^t$ and unramified quotient $\Sigma^{unr}$.
We treat this data as an abstract fixed choice, without a priori connection to the Galois group of any particular extension of 
$p$-adic fields.  

We may fix an abstract root datum and choose an action of $\Sigma$ on the root datum, stabilizing the set of simple roots.
Through this action on the root datum, $\Sigma$ acts on the Weyl group, and we may construct the semidirect product $W\rtimes \Sigma$.

\subsection{transfer factors}

In this section, we assume familiarity with the Langlands-Shelstad transfer factor \cite{langlands1987definition}.

In \cite{gordon}, we showed that the Lie algebra transfer factor is a constructible motivic function.  This proof extends to a proof
that the group transfer factor is constructible in a definable neighborhood of the identity.  There are two potential obstructions to
extending the proof of constructibility from
the neighborhood to the entire group.  The first potential obstruction is the $a$-data.  In the Lie algebra a special choice is available
for the $a$-data that does not work beyond a small neighborhood of the identity in the group.   The second potential obstruction is 
the $\chi$-data.  These are collections of multiplicative characters.  Current versions of motivic integration do not allow the unrestricted
use of multiplicative character.  In the Lie algebra and in a small neighborhood of the identity, the $\chi$-data becomes trivial.

We use the canonical normalization of transfer factors given in \cite[\S7]{hales1993simple}.  The canonical normalization requires
on a choice of an admissible pinning.  The admissible pinning involves a choice of simple root vectors $X_\alpha$ relative fixed Borel subgroup and Cartan.
The choices $X_\alpha$ range over a definable subassignment, and we obtain the canonical normalization by introducing a free
parameter into the transfer factor ranging over the definable subassignment.

\subsection{$a$-data}

To define the transfer factor for $p$-adic  fields, a choice of $a$-data is made, but the transfer-factor is in fact independent of the choice of $a$-data.

This section introduces a definable subassignment of $a$-data and introduces an explicit free variable $a$ into the transfer factor that ranges over
the definable subassignment of $a$-data.  The tuple $a$-data will be indexed by a fixed choice of indexing set.  

We begin wih a review of $a$-data for a $p$-adic field, then show how to make the construction as a definable subassignment.
Let $\Sigma$ be the Galois group of a Galois extension $L/F$.  We assume that $\Sigma$ acts on a finite set $R$ of roots.
The $a$-data are a collection of constants $a_\alpha\in L^\times$ indexed by $\lambda\in R$ such that
\begin{equation}\label{eqn:a}
a_{-\lambda} = -a_\lambda,\quad a_{\sigma\lambda} = \sigma(a_\lambda),\quad \text{ for } \sigma\in \Sigma.
\end{equation}
Let $\epsilon:R\to R$, given by $\epsilon(\lambda)=-\lambda\ne\lambda$.  Let $O$ be the orbit of some $\lambda\in R$ under $\langle \Sigma,\epsilon\rangle$.
The choice of $a$-data can clearly be made orbit by orbit.
If there is no $\sigma\in \Sigma$ such that $\sigma\lambda=-\lambda$, we have a specific choice of $a$-data (selecting a given $\lambda\in O$) given by
\[
a_{\sigma\lambda}=1,\quad a_{-\sigma\lambda}=-1,\quad \sigma\in\Sigma.
\]
If $\sigma_0\in\Sigma$ gives $\sigma_0\lambda=-\lambda$,  then we let $F_{+\lambda}$ be the fixed field of $\Gamma_{+\lambda} = \{\sigma\in\Sigma\mid \sigma\lambda=\lambda\}$
and we let
$F_{\pm\lambda}$ be the fixed field of $\Gamma_{\pm\lambda} = \{\sigma\in\Sigma\mid \sigma\lambda=\pm\lambda\}$.
The extension $F_{+\lambda}/F_{\pm\lambda}$ is quadratic.
We may choose $a$-data by fixing $a_\lambda\in F_{+\lambda}$ such that $\sigma_0(a_\lambda) = -a_\lambda$ then extending uniquely to the entire orbit  by the relation (\ref{eqn:a}).
Specifically, the choice of $a_\lambda$ can be taken to run over units of $F_{+\lambda}$ such that its square is a nonsquare in $F_{\pm\lambda}$, when the quadratic extension is unramified.
We take $a_\lambda$ to run over uniformizers in $F_{\lambda}$ such that its square lies in $F_{\pm\lambda}$, when the quadratic extension is ramified.
We see by these explicit descriptions that $a_\lambda$ is a parameter in a definable subassignment.


\subsection{$\chi$-data}



The treatment of $\chi$-data is not at all obvious from the point of view of constructible motivic functions for two significant reasons.
First, the $\chi$-data are a collection of multiplicative characters, and multiplicative characters are beyond the scope of current versions of motivic integration.
In fact, the problem of multiplicative characters is a major research topic in motivic integration.
Second, they $\chi$-data enter into the transfer factor through the reciprocity law of local class field theory for tori, as developed by Labesse and Langlands.
Local reciprocity is also beyond the scope of current versions of motivic integration.

Nevertheless, although the calculations are not complete, there are reasons to believe that $\chi$-data can be treated in a constructible motivic way.
Explicit calculations show that the multiplicative characters $\chi_\lambda$ can be chosen to have order dividing $4$.  Each coset on which $\chi_\lambda$ is
constant is a definable subassignment.  Also, the reciprocity law is used twice so that we end up with a character $\langle a,\cdot\rangle$ on a $p$-adic torus.
Ultimately, the complex dual group does not appear.  
There are other indications of constructibility: Bouthier's formula for transfer factors for split groups in terms of the Steinberg section, 
and descent formulas for transfer factors for elements in the hyperspecial subgroup in \cite{hales1993simple}.

\subsection{$\Delta_{II}$}
Two terms in the transfer factor rely on multiplicative characters constructed from $\chi$-data: the terms $\Delta_{II}$ and the term $\Delta_2$.

\begin{lemma}  There is a $q$-constructible function representing $\Delta_{II}$ (after introducing some free parameters ranging over definable subassignments).
\end{lemma}

\begin{proof}  We begin with an explicit construction of some characters for a $p$-adic field.  Then we analyze the construction to
see that it can be done constructibly.

Let $F_+/F_\pm$ be a quadratic extension of $p$-adic fields.  Let $\varpi_+$ be a uniformizer in $F_+$.  We define a multiplicative character $\chi_+ = \chi_{F_+/F_{\pm}}:F_+\to \ring{C}^\times$
as follows.  If $F_+/F_\pm$ is unramified, let $\chi_+$ be the unramified character of order two.

If $-1$ is a square in $F_+$, we define $\chi_+$ by $\chi_+(\varpi_+) = i\in\ring{C}$ and $\chi_+$ restricted to units is the unique character of order two.

If $-1$ is a nonsquare in $F_+$, we define $\chi_+$ by $\chi_+(\varpi_+)=1$ and $\chi_+$ restricted to units is the unique character of order two.

In every case, $\chi_+^4 = 1$.  

Now we analyze  constructibility.  The condition that $-1$ is a square or nonsquare is a definable condition.
Assume that $F_+$ and $F_\pm$ are both extensions of $VF$, presented as usual by a definable space of the characteristic polynomial of a generator of the fields.  
Introduce a free parameter $\varpi_+$ that runs over the constructible subassignment of uniformizers in $F_+$.
We claim that $\chi_+$ is a linear combination of characteristic functions
\[
\chi_+ = \sum_{\zeta\in\mu_4(\ring{C})} \zeta\, \op{char}(D_\zeta).
\]
where each $D_\zeta$ is constructible over the space of parameters.  
This is essentially obvious: $F_+/F_\pm$ being unramified is a definable condition on the coefficients of the characteristic polynomial;
the unique character of order two is given in terms of the characteristic function  on squares and nonsquares, etc.

Now we turn to the transfer factor $\Delta_{II}$.  It has the form
\[
\prod_\alpha \chi_\alpha\left(\frac{\alpha(\gamma)-1}{a_\alpha}\right)
\]
It is a constructible function if each factor is a constructible functions. Each morphism $\gamma\to(\alpha(\gamma)-1)/a_\alpha$ is
definable, so we only need to check that each character $\chi_\lambda$ in some choice of $\chi$-data is constructible.  We use the characters
given above to do so.

There is no harm in partitioning the domain of $\Delta_{II}$   according to definable characteristics of the element $\gamma$.  We consider a definable family of
 $L/VF$ that split the centralizer of $\gamma$.  We may assume fixed abstract Galois data $1\to\Sigma^t\to\Sigma\to \Sigma^u\to 1$ with
enumeration $\sigma_i$ of the elements of $\Sigma$ for $L$
and we may assume a fixed action of that data on the root system coming from the centralizer of $\gamma$ (relative to a split torus).   This gives the
indexing set $R$ of roots and action of $\Sigma$ as fixed choices used to partition the domain of $\Delta_{II}$.

Let $\epsilon$ be an automorphism of $R$ that acts as $\lambda\mapsto -\lambda$, and let $O(\lambda)$ be the orbit of $\lambda$ under $\langle \Sigma,\epsilon\rangle$.
If there does not exist $\sigma\in\Sigma$ such that $\sigma\lambda=-\lambda$, then we may take the $\chi$-data for $\mu\in O(\lambda)$ to be the trivial character
(which is constructible).   

Now assume that there exists $\sigma\in\Sigma$ such that $\sigma\lambda = -\lambda$.  Then $F_{+\lambda}/F_{\pm \lambda}$ is a nontrivial quadratic extension.
We set $\chi_\lambda = \chi_+$ for this quadratic extension.  
In more detail,
we include free parameters $\dot\sigma_i$ realizing each abstract automorphism $\sigma$ as a linear transformation of $L/VF$.
The extension $F_{+\lambda}/VF$ and the space of uniformizers $\varpi_+$ in the extension are then described by definable conditions inside $L/VF$ (as in \cite{cluckers2011transfer}).

By transport of structure, we obtain constructible $\chi$-data on the entire orbit of $\lambda$, using the defining properties of $\chi$-data:
$\chi_{\sigma\lambda} = \chi_{\lambda}\cdot \dot\sigma^{-1}$; $F_{+\sigma\lambda}=\dot\sigma F_{+\lambda}$; $\varpi_{+\sigma\lambda}=\dot\sigma\varpi_{+\lambda}$,
and so forth.   Running over all orbits this way, constructible $\chi$-data are obtained.
\end{proof}
% Langlands-Shelstad use Sigma = <Gamma,epsilon>,  Gamma_Langlands-Shelstad = Sigma_this.


\subsection{$\Delta_2$}

We have now treated all terms except $\Delta_2$.
We recall that the term $\Delta_2$ restricts to a multiplicative character on each Cartan subgroup of $G$.  It is constructed from $\chi$-data by
means of class field theory reciprocity for tori.  The following theorem will complete our analysis of the transfer factors on groups.

\begin{theorem}\label{thm:delta2}  There is a $q$-constructible function representing the transfer factor $\Delta$, 
possibly after introducing some free parameters.  These parameters  have no effect upon specialization to a $p$-adic field.
\end{theorem}






\begin{proof}  We give a sketch of a proof.  
The idea of the proof is that multiplicative characters can be chosen to be tamely ramified; that is, they have trivial restricition to topologically unipotent elements.
We have descent formulas for unramified groups that reduce the transfer factor to the case of topologically unipotent elements \cite{langlands2007descent} \cite{hales1993simple}.

We freely use various lemmas on definability from the next section \ref{sec:definability}.

We enumerate the standard Levi components of $G$.  Each is a definable set.  If $\gamma_G$ is conjugate to an element $\gamma_M$ in some proper Levi subgroup, then
by descent formulas for transfer factors we have $\Delta(\gamma_G,\gamma_H) = \Delta_M(\gamma_M,\gamma_{M,H})$.   The element $\gamma_{M,H}$ is a conjugate
of $\gamma_H$ in a Levi of $H$ constructed by descent.
By an induction on the dimension of the group, we may assume that $\Delta_M$ is constructible.  Every regular semisimple element that is not
elliptic is conjugate to a proper Levi subgroup.  We may now assume that $\gamma_G$ belongs to an elliptic Cartan subgroup $T$.

Since $G$ is unramified, the connected center $Z^0$ is also unramified.
$Z^0$ can be naturally identified with a torus in the connected center of $H$.
By Langlands and Shelstad, there is a character $\theta$ on $Z^0$ such that
\[
\Delta(z\gamma_G,z\gamma_H) = \theta(z)\Delta(\gamma_G,\gamma_H).
\]
The character is unramified \cite{hales1993simple}.
The character $\theta$ depends on $(\gamma_G,\gamma_H)$ only through the endoscopic data $(G,H)$.
To check constructibility of $\theta$, it is convenient and allowable to temporarily assume that $\gamma_H$ belongs to the maximally split Cartan subgroup of $H$.
We can apply Levi descent, unless the image $\gamma_G$ of $\gamma_H$ lies in an elliptic torus.

If $G=H$ is a torus, then there is a tautological embedding $\xi_0:{}^LH \to {}^LG$ for which the character $\theta$ is trivial.  Any other choice
of embedding $\xi$ that factors through a finite unramified extension $\op{Gal}(E/F)$.  The cocycles attached to $\xi_0$ and $\xi$ 
differ by a cocycle of $H^1(\op{Gal}(E/F),\hat T)$,
determined by the value $\phi$ on the Frobenius (or in the constructible context, the quasi-Frobenius)
\[
\phi \in \hat T = X^*(T)\otimes \ring{C}^\times = \op{Hom}(X_*(T),\ring{C}^\times)
\]
By reciprocity, this corresponds to the unramified character of $T$ given by $\varpi_F^\lambda\mapsto \phi(\lambda)$, for $\lambda\in X_{*,F}(T)$.
This clearly extends to a constructible function $\theta$ on $T$.

If $G$ has a nontrivial root system and $\gamma_G$ is elliptic, then it is known that the adjoint group of $G$ is $PGL(n)$ and $H$ is the elliptic
unramified torus given by twisting by the longest element of the Weyl group of $PGL(n)$.  In this case Kazhdan gave a formula for the transfer factor 
(later repeated later in Hales and Waldspurger), and by inspection, it is constructible \cite{kazhdan1983lifting}.  It is essentially the
quadratic character attached to the unramified quadratic extension of $F$,
evaluated on a resultant polynomial constructed from $\gamma_H$.   This completes the proof that $\theta$ is constructible.

Now we drop the temporary assumption on $\gamma_H$; it is no longer assumed to lie in a maximally split Cartan subgroup.


By adjusting $(\gamma_G,\gamma_H)$ by a central element, we may reduce the proof of constructibility to the special case  where $\gamma_G$ and
$\gamma_H$ lie in the maximal bounded subgroup of their Cartan subgroups $T$ and $T_H$.
We take a definable topological Jordan decompositon $\gamma_G = \gamma_s \gamma_u$, described in Section \ref{sec:definability}.
Replacing $\gamma_G$ and $\gamma_s$ by stable conjugates $\gamma'_G=\gamma_G^h$ and $\gamma_s^h$ (same $h$), we may assume that
$\gamma_s\in G_u$; that is, its centralizer is unramified.  We may do the same on the endoscopic side.  We have 
\[
\Delta_G(\gamma_G,\gamma_H) = c \Delta_G(\gamma'_G,\gamma'_H)
\]
where $c$ is the ratio of terms coming from $\Delta_{III}$.  The $\Delta_{III}$ terms are constructible, so the proof of constructibility reduces to
the case where we may now drop primes and assume that $\gamma_s$ has an unramified centralizer.
We construct descent data $(G_s,H_s)$ for the centralizer of $\gamma_s$ in $G$ and the corresponding centralizer in $H$.
By \cite{hales1993simple}, the normalized transfer factors satisfy
\[
\Delta(\gamma_G,\gamma_H) = \Delta_s(\gamma_G,\gamma_H),
\]
where the right-hand side is computed with respect to the endoscopic data $(G_s,H_s)$.
(In that reference, it is assumed that $\gamma_G\in G(O_F)$ and $\gamma_H\in H(O_F)$, but that assumption is only needed 
to prove the fact that the centralizer of $\gamma_s$ is unramified.  Since we have a separate argument of that fact, the descent formula
holds in our context as well.)
By an induction on the dimension of the group, the right-hand side is constructible, and the proof is complete except in the case
when $\gamma_s$ is central.

We now assume that $\gamma_s$ is central and strongly compact.  Then $\gamma_s\in K$ because $K$ is a maximal compact.
It is known that $\theta$ is trivial on $K$ \cite[Lemma 3.2]{hales1995fundamental}. Thus again adjusting by an element in the center,
we may assume that $\gamma_s=1$.  That is, we are reduced to proving the constructibility of transfer factors on the set of topologically
unipotent elements.  We pick our $\chi$-data to be tamely ramified.  This implies that the characters $\Delta_2$ are trivial
on topologically unipotent elements.  This reduces constructibility to the analysis of factors $\Delta_I$, $\Delta_{II}$, $\Delta_1$, and $\Delta_{IV}$.
This has already been done.  This completes the proof.
\end{proof}

\subsection{definability results}\label{sec:definability}

In this section we assume that $G$ is an unramified connected reductive group.  It is treated  as definable subassignment
over a definable cocycle space $Z$.

\begin{lemma}  Let $G$ be an unramified reductive group.  There exists a definable subassignment of $G\times G$ of all pairs
$(\gamma,x)$ such that $\gamma$ is semisimple (possibly singular) and $x$ lies in the connected component of the centralizer of $\gamma$.
\end{lemma}

\begin{lemma} Let $G$ be an unramified reductive group.  There exists a definable subassignement of $G\times G$ of all pairs
$(\gamma,\gamma')$ such that $\gamma$ is semisimple (possibly singular) and $\gamma'$ is stably conjugate to $\gamma$.
\end{lemma}

\begin{lemma} Let $G$ be an unramified reductive group, given as a definable subassignment over a cocycle space $Z$.
There is a definable subassignment $G^u$ of $G$ consisting of strongly regular semisimple elements $\gamma$ such that
the connected component of the centralizer of $\gamma$ is unramified. 
\end{lemma}

\begin{lemma} Let $G$ be an unramified reductive group with unramifed endoscopic group $H$, given as a definable subassignments over a common cocycle space $Z$.
There is a definable subassignment $GH$ of all pairs $(\gamma,\gamma_H)$ such that $\gamma_H$ is strongly $G$-regular and $\gamma\in G^u$ is an image of $\gamma_H$.
Consider the Denef-Pas statement $\psi$ that asserts that for all strongly $G$-regular elements $\gamma_H$, there exists an image $\gamma\in G^u$ that is an image of $\gamma_H$.
There exists $N$ such that $\psi_F$ is true for all $F\in\C_N$.
\end{lemma}

\begin{lemma} The set of topologically unipotent elements in a reductive group is a definable subassignement.
\end{lemma}

\begin{lemma} Let $T$ be a torus defined over a cocycle space $Z$. Assume that $T$ is given by twisting a split torus by some cocycle the Galois group
of a splitting field $L/VF$.  Let $S\subset T$ be a maximal unramified subtorus of $T$.  Consider the Denef-Pas statement asserting that (for every $z$ and) for
every $t\in T$, there exists $s\in S$ and a topologically unipotent element $u\in T$ such that $t =s u$.  Then there exists $N$ such that the statement is
true in $F$ for all $F\in \C_N$.
\end{lemma}

\begin{lemma}[topological Jordan decomposition] 
Let $G$ be an unramified reductive group.  There is a definable subassignment of triples $\{(\gamma,\gamma_s,\gamma_u)\in G^3 \mid \cdots\}$,
consisting of triples such that $\gamma$ is regular semisimple and strongly compact, 
$\gamma = \gamma_s \gamma_u = \gamma_u\gamma_s$, $\gamma_u$ is topologically unipotent, and
\[
\alpha(\gamma_s)=1,\quad\text{ or } \op{ord}(\alpha(\gamma_s)-1)=0,
\]
for all absolute roots $\alpha$ of the Cartan subgroup $C_G(\gamma)$.
\end{lemma}

We do not need the following. It was used in an earlier version of this research, and we record it for possible future uses.

\begin{lemma} The approximate exponential is a definable subassignment.  An approximate exponential exists in large residue characteristic.
\end{lemma}

\XX{expand the sketch to a full proof}

\subsection{fundamental lemma}

We conclude this article with a proof of the fundamental lemma for the spherical Hecke algebra for unramified groups in large positive characteristic
in the following form.

\begin{theorem}  Assume Kato-Lusztig constructibility (Conjecture \ref{conj:kato}).
For each absolute root system $R$,
there is a constant $N=N_R\in\ring{N}$ such that
the Langlands-Shelstad fundamental lemma holds for all unramified connected reductive groups $G$ with absolute root system $R$ 
and all of its unramified endoscopic groups $H$ over $F$ 
for all fields $p$-adic fields $F\in \C_N$.
\end{theorem}

By abstract unramified Galois group we mean a fixed finite cyclic group $\Sigma=\Sigma^u$ with choice of generator $\op{qFrob}$
that we call the quasi-Frobenius element.  It is not tied to any particular $p$-adic field.   The abstract dual group is the Langlands
dual constructed with respect to $\Sigma$ and $\op{qFrob}$ rather than the Galois (or Weil) group of a field.

It is an unfortunate limitation of the method that we are not able to be explicit about the restriction $N$ on the residue characteristic of the field.

\begin{proof}
The fundamental lemma takes the form
\begin{equation}\label{eqn:fl}
\sum_{\gamma_G}\Delta_0(\gamma_H,\gamma_G,\cdots)\op{O}(\gamma_G,f_\lambda) - \op{SO}(\gamma_H,b_\xi(f_\lambda)) = 0.
\end{equation}
Stable orbits of regular semisimple elements are definable as fibers of the Chevalley morphism $G\to T/W$.  The invariant motivic measure
on stable orbits is the volume form attached to a Leray residue of an invariant differential form on the group with respect to the canonical
form on $T/W$.  We have shown that the transfer factor and the homomorphism $b_\xi$ can be lifted to a $q$-constructible motivic functions.
The ellipsis $(\cdots)$ indicates extra parameters such as a parameter running over $a$-data, a parameter running over admissible pinnings
for the canonical normalization, and uniformizing parameters used in our explicit treatment of the $\chi$-data.  The $p$-adic transfer factor is independent of these choices,
but in dealing with constructible motivic functions, it is best to make the dependence on the parameters explicit (or at least honor them with an ellipsis).

We may consider the left-hand side of Equation \ref{eqn:fl} as a $q$-constructible function of $(\lambda,\gamma_H,\cdots)\in P^+\times H\times\cdots$, all over a definable
cocycle space $Z$ used to parameterize an unramified splitting field of $G$ and $H$.  

The fundamental lemma holds for the unit element in positive characteristic by the work of Ng\^o \cite{ngo2010lemme}.
This can be lifted to characteristic zero \cite{cluckers2011transfer}, \cite{waldspurger2006endoscopie}.  
It extends to the full Hecke algebra in characteristic zero \cite{hales1995fundamental}.
Hence the identity (\ref{eqn:fl}) holds in characteristic zero.
By the transfer principle, 
there exists $N$ such that the fundamental lemma also holds for all fields $F\in\C_N$.

It is important for the left-hand side of the equation to be viewed as a 
single identity with $P^+$ forming a factor of the definable subassignment, rather than viewed as an infinite collection of identities indexed by $\lambda\in P^+$.
This allows us to invoke the transfer principle a single time, rather than once for each $\lambda\in P^+$.
\end{proof}

\newpage 
\section{easy Kato-Lusztig}

\subsection{twisted Weyl character formula}

We review the proof of the Weyl-character formula, as presented in \cite{kostant1961lie}, \cite{jantzen1977darstellungen}, and \cite{wendt2001weyl}.
At the same time, we consider various $q$-deformations of the standard formulas.

Let $\hat G$ be a complex group, with outer automorphism $\sigma$ that preserves a pinning, including
$\hat T$ and $\hat B = \hat T \hat N$.

Let $\lambda$ be a dominant weight in $X^*(\hat T)^\sigma$.  Let $V_\lambda$ be the irreducible module
of $\hat G$ with highest weight $\lambda$.  The $\sigma$-invariance of $\lambda$ implies that $V_\lambda$
extends uniquely to a representation of $\hat G \rtimes \g{\sigma}$ such that $\sigma v = v$ for $v$ in the
highest weight space of $V_\lambda$.  We let $\tau_\lambda$ be the character on $V_\lambda$, restricted to $\hat T\rtimes\g{\sigma}$.

Let $\n$ be the Lie algebra of $\hat N$, considered as a module of ${}^L\hat T=\hat T\rtimes\g{\sigma}$ by the adjoint representation,
and let $\n'$ be its contragredient module.  We write $\chi_j$ for the character of the exterior power $\Lambda^j \n'$.
  We write $\tilde\chi_q = \sum_j (-q)^j\chi_j$
for the $q$-graded virtual character on the sum of $\Lambda^j \n'$, with grading $(-q)^j$ on the $j$th summand.  

The character $\tilde\chi_q$ evaluated at $\sigma t$ depends only on the image $s\in\hat S$ of $t\in \hat T$.
We have 
\[
P(E^{-1},q)^{-1}(s) = \det(1 - q\sigma t;\n') = \sum_j (-q)^j \chi_j = \tilde\chi_q,
\]
because the determinant can be evaluated by picking a basis of eigenvectors of $\sigma t$ on $\n'$ and expanding in a sum.
We have $E^{-1}$ rather than $E$ because $\n'$ is the contragredient.

There is an explicit product formula for the character $\tilde\chi_q$ as given by the following lemma:

\begin{lemma} \label{eqn:weyl-product}
Let $B_1,\ldots,B_k$ be the orbits of $\Phi^+$ under $\g{\sigma}$, with cardinalities $b_i = \card(B_i)$.
Let $\beta_i\in B_i$ be a representative of each orbit.
Then
\[
\det(1-\sigma  q E;\n) = \prod_{i=1}^k (1- \epsilon_i q^{b_i} e^{N\beta_i}),
\]
where 
$N\beta = \sum_{\beta'\in \g{\sigma}\beta} \beta'$ is the norm and
\[
\epsilon_i=\begin{cases} 1 & \text{type I, II},\\
     -1 & \text{type III}.
\end{cases}
\]
\end{lemma}

\begin{proof} This can be found in the special case $q=1$, 
for example, in \cite{jantzen1977darstellungen}, \cite{wendt2001weyl}.  Here is a sketch.
The determinant is block diagonal, with a block for each orbit of $\g{\sigma}$ on $\Phi^+$.

Let $\beta\in \Phi^+$ be any positive root.  Let $B$ be its orbit under $\g{\sigma}$.
Then $B$ is the set of simple roots in a root system.   Since $\g{\sigma}$ acts transitively on the simple roots,
each connected component of the Dynkin diagram must have type $A_1$ or $A_2$.  

When the connected components have type $A_2$, the simple roots of the Dynkin diagram of $B$ are said to have type II.
We have $\sigma^m \beta = \beta'$, for some $m$, where $\beta$ and $\beta'$ are the two simple roots in the $A_2$-component of $\beta$.
Also $\gamma = \beta + \beta'$ is a positive root of highest weight in $A_2$.  We say that $\gamma$ has type III.
Let $C$ be the orbit of $\gamma$ under $\g{\sigma}$.  

On the block $B$, we can pick a basis $X_i$ of the
root spaces $\n_{\sigma^i\alpha}$ such that $\sigma$ acts as $\sigma X_i = X_{i+1}$, with indices mod $\card{B}$.
The $\det$ on this block is $1- \prod_B {q e^{\beta}} = 1- q^{\card B} e^{N \beta}$.

On block $C$, we have that $\sigma^m$ acts on $X_\gamma = [X_\beta,X_{\beta'}]$ 
as $\sigma^m X_\gamma = [X_{\beta'},X_\beta] = -X_\gamma$, and we have acquired a sign.  
It follows that the determinant on block $C$
is $1+\prod_C {q e^\gamma} = 1+q^{\card C} e^{N\gamma}$.

When the connected
components have type $A_1$, and not type III as above, the root $\beta$ is said to be of type I.  
The $\det$ on this block is computed as in  type II to be $1- \prod_B q e^{\beta} = 1- q^{\card{B}} e^{N \beta}$.
\end{proof}

\begin{corollary} The character $\tilde \chi_q$, for $q=1$ is given by
\[
P(E^{-1},1)^{-1} = \prod_{\alpha\in\Sigma_1} (1-e^{-\alpha}).
\]
\end{corollary}

\begin{proof}  The product formula of the lemma for $q=1$ reduces to
\[
\prod_{i=1}^k (1 - \epsilon_i e^{N\beta_i}),
\]
as $\beta_i$ runs over a representative in $\Phi^+$ of each $\g{\sigma}$-orbit.
When the root $\beta$ has type I, then $N\beta = N'\beta \in \Sigma^+_1$.
When the root $\beta$ has type II, then $N\beta = N'\beta/2$.
Also $\gamma = \beta+\sigma^m\beta$ has type III with norm $N\gamma = N\beta = N'\beta/2$.
Thus, with type II and III there are two orbits with the same norm, but opposite signs $\epsilon_i$.  
The combined contribution from the
two orbits is
\[
(1- e^{N'\beta/2})(1+e^{N'\beta/2}) = (1-e^\alpha),
\]
where $\alpha = N'\beta$.
\end{proof}
We see that the modification $N'$ of the norm $N$  accommodates the
two orbits with the same norm.  The set $\Sigma_1$ of modified norms $N'\beta$ is the
correct root system to use for the twisted Weyl-character formula, because of its relation to $P(E^{-1},1)$.


The spaces $C^j=\Lambda^j \n'\otimes V_\lambda$ are the terms of a cochain complex of ${}^LT$-modules.
We consider the $\ring{Z}/2\ring{Z}$-graded virtual character $\tilde \tau_\lambda  = \tau_\lambda \tilde \chi_{1}$ on  
this cochain complex.
By an Euler-Poincar\'e argument, 
$\tilde\tau_\lambda$ equals the character on the cohomology of the complex.  The cohomology has 
been computed explicitly \cite{kostant1961lie}.
These computations show that for each $w\in W^\sigma$, the term $e^{w(\lambda+\rho)-\rho}$ 
occurs once in cohomology with sign $(-1)^{\ell w}$.
Thus, in terms of our alt-symmetrizer $J$, we have
\[
\tilde \tau_\lambda = J(e^\lambda).
\]
Solving $\tilde \tau_\lambda = \tau_\lambda \tilde \chi_1$ for $\tau_\lambda$, we obtain the Weyl character formula
\[
\tau_\lambda = J(e^\lambda) P(E^{-1},1).
\]
It is natural to extend $\tau_\lambda$ to a $q$-character by defining $\tau_{\lambda,q}$ by
$\tilde \tau_\lambda = \tau_{\lambda,q} \tilde \chi_q$, so that
\[
\tau_{\lambda,q} = J(e^\lambda) P(E^{-1},q).
\]
We leave it as a problem for the reader to find natural interpretations of $\tau_{\lambda,q}$, along the lines of Kazhdan-Lusztig
polynomials.  Kato and Lusztig give an answer when $\sigma=1$.


\subsection{partition function}

Here I describe what I hope the partition function should be $\ldots$

Let $G$ be an unramified group and $K$ a hyperspecial maximal compact subgroup.  Let $A$ be a maximal split torus,
with centralizer $M$.

Let $(X_*,X^*,\Phi,\Phi^\vee)$ be the absolute root datum attached to $G$ over a separable closure.

We have identifications
\[
M/M(0)=A/A(0)=X_*(A)=X_*(M)^\sigma  =X^*(\hat T)^\sigma = Y^* = X^*(\hat S),
\]
where $\hat S = \hat T/(1-\sigma \hat T)$.

Each unramified character $\chi:M\to \ring{C}^\times$, by the identifications given above, is a homomorphism
\[
\chi\in\op{Hom}(M/M(O),\ring{C}^\times) = \op{Hom}(X^*(\hat S),\ring{C}^\times) = X_*(\hat S)\otimes \ring{C}^\times = \hat S.
\]
We write $\chi = \chi_s$, for $s\in\hat S$.


Each absolute root $\beta$ defines by restriction to $A$ a root $\alpha$ of $A$.  We use the modified
restriction map $\op{res}'$ of Haines, so that the root lands in the reduced root system $\Sigma_0$ used by Macdonald.
(This corresponds to picking the long roots out of the nonreduced system.)
Casselman constructs an element $a_\alpha \in M$, which gives an element $\beta'\in X^*(\hat S)$,
by the identifications given above.  

On the other hand, again starting with $\beta\in\Phi$, there is a dual root $\beta^\vee\in \Phi^\vee$, which may be considered
a root of the dual group $\hat G$.   The modified norm map $N'$ is defined in Chriss, Haines, and Kottwitz-Shelstad (foundations twisted endoscopy).
It is the usual norm $N\beta$, except when $\beta^\vee$ has type II and then it is multiplied by an extra factor of $2$ (picking out the long root in 
the nonreduced system).   Then $N'\beta^\vee$ is $\sigma$-fixed: $N'\beta^\vee\in X*(\hat T)^\sigma = X^*(\hat S)$.

The following conjecture should not be hard to confirm or refute.  It is a matter of tracing all the maps through carefully.
(Intuitively, Casselman's  construction of $a_\alpha$ seems to involve switching over to the corresponding
short root in the nonreduced system??  So going to the Langlands dual, we should be picking out the long roots from the nonreduced root system, which is what the 
modified norm map does.)  

\begin{conjecture} For every $\beta\in\Phi$, we have $\beta' = N'\beta^\vee\in X^*(\hat S)$.
\end{conjecture}

The set of modified norms of simple roots $N'\beta^\vee$ are the roots of a root system called $\Sigma_1$ in Chriss (that is not to be confused
with Macdonald's $\Sigma_1$).   It is the root system $\Sigma_1$ that is used for the twisted Weyl character formula on $\hat G$.

We now assume this conjecture, and
rewrite Macdonald's formula in terms of the root system  $\Sigma_1$ in the dual group instead of the root system of $G$.
We identify elements of $\ring{C}[Y^*]$ with functions on $\hat S$.  We also wish to express the constants $q_\alpha$ and $q_{\alpha/2}$
in the formula by data in the dual group.

Assume the outer automorphism $\sigma$ acts on $\hat G$ fixing a pinning including $\hat T$, $\hat B = \hat T \hat N$.  We now take $\hat G$ to be the
primary object and drop the $\vee$ from notation, so that $\Phi$ denotes the  root system of the complex group $\hat G$.
Let $\n$ be the Lie algebra of $\hat N$ and
let $\n = \oplus_{\beta\in \Phi^+} \n_\beta$ be the root space decomposition.  We define a symbolic operator $E$ on $\n$ that is diagonal with
respect to the root space decomposition:
%\[
%E v = e^{k \beta} v, \text{ for } v \in \n_\beta,\quad \text{where } 
%k = k_\beta = \begin{cases} 1,&\beta \text{ type 1}\\
%\frac{1}{2},&\beta \text{ type II,III}.
%\end{cases}
%\]
\[
E v = e^{\beta} v, \text{ for } v \in \n_\beta.
\]
For $w\in W$, we define $w(E)$ formally, by $w(E) v = e^{ w \beta} v$, for $v\in \n_\beta$.
The automorphism $\sigma$ also acts on $\n$.  We define the $q$-partition function as
\[
P(E,q) = 1/\det(1 - q \sigma E;\n).
\]


This determinant encodes all the constants $q_\alpha$, $q_{\alpha/2}$ that occur in the Macdonald formula.
It also encodes the norms of the absolute roots that give the root system $\Sigma_1$.
In particular, $P(E^{-1},1)$ is the denominator of the twisted Weyl integration formula for the root system $\Sigma_1$.

It can help to consider a couple of examples.  If $\sigma=1$, then $q \sigma E$ is diagonal, every root has type I, and the partition function
is a product over root spaces $P(E,q) = \prod_{\beta\in\Phi} (1- q e^\beta)^{-1}$.  This is the classical $q$-partition function.

Now take $\sigma$ to be the outer automorphism of $GL(3,\ring{C})$ preserving a pinning.  The roots $\alpha_1 = t_1-t_2$ and $\alpha_2 = t_2-t_3$ have type II,
and $\alpha_3=t_1-t_3$ has type $III$.  The automorphism $\sigma$ acts as $\sigma X_1 = X_2$, $\sigma X_2 = X_1$, and $\sigma X_3 = - X_3$, for $X_i\in \n_{\alpha_i}$.
Thus, 
\[
\det(1 - q^{-1} \sigma E) = 
\begin{pmatrix} 1 & -q^{-1} e^{\alpha_1/2}\\ -q^{-1} e^{\alpha_2/2} & 1\end{pmatrix} (1 + q^{-1} e^{\alpha_3/2}) = (1 - q^{-2} e^{\alpha/2}) (1 + q^{-1} e^{\alpha/2}),
\]
where $\alpha\in \Sigma_1^+$ is the unique positive root in the root system $\Sigma_1$.
This is  the numerator of $\gamma_\chi$ in the Macdonald formula for $U_3$, expressed in terms of dual group data.

Macdonald's formula takes the form
\[
\hat f_\mu = (*) \sum_{w\in W^\sigma} e^{w\mu} \frac{P(w(E)^{-1},1)}{P(w(E)^{-1},q^{-1})}
\]
where I have omitted the leading factors $(*)$ that appear.



Let $\hat S_1$ be the maximal compact subgroup of $\hat S_1$.  Let $ds$ be the Haar measure on $\hat S_1$ normalized
so that $\hat S_1$ has volume $1$.
We ``define'' the Plancherel measure to be
\[
(*)\frac{P(E,q^{-1}) P(E^{-1},q^{-1})}{P(E,1) P(E^{-1},1)} ds.
\]
where again I omit standard normalizing factors (*).  (We should check that this is $W^\sigma$-invariant.)


Equipped with Plancherel and Macdonald, we obtain the easy Kato-Lusztig formula for the inverse Satake transform, exactly as in Kato's article.   Recall that we write
\[
\tau_\lambda = \sum t_{\lambda,\mu} \hat f_\mu,
\]
for inverse Satake.  The functions $\hat f_\mu$ form an orthogonal basis with respect to the Plancherel measure.
So $t_{\lambda,\mu} = (*) \langle\tau_\lambda,\hat f_\mu\rangle$ up to a normalizing factor $(*)$.
Also, Plancherel and $\tau_\lambda$ are both $W^\sigma$-invariant, which means that we can replace $\hat f_\mu$, which is
expressed by an average over $W$ by the $w=1$ contribution to $\hat f_\mu$.
In the inner product $\langle f,g\rangle$ we conjugate the second term $g$.
The conjugate of $e^\mu$ is $e^{-\mu}$ and of $E$ is $E^{-1}$.
Thus, the coefficient $t_{\lambda,\mu}$, up to trivial factors $(*)$ is given by
\begin{align*}
&~\left\langle \tau_\lambda, e^{\mu} \frac{P(E^{-1},1)}{P(E^{-1},q^{-1})}\right\rangle \hfill\\
&=\int_{\hat S_1} \tau_\lambda \left(e^{-\mu} \frac{P(E,1)}{P(E,q^{-1})}\right) dm(s)\hfill \\
&=
(*)\int_{\hat S_1} \left(J(e^\lambda) P(E^{-1},1)\right)  
\left(e^{-\mu} \frac{P(E,1)}{P(E,q^{-1})}\right)
\left(\frac{P(E,q^{-1}) P(E^{-1},q^{-1})}{P(E,1) P(E^{-1},1)}\right) ds\\
&=
(*)\int_{\hat S_1} J(e^\lambda) P(E^{-1},q^{-1}) e^{-\mu} ds\\
&=(*)(\tau_{\lambda,q^{-1}},e^\mu).
\end{align*}
Here $\tau_{\lambda,q}$ is the $q$-twisted character introduced above.
  Also, in the last line of the equality, $(\cdot,\cdot)$ is the inner product with respect to the Haar (not Plancherel) measure on $\hat S_1$.
With respect to this measure, $(e^\lambda,e^\mu) = \delta_{\lambda,\mu}$.  
We conclude that $t_{\lambda,\mu}$ up to inconsequential factors is given by the coefficient of $e^\mu$ in $\tau_{\lambda,{q^{-1}}}$.
Thus, $t_{\lambda,\mu}$ are simply $q$-multiplicities.  In particular, they are constructible Presburger functions.





\newpage
