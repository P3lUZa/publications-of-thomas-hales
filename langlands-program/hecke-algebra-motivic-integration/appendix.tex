


\section{Inner type and simply laced}  

This appendix give extra details about the endoscopic partition function
for various simple groups.  This appendix will not be included in the published version.

We will say that $w\theta$ has {\it inner type} if $\theta=1$, and $w$
acts as usual on the extended Dynkin diagram.  This subsection
considers data of inner type and simply laced.  Choose $\hat
B(w\theta)$ to be adapted as in Lemma \ref{lemma:adapted} and let
$\Psi^{++} = \Psi_{w\theta}^+(\hat T,\hat B(w\theta))$.

\begin{proposition}\label{lemma:inner} 
  Suppose that $\theta=1$ and the $\hat G$ is simple and simply laced.
  Let $N_1\Psi=\{N_1\beta\mid \beta\in \Psi\}$ be the norm root system
  of $\hat T_1$. Let $\ell$ be the order of $w$.  Let $m:\Psi\to
  \ring{N}$ be given by $m(\alpha)=\ell$ if $\alpha$ is a long root in
  $N_1\Psi$ (or if $N_1\Psi$ is simply laced).  Let $m(\alpha) = 1$,
  otherwise.  Then there exists a lift $\dotw\in N_{\hat G}(\hat T)$
  such that
\begin{equation}\label{eqn:laced}
D(\hat G,\Psi^{++},\dotw,E,q) 
= \prod_{\alpha\in N_1\Psi^{++}} (1- q^{m(\alpha)} e^\alpha)^{m(\alpha)}.
\end{equation}
\end{proposition}

\begin{proof} 
  We make a case-by-case calculation.  As we will see, $m(\alpha)$ is
  the orbit size of $\alpha$.

  If $w=1$, then we set $\dotw=1$.  The determinant in the lemma is
  the determinant $\det(1- q E;\n)$, the root system is $N_1\Psi
  =\Psi$, $\hat B(w\theta)=\hat B_1$, $\Psi^{++}=\Psi^+(\hat T,\hat B)$,
$m(\alpha)=1$, and we reduced to the classical partiton function
\[
\prod_{\Psi^+}(1-q e^\alpha).
\]
 This case is immediate.  (Note that these
  cases are both inner and outer type.)

  There is no non-trivial automorphism $w$ of the extended Dynkin
  diagram when $\hat G_{sc} = \hat G_{adj}$.  Thus, we assume that
  $\hat G$ has nontrivial center.  The simply laced Dynkin diagram of
  $\hat G$ has type $A_{n-1}$, $D_n$, $E_6$, or $E_7$.

\subsection{Types $E_6$ and $E_7$}
  We start with the hardest cases $E_6$ and $E_7$.  We work with the adjoint
  groups $E_{6,adj}$ and $E_{7,adj}$.  The extended
  Dynkin diagram of $E_6$ has an automorphism of order $3$ and the
  extended Dynkin diagram of $E_7$ has an automorphism of order $2$.
  The root system $N_1\Psi$ has type $G_2$ or $F_4$, respectively,
  when the original root system $\Psi$ has type $E_6$ or $E_7$.

  We choose the representative $\dotw$ of $w\in W$ as follows.  Let
  $w=r_1\ldots r_k$ be a reduced word for $w$, where each $r_i$ is a
  simple reflection.  Tits constructs a choice of lift $\dot r_i$ of
  $r_i$ to an element in the normalizer, depending only on the pinning
  $(\hat B,\hat T,X)$.  The product $\dotw = \dot r_1\ldots \dot r_k$
  is independent of the reduced word for $w$.

  Let $\{ e_\gamma \mid \gamma\in \Psi\}\cup \{h_\alpha\mid
  \alpha\in\Delta\}$ be the Kottwitz basis of the Lie algebra of $\hat
  G$.  (A Kottwitz basis is a particularly nice Chevalley basis of the
  Lie algebra, introduced in \cite{cassstructure}.)  For a
  simply-laced root system, there is a single choice involved in
  constructing the basis: a choice of simple root, which we take to be
  the node of valence three in the Dynkin diagram (types $E_6$ and
  $E_7$).  We make some computer assisted calculations, using
  Mathematica.  No specialized Lie package is required; the
  calculations are short and rather simple to implement directly.  An
  algorithmic formula for $\op{Ad}(\dotw) e_\gamma$ appears in
  \cite{cassstructure}.  In our situation, a short computer
  calculation gives the remarkably simple answer:
\[
\op{Ad}(\dotw) e_\gamma 
= e_{w\gamma}\text{ for all } \gamma\in \Psi;
\qquad (E_6 \text{ and } E_7).
\]
This implies that the roots of unity $\zeta$ in Lemma \ref{lemma:fact}
are all $1$, and that the determinant is a product of factors $(1-q^b
e^{N_1\alpha})$.  Then, the lemma follows by counting the number of
times each norm $N_1\alpha$ appears.  The numerical data for the
partition function can be summarized by two equations
\begin{equation}\label{eqn:count}
\card(\Psi) = m + p + \ell^2 p,\quad m = \ell(n-a),
\end{equation}
\begin{align*}
\begin{matrix}
n & m&\ell & a & p & \card(\Psi)\\
6 & 12&3 & 2 & 6 & 126\\
7 & 6& 2 & 4 & 24 & 72,
\end{matrix}
\end{align*}
where $n=6,7$ is the rank, $\ell$ is the order of $w$, $a$ is the
multiplicity of the eigenvalue $1$ for the linear transformation $w$
of $\op{Lie}(\hat T)$, $p$ is the cardinality of the norm positive
root system $p = N_1\Psi^{++}$ (either $G_2$ or $F_4$).  Also, $m$ is
the number of roots $\alpha\in\Psi$ such that $N_1\alpha=0$, and $p$
is the number of roots fixed by $w$ (giving terms such that
$m(\alpha)=1$ in the determinant and giving short roots $N_1\alpha$),
and $\ell^2p$ is the number of the roots in orbits of $w$ of
cardinality $\ell$ (giving terms such that $m(\alpha)=\ell$ in the
determinant and giving long roots $N_1\alpha$).  This gives the
partition function for $E_6$ and $E_7$.

Next, we compute $\epsilon,\phi$ for $E_6$ and $E_7$.  We describe the
root systems of $E_6$ and $E_7$ as explicit subsets of the root
systems of $E_8$.  Let $e_0=(1/2,1/2,\ldots)\in\ring{R}^8$.  The root
system of $E_8$ consists of all vectors $v$ with $v\cdot v = 2$ with
coefficents in
\[
\ring{Z}^8 \cup (e_0 + \ring{Z}^8) \subset \ring{R}^8,
\]
and such that $v\cdot 2 e_0 \in2\ring{Z}$.  The root system $E_7$
consists of those roots of $E_8$ that are orthogonal to $e_0$.  The
root system of $E_6$ consists of those roots of $E_7$ that are
orthogonal to $(0,0,0,0,0,0,1,1)$.  The maximal torus of $E_8$ is
$\ring{C}^{\times,8}$, and of $E_7$ and $E_6$:
\[
\{(t_1,\ldots,t_8)\mid t_1\cdots t_8 = 1\},\quad 
(\text{and also for $E_6$: } t_7 t_8 = 1).
\]
For $E_7$, we can take $\phi:\hat U\to\hat T$ and $\iota:\hat U\to\hat
T_1 = \ring{C}^{\times,4}$.
\[
\epsilon\phi(\uu) = (-\zeta\uu_1,\zeta\uu_2,\zeta\uu_3,\zeta\uu_4,
\zeta\uu_4^{-1},\zeta\uu_3^{-1},\zeta\uu_2^{-1},-\zeta\uu_1^{-1}),
\]
where $\zeta$ is a primitive eighth root of unity, and $\iota:\hat
U\to\hat T_1$, $(\uu_1,\ldots)\mapsto (\uu_1^2,\ldots)$.  For $E_6$,
we can take
\[
\epsilon\phi(\uu) = (\zeta \uu_1,\zeta^2 \uu_1,\uu_1,\zeta\uu_1^{-1},\uu_1^{-1},
\zeta^2\uu_1^{-1},\uu_2,\uu_2^{-1}),
\]
where $\zeta$ is a primitve cube root of unity.

Regularity follows from the observation that no root vanishes identically on
$\epsilon\phi(\hat U)$.
Choose $\hat B_1 = \hat B(\theta_1)$, 
\[
\Psi_{w\theta}^+(\hat T,\hat B(w\theta)) = \{
\alpha\in \Psi(\hat T,\hat B(w\theta)\mid \phi^*\alpha\ne0\}.
\]

\XX{rework}

Before continuing with the proof, we pause to describe the general
strategy to compute partition functions in the remaining cases.  The
strategy applies to this proof and to other proofs that follow.  The
cases we have considered already (outer, $E_6$, $E_7$) give the
partition function in all exceptional cases.  This means that the
remaining cases are $A_{n-1}$, $B_n$, $C_n$, $D_n$.  When $n=4$, the
triality automorphism of $D_4$ has outer type, which has been
considered already.  Thus, we can assume that the automorphism of the
extended diagram of $D_n$ has order $2$ or $4$.  By cases considered,
we may assume the automorphism $\theta$ is trivial except in the case
$D_n$.

We calculate the data in the same manner.  We work in the standard
representation of the each classical group.  For each automorphism of
the extended Dynkin diagram (excluding those of outer type), we
construct a lift $\dotw \in N_{\hat G}(\hat T)$ in the normalizer, and
compute the partition function with respect to $\theta_1=\dotw
\theta$.  For $\tau\in \hat T_1$, we diagonalize $\tau\dotw\theta$ to
$s\theta$, with $s\in\hat S$ by computing its characteristic
polynomial and eigenvalues.  From this diagonal form, we can read off
the determinant.  \XX{insert a formula.}  We write $s = \epsilon
\phi(\uu)$, where $\epsilon \in \hat S$ is independent of $\tau$ and
$\phi(\uu)$ gives the $\tau$-dependence, for some $\iota:\uu\in\hat
U\to \hat T_1$, and $\phi:\hat U\to \hat S$.


\subsection{Type $A_{n-1}$}
We return to the proof of the lemma and do the computation for a split
group of type $A_{n-1}$.  The elements $w$ that act on the extended
Dynkin diagram are powers of the Coxeter element.

Let $\hat G = \op{SL}(n,\ring{C})$, and let $\theta=1$.  Assume that
$n=m \ell$.  We may assume that $w$ acts on
\[
\hat T = \{(t_1,\ldots,t_{n})\mid t_1t_2\cdots t_{n}=1\},
\quad \text{ by } w(t_1,\ldots,t_{n}) = (t_{m+1},t_{m+2},\ldots).
\]
The element $w$ has order $\ell$ and acts on the extended Dynkin
diagram.  We lift $w$ to an element $\dotw = \zeta \dotw'$, where
$\dotw'$ is an orthogonal matrix with entries in $\{0,1\}$ that
realizes the permutation, and the root of unity $\zeta$ is chosen so
that $\dotw \in \op{SL}(n,\ring{C})$.  We have
\[
\hat T \to \hat T_1 = 
\{(\tau_1,\ldots,\tau_m)\mid \tau_1\ldots\tau_m=1\},
\quad \op{diag}(t_1,\ldots,t_n)
\mapsto (t_1t_{m+1}\cdots,t_2 t_{m+2}\cdots,\cdots).
\]
A calculation of the characteristic polynomial of $t\dotw$, for $t\in
\hat T_1$ shows that $t\dotw$ can be described as follows.  Let
$\zeta$ be a primitive $2\ell^\op{th}$ root of unity in
$\ring{C}^\times$.  We set
\begin{equation}
s=\epsilon\phi(\uu)=\op{diag}(z\uu_1,z\uu_2,\ldots,z\uu_m)
\in \hat T\subset \ring{C}^{\times,m\ell},
\end{equation}
where $z\in \ring{C}^{\times,\ell}$ is
\begin{align*}
z &= \op{diag}(\zeta^{2k-1},\zeta^{2k-3},\ldots,\zeta^{1-2k}), &(\ell=2k);\\
      &= \op{diag}(\zeta^{2k},\zeta^{2k-2},\ldots,\zeta^{-2k}), &(\ell=2k+1)
\end{align*}
and
\[
\uu=(\uu_1,\ldots,\uu_m)\in \hat U,
\]
with isogeny $\uu\mapsto
\iota(\uu)=(\uu_1^\ell,\uu_2^\ell,\ldots,\uu_m^\ell) = \tau\in \hat
T_1$.  Since $s$ is in diagonal form, $\op{Ad}(s) e_\gamma = \gamma(s)
e_\gamma$.  Let $N_1\Psi$ be the root system of $\op{SL}(m,\ring{C})$.
The determinant is as stated in the lemma.  This completes the
calculation for $A_{n-1}$.

To finish the lemma, we need to compute the determinant for a groups
of type $D_n$.  There are several cases that occur for $D_n$.  They
are best considered together.  The proof is completed in section
\ref{sec:dn}.
\end{proof}


\subsection{Type $C_n$}

The outer automorphism $\theta$ of $C_n$ is trivial.  We have $\hat S
= \hat T$ and the root system is $N_1\Psi =\Psi$.  We work in the
standard representation of $\op{Sp}(2n)$ of dimension $2n$, defined
with respect to a skew form along the anti-diagonal $J =
\op{anti}(-1,-1,\ldots,\ 1,1\ldots,1)$.  The extended Dynkin diagram
has a nontrivial automorphism of order $2$ that exchanges the two ends
of the diagram.  It is represented in the normalizer as a block matrix
of block size $n$:
\[
\dotw = \begin{pmatrix} 0 &I \\ -I & 0\end{pmatrix}.
\]
Write $n = 2k$ or $n = 2k+1$, depending on the parity of $n$.  The
morphism $\hat T\to \hat T_1=\ring{C}^{\times,k}$ is
\[
t = \op{diag}(t_1,\ldots,t_n,t_n^{-1},\ldots,t_1^{-1})
\mapsto \tau=(\tau_1,\ldots,\tau_k)=(t_1/t_n,t_2/t_{n-1},\ldots).
\]
Define an isogeny $\iota:\hat U=\ring{C}^{\times,k} \to \hat T_1$,
where $(\uu_1,\ldots,\uu_k)\mapsto \tau=(\uu_1^2,\ldots,\uu_k^2)$.
Diagonalizing $t\dotw$, we obtain
\begin{align*}
\epsilon\phi(\uu) 
&= \op{diag}(i \uu_1,-i\uu_1,i\uu_2,-i\uu_2,\ldots,i\uu_k,
-i\uu_k,\ i\uu_k^{-1},\ldots, -i\uu_1^{-1})\in \hat S,
& (n = 2k);\\
    &= \op{diag}(i \uu_1,-i\uu_1,i\uu_2,-i\uu_2,\ldots,i\uu_k,-i\uu_k,i,
  -i,\ i\uu_k^{-1},\ldots, -i\uu_1^{-1}),
& (n = 2k+1).\\
\end{align*}
We compute the determinant/Weyl denominator/partition function:
\begin{equation}
D_+ = \prod_{\alpha\in \Sigma^+} (1-e^\alpha)^2.
\end{equation}
When $n=2k$, the root system $\Sigma$ is that of the endoscopic group
$H=\op{Sp}(2k,E)$, with $E/F$ quadratic unramified.  When $n=2k+1$,
the root system $\Sigma$ is the non-reduced root system $BC_k$ of rank
$k$ with underlying torus $\hat T_1$ and positive roots $\tau_i
\tau_j^{\pm 1}$, $\tau_i^2$, $\tau_i$.

\subsection{Type $B_n$}

The outer automorphism $\theta$ of $B_n$ is trivial.  We have $\hat S
= \hat T$ and the root system is $N_1\Psi =\Psi$.  We work in the
standard representation of $\op{SO}(2n+1)$ of dimension $2n+1$,
defined with respect to a symmetric form along the anti-diagonal $J =
\op{anti}(1,1,\ldots,1)$.  The extended Dynkin diagram has type $B_n$
at one end and has a fork of type $D_n$ at the other.  There is an
automorphism of order $2$ that exchanges the two prongs of the fork.
The morphism $\hat T\to \hat T_1=\ring{C}^{\times,n-1}$ is
\[
t = \op{diag}(t_1,\ldots,t_n,1,t_n^{-1},\ldots,t_1^{-1})
\mapsto \tau = (t_2,\ldots,t_n).
\]
In this case, the isogeny $\iota$ is trivial: $\hat U= \hat T_1$,
$\uu=\tau$.  Diagonalizing $t\dotw$, we obtain
\begin{equation}
\epsilon\phi(\tau) 
= \op{diag}(-t_2,\ldots,-t_n,-1,1,-1,-t_n^{-1},\ldots,-t_1^{-1}).
\end{equation}
We compute the determinant
\begin{equation}
D_+ = \prod_{\alpha\in \Sigma^+} (1-e^\alpha).
\end{equation}
Here $\Sigma$ is the non-reduced root system $BC_{n-1}$ of rank $n-1$,
with underlying torus $\hat T_1$ and positive roots $s_i s_j^{\pm 1}$,
$s_i^2$, $s_i$, with $1<i<j\le n$.

\subsection{Type $D_n$}\label{sec:dn}

Thus, we can confine our attention to classical automorphisms of the
extended Dynkin diagram, of order $2$ or $4$.  The extended Dynkin
diagram of $D_n$ is forked at both ends.  There are various
possibilities for $w\theta$, according to the symmetries of the
extended Dynkin diagram.  We compute all cases (other than outer
type).

We work in the standard representation $\op{SO}(2n)$ of dimension
$2n$, defined with respect to a symmetric form along the anti-diagonal
$J = \op{anti}(1,1,\ldots,1)$.

An outer automorphism $\theta$ exchanges the two prongs of one of the
forks.  When $\theta=1$, then $\hat S = \hat T$ and the root system is
$N_1\Psi = \Psi$.  When the outer automorphism $\theta$ is nontrivial,
then $\hat T\to \hat S =\ring{C}^{\times,n-1}$ is
$\op{diag}(t_1,\ldots,t_n,t_n^{-1},\ldots,t_1^{-1})\mapsto
(t_1,\ldots,t_{n-1})$.  In this case, the root system $\Sigma_1$ with
underlying torus $\hat S$ has type $C_{n-1}$ with positive roots $t_i
t_j^{\pm 1}$, for $1\le i\le j<n$.

Suppose the automorphism is of order $2$, preserving both ends, and
acting non-trivially on the prongs at both ends.  In this case
$\theta=1$ and $\hat S = \hat T$.  Then $\hat T\to \hat T_1 =
\ring{C}^{\times,n-2}$, where $(t_1,\ldots,t_n,\ldots)\mapsto \tau =
(t_2,\ldots,t_{n-1})$.  We can take the isogeny $\iota:\hat U\to \hat
T_1$ to be the identity map, and
\[
\epsilon\phi(\tau) 
= \op{diag}(t_2,\ldots,t_{n-1},-1,1,\ 1,-1,t_{n-1}^{-1},\ldots,t_2^{-1}).
\]
This case has inner type.  We compute the determinant and find that it
has the form described in Lemma \ref{lemma:inner}.

Next, suppose the automorphism is of order $2$, swapping the two ends.
In this case $\theta$ is trivial exactly when $n$ is even.  Write
$n=2k$ or $n=2k+1$, depending on the parity.  We have $\hat T\to \hat
T_1 = \ring{C}^{\times,k}$, where $(t_1,\ldots,t_n,\ldots)\mapsto
\tau=(t_1/t_n,t_2/t_{n-1},\ldots)$.  We can take the isogeny to be
$\iota:\hat U =\ring{C}^{\times,k}\to \hat T_1$, $\uu =
(\uu_2,\ldots,\uu_k)\mapsto \tau=(\uu_2^2,\ldots,\uu_k^2)$.
\begin{align*}
\epsilon \phi(\uu)
&= (\uu_1,-\uu_1,\ldots,\uu_k,-\uu_k) \in \hat T = \hat S, 
&(n=2k);\\
&= (\uu_1,-\uu_1,\ldots,\uu_k,-\uu_k)\in \hat S = \ring{C}^{\times,n-1}, 
&(n=2k+1).
\end{align*}
We compute the determinant
\begin{equation}
D_+ = \prod_{\alpha\in\Sigma^+} (1-e^\alpha)^2.
\end{equation}
Here $\Sigma$ is a root system of type $C_k$ with underlying torus
$\hat T_1$ and positive roots $s_i s_j^{\pm 1}$, for $1\le i\le j\le
k$.

Finally, assume that the automorphism has order $4$.  The automorphism
$\theta$ is trivial when $n$ is odd.  Write $n=2k+1$ or $n=2k$.  We
have $\hat T\to \hat T_1 = \ring{C}^{\times,k-1}$ given by
$(t_1,\ldots,t_n)\mapsto \tau = (t_2/t_{n-1},t_3/t_{n_2},\ldots)$.  We
take the isogeny $\iota:\hat U =\ring{C}^{\times,k-1}\to \hat T_1$, to
be given by $\uu = (\uu_2,\ldots,\uu_k)\mapsto \tau =
(\uu_2^2,\ldots,\uu_k^2)$.  Diagonalizing $t\dotw\theta$, we find
\begin{align}
\epsilon\phi(\uu) 
&= (\uu_2,-\uu_2,\ldots,\uu_k,-\uu_k,i)\in \hat S=\ring{C}^{n-1},
&(n=2k);\\
&= (\uu_2,-\uu_2,\ldots,\uu_k,-\uu_k,i,-1,1)\in \hat T = \ring{C}^n,
&(n=2k+1).
\end{align}
We compute the determinant
\begin{align*}
D_+ = \begin{cases}\prod_{\alpha\in \Sigma^+} 
(1-e^\alpha)^2 \prod_{\alpha\in \Sigma^+_{long}} (1+e^{\alpha/2})^2,&(n=2k);\\
          \prod_{\alpha\in\Sigma^+} (1-e^\alpha)^2,&(n=2k+1).
          \end{cases}
\end{align*}
In both cases, the root system $\Sigma$ has type $C_{k-1}$ on torus
$\hat T_1$.
