\XX{stray}

Here
$\dotw\in N_{\hat G}(\hat T)$ over $w\in W$ gives
$\theta_1=\dotw\theta$ determining the $L$-morphism ${}^LH\to {}^LG$
as described above.  

The next propositon gives a concrete representation of the image under
$\xi$ of a $\theta_H$-conjugacy class in ${}^LH$.  

Let $\xi:{}^LH\to {}^LG$ be an embedding of
  $L$-groups that factors over a finite unramifed extension $E/F$.

\section{X constructibility of B}




By Lemma~\ref{lemma:satake}, the change of basis
  $L_G(\lambda,\cdot)$ to the basis $\tau_\lambda$ is constructible.
  By Theorem \ref{thm:branch} and Lemma \ref{lemma:branch}, the
  restriction of $\tau_\lambda$ is expressed in terms of the basis
  $\sigma_\mu$ through constructible coefficients $m(\lambda,\mu)$:
\[
\phi^*_\epsilon\tau_\lambda = \sum_{\mu} m(\lambda,\mu) \sigma_\mu.
\]
By the Kato-Lusztig constructibility result, the change of basis from
$\sigma_\mu$ to $\hat f^H_\mu$ or $L_H(\lambda,\cdot)$ is
constructible.

Composition of change of basis is given
by matrix multiplication of the transformation matrices.  This matrix
multiplication is constructible by Remark \ref{rem:matrix}.  Thus the
transformation $b_\xi$ expresses each $L_G(\lambda,\cdot)$ as a linear
combination with constructible coefficients of the $L_H(\mu,\cdot)$.
In other words, it is a linear combination of some constructible
functions $d(\lambda,\mu)$, indexed by $\lambda$ and summed over
$\mu\in X_*(A_H)$.  This sum is a fiber integral of the constructible
function $d$ on $X_*(A_G)\times X_*(A_H)$ for the projection to
$X_*(A_H)$.  It is therefore a constructible function.

In fact, the support of $d$ lies in a definable set $E$ such that the
fibers of $E\to X_*(A_H)$ are finite, so there are no integrability
issues.