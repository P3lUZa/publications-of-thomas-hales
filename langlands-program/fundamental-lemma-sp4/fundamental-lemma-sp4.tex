% Paper: The fundamental lemma for $Sp(4)$
% Author: Thomas C. Hales
% Date: July 10, 1995
% Format: Ams-Tex, amsppt
% to be published by the Proc. of the AMS
% Reference Number:  PROC3546.


%\magnification=\magstep1
%\parskip=\baselineskip
%\baselineskip=1.15\baselineskip


%\input amstex
\documentclass{amsart}

\usepackage{amssymb}
\usepackage{amsthm}

%\loadmsbm
%\loadeufm

\newcommand\cL{{\mathcal L}}
\newcommand\g{{\mathfrak g}}
\newcommand\h{{\mathfrak h}}
\newcommand\cH{{\mathcal H}}
\newcommand\G{{\Bbb G}}
%\newcommand\qed{{\it q.e.d.}}

\parskip=1.15\baselineskip

\pretolerance=10000
\raggedbottom


%\rightheadtext{The fundamental lemma}
%\leftheadtext{Thomas C. Hales}

\begin{document}
%\topmatter\
\title{The Fundamental Lemma for $Sp(4)$}
%\endtitle
\author{Thomas C. Hales}
%\affil University of Michigan, Ann Arbor, Michigan\endaffil
\address{Ann Arbor, Michigan}
\thanks{Research supported by the National Science Foundation}
\date{final version: 3/20/1995}


\begin{abstract}
The fundamental lemma is a conjectural identity between the orbital
integrals on two reductive groups.  The fundamental lemma is required
for the stabilization of the trace formula and for various
applications to automorphic forms.  This paper proves the fundamental
lemma for the group $Sp(4)$ and its endoscopic groups.
\end{abstract}

\subjclass[2000]{22E50, 22E35, 20G25}

%\endtopmatter


\maketitle

%\vsize=0.85\vsize
%\parskip=0.4\baselineskip
%\parindent=0pt




Let $F$ be a $p$-adic field of characteristic zero with
ring of integers $O_F$.  Let $G$ be an 
 unramified reductive
group defined over $O_F$, and let $H$ be a standard
(i.e., untwisted) unramified
endoscopic group of $G$ (also defined over $O_F$). 
Fix an embedding $\xi: {}^LH\to {}^LG$ of $L$-groups, which
satisfies the hypotheses for unramified endoscopic data
in [H6,1].
The embedding $\xi$, by the Satake transform, determines a map
$b: \cH_G \to \cH_H$ between the spherical Hecke algebras $\cH_G$
and $\cH_H$ on $G$ and $H$.  We may assume that $\cH_G$ and $\cH_H$
are defined relative to the hyperspecial maximal compact subgroups
$G(O_F)$ and $H(O_F)$.

If $f\in C_c^\infty(G)$ 
and $\gamma_G$ is a strongly regular element of $G$,
we form the orbital integral $\Phi(\gamma_G,f)$ over the conjugacy
class of $\gamma_G$.  Similarly, for $f^H\in C_c^\infty(H)$ 
and strongly
$G$-regular elements $\gamma_H\in H(F)$, we form the orbital integral
$\Phi(\gamma_H,f^H)$.  Various normalizations of measures
enter into the definition of orbital integrals.  
Fixing invariant 
differential forms $\omega_G$
on $G$ and $\omega_T$ on $T=C_G(\gamma_G)$ (the centralizer
of $\gamma_G$) that are defined over $F$,
we may form the orbital integrals $\Phi(\gamma_G,f)$ with
respect to the measure $|\omega_G/\omega_T|$ on the orbit of $\gamma_G$.
We assume that the choices $\omega_T$ and $\omega_{T'}$ for various
Cartan subgroups $T$, $T'$ are conjugate over the algebraic closure
$\bar F$ of $F$.  We make similar selections of measures on $H$.

Let $\Delta(\gamma_H,\gamma_G)$ be the Langlands-Shelstad transfer
factor [LS2] with the canonical normalization of [H4,7].
We form the expression
\begin{equation}
\Lambda(\gamma_H,f) = \sum_{\gamma_G}\Delta(\gamma_H,\gamma_G)
\Phi(\gamma_G,f) - \sum_{\gamma'_H} \Phi(\gamma_H',b(f)). % \tag1
\end{equation}
The first sum runs over representatives of all regular
semisimple conjugacy classes in $G$ and the second sum runs
over representatives of the conjugacy classes in $H$ that are
stably conjugate to the class of $\gamma_H$.  Both sums have
only finitely many nonzero terms.  Both $b$ and $\Delta$ depend
on the choice of embedding $\xi$.

The measures on $G$ and $H$ must be compatible. This is
achieved by fixing a
strongly $G$-regular element $\gamma_H\in H(F)$ such that both
terms of the sum defining $\Lambda(\gamma_H,f)$ are nonzero for
some $f_0\in \cH_G$.  Rescale $|\omega_H|$, so as to modify the
second sum of 
(1) by a scalar so that $\Lambda(\gamma_H,f_0)=0$.  Equivalently, 
we may use the normalizations of [H4,14.2].

The following result, known as the fundamental lemma, is conjectured
to hold for any unramified 
reductive group $G$ in the setting described above.  
It is indispensable
to various applications of the trace formula. 

{\bf Theorem.}  Suppose $G = Sp(4)$.  Then $\Lambda(\gamma_H,f)=0$
for all $f\in \cH_G$ and all strongly $G$-regular elements $\gamma_H\in H(F)$.

{\bf Remark.}  The fundamental lemma for $GSp(4)$ follows from this
theorem together with a series of reductions made in [H6,3.6].
  The fundamental lemma for $GSp(4)$
was described in a lecture 
I gave at Luminy in 1992 as a consequence of the
results of J.-L.
Waldspurger on the homogeneity of Shalika germs and
my earlier work on $GSp(4)$ [H1], [Wa1], [Wa2].  We require
the more recent matching results of [H5] for $Sp(4)$.
A double coset argument discovered by M. Schr\"oder makes
it possible to give another proof of the fundamental lemma
for $GSp(4)$ [S1], [S2].  The details of this approach have been
carried out in a series of preprints by R. Weissauer [We].

A slightly stronger form of the theorem actually holds.
A general argument of Langlands and Shelstad shows that
$\gamma_H$ can be taken to be any $(G,H)$-regular element
[LS3,2.4].

{\bf Proof.}  We will begin with a general reductive group $G$ and
will introduce as hypotheses the parts of the proof of the fundamental
lemma that have been verified only in special cases.  Our approach
to the fundamental lemma is by induction on the dimension
of the group.  It
is therefore natural to assume that the fundamental
lemma is known for groups of smaller dimension.

Let $K$ be any number field that has a completion at some
place $v$ isomorphic to $F$.  Let $G'$ be any reductive
group over $K$ with endoscopic group $H'$ such that at
$v$ the pair $(G',H')$ is equivalent to a proper Levi factor of
$G$ together with the endoscopic group obtained from $H$
by descent, or to the connected centralizer of a noncentral
absolutely semisimple element in $G$ together with the
endoscopic group obtained from $H$ by descent.  
Absolutely semisimple elements, also called ${\Bbb F_q}$-semisimple
elements, are defined in [K].
See also [H4,3].

{\bf Hypothesis 1.}  For any of the reductive groups $G'$
and endoscopic groups $H'$ obtained as above,
the fundamental lemma is true at almost all places of $K$.

A narrower formulation of this hypothesis is possible.  It
is sufficient to verify the hypothesis for a single
carefully chosen number field $K$ (depending on $F$, $G$, and $H$).
Further, 
the fundamental lemma can be reduced to the adjoint
group of $G'$, provided unramified quasicharacters
of $G'_{adj}$ are introduced.  See [H6,3.6].
In the case of $Sp(4)$ the centralizers and Levi factors all have
ranks at most one, so the adjoint group, if not trivial, is $PGL(2)$.
This special case of the
fundamental lemma has several proofs,  for example, [K].

Hypothesis 1 allows us
to apply the results of [H6].  
In particular,
we may assume that the residual characteristic of $F$ is as large
as we please.  It also follows from [H6] that it is sufficient to
prove that $\Lambda(\gamma_H,1_G)=0$, where $1_G$ is the unit element
of the Hecke algebra $\cH_G$.  Of course, $b(1_G)$ is then
the  unit element
of $\cH_H$.  

By applying descent to Levi factors, we may assume that
$H$ is elliptic [H4,12].

Consider a strongly $G$-regular element $\gamma_H \in G(F)$.  If
$\gamma_H$ is not topologically unipotent, Kazhdan's lemma may be
invoked to reduce to a lower-rank case.  
The proof of this in [H4,13] makes the assumption that $G_{der}$ is
simply connected.  This restriction is unnecessary:
the result is proved in [Ko,7.1] without that restriction. 
Thus, we assume
that $\gamma_H$ is topologically unipotent.

Let $\g_{tn}$ be the set of topologically nilpotent elements of the
Lie algebra of $G$.  Let $G_{tu}$ be the set of topologically unipotent
elements of $G$.  Similarly, we consider $\h_{tn} = \text{Lie}(H)_{tn}$ and $H_{tu}$.
If the residual characteristic is sufficiently large,
then the exponential map gives an analytic isomorphism between
$\g_{tn}$ and $G_{tu}$.

Let $\h'\subset \h$ be the set of strongly $G$-regular elements
in $\h$.  In everything that follows, it is understood
that $X_H\in \h'$.  
If $X_H\in \h'_{tn} = \h'\cap \h_{tn}$, the
transfer factor
$\Delta(\exp(X_H),\gamma_G)$ is zero unless $\gamma_G$ is
also topologically unipotent.  So we may pass to the
Lie algebras of $G$ and $H$.

{\bf Hypothesis 2.}  Suppose that the residual characteristic is
sufficiently large.  For $|t|\le 1$, and $X_H\in \h'_{tn}$ fixed,
$\Lambda(\exp(t^2 X_H),1_G)$ is
a polynomial in $|t|$. 

It is known that for $X_H$ fixed, the transfer factor 
for $\exp(t^2 X_H)$ is a constant
times a power of $|t|$, if the residual characteristic is not two [H4].
Hypothesis 2 is then true for the classical groups by the results
of Waldspurger, which express orbital integrals 
of the unit element on the topologically
unipotent set as a sum of terms, each of which is homogeneous, that is,
a monomial in $|t|$.  See [Wa1], [Wa2,V]
 for a precise description of the
class of groups treated and the restrictions on the residual characteristic.
As a very special case of Waldspurger's results, 
we obtain Hypothesis
2 for $Sp(4)$.

Any polynomial in $|t|$ is zero if it is zero for $|t|$ sufficiently
small.  Hypothesis 2 allows us to restrict our attention to a small
neighborhood of the identity where the Shalika germ expansion is valid. 
Consider the Shalika germ expansion of the first sum defining $\Lambda$ when $X_H$ is sufficiently small in $\h'$.
Set $D(X_H) = |\prod_\alpha \alpha(X_H)|^{1/2}$, with the
sum running over the roots of $\h$ with respect to a Cartan
subalgebra containing $X_H$.
We have for $f\in C_c^\infty(G)$,
$$D(X_H)\sum_{X_G} \Delta(\exp(X_H),\exp(X_G))\Phi(\exp(X_G),f)=
 \sum_{O} \Gamma^G_O(X_H)\mu_O(f),$$
for some collection
of functions $\Gamma_{O}^G(X_H)$ defined in a suitable
neighborhood of $0$ in
$\h'_{tn}$.  Up to the factor $D$, 
the left-hand side is the first sum in (1) defining
$\Lambda$.
The functions $\Gamma^G_O$ are combinations
of Shalika germs on $G$.
The sum on the right runs over unipotent
classes in $G$.  We consider this as an identity which
defines $\Gamma_O^G$ in a small neighborhood of $0$ in $\h'$,
and then we extend the functions $\Gamma_O^G$ to all of $\h'$ by homogeneity.

Similarly, we consider a {\it stable} Shalika germ expansion 
on $\h'_{tn}$.  For $X_H$ sufficiently small, 
and $f^H\in C_c^\infty(H)$, write the second sum in (1) in the form
$$D(X_H)\sum_{X_H'} \Phi(\exp(X_H'),f^H) = \sum_{O'}
\Gamma_{O'}^H (X_H)\mu_{O'} (f^H),$$
where the sum on the right
now runs over unipotent conjugacy classes of $H$.
The functions $\Gamma_{O'}^H$ are stable versions of
Shalika germs.
We extend these functions by homogeneity to $\h'$.

{\bf Hypothesis 3.}  There exists a linear map 
$b_0:C_c^\infty(G) \to C_c^\infty(H)$,
$f\mapsto b_0(f)$ satisfying
$$\sum_{O} \Gamma^G_O (X_H) \mu_O(f) = \sum_{O'}\Gamma^H_{O'}(X_H)
\mu_{O'}(b_0(f)),$$
for all $X_H\in \h'$.

This hypothesis is called {\it local $\Delta$-transfer
at the identity\/}
 in [LS3].  It has
been verified for $GSp(4)$ in [H1] and for $Sp(4)$ in [H5]
when the residual characteristic is odd.
Hypothesis 3 is stronger than what is
required for the fundamental lemma. 

For each $X_H$, the right-hand side of the identity of Hypothesis 3
defines an invariant distribution on $C_c^\infty(H)$, which is supported on the
unipotent set, namely
\begin{equation}
f^H \mapsto \sum_{O'} \Gamma^H_{O'}(X_H)\mu_{O'}(f^H).
%\tag2
\end{equation}
Let $\cL$ be the span of these distributions.
The linear map $b_0$ gives, by duality, a linear map 
$\mu\mapsto\mu^G$ from $\cL$ 
to invariant distributions
supported on the unipotent set of $G$.  Specifically, for $\mu\in\cL$,
define $\mu^G$ by $\mu^G(f) = \mu(b_0(f))$, for $f\in C_c^\infty(G)$.

{\bf Hypothesis 4.}  $\mu^G(1_G) = \mu(b(1_G))$, for all $\mu \in \cL$.

If $X_H$ is not elliptic, the unipotent distribution $\mu$
 on $C^\infty_c(H)$ defined by
 (2) satisfies the condition of Hypothesis 4 by
descent and induction (Hypothesis 1). See [H4,12].
Call these distributions {\it stably induced.}

{\bf Proposition.}  If Hypotheses 1, 2, 3, and 4 
hold, then the fundamental lemma is true for $G$ and the
endoscopic group $H$.

We emphasize that all four hypotheses are assertions
about what happens almost everywhere for various number
fields $K$.  In Hypothesis 1, this is explicit.  After
Hypothesis 1, we shifted notation, so that $F$ is no
longer a fixed $p$-adic field for which we wish to prove
the fundamental lemma.  It becomes an indefinite $p$-adic field
of sufficiently large residual characteristic, obtained
by completing various number fields $K$.

{\bf Proof.} It is clear by the preceding remarks that
the first two hypotheses reduce the fundamental lemma
to a statement in a small neighborhood of the identity.
We must verify that $b_0(1_G)$ has the same stable
germ expansion as $b(1_G)$, that is, that $\mu(b_0(1_G))=
\mu(b(1_G))$ for all $\mu\in\cL$.  Hypothesis 4 asserts
this, and the result follows. \qed

A unipotent class $O$ is $r$-{\it regular} if $2r=\dim(C_G(u))-
\text{rank}(G)$, for $u\in O$.  The $0$-regular classes are 
regular, and the $1$-regular classes are subregular.
We call the partial sum 
$$\sum_O\Gamma_O^G(X_H)\mu_O(1_G),$$ where the sum ranges
over all $r$-regular unipotent classes $O$ of $G$, the
{\it $r$-regular term\/} on $G$.  Similarly, the corresponding partial
sum
$$\sum_{O'}\Gamma_{O'}^H(X_H)\mu_{O'}(b(1_G))$$
over $r$-regular orbits $O'$ will be called the
{\it stable $r$-regular term\/} on $H$.  This
distinction between {\it term\/} and {\it germ\/} is
crucial. To verify Hypothesis
4, we must show that the $r$-regular terms on $G$ and $H$
coincide for a finite set of elements $\{X_H\}\subset \h'$ whose
distributions (2) span $\cL$.

\bigskip

Now we restrict our attention to the group $G=Sp(4)$.  
We may assume that the residual characteristic is odd.
The elliptic unramified endoscopic groups
of $G$ are $SO(4)$, the quasisplit form $SO^*(4)$ of $SO(4)$ 
that splits over
an unramified quadratic extension $E/F$, and the product
$SL(2)\times U_E(1)$, where $U_E(1)$ denotes a one-dimensional
nonsplit torus that is split by $E$,
again with $E/F$ an unramified quadratic extension.
When $H=SO(4)$, the regular and both subregular classes give stably
induced distributions $\mu$, so that, by our earlier
comment, Hypothesis 4 is satisfied for
them.  The remaining unipotent conjugacy class of $SO(4)$ is treated
in Lemma 1.  When $H=SO^*(4)$, the regular class is stably
induced. There is no subregular class with
rational points. Again, it is enough to consider the two-regular
class.  If $H=SL(2)\times U_E(1)$, the regular class is
stably induced. It is necessary to consider
the subregular unipotent class.  (In this case, the two-regular
germs vanish [H5].)

We take an image of $X_H$ in $Sp(4)$
and let $\pm t_1$ and $\pm t_2$ in $\bar F$ be the eigenvalues
of the image in $Sp(4)\subset GL(4)$.

The two-regular term in $GSp(4)$
is a product of three factors:

(i)  The constant $A_1(M)$ of Langlands [L,page 470]

(ii)  The Shalika germ. Formulas appear in [H1,6]

(iii) The unipotent orbital integral $\mu_O(1_G)$ computed
relative to the measure 
$|\omega|=|z\,dz\,dy_1\,dy_2\,dy_3|$ with coordinates
on an open set of $O$:
\begin{equation}
\begin{pmatrix} 1&0&0&0\\ y_1&1&0&0 \\
          y_2&0&1&0 \\ y_3&y_2&-y_1&1 \end{pmatrix}
\begin{pmatrix} 1&0&0&z\\ 0&1&0&0 \\
          0&0&1&0 \\ 0&0&0&1 \end{pmatrix}
\begin{pmatrix} 1&0&0&0\\ -y_1&1&0&0 \\
          -y_2&0&1&0 \\ -y_3&-y_2&y_1&1 \end{pmatrix}
\subseteq O.
%\tag3
\end{equation}
Here $Sp(4)$ is represented as $4\times4$ matrices preserving
the form $x_1\wedge x_4 + x_2\wedge x_3$.

The constant $A_1(M)$ evaluates, by the
formula of Langlands, to $1/2$.  (This factor of $1/2$
stems from the fact that the irreducible divisor that gives
the contribution to the two-regular term has multiplicity 2 in
the Igusa variety.)

We define a constant $[G]$, for any reductive group $G$ over
a finite field $k$, to be the cardinality of $G(k)$ divided by
$q^d$, where $d = \dim(G)$, and $q$ is the cardinality of $k$.
For example, the multiplicative group $\G_m$
gives $[\G_m] = (1-1/q)$, $[SO(4)] = (1-1/q^2)^2$,
and $[Sp(4)] = (1-1/q^2)(1-1/q^4)$. If $G$ is defined over
$F$, we also let $[G]$ denote the corresponding constant
over the residue field $k$ of $F$.
As in [H4,14], we normalize the unit element of the Hecke
algebra of (any) $G$ 
to be $\text{\it ch}/[G]$, where {\it ch\/} is the
characteristic function of $G(O_F)$.  

If $G = SL(2)$, then
the stable subregular term is the product of the Shalika
germ
and the subregular orbital integral. The subregular
orbital integral is merely
$f\mapsto f(1)$.  Let $t=t(X)$ be an eigenvalue of 
$X\in {\mathfrak sl}(2)$.
By [LS1], the subregular term evaluated at $X$ is then
$$|t|I_F/[SL(2)],$$
where the constant $I_F$ is expressed as a principal-value
integral
$$I_F = \int_{Q(F)} |\omega_Q|.$$
The surface $Q$ is a twisted form of a product of two projective lines.
The volume form $\omega_Q$ is $da\wedge db/(a-b)^2$, for
an appropriate choice of coordinate $a$, $b$.  See [LS1] for details.
The surface $Q$ and integral depend on a choice of 
Cartan subalgebra containing $X$,
although
the notation does not show this.  

Since ${\mathfrak so}(4)$ is isomorphic to a direct sum of
two copies of ${\mathfrak sl}(2)$, we may also conclude from
the $SL(2)$ calculation that
the stable two-regular term of $SO(4)$, evaluated at
$X_H\in {\mathfrak so}(4) '$, is
\begin{equation}
|t_1^2 -t_2^2| I_F^2/[SO(4)].
%\tag4
\end{equation}
Similarly, since ${\mathfrak so}^*(4)$ is a restriction of
scalars of ${\mathfrak sl}(2)$ over $E$, we conclude that the
stable
two-regular term of $SO^*(4)$, evaluated at 
$X_H\in {\mathfrak so}^*(4)'$,
is
\begin{equation}
|t_1^2 -t_2^2| I_E/[SO^*(4)].
%\tag5
\end{equation}
($t_1-t_2$ and $t_1+t_2$ are to be identified with the
positive roots of ${\mathfrak so}(4)$.)
For similar reasons the stable subregular term of
$SL(2)\times U_E(1)$ at $X_H$ is
\begin{equation}
|t_2| I_F/[SL(2)\times U_E(1)].
%\tag6
\end{equation}
($2t_2$ is to be identified with the positive root of ${\mathfrak sl}(2)
\times {\mathfrak u}_E(1)$.)

To prove Hypothesis 4 we must show that these expressions
coincide with the corresponding terms in $Sp(4)$.
We begin with the endoscopic group $SO(4)$.

{\bf Lemma 1.}  If $H=SO(4)$, then the two-regular term of
$G$, for $X_H$ elliptic, is
$|t_1^2 -t_2^2| I_F^2/[SO(4)]$.

{\bf Proof.}   For
this endoscopic
group we may rely
on the results of [H1].  In $GSp(4)$ there is a single
two-regular unipotent conjugacy class.  

It follows from Hypothesis 3 and Expression 4 that the formula is
correct up to a nonzero scalar. To check the scalar,
we pick any convenient elliptic element $X_H$.
Consider the Cartan subgroup $(U_E(1)\times U_E(1))/\{\pm1\}$ 
in $H$,
with $E/F$ unramified. Fix $X_H$ in its Lie algebra.

In the notation of [H1,6], we set
$w' = w_2 \xi_1 \epsilon/f$, where $\epsilon$ is a unit in $E$ of
trace zero.  Also set $\sigma = \sigma_0\sigma_\alpha\sigma_\beta
\sigma_\alpha\sigma_\beta$.  Let
$\eta$ be the unramified quadratic character of $F^\times$.
Equation 6.8 of [H1] gives the
formula
$$|t_1^2 -t_2^2| \int |dw'/w'| |d \ell_2/\ell_2| \eta(1-\ell_2^2)
\int_{Q(F)} |d\xi_1 d\xi_2/(\xi_1-\xi_2)^2|,$$
for the Shalika germ, where $\sigma$ acts on the coordinates
by $\sigma(w')=1/w'$, $\sigma(\ell_2)= \ell_2$, $\sigma(\xi_1)=-1/\xi_2$,
and $\sigma(\xi_2)=-1/\xi_1$.
All integrals extend over ${\Bbb P}^1$ unless indicated
otherwise. The second integral is $I_F$.
Section 11 of [H2] shows that the
first integral is equal to $2I_F$.
The germ is then $2I_F^2|t_1^2 -t_2^2|$.

The integral $\mu_O(1_G)$ is computed by the method of [H2,12].
Details of closely related calculations are 
also found in [H3,3.9], so
we will simply state the result.  We find
$$\mu_O(1_G) = (1-q^{-2})^{-2} = 1/[SO(4)].$$
The product of the factors (i), (ii), and (iii)
is then
$$|t_1^2 -t_2^2| I_F^2/[SO(4)],$$
as desired.  This completes Lemma 1.
Since this term coincides with the stable two-regular term (4)
of $H$, this also completes the proof of the fundamental
lemma for the endoscopic group $SO(4)$. \qed

Now we turn to the two-regular term 
in the nonsplit case $H = SO^*(4)$. Again it is the
product of three factors: the constant
$A_1(M) = 1/2$, the Shalika germ, and a unipotent
orbital integral $\mu^\kappa_O(1_G)$.
The germ is given by [H5,5.2]. It is
$2 |t_1^2 -t_2^2| I_E$.
Again, by the methods of [H2],
the unipotent orbital integral evaluates to
$$\mu_O^\kappa (1_G) = (1-1/q^2)/[Sp(4)] = 1/[SO^*(4)].$$
(As in [H5], the relevant measure to take is
$$\mu_O^\kappa (1_G) = \int_O \eta (z) |\omega|,$$
where $z$ and $|\omega|$ are as in (3).)
This  is a $\kappa$-orbital
integral on the two-regular conjugacy class in $Sp(4)$. 
The result is two-regular term on $G$
$$|t_1^2 - t_2^2| I_E/[SO^*(4)].$$
This is precisely the stable two-regular
term (5) on $SO^*(4)$.

Finally, we turn to the endoscopic group $H = SL(2) \times U_E(1)$.
We must compare the subregular terms of $G$ and $H$.
The subregular term on $Sp(4)$ comes from
a single $GSp(4)$-orbit (which breaks into two orbits in $Sp(4)$).
Recall that the subregular 
unipotent orbits in $GSp(4)$ are parametrized
by quadratic extensions.
The $GSp(4)$-orbit that enters here is 
the one that corresponds to the unramified quadratic
extension $E/F$.  We label the two orbits in $Sp(4)$ as
$O_+$ and $O_-$.  We consider the germ for the Cartan subgroup
$U_E(1)\times U_E(1)$ of $Sp(4)$.  In the notation of [H1,page
 230],
the Shalika germ
is
$$\int|d\xi/\xi| \int \eta(w^2/w_B)|dw/w^2|.$$
Under the substitution $w = x/(t_2 x+t_2)$, this becomes
$$|t_2| \int |d\xi/\xi| 
\int \eta(1-x^2) |dx/x^2| = 2 |t_2| I_F.$$
(The last equality follows from [H2,11].)
The subregular term on $G$ is then 
$$ 2 |t_2| I_F [\mu_{O_+}(1_G) - \mu_{O_-} (1_G)].$$
Rather than evaluate these unipotent orbital integrals directly,
we note that with a different normalization of measures, they
have already been evaluated by Assem [A,2.3.17].  We add primes to
the distributions (and the characteristic function $1_G$) to
indicate Assem's normalizations.  Assem's result is
\begin{equation}
\mu'_{O_+}(1_G') - \mu'_{O_-}(1_G') = {1-q^{-1}\over 2(1+q^{-1})} =
{[\G_m]\over 2[U_E(1)]}.
%\tag7
\end{equation}
(The orbits $O_+$ and $O_-$ are denoted $(\epsilon,1)$
and $(\epsilon,-1)$ in [A].)
To compare the normalizations of measures, we work with a linear
combination that is easier to treat.  The sum over all the subregular
unipotent classes is the Richardson orbit of a parabolic subgroup.
The corresponding distribution $\sum \mu_O$ is easy to compute.
We have
\begin{equation}
{\sum \mu_O (1_G)\over \sum \mu'_O(1'_G)} =
  {\mu_{O_+}(1_G) - \mu_{O_-}(1_G) \over 
   \mu'_{O_+} (1'_G) - \mu'_{O_-} (1'_G)}.
%\tag8
\end{equation}
Assem finds that $\sum\mu_O'(1_G') = 1$.
By parabolic descent we find $\sum\mu_O(f) = \bar f^P(1)$, where
$\bar f^P$ is the usual function on the Levi of $P=MN$ obtained by
descent.  For the function $f = 1_G$, this gives
$\bar f^P(1) = 1/[M] = 1/[SL(2)\times \G_m]$.
Thus, (7) and (8) give
$$\mu_{O_+}(1_G) - \mu_{O_-}(1_G) = 
  {1\over 2} {1 \over [SL(2)\times \G_m]} {[\G_m]\over [U_E(1)]} =
{1\over 2[H]}.$$
The subregular term of $G$ is then
$|t_2|I_F/[H]$.  This is precisely the stable subregular 
term (6)
on $H$.  The proof of the fundamental lemma 
for $Sp(4)$ is complete. \qed

\bigskip
\centerline{\bf References}
\bigskip

{
\baselineskip = 0.8\baselineskip
\parskip=.3\baselineskip
\everypar={\hangindent=.5in\hangafter=1}

[A]  M. Assem, Unipotent orbital integrals of spherical
	functions on $p$-adic $4\times 4$ symplectic
	groups, preprint.

[H1] T. Hales, Shalika Germs on $GSp(4)$, Ast\'erisque 171--172,
	(1989), 195--256.

[H2] T. Hales, Orbital integrals on $U(3)$, The Zeta Function
	of Picard Modular Surfaces, Les Publications CRM,
	(R.P. Langlands and D. Ramakrishnan, eds.), (1992).

[H3] T. Hales, Unipotent representations and unipotent classes
	in $SL(n)$, Amer. J. Math. {\bf 115}, no. 6, (1993),
	1347--1383.

[H4] T. Hales, A simple definition of transfer factors for unramified
	groups, Contemporary Math {\bf 145} (1993), 109--134.

[H5] T. Hales, Twisted endoscopy of $GL(4)$ and $GL(5)$: transfer
	of Shalika germs, Duke Math. J. {\bf 76}, no. 2, (1994),
	595--632.

[H6] T. Hales, The fundamental lemma for standard endoscopy:
	reduction to unit elements, to appear in the Canad. J. Math.

[K] D. Kazhdan, On lifting, Lie group representations, II,
	Lecture Notes in Math., vol 1041, Springer-Verlag,
	New York, 1984.

[Ko] R. Kottwitz, Stable trace formula: elliptic singular
	terms, Math. Ann. {\bf 275}, (1986), 365--399.

[L]   R. Langlands, Orbital integrals on forms of $SL(3)$, I,
	Amer. J. Math. {\bf 105}, (1983), 465--506.

[LS1] R. Langlands, D. Shelstad, On principal values on $p$-adic
	manifolds, Lie Group Representations II, 
	Lecture Notes in Math. vol 1041, Springer-Verlag, 1984.

[LS2]  R. Langlands, D. Shelstad,  On the definition of transfer factors, Math. Ann.
	{\bf 278}, (1987), 219--271.

[LS3]  R. Langlands, D. Shelstad,  Descent for transfer factors, The Grothendieck
	festschrift, Progress in Math., Birkh\"auser, 1990.

[S1]  M. Schr\"oder, Z\"ahlen der Punkte mod $p$ einer
	Shimuravariet\"at zu $GSp(4)$, thesis, Mannheim, 1993.

[S2]  M. Schr\"oder, Calculating $p$-adic orbital integrals on groups
	of symplectic similitudes in four variables,
	preprint.

[Wa1]  J.-L. Waldspurger, Quelques r\'esultats de finitude
	concernant les distributions invariantes sur
	les alg\`ebres de Lie $p$-adiques, preprint.


[Wa2] J.-L. Waldspurger, Homog\'en\'eit\'e de certaines
	distributions sur les groupes $p$-adiques, preprint.

[We]  R. Weissauer, A special case of the fundamental
	lemma, Parts I, II, III, IV, preprints.

}


\end{document}

